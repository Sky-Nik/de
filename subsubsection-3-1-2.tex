\begin{theorem}
	Якщо $y = y_1(x)$ є розв'язком однорідного лінійного рівняння, то і $y = C y_1 (x)$, де $C$ --- довільна стала, теж буде розв'язком однорідного лінійного рівняння.
\end{theorem}

\begin{proof}
	Справді, нехай $y = y_1(x)$ --- розв'язок лінійного однорідного рівняння, тобто
	\begin{equation*}
		a_0(x) y_1^{(n)} (x) + a_1(x) y_1^{(n - 1)} (x) + \ldots + a_n(x) y_1(x) \equiv 0.
	\end{equation*}

	Тоді і
	\begin{multline*}
		a_0(x) (C y_1)^{(n)}(x) + a_1(x) (C y_1)^{(n - 1)}(x) + \ldots + a_n(x) (C y_1)(x) = \\
		= C \left( a_0(x) y_1^{(n)} (x) + a_1(x) y_1^{(n - 1)} (x) + \ldots + a_n(x) y_1(x) \right) \equiv 0,
	\end{multline*}
	оскільки вираз в дужках дорівнює нулю.
\end{proof}

\begin{theorem}
	Якщо $y_1(x)$ і $y_2(x)$ є розв'язками лінійного однорідного рівняння, то і $y = y_1(x) + y_2(x)$ теж буде розв'язком лінійного однорідного рівняння.
\end{theorem}

\begin{proof}
	Справді, нехай $y_1(x)$ і $y_2(x)$ --- розв'язки лінійного рівняння, тобто
	\begin{align*}
		a_0(x) y_1^{(n)} (x) + a_1(x) y_1^{(n - 1)} (x) + \ldots + a_n(x) y_1(x) &\equiv 0, \\
		a_0(x) y_2^{(n)} (x) + a_1(x) y_2^{(n - 1)} (x) + \ldots + a_n(x) y_2(x) &\equiv 0.
	\end{align*}

	Тоді і
	\begin{multline*}
		a_0(x) (y_1 + y_2)^{(n)} (x) + a_1(x) (y_1 + y_2)^{(n - 1)} (x) + \ldots + a_n(x) (y_1 + y_2) (x) = \\
		= \left( a_0(x) y_1^{(n)} (x) + a_1(x) y_1^{(n - 1)} (x) + \ldots + a_n(x) y_1(x) \right) + \\
		+ \left( a_0(x) y_2^{(n)} (x) + a_1(x) y_2^{(n - 1)} (x) + \ldots + a_n(x) y_2(x) \right) \equiv 0,
	\end{multline*}
	оскільки обидві дужки дорівнюють нулю.
\end{proof}

\begin{theorem}
	Якщо $y_1(x), y_2(x), \ldots, y_n(x)$ --- розв'язки однорідного лінійного рівняння, то і  $y = \sum_{i=1}^n C_i y_i(x)$, де $C_i$ --- довільні сталі, також буде розв'язком лінійного однорідного рівняння.
\end{theorem}

\begin{proof}
	Справді, нехай $y_1(x), y_2(x), \ldots, y_n(x)$ --- розв'язки лінійного однорідного рівняння, тобто
	\begin{equation*}
		a_0(x) y_i^{(n)} (x) + a_1(x) y_i^{(n - 1)} (x) + \ldots + a_n(x) y_i(x) \equiv 0, \quad i = \overline{1, n}.
	\end{equation*}
	
	Тоді і   
 	\begin{multline*}
 		a_0(x) \left(\sum_{i=1}^n C_i y_i\right)^{(n)} (x) + a_1(x) \left(\sum_{i=1}^n C_i y_i\right)^{(n - 1)} (x) + \ldots \\
 		\ldots + a_{n-1}(x) \left(\sum_{i=1}^n C_i y_i\right)'(x) + a_n(x) \left(\sum_{i=1}^n C_i y_i\right)(x) = \\
 		= \sum_{i=1}^n C_i  \left( a_0(x) y_i^{(n)} (x) + a_1(x) y_i^{(n - 1)} (x) + \ldots + a_n(x) y_i(x) \right) \equiv 0,
 	\end{multline*}
	оскільки кожна дужка дорівнює нулю.
\end{proof}

\begin{theorem}
	Якщо комплексна функція дійсного аргументу, тобто $y = u(x) + i v(x)$ є розв'язком лінійного однорідного рівняння, то окремо дійсна частина $u(x)$ і уявна $v(x)$ будуть також розв'язками цього рівняння.
\end{theorem}

\begin{proof}
	Справді, нехай $y = u(x) + i v(x)$ є розв'язком лінійного однорідного рівняння, тобто
	\begin{multline*}
		a_0(x) (u) + i v)^{(n)} (x) + a_1(x) (u + i v)^{(n - 1)} (x) + \ldots \\
		\ldots + a_{n - 1}(x) (u + i v)' (x) + a_n(x) (u + i v) (x) \equiv 0.
	\end{multline*}

	Розкривши дужки і перегрупувавши члени, одержимо
	\begin{multline*}
		\left( a_0(x) u^{(n)}(x) + a_1(x) u^{(n - 1)} (x) + \ldots + a_n(x) u(x) \right) + \\
		+ i \left( a_0(x) v^{(n)}(x) + a_1(x) v^{(n - 1)} (x) + \ldots + a_n(x) v(x) \right) \equiv 0.
	\end{multline*}

	Комплексний вираз дорівнює нулю тоді і тільки тоді, коли дорівнюють нулю дійсна і уявна частини, тобто
 	\begin{align*}
		a_0(x) u^{(n)}(x) + a_1(x) u^{(n - 1)} (x) + \ldots + a_n(x) u(x) &\equiv 0, \\
		a_0(x) v^{(n)}(x) + a_1(x) v^{(n - 1)} (x) + \ldots + a_n(x) v(x) &\equiv 0,
	\end{align*}
	або функції $u(x)$, $v(x)$ є розв'язками рівняння, що і було потрібно довести.
\end{proof}
