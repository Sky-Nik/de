Розглянемо ряд диференціальних рівнянь, що інтегруються в квадратурах.
\begin{enumerate}
	\item Рівняння вигляду 
	\begin{equation*}
		%\label{eq:1.5.2}
		F(y') = 0.
	\end{equation*}
	Нехай алгебраїчне рівняння $F(k) = 0$ має принаймні один дійсний корінь $k = k_0$. Тоді, інтегруючи $y' = k_0$, одержимо $y = k_0 \cdot x + C$. Звідси $k_0 = (y - C) / x$ і вираз
	\begin{equation*}
		%\label{eq:1.5.3}
		F \left( \frac{y - c}{x} \right) = 0	
	\end{equation*}
	містить всі розв’язки вихідного диференціального рівняння.
	\item Рівняння вигляду 
	\begin{equation*}
		%\label{eq:1.5.4}
		F(x, y') = 0.
	\end{equation*}
	Нехай це рівняння можна записати у параметричному вигляді
	\begin{equation*}
		%\label{eq:1.5.5}
		\left\{\begin{aligned}
			x &= \phi(t), \\
			y' &= \psi(t).
		\end{aligned}\right.
	\end{equation*}
	Використовуючи співвідношення $\diff y = y ' \cdot \diff x$, одержимо 
	\begin{equation*}
		%\label{eq:1.5.6}
		\diff y = \psi(t) \cdot \phi'(t) \cdot \diff t.
	\end{equation*}
	Проінтегрувавши, запишемо
	\begin{equation*}
		%\label{eq:1.5.7}
		y = \int \psi(t) \cdot \phi'(t) \cdot \diff t + C.
	\end{equation*}
	І загальний розв’язок в параметричній формі має вигляд
	\begin{equation*}
		%\label{eq:1.5.8}
		\left\{\begin{aligned}
		x &= \phi(t), \\
		y &= \int \psi(t) \cdot \phi'(t) \cdot \diff t + C.
		\end{aligned}\right.
	\end{equation*}
	\item Рівняння вигляду 
	\begin{equation*}
		%\label{eq:1.5.9}
		F(y, y') = 0.
	\end{equation*}
	Нехай це рівняння можна записати у параметричному вигляді
	\begin{equation*}
		%\label{eq:1.5.10}
		\left\{\begin{aligned}
			y &= \phi(t), \\
			y' &= \psi(t).
		\end{aligned}\right.
	\end{equation*}
	Використовуючи співвідношення $\diff y = y ' \cdot \diff x$, одержимо 
	\begin{equation*}
		%\label{eq:1.5.11}
		\phi'(t) \cdot \diff t = \psi(t) \cdot \diff x
	\end{equation*}
	і
	\begin{equation*}
		%\label{eq:1.5.12}
		\diff x = \frac{\phi'(t)}{\psi(t)} \cdot \diff t
	\end{equation*}
	Проінтегрувавши, запишемо
	\begin{equation*}
		%\label{eq:1.5.13}
		x = \int \frac{\phi'(t)}{\psi(t)}\cdot \diff t + C.
	\end{equation*}
	І загальний розв’язок в параметричній формі має вигляд
	\begin{equation*}
		%\label{eq:1.5.14}
		\left\{\begin{aligned}
		x &= \int \frac{\phi'(t)}{\psi(t)}\cdot \diff t + C, \\
		y &= \phi(t).
		\end{aligned}\right.
	\end{equation*}
	\item Рівняння Лагранжа
	\begin{equation*}
		%\label{eq:1.5.15}
		y = \phi(y') \cdot x + \psi(y').
	\end{equation*}
	Введемо параметр $y' = \frac{\diff y}{\diff x} = p$ і отримаємо
	\begin{equation*}
		%\label{eq:1.5.16}
		y = \phi(p) \cdot x + \psi(p).
	\end{equation*}
	Продиференціювавши, запишемо
	\begin{equation*}
		%\label{eq:1.5.17}
		\diff y = \phi'(p) \cdot x \cdot \diff p + \phi(p) \cdot \diff x + \psi'(p) \cdot \diff p.
	\end{equation*}
	Замінивши $\diff y = p \cdot \diff x$ одержимо
	\begin{equation*}
		%\label{eq:1.5.18}
		p \cdot \diff x = \phi'(p) \cdot x \cdot \diff p + \phi(p) \cdot \diff x + \psi'(p) \cdot \diff p.
	\end{equation*}
	Звідси
	\begin{equation*}
		%\label{eq:1.5.19}
		(p - \phi(p)) \cdot \diff x - \phi'(p) \cdot x \cdot \diff p = \psi'(p) \cdot \diff p.
	\end{equation*}
	І отримали лінійне неоднорідне диференціальне рівняння
	\begin{equation*}
		%\label{eq:1.5.20}
		\frac{\diff x}{\diff p} + \frac{\phi'(p)}{\phi(p)-p} \cdot x = \frac{\phi'(p)}{p-\phi(p)}.
	\end{equation*}
	Його роз\-в'яз\-ок
	\begin{multline*}
		%\label{eq:1.5.21}
		x = \exp\left\{\int \frac{\phi'(p)}{p-\phi(p)} \cdot \diff p\right\} \cdot \\
		\cdot \left(\int \frac{\phi'(p)}{p-\phi(p)} \cdot \exp\left\{\int \frac{\phi'(p)}{\phi(p)-p} \cdot \diff p\right\} \diff p + C \right) = \Psi(p, C).
	\end{multline*}
	І остаточний розв’язок рівняння Лагранжа в параметричній формі запишеться у вигляді
	\begin{equation*}
		%\label{eq:1.5.22}
		\left\{\begin{aligned}
			x &= \Psi(p,C), \\
			y &= \phi(p) \cdot \Phi(p, C) + \psi(p).
		\end{aligned}\right.
	\end{equation*}
	\item Рівняння Клеро. \\

	Частинним випадком рівняння Лагранжа, що відповідає $\phi(y') = y'$ є рівняння Клеро
 	\begin{equation*}
 		%\label{eq:1.5.23}
 		y = y' x + \psi(y').
 	\end{equation*}
	Поклавши $y' = \frac{\diff y}{\diff x} = p$, отримаємо $y = p x + \psi(p)$. Продиференціюємо 
	\begin{equation*}
		%\label{eq:1.5.24}
		\diff y = p \cdot \diff x + x \cdot \diff p + \psi'(p) \cdot \diff p.
	\end{equation*}
	Оскільки $\diff y = p \cdot \diff x$, то
	\begin{equation*}
		%\label{eq:1.5.25}
		p \cdot \diff x = p \cdot \diff x + x \cdot \diff p + \psi'(p) \cdot \diff p.
	\end{equation*}
	Скоротивши, одержимо
	\begin{equation*}
		%\label{eq:1.5.25}
		(x + \psi'(p)) \cdot \diff p = 0.
	\end{equation*}
	Можливі два випадки.
	\begin{enumerate}
		\item $x + \psi'(p) - 0$ і розв’язок має вигляд
		\begin{equation*}
			%\label{eq:1.5.26}
			\left\{\begin{aligned}
				x &= - \psi'(p), \\
				y &= -p \cdot \psi'(p) + \psi(p).
			\end{aligned}\right.
		\end{equation*}
		\item $\diff p = 0$, $p = C$ і розв’язок має вигляд
		\begin{equation*}
			%\label{eq:1.5.27}
			y = C x + \psi(C).
		\end{equation*}
	\end{enumerate}
	Загальним розв’язком рівняння Клеро буде сім'я ``прямих''. Її огинає особлива крива.
	\item Параметризація загального вигляду. Нехай диференціальне рівняння $F(x, y, y') = 0$ вдалося записати у вигляді системи рівнянь з двома параметрами
	\begin{equation*}
		%\label{eq:1.5.28}
		x = \phi(u, v), \quad y = \psi(u, v), \quad y' = \theta(u, v).	
	\end{equation*}
	Використовуючи співвідношення $\diff y = y' \cdot \diff x$, одержимо
	\begin{multline*}
		%\label{eq:1.5.29}
		\frac{\partial \psi(u,v)}{\partial u} \cdot \diff u + \frac{\partial \psi(u, v)}{\partial v} \cdot \diff v = \\
		= \theta(u,v) \cdot \left( \frac{\partial \phi(u,v)}{\partial u} \cdot \diff u + \frac{\partial \phi(u, v)}{\partial v} \cdot \diff v\right)
	\end{multline*}
	Перегрупувавши члени, одержимо
	\begin{multline*}
		%\label{eq:1.5.30}
		\left( \frac{\partial \psi(u,v)}{\partial u} - \theta(u, v) \cdot \frac{\partial \phi(u,v)}{\partial u} \right) \diff u = \\
		= \left( \theta(u,v) \cdot \frac{\partial \phi(u, v)}{\partial v} - \frac{\partial \psi(u, v)}{\partial v} \right) \diff v.
	\end{multline*}
	Звідси
	\begin{equation*}
		%\label{eq:1.5.32}
		\frac{\diff u}{\diff v} = \frac{\theta(u,v) \cdot \frac{\partial \phi(u, v)}{\partial v} - \frac{\partial \psi(u, v)}{\partial v}}{\frac{\partial \psi(u,v)}{\partial u} - \theta(u, v) \cdot \frac{\partial \phi(u,v)}{\partial u}}.
	\end{equation*}
	Або отримали рівняння вигляду
	\begin{equation*}
		%\label{eq:1.5.33}
		\frac{\diff u}{\diff v} = f(u, v).
	\end{equation*}
	Параметризація загального вигляду не дає інтеграл диференціального рівняння. Вона дозволяє звести диференціальне рівняння, не роз\-в'яз\-а\-не відносно похідної, до диференціального рівняння, роз\-в'яз\-а\-но\-го відносно похідної.
	\item Нехай рівняння $F(x, y, y') = 0$ можна розв’язати відносно $y'$ і воно має $n$ коренів, тобто його  можна записати у вигляді  
	\begin{equation*}
		%\label{eq:1.5.34}
		\prod_{i=1}^n (y' - f_i(x, y)) = 0.
	\end{equation*}
	Розв’язавши кожне з рівнянь $y' = f_i(x, y)$, $i=\overline{1,n}$, отримаємо $n$ загальних розв’язків (або інтервалів) $y = \phi_i(x, C)$, $i=\overline{1,n}$ (або $\phi_u(x,y)=C$, $i=\overline{1,n}$). І загальний розв’язок вихідного рівняння, не розв’язаного відносно похідної має вигляд
	\begin{equation*}
		%\label{eq:1.5.35}
		\prod_{i=1}^n (y - \phi_i(x, C)) = 0,
	\end{equation*}
	або
	\begin{equation*}
		%\label{eq:1.5.36}
		\prod_{i=1}^n (\phi_i(x, y) - C) = 0.
	\end{equation*}
\end{enumerate}
