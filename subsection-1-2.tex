\subsubsection{Загальна теорія}
Нехай рівняння має вигляд
\begin{equation*}
	%\label{eq:1.2.1}
	M(x, y) \cdot \diff	x + N(x, y) \cdot \diff y = 0.
\end{equation*}

Якщо функції $M(x, y)$ та $N(x, y)$ однорідні одного ступеня, то рівняння називається однорідним. Нехай функції $M(x, y)$ та $N(x, y)$ однорідні ступеня $k$, тобто
\begin{equation*}
	%\label{eq:1.2.2}
	M(t \cdot x, t \cdot y) = t^k \cdot M(x, y), \qquad N(t \cdot x, t \cdot y) = t^k \cdot N(x, y).
\end{equation*}

Робимо заміну 
\begin{equation*}
	%\label{eq:1.2.2_5}
	y = u x, \quad \diff y = u \diff x + x \diff u.
\end{equation*}
Після підстановки одержуємо
\begin{equation*}
	%\label{eq:1.2.3}
	M(x, u x) \cdot \diff x + N(x, u x) \cdot (u \diff x + x \diff u) = 0,
\end{equation*}
або 
\begin{equation*}
	%\label{eq:1.2.4}
	x^k M(1, u) \cdot \diff x + x^k N(1, u) \cdot (u \diff x + x \diff u) = 0.
\end{equation*}

Скоротивши на $x^k$ і розкривши дужки, запишемо 
\begin{equation*}
	%\label{eq:1.2.5}
	M(1, u) \cdot \diff x + N(1, u) \cdot u \diff x + N(1, u) \cdot x \diff u = 0.
\end{equation*}
Згрупувавши, одержимо рівняння зі змінними, що розділяються
\begin{equation*}
	%\label{eq:1.2.6}
	(M(1, u) + N(1, u) \cdot u) \diff x + N(1, u) \cdot x \diff u = 0,
\end{equation*}
або 
\begin{equation*}
	%\label{eq:1.2.7}
	\int \frac{\diff x}{x} + \int \frac{N(1, u) \cdot \diff u}{M(1, u) + N(1, u) \cdot u} = C.
\end{equation*}

Взявши інтеграли та замінивши $u = y / x$, отримаємо загальний інтеграл $\Phi(x, y / x) = C$.

\subsubsection{Рівняння, що зводяться до однорідних}

Нехай маємо рівняння вигляду
\begin{equation*}
	%\label{eq:1.2.8}
	\label{eq:linear-fractional-equation}
	\frac{\diff y}{\diff x} = f \left( \frac{a_1 x + b_1 y + c_1}{a_2 x + b_2 y + c_2} \right).
\end{equation*}

Розглянемо два випадки
\begin{enumerate}
	\item 
	\begin{equation*}
		%\label{eq:1.2.9}
		\Delta = \begin{vmatrix} a_1 & b_1 \\ a_2 & b_2 \end{vmatrix} \ne 0.
	\end{equation*}

	Тоді система алгебраїчних рівнянь
	\begin{equation*}
		%\label{eq:1.2.10}
		\left\{
			\begin{aligned}
				a_1 x + b_1 y + c_1 &= 0, \\
				a_2 x + b_2 y + c_2 &= 0,
			\end{aligned}
		\right.
	\end{equation*}
	має єдиний розв’язок $(x_0, y_0)$. Проведемо заміну 
	\begin{equation*}
		%\label{eq:1.2.10_5}
		\left\{\begin{aligned}
			x &= x_1 + x_0, \\
			y &= y_1 + y_0
		\end{aligned}\right.
	\end{equation*}
	та отримаємо
	\begin{multline*}
		%\label{eq:1.2.11}
		\frac{\diff y_1}{\diff x_1} = f \left( \frac{a_1 \cdot (x_1 + x_0) + b_1 \cdot (y_1 + y_0) + c_1}{a_2 \cdot (x_1 + x_0) + b_2 \cdot (y_1 + y_0) + c_2} \right) = \\
		= f \left( \frac{a_1 x_1 + b_1 y_1 + (a_1 x_0 + b_1 y_0 + c_1)}{a_2 x_1 + b_2 y_1 + (a_2 x_0 + b_2 y_0 + c_2)} \right)
	\end{multline*}

	Оскільки $(x_0, y_0)$ -- розв’язок алгебраїчної системи, то диференціальне рівняння набуде вигляду
	\begin{equation*}
		%\label{eq:1.2.12}
		\frac{\diff y_1}{\diff x_1} = f \left( \frac{a_1 x_1 + b_1 y_1}{a_2 x_1 + b_2 y_1} \right)
	\end{equation*}
	і є однорідним нульового ступеня. Робимо заміну 
	\begin{equation*}
		%\label{eq:1.2.12_5}
		y_1 = u x_1, \quad \diff y_1 = u \cdot \diff x_1 + x_1 \cdot \diff u.
	\end{equation*}

	Підставимо в рівняння
	\begin{equation*}
		%\label{eq:1.2.13}
		u + x_1 \cdot \frac{\diff u}{\diff x_1} = f \left( \frac{a_1 x_1 + b_1 u x_1}{a_2 x_1 + b_2 u x_1} \right).
	\end{equation*}
	
	Одержимо
	\begin{equation*}
		%\label{eq:1.2.14}
		x_1 \cdot \diff u + \left( u - f \left( \frac{a_1 x_1 + b_1 u x_1}{a_2 x_1 + b_2 u x_1} \right) \right) \diff x_1 = 0.
	\end{equation*}

	Розділивши змінні, маємо
	\begin{equation*}
		%\label{eq:1.2.15}
		\int \frac{\diff u}{u - f \left( \frac{a_1 x_1 + b_1 u x_1}{a_2 x_1 + b_2 u x_1} \right)} + \ln (x_1) = C.
	\end{equation*}

	І загальний інтеграл рівняння має вигляд $\Phi(u, x_1) = C$. Повернувшись до вихідних змінних, запишемо
	\begin{equation*}
		%\label{eq:1.2.16}
		\Phi \left( \frac{y - y_0}{x - x_0}, x - x_0 \right) = C.
	\end{equation*}

	\item Нехай 
	\begin{equation*}
		%\label{eq:1.2.17}
		\Delta = \begin{vmatrix} a_1 & b_1 \\ a_2 & b_2 \end{vmatrix} = 0,
	\end{equation*}
	тобто коефіцієнти рядків лінійно залежні і
	\begin{equation*}
		%\label{eq:1.2.18}
		a_1 x + b_1 y = \alpha \cdot (a_2 x + b_2 y).
	\end{equation*}

	Робимо заміну $a_2 x + b_2 y = z$. Звідси $\frac{\diff y}{\diff x} = \frac{1}{b_2} \cdot \left( \frac{\diff z}{\diff x} - a_2 \right)$. \\

	Підставивши в диференціальне рівняння, одержимо
	\begin{equation*}
		%\label{eq:1.2.19}
		\frac{1}{b_2} \cdot \left( \frac{\diff z}{\diff x} - a_2 \right) = f \left ( \frac{\alpha z + c_1}{z + c_2} \right),
	\end{equation*}
	або
	\begin{equation*}
		%\label{eq:1.2.20}
		\frac{\diff z}{\diff x} = a_2 + b_2 \cdot f \left ( \frac{\alpha z + c_1}{z + c_2} \right),
	\end{equation*}
	Розділивши змінні, отримаємо
	\begin{equation*}
		%\label{eq:1.2.21}
		\int \frac{\diff z}{a_2 + b_2 \cdot f \left ( \frac{\alpha z + c_1}{z + c_2} \right)} - x = C,
	\end{equation*}

	Загальний інтеграл має вигляд $\Phi(a_2 x + b_2 y, x) = C$
\end{enumerate}

\subsubsection{Вправи для самостійної роботи}

Однорідні рівняння можуть бути записані у вигляді \[y' = f \left( \frac{y}{x} \right)\] або  \[ M(x, y)\cdot\diff y + N(x, y) \cdot \diff y = 0, \] де $M(x, y)$ і $N(x, y)$ -- однорідні функції одного й того ж ступеня. Для того, щоб розв’язати однорідне рівняння, необхідно провести заміну \[y = u x, \quad \diff y = u \cdot \diff x + x \cdot \diff u,\] в результаті якої отримаємо рівняння зі змінними, що розділяються. 

\begin{example}
	Розв’язати рівняння $x \cdot \diff y = (x + y) \cdot \diff y$. 
\end{example}

\begin{solution}
	Дане рівняння однорідне, оскільки $x$ та $x + y$ є однорідними функціями першого ступеня. \\

	Проведемо заміну: $y = u x$. Тоді $\diff y = u \diff x + x \diff y$. Підставивши $y$ та $\diff y$ в задане рівняння, отримаємо  
	\begin{align*}
		x \cdot (x \diff u + u \diff x) &= (x + x u) \diff x, \\
		x^2 \diff u &= x \diff x
	\end{align*}

	Розв’яжемо це рівняння зі змінними, що розділяються:
	\begin{align*}
		\diff u &= \frac{\diff x}{x}, \\
		u &= \ln |x| + C.
	\end{align*}
	Повернувшись до вихідних змінних $u = y / x$, отримаємо \[y = x \cdot (\ln |x| + C).\] Крім того розв’язком є $x = 0$, що було загублене при поділенні рівняння на $x$.
\end{solution}

Розв’язати рівняння:
\begin{multicols}{2}
\begin{problem}
	\[ (x + 2y) \cdot \diff x - x \diff y = 0; \]
\end{problem}
\begin{problem}
	\[ (x - y) \cdot \diff x + (x + y) \cdot \diff y = 0; \]
\end{problem}
\begin{problem}
	\[y^2 + x^2 y' = x y y'; \]
\end{problem}
\begin{problem}
	\[ (x^2 + y^2) \cdot y' = 2 xy; \]
\end{problem}
\begin{problem}
	\[ xy' - y = x \cdot \tan \left( \frac{y}{x} \right); \]
\end{problem}
\begin{problem}
	\[ x y' = y - x e^{y / x}; \]
\end{problem}
\begin{problem}
	\[x y' - y = (x + y) \cdot \ln \left( \frac{x + y}{x} \right); \]
\end{problem}
\begin{problem}
	\[ (3x + y) \cdot \diff x - (2x + 3y) \cdot \diff y = 0; \]
\end{problem}
\begin{problem}
	\[ x y' = y \cos \left(\ln \left(\frac{y}{x} \right)\right); \]
\end{problem}
\begin{problem}
	\[ \left(y+\sqrt{xy}\right)\cdot \diff x = x \diff y; \]
\end{problem}
\begin{problem}
	\[ xy' = \sqrt{x^2 - y^2} + y; \]
\end{problem}
\begin{problem}
	\[ x^2 y' = y \cdot (x + y); \]
\end{problem}
\begin{problem}
	\[ y \cdot (-y + xy') = \sqrt{x^4 + y^4}; \]
\end{problem}
\begin{problem}
	\[ x \diff y - y \diff x = \sqrt{x^2 + y^2} \diff x; \]
\end{problem}
\begin{problem}
	\[ (y^2 - 2xy) \cdot \diff x + x^2 \cdot \diff y = 0; \]
\end{problem}
\begin{problem}
	\[ 2x^3 y' = y \cdot (2x^2 - y^2); \]
\end{problem}
\begin{problem}
	\[ \left( x - y \cos \left(\rfrac{y}{x}\right)\right) \diff x = - x \cos \left(\rfrac{y}{x}\right) \diff y; \]
\end{problem}
\begin{problem}
	\[ y' (xy - x^2) = y^2; \]
\end{problem}
\begin{problem}
	\[ 2xyy' = x^2 + y^2; \]
\end{problem}
\begin{problem}
	\[ (6x + 3y) \cdot \diff x = (7x - 2y) \cdot \diff y; \]
\end{problem}
\begin{problem}
	\[ y^2 x \diff x = y \cdot (xy - 2y^2) \cdot \diff y; \]
\end{problem}
\begin{problem}
	\[ x^2 y \diff x = y \cdot (xy - 2y^2) \cdot \diff y; \]
\end{problem}
\begin{problem}
	\[ 2y^3 = xy' \cdot (2y^2 - x^2); \]
\end{problem}
\begin{problem}
	\[ \left( x + \sqrt{xy}\right) \cdot \diff y = y \diff x; \]
\end{problem}
\begin{problem}
	\[ y = \left( \sqrt{y^2 - x^2} + x\right) y'; \]
\end{problem}
\begin{problem}
	\[ (3x - 2y) \cdot \diff x - (2x + y) \cdot \diff y = 0; \]
\end{problem}
\begin{problem}
	\[ (7x + 6y) \cdot \diff x - (x + 3y) \cdot \diff y = 0; \]
\end{problem}
\begin{problem}
	\[ xy' = y + x \cdot \cot \left(\frac{y}{x}\right). \]
\end{problem}
\end{multicols}

Знайти частинні розв’язки, що задовольняють задані початкові умови:
\begin{problem}
	\[ x y' = 4 \sqrt{2x^2 + y^2} + y, \quad y(1) = 2; \]
\end{problem}
\begin{problem}
	\[ (2y^2 + 3x^2) \cdot xy' = 3y^3 + 6yx^2, \quad y(2) = 1; \]
\end{problem}
\begin{problem}
	\[ y' (x^2 - 2xy) = x^2 + xy - y^2, \quad y(3) = 0; \]
\end{problem}
\begin{problem}
	\[ 2 y' = \frac{y^2}{x^2} + 8 \cdot \frac{y}{x} + 8, \quad y(1) = 1; \]
\end{problem}
\begin{problem}
	\[ y' (x^2 - 4xy) = x^2 + xy - 3 y^2, \quad y(1) = 1; \]
\end{problem}
\begin{problem}
	\[ xy' = 3 \sqrt{2x^2 + y^2} + y, \quad y(1) = 1; \]
\end{problem}
\begin{problem}
	\[ (2y^2 + 7x^2)\cdot x y' = 3y^3 + 14yx^2, \quad y(1) = 1; \]
\end{problem}
\begin{problem}
	\[ 2y' = \frac{y^2}{x^2} + 6 \cdot \frac{y}{x} + 3, \quad y(3) = 1; \]
\end{problem}
\begin{problem}
	\[ x^2 y' = y^2 + 4xy + 2x^2, \quad y(1) = 1; \]
\end{problem}
\begin{problem}
	\[ xy' = \sqrt{2x^2 + y^2} + y, \quad y(1) = 1; \]
\end{problem}
\begin{problem}
	\[ xy' = 3 \sqrt{x^2 + y^2} + y, \quad y(3) = 4; \]
\end{problem}
\begin{problem}
	\[ xy' = 2 \sqrt{x^2 + y^2} + y, \quad y(4) = 3; \]
\end{problem}
\begin{problem}
	\[ 2y' = \frac{y^2}{x^2} + 8 \cdot \frac{y}{x} + 8, \quad y(1) = 1; \]
\end{problem}
\begin{problem}
	\[ y' = \frac{x + 2y}{2x - y}, \quad y(3) = 8. \]
\end{problem}