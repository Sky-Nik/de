\section*{Заняття 8: Інтегрувальний множник. Випадки знаходження інтегрувального множника}

\subsection*{Рекомендовані приклади для аудиторної роботи}

Розв'язати диференціальні рівняння методом інтегрувального множника, знаючи, що вони мають вигляд $\mu = f(x)$ або $\mu = f(y)$:

\begin{problem}
	$(2 y + x y^3) \cdot \diff x + (x + x^2 y^2) \cdot \diff y = 0$.
\end{problem}

\begin{problem}
	$y^2 \cdot (x - 3 y) \cdot \diff x + (1 - 3 x y^2) \cdot \diff y = 0$.
\end{problem}

\begin{problem}
	$2 y \cdot \diff x + (y^2 - 6 x) \cdot \diff y = 0$.
\end{problem}

Зінтегрувати рівняння за допомогою множників $\mu(x + y)$, $\mu(x y)$, або $\mu(x - y)$

\begin{problem}
	$(y - a y / x + x) \cdot \diff x + a \cdot \diff y = 0$.
\end{problem}

\begin{problem}
	$y^2 \cdot \diff x + (x y - 1) \cdot \diff y = 0$.
\end{problem}

\subsection*{Рекомендовані приклади для домашнього завдання}

Розв'язати диференціальні рівняння методом інтегрувального множника, знаючи, що вони мають вигляд $\mu = f(x)$ або $\mu = f(y)$:

\begin{problem}
	$(1 + x^2 y) \cdot \diff x + x^2 \cdot (x + y) \cdot \diff y = 0$.
\end{problem}

\begin{problem}
	$(2 x y + a x) \cdot \diff x + \diff y = 0$.
\end{problem}

\begin{problem}
	$\diff x + (x + e^{-y} \cdot y^2) \cdot \diff y = 0$.
\end{problem}

Зінтегрувати рівняння за допомогою множників $\mu(x + y)$, $\mu(x y)$, або $\mu(x - y)$

\begin{problem}
	$\diff x + x \cdot \cot (x + y) \cdot (\diff x + \diff y) = 0$.
\end{problem}

\begin{problem}
	$(2 x^2 y + x) \cdot \diff y + (y + 2 x y^2 - x^2 y^3) \cdot \diff x = 0$.
\end{problem}
