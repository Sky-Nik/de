\section*{Заняття 17--18: Методи розв'язування неоднорідних систем з постійним коефіцієнтами. Застосування методу невизначених коефіцієнтів}

\subsection*{Рекомендовані приклади для аудиторної роботи}

\begin{problem}
	\[ \left\{ \begin{aligned} \dot x &= y + 2 e^t, \\ \dot y &= x + t^2. \end{aligned} \right. \]
\end{problem}

\begin{problem}
	\[ \left\{ \begin{aligned} \dot x &= 3 x + 2 y + 4 e^{5 t}, \\ \dot y &= x + 2 y. \end{aligned} \right. \]
\end{problem}

\begin{problem}
	\[ \left\{ \begin{aligned} \dot x &= 4 x + y - e^{2 t}, \\ \dot y &= y - 2 x. \end{aligned} \right. \]
\end{problem}

\begin{problem}
	\[ \left\{ \begin{aligned} \dot x &= 2 x - y, \\ \dot y &= y - 2 x + 18. \end{aligned} \right. \]
\end{problem}

\begin{problem}
	\[ \left\{ \begin{aligned} \dot x &= x - y + 8 t, \\ \dot y &= 5 x - y. \end{aligned} \right. \]
\end{problem}

\begin{problem}
	\[ \left\{ \begin{aligned} \dot x &= t + \tan^2 (t) - 1, \\ \dot y &= - x + \tan (t). \end{aligned} \right. \]
\end{problem}

\begin{problem}
	\[ \left\{ \begin{aligned} \dot x &= - 4 x - 2 y + 2 / (e^t - 1), \\ \dot y &= 6 x + 3 y - 3 / (e^t - 1). \end{aligned} \right. \]
\end{problem}

\begin{problem}
	\[ \left\{ \begin{aligned} \dot x &= x - y + 1 / \cos (t), \\ \dot y &= 2 x - y. \end{aligned} \right. \]
\end{problem}

\subsection*{Рекомендовані приклади для домашнього завдання}

\begin{problem}
	\[ \left\{ \begin{aligned} \dot x &= y - 5 \cos (t), \\ \dot y &= 2 x + y. \end{aligned} \right. \]
\end{problem}

\begin{problem}
	\[ \left\{ \begin{aligned} \dot x &= 2 x - 4 y + 4 e^{-2t}, \\ \dot y &= 2 x - 2 y. \end{aligned} \right. \]
\end{problem}

\begin{problem}
	\[ \left\{ \begin{aligned} \dot x &= 2 y - x + 1, \\ \dot y &= 3 y - 2x. \end{aligned} \right. \]
\end{problem}

\begin{problem}
	\[ \left\{ \begin{aligned} \dot x &= x + 2 y + 16 t \cdot e^t, \\ \dot y &= 2 x - 2 y. \end{aligned} \right. \]
\end{problem}

\begin{problem}
	\[ \left\{ \begin{aligned} \dot x &= 2 x - y, \\ \dot y &= 2 y - x - 5 e^t \cdot \sin (t). \end{aligned} \right. \]
\end{problem}

\begin{problem}
	\[ \left\{ \begin{aligned} \dot x &= 2 y - x, \\ \dot y &= 4 y - 3 x - e^{3 t} / (e^{2 t} + 1). \end{aligned} \right. \]
\end{problem}
