Рівняння вигляду
\begin{equation*}
	%\label{eq:1.3.8}
	\frac{\diff y}{\diff x} + p(x) \cdot y = q(x) \cdot y^m, \quad m \ne 1
\end{equation*}
називається рівнянням Бернуллі. Розділимо на $y^m$ і одержимо 
\begin{equation*}
	%\label{eq:1.3.9}
	y^{-m} \cdot \frac{\diff y}{\diff x} + p(x) \cdot y^{1-m} = q(x).
\end{equation*}

Зробимо заміну: 
\begin{equation*}
	%\label{eq:1.3.9_5}
	y^{1-m} = z, \quad (1 - m) \cdot y^{-m} \cdot \frac{\diff y}{\diff x} = \diff z.
\end{equation*}

Підставивши в рівняння, отримаємо
\begin{equation*}
	%\label{eq:1.3.10}
	\frac{1}{1-m} \cdot \frac{\diff z}{\diff x} + p(x) \cdot z = q(x).
\end{equation*}

Одержали лінійне диференціальне рівняння. Його розв’язок має вигляд
\begin{multline*}
	%\label{eq:1.3.11}
	z = \exp\left\{ -(1 - m) \cdot \int p(x) \diff x \right\} \cdot \\
	\cdot \left( (1-m) \cdot \int q(x) \cdot \exp\left\{ (1 - m) \cdot \int p(x) \diff x \right\} + C\right).
\end{multline*}
