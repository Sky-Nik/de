Співвідношення вигляду
\begin{equation*}
	\left\{
		\begin{aligned}
			F_1(\dot x_1, \dot x_2, \ldots, \dot x_n, x_1, x_2, \ldots, x_n, t) &= 0, \\
			F_2(\dot x_1, \dot x_2, \ldots, \dot x_n, x_1, x_2, \ldots, x_n, t) &= 0, \\
			\ldots \ldots \ldots \ldots \ldots \ldots \ldots \ldots \ldots \ldots \ldots & \ldots . \\
			F_n(\dot x_1, \dot x_2, \ldots, \dot x_n, x_1, x_2, \ldots, x_n, t) &= 0
		\end{aligned}
	\right.
\end{equation*}
називається системою $n$ звичайних диференціальних рівнянь першого порядку. \\

Якщо система розв'язана відносно похідних і має вигляд
\begin{equation*}
	\left\{
		\begin{aligned}
			\dot x_1 &= f_1(x_1, x_2, \ldots, x_n, t) &= 0, \\
			\dot x_2 &= f_2(x_1, x_2, \ldots, x_n, t) &= 0, \\
			\ldots & \ldots \ldots \ldots \ldots \ldots \ldots \ldots \ldots &\ldots . \\
			\dot x_n &= f_n(x_1, x_2, \ldots, x_n, t) &= 0
		\end{aligned}
	\right.
\end{equation*}
то вона називається системою в нормальній формі.

\begin{definition}
	Розв'язком системи диференціальних рівнянь на\-зи\-ва\-є\-ть\-ся набір $n$ неперервно диференційованих функцій $x_1(t), \ldots, x_n(t)$ що тотожно задовольняють кожному з рівнянь системи.
\end{definition}

У загальному випадку розв'язок системи залежить від $n$ довільних сталих і має вигляд  $x_1(t, C_1, \ldots, C_n), \ldots, x_n(t, C_1, \ldots, C_n)$ і задача Коші для системи звичайних диференціальних рівнянь першого порядку ставиться в такий спосіб. Потрібно знайти розв'язок, що задовольняє початковим умовам (умовам Коші):
\begin{equation*}
	x_1(t_0) = x_1^0, \quad x_2(t_0) = x_2^0, \quad \ldots, \quad x_n(t_0) = x_n^0
\end{equation*}

\begin{definition}
	Розв'язок $x_1(t, C_1, \ldots, C_n)$, $\ldots$, $x_n(t, C_1, \ldots, C_n)$ на\-зи\-ва\-є\-ть\-ся загальним, якщо за рахунок вибору сталих $C_1, \ldots, C_n$ можна роз\-в'яз\-а\-ти довільну задачу Коші.
\end{definition}

Для систем звичайних диференціальних рівнянь досить важливим є поняття інтеграла системи. В залежності від гладкості (тобто диференційованості) можна розглядати два визначення інтеграла.

\begin{definition}
	\begin{enumerate}
		\item Функція $F(x_1, x_2, \ldots, x_n, t)$ стала уздовж розв'язків системи, називається інтегралом системи.
		\item Функція $F(x_1, x_2, \ldots, x_n, t)$ повна похідна, якої в силу системи тотожно дорівнює нулю, називається інтегралом системи.
	\end{enumerate}
\end{definition}

Для лінійних рівнянь існує поняття лінійної залежності і незалежності розв'язків. Для нелінійних рівнянь (систем рівнянь) аналогічним поняттям є функціональна незалежність.

\begin{definition}
	Інтеграли 
	\begin{equation*}
		F_1(x_1, x_2, \ldots, x_n, t), \quad F_2(x_1, x_2, \ldots, x_n, t), \quad \ldots, \quad F_n(x_1, x_2, \ldots, x_n, t)
	\end{equation*}
	називаються функціонально незалежними, якщо не існує функції $n$ змінних $\Phi(z_1, z_2, \ldots, z_n)$ такої, що
	\begin{equation*}
		\Phi(F_1(x_1, x_2, \ldots, x_n, t), F_2(x_1, x_2, \ldots, x_n, t), \ldots, F_n(x_1, x_2, \ldots, x_n, t)) = 0.
	\end{equation*}
\end{definition}

\begin{theorem}
	Для того щоб інтеграли 
	\begin{equation*}
		F_1(x_1, x_2, \ldots, x_n, t), \quad F_2(x_1, x_2, \ldots, x_n, t), \quad \ldots, \quad F_n(x_1, x_2, \ldots, x_n, t)
	\end{equation*}
	системи звичайних диференціальних рівнянь були функціонально незалежними, необхідно і достатньо, щоб визначник Якобі був відмінний від тотожного нуля, тобто 
	\begin{equation*}
		\frac{D(F_1, F_2, \ldots, F_n)}{D(x_1, x_2, \ldots, x_n)} \ne 0.
	\end{equation*}
\end{theorem}

\begin{definition}
	Якщо $F(x_1, x_2, \ldots, x_n, t)$ -- інтеграл системи диференціальних рівнянь, то рівність $F(x_1, x_2, \ldots, x_n, t) = C$ називається першим інтегралом.
\end{definition}

\begin{definition}
	Сукупність $n$ функціонально незалежних інтегралів називається загальним інтегралом системи диференціальних рівнянь.
\end{definition}

Власне кажучи загальний інтеграл -- це загальний роз\-в'яз\-ок системи диференціальних рівнянь у неявному вигляді.

\begin{theorem}[існування та єдиності розв'язку задачі Коші]
	Щоб система диференціальних рівнянь, розв'язаних відносно похідної, мала єдиний роз\-в'яз\-ок, що задовольняє умовам Коші: 
	\begin{equation*}
		x_1(t_0) = x_1^0, \quad x_2(t_0) = x_2^0, \quad \ldots, \quad x_n(t_0) = x_n^0
	\end{equation*}
	досить, щоб:
	\begin{enumerate}
		\item функції $f_1, f_2, \ldots, f_n$ були неперервними по змінним $x_1, x_2, \ldots, x_n, t$ в околі точки $\left(x_1^0, x_2^0, \ldots, x_n^0, t_0\right)$;
		\item функції $f_1, f_2, \ldots, f_n$ задовольняли умові Ліпшиця по аргументах $x_1, x_2, \ldots, x_n$ у тому ж околі.
	\end{enumerate}
\end{theorem}

\begin{remark}
	Умову Ліпшиця можна замінити більш грубою умовою, але такою, що перевіряється легше, існування обмежених частинних похідних, тобто 
	\begin{equation*}
		\left| \frac{\partial f_i}{\partial x_j} \right| \le M, \quad i,j = 1, 2, \ldots, n.
	\end{equation*}
\end{remark}