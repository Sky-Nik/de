Рівняння вигляду
\begin{equation*}
	%\label{eq:3.1}
	a_0(x) \cdot y^{(n)} + a_1(x) \cdot y^{(n - 1)} + \ldots + a_n(x) \cdot y = b(x)
\end{equation*} 
називається лінійним неоднорідним диференціальним рівнянням $n$-го порядку. \\

Рівняння вигляду
\begin{equation*}
	%\label{eq:3.2}
	a_0(x) \cdot y^{(n)} + a_1(x) \cdot y^{(n - 1)} + \ldots + a_n(x) \cdot y = 0
\end{equation*} 
називається лінійним однорідним диференціальним рівнянням $n$-го порядку. \\

Якщо при $x \in [a, b]$, $a_0(x) \ne 0$ коефіцієнти $b(x)$, $a_i(x)$, $i=\overline{0,n}$ неперервні, то для рівняння
\begin{equation*}
	%\label{eq:3.3}
	y^{(n)} = - \frac{a_1(x)}{a_0(x)} \cdot y^{(n - 1)} - \ldots - \frac{a_n(x)}{a_0(x)} \cdot y + \frac{b(x)}{a_0(x)}.
\end{equation*}
виконуються умови теореми існування та єдиності і існує єдиний роз\-в'яз\-ок $y = y(x)$, що задовольняє початковим умовам
\begin{equation*}
	%\label{eq:3.4}
	y(x_0) = y_0, \quad y'(x_0) = y_0', \quad \ldots, \quad y^{(n - 1)} = y_0^{(n - 1)}.
\end{equation*}

\subsection{Лінійні однорідні рівняння}

\subsubsection{Властивості лінійних однорідних рівнянь}

\begin{theorem}
	Лінійність і однорідність зберігаються при довільному перетворенні незалежної змінної $x = \phi(t)$.
\end{theorem}
\begin{proof}
	Справді, після заміни $x = \phi(t)$, одержимо
	\begin{align*}
		%\label{eq:3.1.1}
		y_x' &= \frac{\diff y}{\diff x} = \frac{\diff y}{\diff t} \cdot \frac{\diff t}{\diff x} = \frac{1}{\phi'(t)} \cdot \frac{\diff y}{\diff t}, \\
		%\label{eq:3.1.2}
		y_{x^2}'' &= \frac{\diff}{\diff x} \cdot y_x' = \frac{\diff}{\diff t} \left( \frac{1}{\phi'(t)} \cdot \frac{\diff y}{\diff t} \right) \cdot \frac{1}{\phi'(t)} = \\
		&= - \frac{\phi''(t)}{(\phi'(t))^2} \cdot \frac{\diff y}{\diff t} + \frac{1}{(\phi'(t))^2} \cdot \frac{\diff^2 y}{\diff t^2}, \nonumber
	\end{align*}
	і так далі до $n$-го порядку. Після підстановки і приведення подібних, знову отримуємо лінійне однорідне рівняння
	\begin{equation*}
		%\label{eq:3.1.3}
		A_0(t) \cdot \frac{\diff^n y}{\diff t^n} + A_1(t) \cdot \frac{\diff^{n - 1} y}{\diff t^{n - 1}} + \ldots + A_n(t) \cdot y = 0.
	\end{equation*}
\end{proof}

\begin{theorem}
	Лінійність і однорідність зберігаються при лінійному перетворенні невідомої функції $y = \alpha (x) \cdot z$.
\end{theorem}
\begin{proof}
	Справді, після заміни $y = \alpha (x) \cdot z$, одержимо
	\begin{align*}
		%\label{eq:3.1.4}
		y_x' &= \alpha'(x) \cdot z + \alpha(x) \cdot z', \\
		%\label{eq:3.1.5}
		y_{x^2}'' &= \alpha''(x) \cdot z + 2 \alpha'(x) \cdot z' + \alpha(x) \cdot z'',
	\end{align*}
	і так далі до $n$-го порядку. Після підстановки знову отримаємо лінійне однорідне рівняння
	\begin{equation*}
		%\label{eq:3.1.6}
		\bar A_0(x) \cdot z^{(n)} + \bar A_1(x) \cdot z^{(n - 1)} + \ldots + \bar A_n(x) \cdot z = 0.
	\end{equation*}
\end{proof}