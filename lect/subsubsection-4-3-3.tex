При розв'язуванні систем методом Ейлера складають характеристичне рівняння, і в залежності від його коренів для кожного $\lambda_i$,    $i = \overline{1, n}$ знаходять відповідний лінійно незалежний розв'язок.

\begin{example}
    Розв'язати систему:
    \[ \left\{ \begin{aligned}
        \dot x &= 2 x + 3 y, \\
        \dot y &= 3 x + 4 y.
    \end{aligned} \right. \]
\end{example}

\begin{solution}
    Характеристичне рівняння має вигляд
    \[ \begin{vmatrix}
        2 - \lambda & 1 \\
        3 & 4 - \lambda 
    \end{vmatrix} = 0, \]
    або $\lambda^2 - 6 \lambda + 5 = 0$. \parvskip
    
    Коренями будуть $\lambda_1 = 1$, $\lambda_2 = 5$.
    
    \begin{enumerate}
        \item Знайдемо власний вектор, що відповідає $\lambda_1 = 1$. Підставивши в систему
        \[ \left\{ \begin{aligned}
            (2 - \lambda) \alpha_1 + \alpha_2 &= 0, \\
            3 \alpha_1 + (4 - \lambda) \alpha_2 &= 0, 
        \end{aligned} \right. \]
        одержимо 
        \[ \left\{ \begin{aligned}
            \alpha_1 + \alpha_2 &= 0, \\
            3 \alpha_1 + 3 \alpha_2 &= 0.
        \end{aligned} \right. \]
        
        Звідси $\alpha_1 = 1$, $\alpha_2 = -1$.
        
        \item
        Знайдемо власний вектор, що відповідає $\lambda_2 = 5$. Підставивши в систему, одержимо
        \[ \left\{ \begin{aligned}
            - 3 \alpha_1 + \alpha_2 &= 0, \\
            3 \alpha_1 - \alpha_2 &= 0.
        \end{aligned} \right. \]
        
        Звідси $\alpha_1 = 1$, $\alpha_2 = 3$.
    \end{enumerate}
    
    Таким чином, одержимо розв'язок системи у вигляді
    \[ \begin{pmatrix} x \\ y \end{pmatrix} = c_1 e^t \begin{pmatrix} 1 \\ -1 \end{pmatrix} + C_2 e^{5t} \begin{pmatrix} 1 \\ 3 \end{pmatrix} = \begin{pmatrix} e^t & e^{5t} \\ - e^t & 3 e^{5t} \end{pmatrix} \begin{pmatrix} c_1 \\ c_2 \end{pmatrix}. \]
\end{solution}

\begin{example}
    Розв'язати систему:
    \[ \left\{ \begin{aligned}
        \dot x &= x + y, \\
        \dot y &= -2 x + 3 y.
    \end{aligned} \right. \]
\end{example}

\begin{solution}
    Характеристичне рівняння має вигляд
    \[ \begin{vmatrix}
        1 - \lambda & 1 \\
        -2 & 3 - \lambda 
    \end{vmatrix} = 0, \]
    або $\lambda^2 - 4 \lambda + 5 = 0$. \parvskip
    
    Коренями будуть $\lambda_{1,2} = 2 \pm i$. \parvskip
    
    Візьмемо $\lambda_1 = 2 + i$. Підставивши в систему
    \[ \left\{ \begin{aligned}
        (1 - \lambda) \alpha_1 + \alpha_2 &= 0, \\
        -2 \alpha_1 + (3 - \lambda) \alpha_2 &= 0, 
    \end{aligned} \right. \]
    одержимо 
    \[ \left\{ \begin{aligned}
        (-1 - i) \alpha_1 + \alpha_2 &= 0, \\
        -2 \alpha_1 + (1 - i) \alpha_2 &= 0.
    \end{aligned} \right. \]
    
    Звідси $\alpha_1 = 1$, $\alpha_2 = 1 + i$. \parvskip
    
    Запишемо вектор розв'язку
    \begin{multline*} \begin{pmatrix} x \\ y \end{pmatrix} = \begin{pmatrix} e^{(2 + i) t} \\ (1 + i) e^{(2 + i) t} \end{pmatrix} = \begin{pmatrix} e^{2 t} (\cos t + i \sin t) \\ e^{2 t} (1 + i) (\cos t + i \sin t) \end{pmatrix} = \\ = \begin{pmatrix} e^{2 t} \cos t \\ e^{2 t} (\cos t - \sin t) \end{pmatrix} + i \begin{pmatrix} e^{2 t} \sin t \\ e^{2 t} (\cos t + \sin t) \end{pmatrix}. \end{multline*}
    
    Оскільки комплексно-спряженому розв'язку відповідають два лінійно незалежних розв'язки, то загальний розв'язок має вигляд
    \begin{multline*} \begin{pmatrix} x \\ y \end{pmatrix} = c_1 \begin{pmatrix} e^{2 t} \cos t \\ e^{2 t} \cos t - \sin t \end{pmatrix} + c_2 \begin{pmatrix} e^{2 t} \sin t \\ e^{2 t} (\cos t + \sin t) \end{pmatrix} = \\ = \begin{pmatrix} e^{2 t} \cos t & e^{2 t} \sin t \\ e^{2 t} (\cos t - \sin t) & e^{2 t} (\cos t + \sin t) \end{pmatrix} \begin{pmatrix} c_1 \\ c_2 \end{pmatrix}. \end{multline*}
\end{solution}

\begin{example}
    Розв'язати систему:
    \[ \left\{ \begin{aligned}
        \dot x &= 2 x + y, \\
        \dot y &= -x + 4 y.
    \end{aligned} \right. \]
\end{example}
 
\begin{solution}
    Характеристичне рівняння має вигляд
    \[ \begin{vmatrix}
        2 - \lambda & 1 \\
        -1 & 4 - \lambda 
    \end{vmatrix} = 0, \]
    або $\lambda^2 - 6 \lambda + 9 = 0$. \parvskip
    
    Коренями будуть $\lambda_1 = \lambda_2 = 3$. Оскільки
    \[ \rang \left. \begin{pmatrix} 
        2 - \lambda & 1 \\
        -1 & 4 - \lambda 
    \end{pmatrix} \right|_{\lambda = 3} 
    = 
    \rang \begin{pmatrix} 
        -1 & 1 \\
        -1 & 1
    \end{pmatrix} = 1, \]
    то матриця має один власний вектор. Тому розв'язок шукаємо у вигляді
    \[ x = (a_1^1 + a_1^2 t) e^{3t}, \quad y = (a_2^1 + a_2^2 t) e^{3t}. \]
    
    Підставимо в систему
    \[ \left\{ \begin{aligned}
        3 e^{3t} (a_1^1 + a_1^2 t) + a_1^2 e^{3t} &= 2 (a_1^1 + a_1^2 t) e^{3t} + (a_2^1 + a_2^2) e^{3t}, \\
        3 e^{3t} (a_2^1 + a_2^2 t) + a_2^2 e^{3t} &= - (a_1^1 + a_1^2 t) e^{3t} + 4 (a_2^1 + a_2^2) e^{3t}.
    \end{aligned} \right. \]
    
    Прирівнявши коефіцієнти при однакових членах, одержимо дві системи
    \[ \left\{ \begin{aligned} 
        3 a_1^2 &= 2 a_1^2 + a_2^2, \\
        3 a_2^2 &= -a_1^2 + 4 a_2^2,
    \end{aligned} \right. 
    \qquad
    \left\{ \begin{aligned} 
        3 a_1^1 + a_1^2 &= 2 a_1^1 + a_2^1, \\
        3 a_2^1 + a_2^2 &= -a_1^1 + 4 a_2^1.
    \end{aligned} \right.\]

    Або
    \[ \left\{ \begin{aligned} 
        -a_1^2 + a_2^2 &= 0, \\
        -a_1^2 + a_2^2 &= 0,
    \end{aligned} \right. 
    \qquad
    \left\{ \begin{aligned} 
        -a_1^1 + a_2^1 &= a_1^2, \\
        -a_1^1 + a_2^1 &= a_1^2.
    \end{aligned} \right.\]

    З першої системи одержуємо $a_1^2 = a_2^2 = c_1$. Підставивши в другу, одержимо $-a_1^1 + a_2^1 = c_1$. Поклавши $a_1^1 = c_2$, одержимо $c_2^1 = c_1 + c_2$. Таким чином,
    \begin{multline*} \begin{pmatrix} x \\ y \end{pmatrix} = \begin{pmatrix} (c_2 + c_1 t) e^{3 t} \\ (c_1 + c_2 + c_1 t) e^{3 t} \end{pmatrix} = c_1 \begin{pmatrix} t e^{3 t} \\ (1 + t) e^{3 t} \end{pmatrix} + c_2 \begin{pmatrix} e^{3 t} \\ e^{3 t} \end{pmatrix} = \\ = \begin{pmatrix} t e^{3 t} & e^{3 t} \\ (1 + t) e^{3 t} & e^{3 t} \end{pmatrix} \begin{pmatrix} c_1 \\ c_2 \end{pmatrix}. \end{multline*}
\end{solution}

Розв'яжемо ці ж системи матричним методом.

\setcounter{problem}{0}
\begin{example}
    Розв'язати систему:
    \[ \left\{ \begin{aligned}
        \dot x &= 2 x + 3 y, \\
        \dot y &= 3 x + 4 y.
    \end{aligned} \right. \]
\end{example}

\begin{solution}
    Характеристичне рівняння має вигляд
    \[ \begin{vmatrix}
        2 - \lambda & 1 \\
        3 & 4 - \lambda 
    \end{vmatrix} = 0, \]
    або $\lambda^2 - 6 \lambda + 5 = 0$. \parvskip
    
    Його коренями будуть $\lambda_1 = 1$, $\lambda_2 = 5$. Тому 
    \[ \Lambda = \begin{pmatrix} 1 & 0 \\ 0 & 5 \end{pmatrix} \quad e^{\Lambda t} = \begin{pmatrix} e^t & 0 \\ 0 & e^{5 t} \end{pmatrix}. \]
    
    Розв'язуємо матричне рівняння $A S = S \Lambda$, або 
    \[ \begin{pmatrix} 2 & 1 \\ 3 & 4 \end{pmatrix} \begin{pmatrix} a_1^1 & a_1^2 \\ a_2^1 & a_2^2 \end{pmatrix} = \begin{pmatrix} a_1^1 & a_1^2 \\ a_2^1 & a_2^2 \end{pmatrix} \begin{pmatrix} 1 & 0 \\ 0 & 5 \end{pmatrix}. \]
    
    Воно розпадається на два 
    \[ \left\{ \begin{aligned} 
        2 a_1^1 + a_2^1 &= a_1^1, \\
        3 a_1^1 + 4 a_2^1 &= a_2^1,
    \end{aligned} \right. 
    \qquad
    \left\{ \begin{aligned} 
        2 a_1^2 + a_2^2 &= 5 a_1^2, \\
        3 a_1^2 + 4 a_2^2 &= 5 a_2^2,
    \end{aligned} \right.\]
    
    Після перенесення всіх членів уліво, одержимо
    \[ \left\{ \begin{aligned} 
        a_1^1 + a_2^1 &= 0, \\
        3  a_1^1 + 3 a_2^1 &= 0,
    \end{aligned} \right. 
    \qquad
    \left\{ \begin{aligned} 
        -3 a_1^2 + a_2^2 &= 0, \\
        3 a_1^2 - a_2^2 &= 0,
    \end{aligned} \right.\]
    
    Звідси $a_1^1 = 1$, $a_2^1 = - 1$, $a_1^2 = 1$, $a_2^2 = 3$. \parvskip
    
    Таким чином, загальний розв'язок має вигляд
    \[ S = \begin{pmatrix} 1 & 1 \\ -1 & 3 \end{pmatrix}, \quad \begin{pmatrix} x \\ y \end{pmatrix} = \begin{pmatrix} e^t & e^{5t} \\ -e^t & 3 e^{5 t} \end{pmatrix} \begin{pmatrix} c_1 \\ c_2 \end{pmatrix}. \]
\end{solution}

\begin{example}
    Розв'язати систему:
    \[ \left\{ \begin{aligned}
        \dot x &= x + y, \\
        \dot y &= -2 x + 3 y.
    \end{aligned} \right. \]
\end{example}
 
\begin{solution}
    Характеристичне рівняння має вигляд
    \[ \begin{vmatrix}
        1 - \lambda & 1 \\
        -2 & 3 - \lambda 
    \end{vmatrix} = 0, \]
    або $\lambda^2 - 4 \lambda + 5 = 0$. \parvskip
    
    Коренями будуть $\lambda_{1,2} = 2 \pm i$. Тому 
    \[ \Lambda = \begin{pmatrix} 2 & 1 \\ -1 & 2 \end{pmatrix} \quad e^{\Lambda t} = \begin{pmatrix} e^{2 t} \cos t & e^{2 t} \sin t \\ - e^{2 t} \sin t & e^{2 t} \cos t \end{pmatrix}. \]
    
    Матричне рівняння має вигляд $A S = S \Lambda$, чи
    \[ \begin{pmatrix} 1 & 1 \\ -2 & 3 \end{pmatrix} \begin{pmatrix} a_1^1 & a_1^2 \\ a_2^1 & a_2^2 \end{pmatrix} = \begin{pmatrix} a_1^1 & a_1^2 \\ a_2^1 & a_2^2 \end{pmatrix} \begin{pmatrix} 2 & 1 \\ -1 & 2 \end{pmatrix}. \]
    
    Розпишемо його поелементно
    \[ \left\{ \begin{aligned} 
        a_1^1 + a_2^1 &= 2 a_1^1 - a_1^2, \\
        -2 a_1^1 + 3 a_2^1 &= 2 a_2^1 - a_2^2, \\
        a_1^2 + a_2^2 &= a_1^1 + 2 a_1^2, \\
        -2 a_1^2 + 3 a_2^2 &= a_2^1 + 2 a_2^2.
    \end{aligned} \right.\]
    
    На відміну від попереднього пункту (і це істотно ускладнює обчислення) система не розщеплюється  на дві незалежні підсистеми. Після перенесення всіх членів в одну сторону, одержимо систему
    \[ \left\{ \begin{aligned} 
        - a_1^1 - a_1^2 + a_2^1 &= 0, \\
        -2 a_1^1 + a_2^1 + a_2^2 &= 0, \\
        -a_1^1 - a_1^2 + a_2^2 &= 0, \\
        -2 a_1^2 + a_2^1 + a_2^2 &= 0.
    \end{aligned} \right.\]
     
    Помножимо перше рівняння на $-2$ і, склавши з другим, підставимо на місце другого. Далі, помножимо перше рівняння на $-1$ і, склавши з третім, поставимо його на місце третього. Одержуємо систему
    \[ \left\{ \begin{aligned} 
        - a_1^1 + a_1^2 + a_2^1 &= 0, \\
        -2 a_1^2 - a_2^1 + a_2^2 &= 0, \\
        -2 a_1^2 - a_2^1 + a_2^2 &= 0, \\
        -2 a_1^2 - a_2^1 + a_2^2 &= 0.
    \end{aligned} \right.\]
     
    Останні два рівняння можна відкинути. Залишається
    \[ \left\{ \begin{aligned} 
        - a_1^1 + a_1^2 + a_2^1 &= 0, \\
        -2 a_1^2 - a_2^1 + a_2^2 &= 0.
    \end{aligned} \right.\]
    
    Покладаємо $a_1^2 = a_2^2 = 1$. Тоді $a_2^1 = -1$, $a_1^1 = 0$. Таким чином,
    \begin{multline*} S = \begin{pmatrix} 0 & 1 \\ -1 & 1 \end{pmatrix}, \quad \begin{pmatrix} x \\ y \end{pmatrix} = \begin{pmatrix} 0 & 1 \\ -1 & 1 \end{pmatrix} \begin{pmatrix} e^{2 t} \cos t & e^{2 t} \sin t \\ - e^{2 t} \sin t & e^{2 t} \cos t \end{pmatrix} \begin{pmatrix} c_1 \\ c_2 \end{pmatrix} = \\ = \begin{pmatrix} - e^{2 t} \sin t & e^{2 t} \cos t \\ - e^{2 t} (\cos t + \sin t) & e^{2 t} (\cos t - \sin t) \end{pmatrix} \begin{pmatrix} c_1 \\ c_2 \end{pmatrix}. \end{multline*}
\end{solution}

\begin{example}
    Розв'язати систему:
    \[ \left\{ \begin{aligned}
        \dot x &= 2 x + y, \\
        \dot y &= -x + 4 y.
    \end{aligned} \right. \]
\end{example}

\begin{solution}
    Характеристичне рівняння має вигляд
    \[ \begin{vmatrix}
        2 - \lambda & 1 \\
        -1 & 4 - \lambda 
    \end{vmatrix} = 0, \]
    або $\lambda^2 - 6 \lambda + 9 = 0$. \parvskip
    
    Коренями будуть $\lambda_1 = \lambda_2 = 3$. Оскільки
    \[ \rang \left. \begin{pmatrix} 
        2 - \lambda & 1 \\
        -1 & 4 - \lambda 
    \end{pmatrix} \right|_{\lambda = 3} 
    = 
    \rang \begin{pmatrix} 
        -1 & 1 \\
        -1 & 1
    \end{pmatrix} = 1, \]
    то матриця має один власний вектор і клітка Жордана має вигляд
    \[ \Lambda = \begin{pmatrix} 3 & 1 \\ 0 & 3 \end{pmatrix} \quad e^{\Lambda t} = \begin{pmatrix} e^{3 t} & t e^{3 t} \\ 0 t & e^{3 t} \cos t \end{pmatrix}. \]
    
    Матричне рівняння має вигляд $A S = S \Lambda$, чи
    \[ \begin{pmatrix} 2 & 1 \\ -1 & 4 \end{pmatrix} \begin{pmatrix} a_1^1 & a_1^2 \\ a_2^1 & a_2^2 \end{pmatrix} = \begin{pmatrix} a_1^1 & a_1^2 \\ a_2^1 & a_2^2 \end{pmatrix} \begin{pmatrix} 3 & 1 \\ 0 & 3 \end{pmatrix}. \]
    
    Розпишемо його поелементно
    \[ \left\{ \begin{aligned} 
        2 a_1^1 + a_2^1 &= 3 a_1^1, \\
        - a_1^1 + 4 a_2^1 &= 3 a_2^1, \\
    \end{aligned} \right. 
    \qquad
    \left\{ \begin{aligned} 
        2 a_1^2 + a_2^2 &= a_1^1 + 3 a_1^2, \\
        - a_1^2 + 4 a_2^2 &= a_2^1 + 3 a_2^2.
    \end{aligned} \right.\]
    
    На відміну від комплексних коренів, можна розв'язати спочатку першу підсистему, а потім другу. Перша має вид
    \[ \left\{ \begin{aligned} 
        - a_1^1 + a_2^1 &= 0, \\
        - a_1^1 + a_2^1 &= 0, \\
    \end{aligned} \right. \]
    
    Звідси $a_1^1 = a_2^1 = 1$. \parvskip
    
    Підставивши в другу, одержимо
    \[ \left\{ \begin{aligned} 
        - a_1^2 + a_2^2 &= 1, \\
        - a_1^2 + a_2^2 &= 1.
    \end{aligned} \right.\]
    
    Звідси $a_2^2 = 1$, $a_1^2 = 0$. Таким чином одержали
    \begin{multline*} S = \begin{pmatrix} 1 & 0 \\ 1 & 1 \end{pmatrix}, \quad \begin{pmatrix} x \\ y \end{pmatrix} = \begin{pmatrix} 1 & 0 \\ 1 & 1 \end{pmatrix} \begin{pmatrix} e^{3 t} & t e^{3 t} \\ -0 & e^{3 t} \end{pmatrix} \begin{pmatrix} c_1 \\ c_2 \end{pmatrix} = \\ = \begin{pmatrix} e^{3 t} & t e^{3 t} \\ e^{3 t} & (t + 1) e^{3 t} \end{pmatrix} \begin{pmatrix} c_1 \\ c_2 \end{pmatrix}. \end{multline*}
\end{solution}

\begin{remark}
    Якщо власні числа дійсні різні, то обидва методи еквівалентні. Якщо власні числа комплексні, переважніше метод Ейлера, якщо кратні, то матричний метод.
\end{remark}

Розв'язати лінійні однорідні системи методом Ейлера чи матричним методом.
\begin{multicols}{2}
    \begin{problem}
        \[ \left\{ \begin{aligned} 
            \dot x &= x - y, \\
            \dot y &= -4 x + y.
        \end{aligned} \right. \]
    \end{problem}
    
    \begin{problem}
        \[ \left\{ \begin{aligned} 
            \dot x &= -x + 8 y, \\
            \dot y &= x + y.
        \end{aligned} \right. \]
    \end{problem}
    
    \begin{problem}
        \[ \left\{ \begin{aligned} 
            \dot x &= x -3 y, \\
            \dot y &= 3 x + y.
        \end{aligned} \right. \]
    \end{problem}
    
    \begin{problem}
        \[ \left\{ \begin{aligned} 
            \dot x &= - x - 5 y, \\
            \dot y &= x + y.
        \end{aligned} \right. \]
    \end{problem}
    
    \begin{problem}
        \[ \left\{ \begin{aligned} 
            \dot x &= 3 x - y, \\
            \dot y &= 4 x - y.
        \end{aligned} \right. \]
    \end{problem}
    
    \begin{problem}
        \[ \left\{ \begin{aligned} 
            \dot x &= -3 x + 2 y, \\
            \dot y &= -2 x + y.
        \end{aligned} \right. \]
    \end{problem}
    
    \begin{problem}
        \[ \left\{ \begin{aligned} 
            \dot x &= 5 x + 3 y, \\
            \dot y &= -3 x - y.
        \end{aligned} \right. \]
    \end{problem}
\end{multicols}

Розв'язати лінійні однорідні системи методом Ейлера чи матричним методом (після системи вкзані власні числа для спрощення обчислень).
\begin{multicols}{2}
    \begin{problem}
        \[ \left\{ \begin{aligned} 
            \dot x &= x - y + z, \\
            \dot y &= x + y - z, \\
            \dot z &= 2 x - y.
        \end{aligned} \right. \]
        ($\lambda_1 = 1$, $\lambda_2 = 2$, $\lambda_3 = -1$)
    \end{problem}
    
    \begin{problem}
        \[ \left\{ \begin{aligned} 
            \dot x &= x - 2 y - z, \\
            \dot y &= -x + y + z, \\
            \dot z &= x - z.
        \end{aligned} \right. \]
        ($\lambda_1 = 0$, $\lambda_2 = 2$, $\lambda_3 = -1$)
    \end{problem}
    
    \begin{problem}
        \[ \left\{ \begin{aligned} 
            \dot x &= 2 x - y + z, \\
            \dot y &= x + 2 y - z, \\
            \dot z &= x - y + 2 z.
        \end{aligned} \right. \]
        ($\lambda_1 = 1$, $\lambda_2 = 2$, $\lambda_3 = 3$)
    \end{problem}
    
    \begin{problem}
        \[ \left\{ \begin{aligned} 
            \dot x &= 3 x - y + z, \\
            \dot y &= x + y + z, \\
            \dot z &= 4 x - y + 4 z.
        \end{aligned} \right. \]
        ($\lambda_1 = 1$, $\lambda_2 = 2$, $\lambda_3 = 5$)
    \end{problem}
    
    \begin{problem}
        \[ \left\{ \begin{aligned} 
            \dot x &= -3 x - 4 y - 2 z, \\
            \dot y &= x + z, \\
            \dot z &= 6 z - 6 y + 5 z.
        \end{aligned} \right. \]
        ($\lambda_1 = 1$, $\lambda_2 = 2$, $\lambda_3 = -1$)
    \end{problem}
    
    \begin{problem} 
        \[ \left\{ \begin{aligned} 
            \dot x &= x - y - z, \\
            \dot y &= x + y, \\
            \dot z &= 3 x + z.
        \end{aligned} \right. \]
        ($\lambda_1 = 1$, $\lambda_{2, 3} = 1 +\pm 3 i$)
    \end{problem}
    
    \begin{problem}
        \[ \left\{ \begin{aligned} 
            \dot x &= 2 x + y, \\
            \dot y &= x + 3 y - z, \\
            \dot z &= -x + y - z.
        \end{aligned} \right. \]
        ($\lambda_1 = 2$, $\lambda_{2, 3} = 3 \pm i$)
    \end{problem}
    
    \begin{problem}
        \[ \left\{ \begin{aligned} 
            \dot x &= 2 x - y + 2 z, \\
            \dot y &= x + z, \\
            \dot z &= -2 x - y + 2 z.
        \end{aligned} \right. \]
        ($\lambda_1 = 2$, $\lambda_{2, 3} = \pm i$)
    \end{problem}
    
    \begin{problem}
        \[ \left\{ \begin{aligned} 
            \dot x &= 4 x - y - z, \\
            \dot y &= x + 2 y - z, \\
            \dot z &= x - y + 2 z.
        \end{aligned} \right. \]
        ($\lambda_1 = 2$, $\lambda_2 = \lambda_3 = 3$)
    \end{problem}
    
    \begin{problem}
        \[ \left\{ \begin{aligned} 
            \dot x &= 2 x - y - z, \\
            \dot y &= 3 x - 2 y - 3 z, \\
            \dot z &= -x + y + 2 z.
        \end{aligned} \right. \]
        ($\lambda_1 = 0$, $\lambda_2 = \lambda_3 = 1$)
    \end{problem}
    
    \begin{problem}
        \[ \left\{ \begin{aligned} 
            \dot x &= - 2 x + y - 2 z, \\
            \dot y &= x - 2 y + 2 z, \\
            \dot z &= 3 x - 3 y + 5 z.
        \end{aligned} \right. \]
        ($\lambda_1 = 3$, $\lambda_2 = \lambda_3 = -1$)
    \end{problem}
    
    \begin{problem}
        \[ \left\{ \begin{aligned} 
            \dot x &= 3 x - 2 y - z, \\
            \dot y &= 3 x - 4 y + 2 z, \\
            \dot z &= 2 x - 4 y.
        \end{aligned} \right. \]
        ($\lambda_1 = \lambda_2 = 2$, $\lambda_3 = -5$)
    \end{problem}
    
    \begin{problem}
        \[ \left\{ \begin{aligned} 
            \dot x &= x - y + z, \\
            \dot y &= x + y - z, \\
            \dot z &= - y + 2 z.
        \end{aligned} \right. \]
        ($\lambda_1 = \lambda_2 = 1$, $\lambda_3 = 2$)
    \end{problem}

    \begin{problem}
        \[ \left\{ \begin{aligned} 
            \dot x &= - x + y - 2 z, \\
            \dot y &= 4 x + y, \\
            \dot z &= - y + 2 z.
        \end{aligned} \right. \]
        ($\lambda_1 = 1$, $\lambda_2 = \lambda_3 = -1$)
    \end{problem}
\end{multicols}
    
\begin{multicols}{2}
    \begin{problem}
        \[ \left\{ \begin{aligned} 
            \dot x &= 2 x + y, \\
            \dot y &= 2 y + 4 z, \\
            \dot z &= x - z.
        \end{aligned} \right. \]
        ($\lambda_1 = \lambda_2 = 0$, $\lambda_3 = 3$)
    \end{problem}
    
    \begin{problem}
        \[ \left\{ \begin{aligned} 
            \dot x &= 2 x - y - z, \\
            \dot y &= 2 x - y - z, \\
            \dot z &= - x + z.
        \end{aligned} \right. \]
        ($\lambda_1 = \lambda_2 = \lambda_3 = 2$)
    \end{problem}
\end{multicols}
