Рівняння вигляду
\begin{equation*}
    \frac{\diff y}{\diff x} = f(x) g(y),
\end{equation*}
або більш загального вигляду
\begin{equation*}
    f_1(x) f_2(y) \diff x + g_1(x) g_2(y) \diff y = 0
\end{equation*}
називаються рівняннями зі змінними, що розділяються. Розділимо його на $f_2(y) g_1(x)$ і одержимо рівняння з розділеними змінними:
\begin{equation*}
    \frac{f_1(x)}{g_1(x)} \diff x + \frac{g_2(y)}{f_2(y)} \diff y = 0.
\end{equation*}

Узявши інтеграли, отримаємо
\begin{equation*}
    \int \frac{f_1(x)}{g_1(x)} \diff x + \int \frac{g_2(y)}{f_2(y)} \diff y = C,
\end{equation*}
або
\begin{equation*}
    \Phi(x, y) = C.
\end{equation*}

\begin{definition}
    Це кінцеве рівняння, що визначає розв'язок диференціального рівняння як неявну функцію від $x$, називається інтегралом розглянутого рівняння.
\end{definition}

\begin{definition}
    Це ж рівняння, що визначає всі без винятку розв'язки даного диференціального рівняння, називається загальним інтегралом.
\end{definition}

Бувають випадки (в основному), що невизначені інтеграли з рівняння з розділеними змінними не можна записати в елементарних функціях. Попри це, задача інтегрування вважається виконаною. Кажуть, що диференціальне рівняння розв'язне у квадратурах. \parvskip

Можливо, що інтеграл рівняння розв'язується відносно $y$:
\begin{equation*}
    y = y(x, C).
\end{equation*}

Тоді, завдяки вибору $C$, можна одержати всі розв'язки.

\begin{definition}
    Ця залежність, що тотожно задовольняє вихідному диференціальному рівнянню, де $C$ --- довільна стала, називається загальним розв'язком диференціального рівняння.
\end{definition}

Геометрично загальний розв'язок являє собою сім'ю кривих, що не перетинаються, які заповнюють деяку область. Іноді треба виділити одну криву сім'ї, що проходить через задану точку $M(x_0, y_0)$.

\begin{definition}
    Знаходження розв'язку $y = y(x)$, що проходить через задану точку $M(x_0, y_0)$, називається розв'язком задачі Коші.
\end{definition}

\begin{definition}
    Розв'язок, який записаний у вигляді $y = y(x, x_0, y_0)$ і задовольняє умові $y(x, x_0, y_0) = y_0$, називається розв'язком у формі Коші.
\end{definition}