% cd ..\..\Users\NikitaSkybytskyi\Desktop\differential-equations
% pdflatex chapter-3.tex && cls && pdflatex chapter-3.tex && cls && pdflatex chapter-3.tex && del chapter-3.toc, chapter-3.log, chapter-3.aux, chapter-3.out && start chapter-3.pdf

\documentclass[a4paper, 12pt]{article}
\usepackage[utf8]{inputenc}
\usepackage[T2A,T1]{fontenc}
\usepackage[english, ukrainian]{babel}
\usepackage{amsmath, amssymb, natbib, float, multirow, multicol, xcolor, hyperref}

\allowdisplaybreaks
\setlength\parindent{0pt}

\title{Диференціальні рівняння}
\author{Скибицький Нікіта}
\date{\today}

\hypersetup{unicode=true, colorlinks=true, linktoc=all, linkcolor=red}

\usepackage{amsthm}
\newtheorem{theorem}{Теорема}[section]
\newtheorem{lemma}{Лема}[section]
\theoremstyle{definition}
\newtheorem*{definition}{Визначення}
\newtheorem{problem}{Задача}[subsection]
\newtheorem*{example*}{Приклад}
\newtheorem{example}[problem]{Приклад}
\newtheorem{property}{Властивість}
\newtheorem*{solution}{Розв'язок}
\newtheorem*{remark}{Зауваження}

\renewcommand{\phi}{\varphi}
\renewcommand{\epsilon}{\varepsilon}
\newcommand{\RR}{\mathbb{R}}
\newcommand{\NN}{\mathbb{N}}

\DeclareMathOperator{\trace}{tr}

\newcommand{\todo}{\texttt{[TO DO]}}

\newcommand*\diff{\mathop{}\!\mathrm{d}}
\newcommand*\rfrac[2]{{}^{#1}\!/_{\!#2}}

\numberwithin{equation}{section}% reset equation counter for sections
\numberwithin{equation}{subsection}% Omit `.0` in equation numbers for non-existent subsections.
\renewcommand*{\theequation}{%
	\ifnum\value{subsection}=0%
		\thesection%
	\else%
		\thesubsection%
	\fi%
	.\arabic{equation}%
}

\makeatletter
\def\old@comma{,}
\catcode`\,=13
\def,{%
	\ifmmode%
		\old@comma\discretionary{}{}{}%
	\else%
		\old@comma%
	\fi%
}
\makeatother

\newcommand{\parvskip}{\vspace{1em}}

\begin{document}

\setcounter{section}{2}
\section{Лі\-ній\-ні ди\-фе\-рен\-ці\-аль\-ні рів\-ня\-н\-ня ви\-щих \allowbreak по\-ряд\-ків}
Рівняння вигляду
\begin{equation}
	\label{eq:3.1}
	a_0(x) \cdot y^{(n)} + a_1(x) \cdot y^{(n - 1)} + \ldots + a_n(x) \cdot y = b(x)
\end{equation} 
називається лінійним неоднорідним диференціальним рівнянням $n$-го порядку. \\

Рівняння вигляду
\begin{equation}
	\label{eq:3.2}
	a_0(x) \cdot y^{(n)} + a_1(x) \cdot y^{(n - 1)} + \ldots + a_n(x) \cdot y = 0
\end{equation} 
називається лінійним однорідним диференціальним рівнянням $n$-го порядку. \\

Якщо при $x \in [a, b]$, $a_0(x) \ne 0$ коефіцієнти $b(x)$, $a_i(x)$, $i=\overline{0,n}$ неперервні, то для рівняння
\begin{equation}
	\label{eq:3.3}
	y^{(n)} = - \frac{a_1(x)}{a_0(x)} \cdot y^{(n - 1)} - \ldots - \frac{a_n(x)}{a_0(x)} \cdot y + \frac{b(x)}{a_0(x)}.
\end{equation}
виконуються умови теореми існування та єдиності і існує єдиний розв’язок $y = y(x)$, що задовольняє початковим умовам
\begin{equation}
	\label{eq:3.4}
	y(x_0) = y_0, \quad y'(x_0) = y_0', \quad \ldots, \quad y^{(n - 1)} = y_0^{(n - 1)}.
\end{equation}

\subsection{Лінійні однорідні рівняння}

\subsubsection{Властивості лінійних однорідних рівнянь}

\begin{theorem}
	Лінійність і однорідність зберігаються при довільному перетворенні незалежної змінної $x = \phi(t)$.
\end{theorem}
\begin{proof}
	Справді, після заміни $x = \phi(t)$, одержимо
	\begin{align}
		\label{eq:3.1.1}
		y_x' &= \frac{\diff y}{\diff x} = \frac{\diff y}{\diff t} \cdot \frac{\diff t}{\diff x} = \frac{1}{\phi'(t)} \cdot \frac{\diff y}{\diff t}, \\
		\label{eq:3.1.2}
		y_{x^2}'' &= \frac{\diff}{\diff x} \cdot y_x' = \frac{\diff}{\diff t} \left( \frac{1}{\phi'(t)} \cdot \frac{\diff y}{\diff t} \right) \cdot \frac{1}{\phi'(t)} = \\
		&= - \frac{\phi''(t)}{(\phi'(t))^2} \cdot \frac{\diff y}{\diff t} + \frac{1}{(\phi'(t))^2} \cdot \frac{\diff^2 y}{\diff t^2}, \nonumber
	\end{align}
	і так далі до $n$-го порядку. Після підстановки і приведення подібних, знову отримуємо лінійне однорідне рівняння
	\begin{equation}
		\label{eq:3.1.3}
		A_0(t) \cdot \frac{\diff^n y}{\diff t^n} + A_1(t) \cdot \frac{\diff^{n - 1} y}{\diff t^{n - 1}} + \ldots + A_n(t) \cdot y = 0.
	\end{equation}
\end{proof}

\begin{theorem}
	Лінійність і однорідність зберігаються при лінійному перетворенні невідомої функції $y = \alpha (x) \cdot z$.
\end{theorem}
\begin{proof}
	Справді, після заміни $y = \alpha (x) \cdot z$, одержимо
	\begin{align}
		\label{eq:3.1.4}
		y_x' &= \alpha'(x) \cdot z + \alpha(x) \cdot z', \\
		\label{eq:3.1.5}
		y_{x^2}'' &= \alpha''(x) \cdot z + 2 \alpha'(x) \cdot z' + \alpha(x) \cdot z'',
	\end{align}
	і так далі до $n$-го порядку. Після підстановки знову отримаємо лінійне однорідне рівняння
	\begin{equation}
		\label{eq:3.1.6}
		\bar A_0(x) \cdot z^{(n)} + \bar A_1(x) \cdot z^{(n - 1)} + \ldots + \bar A_n(x) \cdot z = 0.
	\end{equation}
\end{proof}

	\subsection{Лінійні однорідні рівняння}
	\input{subsection-3-1.tex}

		\subsubsection{Властивості лінійних однорідних рівнянь}
		\begin{theorem}
	Лінійність і однорідність зберігаються при довільному перетворенні незалежної змінної $x = \phi(t)$.
\end{theorem}
\begin{proof}
	Справді, після заміни $x = \phi(t)$, одержимо
	\begin{align*}
		%\label{eq:3.1.1}
		y_x' &= \frac{\diff y}{\diff x} = \frac{\diff y}{\diff t} \cdot \frac{\diff t}{\diff x} = \frac{1}{\phi'(t)} \cdot \frac{\diff y}{\diff t}, \\
		%\label{eq:3.1.2}
		y_{x^2}'' &= \frac{\diff}{\diff x} \cdot y_x' = \frac{\diff}{\diff t} \left( \frac{1}{\phi'(t)} \cdot \frac{\diff y}{\diff t} \right) \cdot \frac{1}{\phi'(t)} = \\
		&= - \frac{\phi''(t)}{(\phi'(t))^2} \cdot \frac{\diff y}{\diff t} + \frac{1}{(\phi'(t))^2} \cdot \frac{\diff^2 y}{\diff t^2}, \nonumber
	\end{align*}
	і так далі до $n$-го порядку. Після підстановки і приведення подібних, знову отримуємо лінійне однорідне рівняння
	\begin{equation*}
		%\label{eq:3.1.3}
		A_0(t) \cdot \frac{\diff^n y}{\diff t^n} + A_1(t) \cdot \frac{\diff^{n - 1} y}{\diff t^{n - 1}} + \ldots + A_n(t) \cdot y = 0.
	\end{equation*}
\end{proof}

\begin{theorem}
	Лінійність і однорідність зберігаються при лінійному перетворенні невідомої функції $y = \alpha (x) \cdot z$.
\end{theorem}
\begin{proof}
	Справді, після заміни $y = \alpha (x) \cdot z$, одержимо
	\begin{align*}
		%\label{eq:3.1.4}
		y_x' &= \alpha'(x) \cdot z + \alpha(x) \cdot z', \\
		%\label{eq:3.1.5}
		y_{x^2}'' &= \alpha''(x) \cdot z + 2 \alpha'(x) \cdot z' + \alpha(x) \cdot z'',
	\end{align*}
	і так далі до $n$-го порядку. Після підстановки знову отримаємо лінійне однорідне рівняння
	\begin{equation*}
		%\label{eq:3.1.6}
		\bar A_0(x) \cdot z^{(n)} + \bar A_1(x) \cdot z^{(n - 1)} + \ldots + \bar A_n(x) \cdot z = 0.
	\end{equation*}
\end{proof}

		\subsubsection{Властивості роз\-в'яз\-ків лінійних однорідних рівнянь}
		\begin{theorem}
	Якщо $y = y_1(x)$ є розв'язком однорідного лінійного рівняння, то і $y = C y_1 (x)$, де $C$ --- довільна стала, теж буде розв'язком однорідного лінійного рівняння.
\end{theorem}

\begin{proof}
	Справді, нехай $y = y_1(x)$ --- розв'язок лінійного однорідного рівняння, тобто
	\begin{equation*}
		a_0(x) y_1^{(n)} (x) + a_1(x) y_1^{(n - 1)} (x) + \ldots + a_n(x) y_1(x) \equiv 0.
	\end{equation*}

	Тоді і
	\begin{multline*}
		a_0(x) (C y_1)^{(n)}(x) + a_1(x) (C y_1)^{(n - 1)}(x) + \ldots + a_n(x) (C y_1)(x) = \\
		= C \left( a_0(x) y_1^{(n)} (x) + a_1(x) y_1^{(n - 1)} (x) + \ldots + a_n(x) y_1(x) \right) \equiv 0,
	\end{multline*}
	оскільки вираз в дужках дорівнює нулю.
\end{proof}

\begin{theorem}
	Якщо $y_1(x)$ і $y_2(x)$ є розв'язками лінійного однорідного рівняння, то і $y = y_1(x) + y_2(x)$ теж буде розв'язком лінійного однорідного рівняння.
\end{theorem}

\begin{proof}
	Справді, нехай $y_1(x)$ і $y_2(x)$ --- розв'язки лінійного рівняння, тобто
	\begin{align*}
		a_0(x) y_1^{(n)} (x) + a_1(x) y_1^{(n - 1)} (x) + \ldots + a_n(x) y_1(x) &\equiv 0, \\
		a_0(x) y_2^{(n)} (x) + a_1(x) y_2^{(n - 1)} (x) + \ldots + a_n(x) y_2(x) &\equiv 0.
	\end{align*}

	Тоді і
	\begin{multline*}
		a_0(x) (y_1 + y_2)^{(n)} (x) + a_1(x) (y_1 + y_2)^{(n - 1)} (x) + \ldots + a_n(x) (y_1 + y_2) (x) = \\
		= \left( a_0(x) y_1^{(n)} (x) + a_1(x) y_1^{(n - 1)} (x) + \ldots + a_n(x) y_1(x) \right) + \\
		+ \left( a_0(x) y_2^{(n)} (x) + a_1(x) y_2^{(n - 1)} (x) + \ldots + a_n(x) y_2(x) \right) \equiv 0,
	\end{multline*}
	оскільки обидві дужки дорівнюють нулю.
\end{proof}

\begin{theorem}
	Якщо $y_1(x), y_2(x), \ldots, y_n(x)$ --- розв'язки однорідного лінійного рівняння, то і  $y = \sum_{i=1}^n C_i y_i(x)$, де $C_i$ --- довільні сталі, також буде розв'язком лінійного однорідного рівняння.
\end{theorem}

\begin{proof}
	Справді, нехай $y_1(x), y_2(x), \ldots, y_n(x)$ --- розв'язки лінійного однорідного рівняння, тобто
	\begin{equation*}
		a_0(x) y_i^{(n)} (x) + a_1(x) y_i^{(n - 1)} (x) + \ldots + a_n(x) y_i(x) \equiv 0, \quad i = \overline{1, n}.
	\end{equation*}
	
	Тоді і   
 	\begin{multline*}
 		a_0(x) \left(\sum_{i=1}^n C_i y_i\right)^{(n)} (x) + a_1(x) \left(\sum_{i=1}^n C_i y_i\right)^{(n - 1)} (x) + \ldots \\
 		\ldots + a_{n-1}(x) \left(\sum_{i=1}^n C_i y_i\right)'(x) + a_n(x) \left(\sum_{i=1}^n C_i y_i\right)(x) = \\
 		= \sum_{i=1}^n C_i  \left( a_0(x) y_i^{(n)} (x) + a_1(x) y_i^{(n - 1)} (x) + \ldots + a_n(x) y_i(x) \right) \equiv 0,
 	\end{multline*}
	оскільки кожна дужка дорівнює нулю.
\end{proof}

\begin{theorem}
	Якщо комплексна функція дійсного аргументу, тобто $y = u(x) + i v(x)$ є розв'язком лінійного однорідного рівняння, то окремо дійсна частина $u(x)$ і уявна $v(x)$ будуть також розв'язками цього рівняння.
\end{theorem}

\begin{proof}
	Справді, нехай $y = u(x) + i v(x)$ є розв'язком лінійного однорідного рівняння, тобто
	\begin{multline*}
		a_0(x) (u) + i v)^{(n)} (x) + a_1(x) (u + i v)^{(n - 1)} (x) + \ldots \\
		\ldots + a_{n - 1}(x) (u + i v)' (x) + a_n(x) (u + i v) (x) \equiv 0.
	\end{multline*}

	Розкривши дужки і перегрупувавши члени, одержимо
	\begin{multline*}
		\left( a_0(x) u^{(n)}(x) + a_1(x) u^{(n - 1)} (x) + \ldots + a_n(x) u(x) \right) + \\
		+ i \left( a_0(x) v^{(n)}(x) + a_1(x) v^{(n - 1)} (x) + \ldots + a_n(x) v(x) \right) \equiv 0.
	\end{multline*}

	Комплексний вираз дорівнює нулю тоді і тільки тоді, коли дорівнюють нулю дійсна і уявна частини, тобто
 	\begin{align*}
		a_0(x) u^{(n)}(x) + a_1(x) u^{(n - 1)} (x) + \ldots + a_n(x) u(x) &\equiv 0, \\
		a_0(x) v^{(n)}(x) + a_1(x) v^{(n - 1)} (x) + \ldots + a_n(x) v(x) &\equiv 0,
	\end{align*}
	або функції $u(x)$, $v(x)$ є розв'язками рівняння, що і було потрібно довести.
\end{proof}


		\subsubsection{Лінійна залежність і незалежність роз\-в'яз\-ків. Загальний роз\-в'яз\-ок лінійного однорідного рівняння вищого порядку}
		\begin{definition}
	Функції $y_0(x), y_1(x), \ldots, y_n(x)$ називаються лінійно залежними на відрізку $[a,b]$ якщо існують не всі рівні нулю сталі $C_0, \ldots, C_n$ такі, що при всіх $x \in [a,b]$: 
	\begin{equation}
		\label{eq:3.1.18}
		C_0 \cdot y_0(x) + C_1 \cdot y_1(x) + \ldots + C_n \cdot y_n(x) = 0.
	\end{equation}

	Якщо ж тотожність справедлива лише коли $C_0 = C_1 = \ldots = C_n = 0$, то функції $y_1(x), y_2(x), \ldots, y_n(x)$ називаються лінійно незалежними.
\end{definition}

\textbf{Приклади:}
\begin{enumerate}
	\item Функції $1, x, x^2, \ldots, x^n$ -- лінійно незалежні на будь-якому відрізку $[a,b]$, тому що вираз $C_0 + C_1 x + \ldots + C_n x^n$ є многочленом ступеню $n$ і має не більш, ніж $n$ дійсних коренів.
	\item Функції $e^{\lambda_1 x}, e^{\lambda_2 x}, \ldots, e^{\lambda_n x}$, де всі $\lambda_i$ -- дійсні різні числа -- лінійно незалежні. 
	\item Функції $1, \sin x, \cos x, \ldots, \sin nx, \cos nx$ -- лінійно незалежні.
\end{enumerate}

\begin{theorem}[необхідна умова лінійної незалежності функцій]
	Якщо функції $y_0(x), y_1(x), \ldots, y_n(x)$ -- лінійно залежні, то визначник Вронського $W[y_0, y_1, \ldots, y_n](x)$ тотожно дорівнює нулю при всіх $x \in [a,b]$:
	\begin{equation}
		\label{eq:3.1.19}
		W[y_0, y_1, \ldots, y_n](x) = \begin{vmatrix} y_0(x) & y_1(x) & \cdots & y_n(x) \\ y_0'(x) & y_1'(x) & \cdots & y_n'(x) \\ \vdots & \vdots & \ddots & \vdots \\ y_0^{(n)}(x) & y_1^{(n)}(x) & \cdots & y_n^{(n)}(x) \end{vmatrix} = 0.
	\end{equation}
\end{theorem}

\begin{proof}
	Нехай $y_0(x), y_1(x), \ldots, y_n(x)$ -- лінійно залежні. Тоді існують не всі рівні нулю сталі $C_0, \ldots, C_n$ такі, що при $x \in [a,b]$ буде тотожно виконуватися: \eqref{eq:3.1.18}.	Продиференціювавши $n$ разів, одержимо 
	\begin{equation}
		\label{eq:3.1.20}
		\left\{ \begin{aligned}
			C_0 \cdot y_0(x) + C_1 \cdot y_1(x) + \ldots + C_n \cdot y_n(x) &= 0, \\
			C_0 \cdot y_0'(x) + C_1 \cdot y_1'(x) + \ldots + C_n \cdot y_n(x) &= 0, \\
			\ldots \ldots \ldots \ldots \ldots \ldots \ldots \ldots \ldots \ldots \ldots \ldots \ldots & \ldots \ldots \\
			C_0 \cdot y_0^{(n)}(x) + C_1 \cdot y_1^{(n)}(x) + \ldots + C_n \cdot y_n^{(n)}(x) &= 0.
		\end{aligned} \right.
	\end{equation}
 
	Для кожного фіксованого $x \in [a,b]$ одержимо лінійну однорідну систему алгебраїчних рівнянь, що має ненульовий розв’язок $C_0, \ldots, C_n$. А це можливо тоді і тільки тоді, коли визначник системи дорівнює нулю, тобто $W[y_0, y_1, \ldots, y_n](x) = 0$ при всіх $x \in [a,b]$.
\end{proof}

\begin{theorem}[достатня умова лінійної незалежності розв’язків]
	Якщо розв’язки лінійного однорідного рівняння $y_0(x), y_1(x), \ldots, y_n(x)$ -- лінійно незалежні, то визначник Вронського $W[y_0, y_1, \ldots, y_n](x)$ не дорівнює нулю в жодній точці $x \in [a,b]$.
\end{theorem} 

\begin{proof}
	Припустимо, від супротивного, що існує $x_0 \in [a,b]$, при якому $W[y_0, y_1, \ldots, y_n](x_0) = 0$. Оскільки визначник дорівнює нулю, то існує ненульовий розв’язок $C_0^0, C_1^0, \ldots, C_n^0$ лінійної однорідної системи алгебраїчних рівнянь \eqref{eq:3.1.19}. Розглянемо лінійну комбінацію 
	\begin{equation}
		\label{eq:3.1.21}
		y(x) = C_0^0 y_0(x) + C_1 y_1(x) + \ldots + C_n y_n(x)
	\end{equation}
	з отриманими коефіцієнтами. \\

	У силу третьої властивості ця комбінація буде розв’язком. У силу вибору сталих $C_0^0, C_1^0, \ldots, C_n^0$, розв’язок буде задовольняти умовам
	\begin{equation}
		\label{eq:3.1.22}
		y(x_0) = y'(x_0) = \ldots = y^{(n)}(x_0) = 0.
	\end{equation}
 
	Але цим же умовам, як неважко перевірити простою підстановкою, задовольняє і тотожний нуль, тобто $y \equiv 0$. І в силу теореми існування та єдиності ці два розв’язки співпадають, тобто 
	\begin{equation}
		\label{eq:3.1.23}
		y(x) = C_0^0 y_0(x) + C_1 y_1(x) + \ldots + C_n y_n(x) = 0
	\end{equation}
	при $x \in [a,b]$, або система функцій $y_0(x), y_1(x), \ldots, y_n(x)$ лінійно залежна, що суперечить припущенню. Таким чином $W[y_0, y_1, \ldots, y_n](x_0) \ne 0$ у жодній точці $x_0 \in [a,b]$, що і було потрібно довести .
\end{proof}

На підставі попередніх двох теорем сформулюємо необхідні і достатні умови лінійної незалежності розв’язків лінійного однорідного рівняння.

\begin{theorem}
	Для того щоб розв’язки лінійного однорідного диференціального рівняння $y_0(x), y_1(x), \ldots, y_n(x)$ були лінійно незалежними, необхідно і достатньо, щоб визначник Вронського не дорівнював нулю в жодній точці $x \in [a,b]$, тобто $W[y_0, y_1, \ldots, y_n](x) \ne 0$.
\end{theorem}

\begin{theorem}
	Загальним розв’язком лінійного однорідного рівняння
	\begin{equation}
		\label{eq:3.1.24}
		a_0(x) \cdot y^{(n)} + a_1(x) \cdot y^{(n-1)} + \ldots + a_{n-1}(x) \cdot y' + a_n \cdot y = 0
	\end{equation}
 	є лінійна комбінація $n$ лінійно незалежних розв’язків $y = \sum_{i = 1}^n C_i \cdot y_i(x)$.
\end{theorem}

\begin{proof}
	Оскільки $y_i(x)$, $i = 1, 2, \ldots, n$ є розв’язками, то в силу третьої властивості їхня лінійна комбінація також буде розв’язком. \\

	Покажемо, що цей розв’язок загальний, тобто вибором сталих $C_1, \ldots, C_n$ можна розв’язати довільну задачу Коші
	\begin{equation}
		\label{eq:3.1.25}
		y(x_0) = y_0, \quad y'(x_0) = y_0', \quad \ldots, \quad y^{(n - 1)}(x_0) = y_0^{(n - 1)}.
	\end{equation}

	Дійсно, оскільки система розв’язків лінійно незалежна, то визначник Вронського відмінний від нуля й алгебраїчна система неоднорідних рівнянь
	\begin{equation}
		\label{eq:3.1.26}
		\left\{ \begin{aligned}
			C_1 \cdot y_1(x_0) + C_2 \cdot y_2(x_0) + \ldots + C_n \cdot y_n(x_0) &= y_0, \\
			C_1 \cdot y_1'(x_0) + C_2 \cdot y_2'(x_0) + \ldots + C_n \cdot y_n(x_0) &= y_0', \\
			\ldots \ldots \ldots \ldots \ldots \ldots \ldots \ldots \ldots \ldots \ldots \ldots \ldots \ldots & \ldots \ldots \\
			C_1 \cdot y_1^{(n - 1)}(x_0) + C_2 \cdot y_2^{(n - 1)}(x_0) + \ldots + C_n \cdot y_n^{(n)}(x_0) &= y_0^{(n - 1)},
		\end{aligned} \right.
	\end{equation}
	має єдиний розв’язок $C_1^0, C_2^0, \ldots, C_n^0$. І лінійна комбінація $y = \sum_{i = 1}^n C_i^0 \cdot y_i(x)$ є розв’язком, причому, як видно із системи алгебраїчних рівнянь, буде задовольняти довільно обраним умовам Коші.
\end{proof}

Зауважимо, що максимальне число лінійно незалежних розв’язків дорівнює порядку рівняння. Це випливає з попередньої теореми, тому що будь-який розв’язок виражається через лінійну комбінацію $n$ лінійно незалежних розв’язків.

\begin{definition}
	Будь-які $n$ лінійно незалежних розв’язків лінійного однорідного рівняння $n$-го порядку називаються фундаментальною системою розв’язків.
\end{definition}

		\subsubsection{Формула Остроградського-Ліувіля}
		Оскільки максимальне число лінійно незалежних розв’язків дорівнює $n$, то система $y_1(x) , \ldots, y_n(x), y(x)$ буде залежною і $W[y_1,\ldots,y_n,y]\equiv0$, тобто
\begin{equation*}
	%\label{eq:3.1.27}
	\begin{vmatrix}
		y_1 & \cdots & y_n & y \\
		y_1' & \cdots & y_n' & y' \\
		\vdots & \ddots & \vdots & \vdots \\
		y_1^{(n)} & \cdots & y_n^{(n)} & y'
	\end{vmatrix} \equiv 0.
\end{equation*}
 
Розкладаючи визначник по елементах останнього стовпця, одержимо
 
\begin{multline}
	%\label{eq:3.1.28}
	\begin{vmatrix}
		y_1 & y_2 & \cdots & y_n \\
		y_1' & y_2' & \cdots & y_n' \\
		\vdots & \vdots & \ddots & \vdots \\
		y_1^{(n - 1)} & y_2^{(n - 1)} & \cdots & y_n^{(n - 1)} \\
	\end{vmatrix}  y^{(n)} 
	- 
	\begin{vmatrix}
		y_1 & y_2 & \cdots & y_n \\
		\vdots & \vdots & \ddots & \vdots \\
		y_1^{(n - 2)} & y_2^{(n - 2)} & \cdots & y_n^{(n - 2)} \\
		y_1^{(n)} & y_2^{(n)} & \cdots & y_n^{(n)}
	\end{vmatrix}  y^{(n - 1)} + \ldots \\
	\ldots + (-1)^{n - 1}
	\begin{vmatrix}
		y_1 & y_2 & \cdots & y_n \\
		y_1'' & y_2'' & \cdots & y_n'' \\
		\vdots & \vdots & \ddots & \vdots \\
		y_1^{(n)} & y_2^{(n)} & \cdots & y_n^{(n)}
	\end{vmatrix}  y'
	+ (-1)^n  
	\begin{vmatrix}
		y_1' & y_2' & \cdots & y_n' \\
		y_1'' & y_2'' & \cdots & y_n'' \\
		\vdots & \vdots & \ddots & \vdots \\
		y_1^{(n)} & y_2^{(n)} & \cdots & y_n^{(n)}
	\end{vmatrix}  y\equiv 0.
\end{multline}

Порівнюючи з рівнянням 
\begin{equation*}
	%\label{eq:3.1.29}
	a_0(x) \cdot y^{(n)} + a_1(x) \cdot y^{(n - 1)} + \ldots + a_n(x) \cdot y = 0
\end{equation*}
одержимо, що
\begin{equation*}
	%\label{eq:3.1.30}
	\frac{a_1(x)}{a_0(x)} = - \frac{\begin{vmatrix}
		y_1(x) & y_2(x) & \cdots & y_n(x) \\
		\vdots & \vdots & \ddots & \vdots \\
		y_1^{(n - 2)}(x) & y_2^{(n - 2)}(x) & \cdots & y_n^{(n - 2)}(x) \\
		y_1^{(n)}(x) & y_2^{(n)}(x) & \cdots & y_n^{(n)}(x)
	\end{vmatrix}}{W[y_1, y_2, \ldots, y_n](x)}.
\end{equation*}
Але оскільки
\begin{multline}
	%\label{eq:3.1.31}
	\frac{\diff}{\diff x} W[y_1, y_2, \ldots, y_n] = \begin{vmatrix}
		y_1' & y_2' & \cdots & y_n' \\
		y_1' & y_2' & \cdots & y_n' \\
		\vdots & \vdots & \ddots & \vdots \\
		y_1^{(n - 2)} & y_2^{(n - 2)} & \cdots & y_n^{(n - 2)} \\
		y_1^{(n - 1)} & y_2^{(n - 1)} & \cdots & y_n^{(n - 1)}
	\end{vmatrix} + \\ + \begin{vmatrix}
		y_1 & y_2 & \cdots & y_n \\
		y_1'' & y_2'' & \cdots & y_n'' \\
		\vdots & \vdots & \ddots & \vdots \\
		y_1^{(n - 2)} & y_2^{(n - 2)} & \cdots & y_n^{(n - 2)} \\
		y_1^{(n - 1)} & y_2^{(n - 1)} & \cdots & y_n^{(n - 1)}
	\end{vmatrix} + \ldots + \begin{vmatrix}
		y_1 & y_2 & \cdots & y_n \\
		y_1' & y_2' & \cdots & y_n' \\
		\vdots & \vdots & \ddots & \vdots \\
		y_1^{(n - 2)} & y_2^{(n - 2)} & \cdots & y_n^{(n - 2)} \\
		y_1^{(n)} & y_2^{(n)} & \cdots & y_n^{(n)}
	\end{vmatrix} = \\ = 0 + 0 + \ldots + \begin{vmatrix}
		y_1 & y_2 & \cdots & y_n \\
		y_1' & y_2' & \cdots & y_n' \\
		\vdots & \vdots & \ddots & \vdots \\
		y_1^{(n - 2)} & y_2^{(n - 2)} & \cdots & y_n^{(n - 2)}\\
		y_1^{(n)} & y_2^{(n)} & \cdots & y_n^{(n)}
	\end{vmatrix}
\end{multline}   
то, підставивши в попередній вираз, одержимо
\begin{equation*}
	%\label{eq:3.1.32}
	- \frac{a_1(x)}{a_0(x)} = \frac{\frac{\diff}{\diff x} W[y_1, y_2, \ldots, y_n](x)}{W[y_1, y_2, \ldots, y_n](x)}.
\end{equation*}
Розділимо змінні
\begin{equation*}
	%\label{eq:3.1.33}
	- \frac{a_1(x)}{a_0(x)} \diff x = \frac{\diff W[y_1, y_2, \ldots, y_n](x)}{W[y_1, y_2, \ldots, y_n](x)}.
\end{equation*}
Проінтегрувавши, одержимо
\begin{equation*}
	%\label{eq:3.1.33}
	\ln W[y_1, y_2, \ldots, y_n](x) - \ln W[y_1, y_2, \ldots, y_n](x_0) = -\int_{x_0}^x \frac{a_1(x)}{a_0(x)} \diff x
\end{equation*}
або
\begin{equation*}
	%\label{eq:3.1.33}
	W[y_1, y_2, \ldots, y_n](x) = W[y_1, y_2, \ldots, y_n](x_0) \cdot \exp \left\{ -\int_{x_0}^x \frac{a_1(x)}{a_0(x)} \diff x \right\}.
\end{equation*}
Отримана формула називається формулою Остроградського-Ліувілля. Зокрема, якщо рівняння має вид
\begin{equation*}
	%\label{eq:3.1.34}
	y^{(n)} + p_1 \cdot y^{(n - 1)} + \ldots + p_n(x) \cdot y = 0,
\end{equation*}
то формула запишеться у вигляді
\begin{equation*}
	%\label{eq:3.1.35}
	W[y_1, y_2, \ldots, y_n](x) = W[y_1, y_2, \ldots, y_n](x_0) \cdot \exp \left\{ -\int_{x_0}^x p_1(x) \diff x \right\}.
\end{equation*}


		\subsubsection{Формула Абеля}
		Розглянемо застосування формули Остроградського-Ліувіля до рівняння 2-го порядку
\begin{equation*}
	y'' + p_1(x) y' + p_2(x) y = 0.
\end{equation*}

Нехай $y_1(x)$ --- один з розв'язків. Тоді
\begin{equation*}
	\begin{vmatrix}
		y_1(x) & y(x) \\
		y_1'(x) & y'(x)
	\end{vmatrix} = C_2 \exp \left\{ - \int p_1(x) \diff x \right\}.
\end{equation*}

Розкривши визначник, одержимо
\begin{equation*}
	y_1(x) y'(x) - y(x) y_1'(x) = C_2 \exp \left\{ - \int p_1(x) \diff x \right\}.
\end{equation*}
 
Розділивши на $y_1^2(x)$, запишемо
\begin{equation*}
	\frac{y_1(x) y'(x) - y(x) y_1'(x)}{y_1^2(x)} = \frac{C_2}{y_1^2(x)} \exp \left\{ - \int p_1(x) \diff x \right\},
\end{equation*}
або
\begin{equation*}
	\frac{\diff}{\diff x} \left( \frac{y(x)}{y_1(x)} \right) = \frac{C_2}{y_1^2(x)} \exp \left\{ - \int p_1(x) \diff x \right\},
\end{equation*}

Проінтегрувавши, одержимо
\begin{equation*}
	\frac{y(x)}{y_1(x)} = C_2 \int \left( \frac{1}{y_1^2(x)} \exp \left\{ - \int p_1(x) \diff x \right\} \right) \diff x + C_1,
\end{equation*}

Остаточно
\begin{equation*}
	y(x) = C_1 y_1(x) + C_2 y_1(x) \int \left( \frac{1}{y_1^2(x)} \exp \left\{ - \int p_1(x) \diff x \right\} \right) \diff x,
\end{equation*}

Отримана формула називається формулою Абеля. Вона дозволяє по одному відомому роз\-в'яз\-ку знайти загальний роз\-в'яз\-ок однорідного лінійного рівняння другого порядку.


		\subsubsection{Вправи для самостійної роботи}
		Розв’язати лінійне однорядне диференціальне рівняння другого порядку, якщо відомий один розв’язок

\begin{example}
	$(x^2 + 1) \cdot y'' - 2 x \cdot y' + 2 y = 0$, $y_1(x) = x$.
\end{example}
\begin{solution}
	За формулою Абеля маємо
	\begin{multline*}
		y_2(x) = x \cdot \int \left(\frac{1}{x} \exp \left\{ \int \frac{2x\diff x}{x^2+1} \right\} \right) \diff x = x \cdot \int \left(\frac{1}{x} e^{\ln |x^2 + 1|} \right) \diff x = \\ = x \cdot \int \left(\frac{x^2 + 1}{x} \right) \diff x = x \cdot \left( x - \frac1x \right) = x^2 - 1.
	\end{multline*}
	Загальний розв’язок має вигляд \[ y(x) = C_1 \cdot x + C_2 \cdot (x^2 - 1).\]
\end{solution}

Розв’язати рівняння: 
\begin{problem}
	\[x^2\cdot(x+1)\cdot y''-2y=0,\quad y_1(x)=1+\frac1x;\]
\end{problem}
\begin{problem}
	\[x\cdot y''+2y'-x\cdot y=0,\quad y_1(x)=\frac{e^x}{x};\]
\end{problem}
\begin{problem}
	\[y''-2\cdot(1+\tan^2(x))\cdot y=0,\quad y_1(x)=\tan x;\]
\end{problem}
\begin{problem}
	\[(e^x+1)\cdot y''-2y'+e^x\cdot y=0,\quad y_1(x)=e^x-1;\]
\end{problem}
\begin{problem}
	\[y''-y'\cdot\tan x+2y=0,\quad y_1(x)=\sin x;\]
\end{problem}
\begin{problem}
	\[y''+4x\cdot y'+(4x^2+2)\cdot y=0,\quad y_1(x)=e^{a x^2}.\]
\end{problem}

Знайти загальний розв’язок підібравши один частинний
\begin{problem}
	\[(2x+1)\cdot y''+4x\cdot y'-4y=0;\]
\end{problem}
\begin{problem}
	\[x\cdot y''-(2x+1)\cdot y'+(x+1)\cdot y=0;\]
\end{problem}
\begin{problem}
	\[x\cdot(x-1)\cdot y''-x\cdot y'+y=0.\]
\end{problem}

	\subsection{Лінійні однорідні рівняння зі сталими коефіцієнтами}
	
		\subsubsection{Загальна теорія}
		Розглянемо лінійні однорідні диференціальні рівняння з сталими коефіцієнтами
\begin{equation*}
	y^{(n)} + a_1 y^{(n - 1)} + \ldots + a_n y = 0
\end{equation*}

Розв'язок будемо шукати у вигляді $y = e^{\lambda x}$. Продиференціювавши, одержимо 
\begin{equation*}
 	y' = \lambda e^{\lambda x}, \quad y'' = \lambda^2 e^{\lambda x}, \quad \ldots \quad y^{(n)} = \lambda^n e^{\lambda x}.
\end{equation*}

Підставивши $y', y'', \ldots, y^{(n)}$ в диференціальне рівняння, отримаємо
\begin{equation*}
	\lambda^n e^{\lambda x} + a_1 \lambda^{n - 1}e^{\lambda x} + \ldots + a_n e^{\lambda x} = 0.
\end{equation*}

Скоротивши на $e^{\lambda x}$, одержимо характеристичне рівняння
\begin{equation*}
	\lambda^n + a_1 \lambda^{n - 1} + \ldots + a_n = 0.
\end{equation*}

Алгебраїчне рівняння $n$-го степеня має $n$ коренів. У залежності від їхнього вигляду будемо мати різні розв'язки.
\begin{enumerate}
    \item Нехай $\lambda_1, \lambda_2, \ldots, \lambda_n$ --- дійсні і різні. Тоді функції $e^{\lambda_1 x}, e^{\lambda_2 x}, \ldots, e^{\lambda_n x}$ є розв'язками й оскільки всі $\lambda_i$ різні, то $e^{\lambda_i x}$ --- розв'язки лінійно незалежні, тобто $\left\{ e^{\lambda_i x} \right\}_{i = 1}^n$ фундаментальна система розв'язків. Загальним розв'язком буде лінійна комбінація $y = \sum_{i=1}^n C_i e^{\lambda_i x}$.

    \item Нехай маємо комплексно спряжені корені $\lambda=p+iq$, $\bar\lambda=p-iq$. Їм відповідають розв'язки $e^{(p+iq)x}$, $e^{(p-iq)x}$ . Розкладаючи їх по формулі Ейлера, одержимо: 
    \begin{align*}
    	e^{(p+iq)x} &= e^{px} e^{iqx} = e^{px} (\cos qx + i \sin qx) = u(x) + i v(x), \\
    	e^{(p-iq)x} &= e^{px} e^{-iqx} = e^{px} (\cos qx - i \sin qx) = u(x) - i v(x).
    \end{align*}
    
    І, як випливає з властивості 4, функції $u(x)$ й $v(x)$ будуть окремими розв'язками. Таким чином, кореням $\lambda = p + iq$, $\bar\lambda = p - iq$ відповідають два лінійно незалежних розв'язки $u = e^{px} \cos qx$, $v = e^{px} \sin qx$. Загальним розв'язком, що відповідає цим двом кореням, буде $y = C_1 e^{px} \cos qx + C_2 e^{px} \sin x$.
    
    \item Нехай $\lambda$ --- кратний корінь, кратності $k$, тобто $\lambda_1 = \lambda_2 = \ldots = \lambda_k$, $k\le n$.
    
    \begin{enumerate}
        \item Розглянемо випадок $\lambda=0$. Тоді характеристичне рівняння вироджується в рівняння
        \begin{equation*}
        	\lambda^n + a_1 \lambda^{n - 1} + \ldots + a_{n - k} \lambda^k = 0.
        \end{equation*}
         
        Диференціальне рівняння, що відповідає цьому характеристичному, запишеться у вигляді
        \begin{equation*}
        	y^{(n)} + a_1 y^{(n - 1)} + \ldots + a_{n-k} y^{(k)} = 0
        \end{equation*}
        
        Неважко бачити, що частковими, лінійно незалежними роз\-в'я\-з\-ка\-ми цього рівняння, будуть функції $1, x, x^2, \ldots, x^{k-1}$. Загальним роз\-в'я\-з\-ком, що відповідає кореню $\lambda=0$ кратності $k$, буде лінійна комбінація цих функцій $y = C_1 + C_2 x + \ldots + C_k x^{k - 1}$.
        
        \item Нехай $\lambda = \nu \ne 0$ --- корінь дійсний. Зробивши заміну $y = e^{\nu x} z$, на підставі властивості 2 лінійних рівнянь після підстановки знову одержимо лінійне однорідне диференціальне рівняння 
        \begin{equation*}
        	z^{(k)} + b_1 z^{(k-1)} + \ldots + b_k z = 0.
        \end{equation*}
        
        Причому, оскільки $y_i(x) = e^{\lambda_i x}$ а $x_i(x) = e^{\mu_i x}$, то показники $\lambda_i$, $\mu_i$ зв'язані співвідношенням $\lambda_i = \nu + \mu_i$. Звідси кореню $\lambda = \nu$ кратності $k$ відповідає корінь $\mu=0$ кратності $k$. Як випливає з попереднього пункту, кореню $\mu=0$ кратності $k$ відповідає загальний розв'язок вигляду $z = C_1 + C_2 x + \ldots + C_k x^{k - 1}$. \parvskip
        
        З огляду на те, що $y = e^{\nu x} z$, одержимо, що кореню $\lambda=\nu$ кратності $k$ відповідає розв'язок
        \begin{equation*}
        	y = \left(C_1 + C_2 x + \ldots + C_k x^{k - 1} \right) e^{\nu x}.
        \end{equation*}
        
        \item Нехай характеристичне рівняння має корені $\lambda=p+iq$, $\bar\lambda=p-iq$ кратності $k$. Проводячи аналогічні викладки одержимо, що їм відповідають лінійно незалежні розв'язки
        \begin{equation*}
        	e^{px} \cos qx, \quad x e^{px} \cos qx, \quad \ldots, \quad x^{k-1} e^{px} \cos qx,
        \end{equation*}
        
        \begin{equation*}
        	e^{px} \sin qx, \quad x e^{px} \sin qx, \quad \ldots, \quad x^{k-1} e^{px} \sin qx.
        \end{equation*}
        
        І загальним розв'язком, що відповідає цим кореням буде
        \begin{multline*}
        	y = C_1 e^{px} \cos qx + C_2 x e^{px} \cos qx + C_k x^{k-1} e^{px} \cos qx + \\ + C_{k+1} e^{px} \sin qx + C_{k+2} x e^{px} \sin qx + \ldots + C_{2k} x^{k-1} e^{px} \sin qx.
        \end{multline*}
    \end{enumerate}
\end{enumerate}


		\subsubsection{Вправи для самостійної роботи}
		\begin{example}
	Розв’язати рівняння $y'' + y' - 2 y = 0$.
\end{example}
\begin{solution}
	Розв’язок шукаємо у вигляді $y = e^{\lambda x}$. Тоді \[y'=\lambda e^{\lambda x}, \quad y''=\lambda^2 e^{\lambda x}.\] Підставивши в диференціальне рівняння, одержуємо \[ \lambda^2 e^{\lambda x} + \lambda e^{\lambda x} - 2 e^{\lambda x} = 0.\] Скоротивши на $e^{\lambda x}$, одержуємо характеристичне рівняння \[\lambda^2+\lambda-2=0.\] Його коренями будуть $\lambda_1=-1$, $\lambda_2=2$. Їм відповідають два лінійно незалежні розв’язки $e^{-x}$, $e^{2x}$. І загальним розв’язком диференціального рівняння буде \[ y(x) = C_1 \cdot e^{-x} + C_2 \cdot e^{2x}.\]
\end{solution}
\begin{example}
	Розв’язати рівняння $y'' + y' + 2 y = 0$.
\end{example}
\begin{solution}
	Розв’язок шукаємо у вигляді $y = e^{\lambda x}$. Тоді \[ y' = \lambda e^{\lambda x}, \quad y'' =\lambda^2 e^{\lambda x}.\] Підставивши в диференціальне рівняння, одержуємо \[ \lambda^2 e^{\lambda x} + \lambda e^{\lambda x} + 2 e^{\lambda x}=0.\] Скоротимо на $e^{\lambda x}$: \[ \lambda^2+\lambda+2=0.\] Коренями характеристичного рівняння будуть $\lambda_1=-1\pm i$. Їм відповідають два лінійно незалежні розв’язки \[y_1(x)=e^{-x}\cdot\cos x, \quad y_2(x)=e^{-x}\cdot\sin x.\] І загальним розв’язком рівняння буде \[y(x)=C_1\cdot e^{-x}\cdot\cos x+C_2\cdot e^{-x}\cdot\sin x.\]
\end{solution}
\begin{example}
	Розв’язати рівняння $y'' + 4 y' + 4 y = 0$.
\end{example}
\begin{solution}
	Розв’язок шукаємо у вигляді $y = e^{\lambda x}$. Тоді \[y'=\lambda e^{\lambda x},\quad y''=\lambda^2 e^{\lambda x}.\] Підставляємо в диференціальне рівняння, одержуємо \[\lambda^2 e^{\lambda x}+4\lambda e^{\lambda x}+4e^{\lambda x}=0.\] Скоротимо на $e^{\lambda x}$:\[\lambda^2+4\lambda+4=0.\] Коренями характеристичного рівняння будуть $\lambda_1=\lambda_2=-2$. Оскільки вони кратні їм відповідають два лінійно незалежні розв’язки \[y_1(x)=e^{-2x},\quad y_2(x)=x\cdot e^{-2x}.\] І загальним розв’язком рівняння буде \[y(x)=C_1\cdot e^{-2x}+C_2\cdot x\cdot e^{-2x}.\]
\end{solution}

Розв’язати рівняння:
\begin{multicols}{2}
\begin{problem}
	\[y''-5y'+6y=0;\]
\end{problem}
\begin{problem}
	\[y''-9y=0;\]
\end{problem}
\begin{problem}
	\[y''-y'=0;\]
\end{problem}
\begin{problem}
	\[y''+2y'+y=0;\]
\end{problem}
\begin{problem}
	\[2y''+5y'+2y=0;\]
\end{problem}
\begin{problem}
	\[y''-4y=0;\]
\begin{problem}
	\[y''+3y'=0;\]
\end{problem}
\end{problem}
\begin{problem}
	\[y''-y'-2y=0;\]
\end{problem}
\begin{problem}
	\[y''-4y'+2y=0;\]
\end{problem}
\begin{problem}
	\[y''+6y'+13y=0;\]
\end{problem}
\begin{problem}
	\[y''-4y'+15y=0;\]
\end{problem}
\begin{problem}
	\[y''-6y'+34y=0;\]
\end{problem}
\begin{problem}
	\[y''+4y=0;\]
\end{problem}
\begin{problem}
	\[y''+2y'+10y=0;\]
\end{problem}
\begin{problem}
	\[y''+y=0.\]
\end{problem}
\end{multicols}

Знайти частинні розв’язки, що задовольняють зазначеним початковим умовам при $x=0$:
\begin{problem}
	\[y''-5y'+4y=0,\quad y=5,\quad y'=8;\]
\end{problem}
\begin{problem}
	\[y''+3y'+2y=0,\quad y=1,\quad y'=-1;\]
\end{problem}
\begin{problem}
	\[y''+4y=0,\quad y=0,\quad y'=2;\]
\end{problem}
\begin{problem}
	\[y''+2y'=0,\quad y=1,\quad y'=0;\]
\end{problem}
\begin{problem}
	\[y''-4y'+4y=0,\quad y=3,\quad y'=-1;\]
\end{problem}
\begin{problem}
	\[y''+4y'+29y=0,\quad y=0,\quad y'=15;\]
\end{problem}
\begin{problem}
	\[y''+3y=0,\quad y=0,\quad y'=1;\]
\end{problem}
\begin{problem}
	\[y''-2y'+y=0,\quad y=4,\quad y'=2;\]
\end{problem}

Розв’язати рівняння:
\begin{multicols}{2}
\begin{problem}
	\[y'''-13y''+12y'=0;\]
\end{problem}
\begin{problem}
	\[y''-y'=0;\]
\end{problem}
\begin{problem}
	\[y^{(4)}-2y''=0;\]
\end{problem}
\begin{problem}
	\[y'''-3y''+3y-y=0;\]
\end{problem}
\begin{problem}
	\[y^{(4)}+4y=0;\]
\end{problem}
\begin{problem}
	\[y'''+y=0;\]
\end{problem}
\begin{problem}
	\[y^{(4)}+8y''+16y=0;\]
\end{problem}
\begin{problem}
	\[y^{(4)}+y'=0;\]
\end{problem}
\begin{problem}
	\[y^{(4)}-2y''+y=0;\]
\end{problem}
\begin{problem}
	\[y^{(4)}-a^4y=0;\]
\end{problem}
\begin{problem}
	\[y^{(4)}-6y''+9y=0;\]
\end{problem}
\begin{problem}
	\[y^{(4)}+a^2y''=0;\]
\end{problem}
\begin{problem}
	\[y^{(4)}+2y'''+y''=0;\]
\end{problem}
\begin{problem}
	\[y^{(4)}+2y''+y=0;\]
\end{problem}
\begin{problem}
	\[y'''+9y'=0;\]
\end{problem}
\begin{problem}
	\[y'''-3y'-2y=0;\]
\end{problem}
\begin{problem}
	\[y^{(4)}+10y''+9y=0.\]
\end{problem}
\end{multicols}
Знайти частинні розв’язки диференціальних рівнянь:
\begin{problem}
	\[y'''+y'=0, \quad y(0)=2, \quad y'(0)=0, \quad y''(0)=-1;\]
\end{problem}
\begin{problem}
	\[y^{(5)}-y'=0, \quad y(0)=y''(0)=0, \quad y'(0)=1, \quad y'''(0)=1, \quad y^{(4)}=2;\]
\end{problem}
\begin{problem}
	\[y'''+2y''+10y'=0, \quad y(0)=2, \quad y'(0)=y''(0)=1;\]
\end{problem}
\begin{problem}
	\[y'''-y'=0, \quad y(0)=3, \quad y'(0)=-1, \quad y''(0)=1;\]
\end{problem}
\begin{problem}
	\[y'''+y'=0, \quad y(0)=2, \quad y'(0)=0, \quad y''(0)=-1.\]
\end{problem}

	\subsection{Лінійні неоднорідні диференціальні рівняння}
	Загальний вигляд лінійних неоднорідних диференціальних рівнянь наступний
\begin{equation*}
	%\label{eq:3.3.1}
	a_0(x) \cdot y^{(n)}(x) + a_1(x) \cdot y^{(n - 1)}(x) + \ldots + a_n(x) \cdot y(x) = b(x).
\end{equation*}

\subsubsection{Властивості розв’язків лінійних неоднорідних рівнянь. Загальний розв’язок лінійного неоднорідного рівняння}

\begin{property}
	\label{prop:3.3.1}
	Якщо $y_0(x)$ -- розв’язок лінійного однорідного рівняння, $y_1(x)$ -- розв’язок неоднорідного рівняння, то $y(x) = y_0(x) + y_1(x)$ буде розв’язком лінійного неоднорідного диференціального рівняння.
\end{property}

\begin{proof}
	Дійсно, нехай $y_0(x)$ і $y_1(x)$ -- розв’язки відповідно однорідного і неоднорідного рівнянь, тобто
	\begin{align}
		%\label{eq:3.3.2}
		a_0(x) \cdot y_0^{(n)}(x) + a_1(x) \cdot y_0^{(n - 1)}(x) + \ldots + a_n(x) \cdot y_0(x) &= 0, \\
		%\label{eq:3.3.3}
		a_0(x) \cdot y_1^{(n)}(x) + a_1(x) \cdot y_1^{(n - 1)}(x) + \ldots + a_n(x) \cdot y_1(x) &= b(x).
	\end{align}	
	
	Тоді 
	\begin{multline}
		%\label{eq:3.3.4}
		a_0(x) (y_0 + y_1)^{(n)}(x) + a_1(x) (y_0 + y_1)^{(n - 1)}(x) + \ldots + a_n(x) (y_0 + y_1)(x) = \\ = \left( a_0(x) \cdot y_0^{(n)}(x) + a_1(x) \cdot y_0^{(n - 1)}(x) + \ldots + a_n(x) \cdot y_0(x) \right) + \\ + \left( a_0(x) \cdot y_1^{(n)}(x) + a_1(x) \cdot y_1^{(n - 1)}(x) + \ldots + a_n(x) \cdot y_1(x) \right) = \\ = 0 + b(x) = b(x),
	\end{multline}
	тобто $y(x) = y_0(x) + y_1(x)$ -- розв’язок неоднорідного диференціального рівняння.
\end{proof}

\begin{property}[Принцип суперпозиції]
	Якщо $y_i(x)$, $i = \overline{1, n}$ -- розв’язки лінійних неоднорідних диференціальних рівнянь
	\begin{equation*}
		%\label{eq:3.3.5}
		a_0(x) \cdot y^{(n)}(x) + a_1(x) \cdot y^{(n - 1)}(x) + \ldots + a_n(x) \cdot y(x) = b_i(x), \quad i = \overline{1, n}
	\end{equation*}
	то $y(x) = \sum_{i = 1}^n C_i \cdot y_i(x)$ з довільними сталими $C_i$ буде розв’язком лінійного неоднорідного рівняння
	\begin{equation*}
		%\label{eq:3.3.6}
		a_0(x) y^{(n)}(x) + a_1(x) y^{(n - 1)}(x) + \ldots + a_n(x) y(x) = \sum_{i = 1}^n C_i  b_i(x)
	\end{equation*}
\end{property}

\begin{proof}
	Дійсно, нехай $y_i(x)$, $i = \overline{1, n}$ -- розв’язки відповідних неоднорідних рівнянь, тобто
	\begin{equation*}
		%\label{eq:3.3.7}
		a_0(x) \cdot y_i^{(n)}(x) + a_1(x) \cdot y_i^{(n - 1)}(x) + \ldots + a_n(x) \cdot y_i(x) = b_i(x), \quad i = \overline{1, n}
	\end{equation*}

	Склавши лінійну комбінацію з рівнянь і їхніх правих частин з коефіцієнтами $C_i$ одержимо
	\begin{multline}
		%\label{eq:3.3.8}
		\sum_{i = 1}^n C_i \cdot \left( a_0(x) \cdot y_i^{(n)}(x) + a_1(x) \cdot y_i^{(n - 1)}(x) + \ldots + a_n(x) \cdot y_i(x) \right) = \\ = \sum_{i = 1}^n C_i \cdot b_i(x),
	\end{multline}
	або, перегрупувавши, запишемо
	\begin{multline}
		%\label{eq:3.3.9}
		a_0(x) \cdot \left( \sum_{i = 1}^n C_i \cdot y_i^{(n)}(x) \right) + a_1(x) \cdot \left( \sum_{i = 1}^n C_i \cdot y_i^{(n - 1)}(x)\right) + \ldots \\ \ldots + a_n(x) \cdot \left( \sum_{i = 1}^n C_i \cdot y_i(x) \right) = \sum_{i = 1}^n C_i \cdot b_i(x),
	\end{multline}
	що і було потрібно довести.
\end{proof}

\begin{property}
	Якщо комплексна функція $y(x) = u(x) + i v(x)$ з дійсними елементами є розв’язком лінійного неоднорідного рівняння з комплексною правою частиною $b(x) = f(x) + i p(x)$, то дійсна частина $u(x)$ є розв’язком рівняння з правою частиною $f(x)$, а уявна $v(x)$ є розв’язком рівняння з правою частиною $p(x)$.
\end{property}

\begin{proof}
	Дійсно, як випливає з умови,
	\begin{multline}
	 	%\label{eq:3.3.10}
	 	a_0(x) \cdot (u + i v)^{(n)}(x) + a_1(x) \cdot (u + i v)^{(n - 1)}(x) + \ldots + a_n(x) \cdot (u + i v)(x) = \\ = f(x) + i p(x).
	\end{multline}
	
	Розкривши дужки, одержимо
	\begin{multline}
		%\label{eq:3.3.11}
		\left( a_0(x) \cdot u^{(n)}(x) + a_1(x) \cdot u^{(n - 1)}(x) + \ldots + a_n(x) \cdot u(x) \right) + \\ + i \left( a_0(x) \cdot v^{(n)}(x) + a_1(x) \cdot v^{(n - 1)}(x) + \ldots + a_n(x) \cdot v(x) \right) = \\ = f(x) + i p(x).
	\end{multline}

	А комплексні вирази рівні між собою тоді і тільки тоді, коли дорівнюють окремо дійсні та уявні частини, тобто
	\begin{align}
		%\label{eq:3.3.12}
		a_0(x) \cdot u^{(n)}(x) + a_1(x) \cdot u^{(n - 1)}(x) + \ldots + a_n(x) \cdot u(x) &= f(x), \\ 
		%\label{eq:3.3.13}
		a_0(x) \cdot v^{(n)}(x) + a_1(x) \cdot v^{(n - 1)}(x) + \ldots + a_n(x) \cdot v(x) &= p(x),
	\end{align}
	що і було потрібно довести.
\end{proof}

\begin{theorem}
	Загальний розв’язок лінійного неоднорідного диференціального рівняння складається з загального розв’язку лінійного однорідного рівняння і частинного розв’язку неоднорідного рівняння.
\end{theorem}
\begin{proof}
	Нехай $y_{\text{homo}}(x) = \sum_{i = 1}^n C_i \cdot y_i(x)$ -- загальний розв’язок однорідного\footnote{Homogeneous equation* -- однорідне рівняння.} рівняння, а $y_{\text{hetero}}(x)$ -- частинний розв’язок неоднорідного\footnote{Heterogeneous equation* -- неоднорідне рівняння.} рівняння. \\

	Тоді, як випливає з властивості \ref{prop:3.3.1}, $y(x) = \sum_{i = 1}^n C_i \cdot y_i(x) + y_{\text{hetero}}(x)$, буде розв’язком неоднорідного рівняння. Покажемо, що цей розв’язок  загальний, тобто вибором коефіцієнтів $C_i$ можна розв’язати довільну задачу Коші
	\begin{equation*}
		%\label{eq:3.3.14}
		y(x_0) = y_0, \quad y'(x_0) = y_0', \quad \ldots, \quad y^{(n - 1)}(x_0) = y_0^{(n - 1)}.
	\end{equation*}

	Дійсно, оскільки $y_{\text{homo}}$ загальний розв’язок однорідного рівняння, то $y_i$, $i = \overline{1, n}$ лінійно незалежні, а отже визначник Вронського $W[y_1, y_2, \ldots, y_n] \ne 0$. Звідси, неоднорідна система лінійних алгебраїчних рівнянь 
	\begin{equation*}
		%\label{eq:3.3.15}
		\left\{ \begin{aligned}
			C_1 \cdot y_1(x_0) + C_2 \cdot y_2(x_0) + \ldots + C_n \cdot y_n(x_0) &= y_0 - y_{\text{hetero}}(x_0), \\
			C_1 \cdot y_1'(x_0) + C_2 \cdot y_2'(x_0) + \ldots + C_n \cdot y_n'(x_0) &= y_0' - y_{\text{hetero}}'(x_0), \\
			\ldots \ldots \ldots \ldots \ldots \ldots \ldots \ldots \ldots \ldots \ldots \ldots \ldots \ldots & \ldots \ldots \ldots \ldots \ldots \ldots 	\\
			C_1 \cdot y_1^{(n - 1)}(x_0) + C_2 \cdot y_2^{(n - 1)}(x_0) + \ldots + C_n \cdot y_n^{(n - 1)}(x_0) &= y_0 - y_{\text{hetero}}^{(n - 1)}(x_0),
		\end{aligned} \right.
	\end{equation*}
	має єдиний розв’язок для довільних наперед обраних $y_0, y_0', \ldots, y_0^{(n - 1)}$. Нехай розв’язком системи буде $C_1^0, C_2^0, \ldots, C_n^0$. Тоді, як випливає з вигляду системи, функція $y(x) = \sum_{i = 1}^n C_i^0 \cdot y_i(x) + y_{\text{hetero}}$ є розв’язком поставленому задачі Коші.
\end{proof}

Як випливає з теореми для знаходження загального розв’язку лінійного неоднорідного рівняння треба шукати загальний розв’язок однорідного рівняння, тобто будь-які $n$ лінійно незалежні розв’язкі і якийсь частинний розв’язок неоднорідного рівняння. Розглянемо три методи побудови частинного розв’язку лінійного неоднорідного рівняння.

		\subsubsection{Властивості розв'язків лінійних неоднорідних рівнянь. Загальний розв'язок лінійного неоднорідного рівняння}
		\begin{property}
	Якщо $y_0(x)$ --- розв'язок лінійного однорідного рівняння, $y_1(x)$ --- розв'язок неоднорідного рівняння, то $y(x) = y_0(x) + y_1(x)$ буде розв'язком лінійного неоднорідного диференціального рівняння.
\end{property}

\begin{proof}
	Дійсно, нехай $y_0(x)$ і $y_1(x)$ --- розв'язки відповідно однорідного і неоднорідного рівнянь, тобто
	\begin{align*}
		a_0(x) y_0^{(n)}(x) + a_1(x) y_0^{(n - 1)}(x) + \ldots + a_n(x) y_0(x) &= 0, \\
		a_0(x) y_1^{(n)}(x) + a_1(x) y_1^{(n - 1)}(x) + \ldots + a_n(x) y_1(x) &= b(x).
	\end{align*}	
	
	Тоді 
	\begin{multline*}
		a_0(x) (y_0 + y_1)^{(n)}(x) + a_1(x) (y_0 + y_1)^{(n - 1)}(x) + \ldots + a_n(x) (y_0 + y_1)(x) = \\ = \left( a_0(x) y_0^{(n)}(x) + a_1(x) y_0^{(n - 1)}(x) + \ldots + a_n(x) y_0(x) \right) + \\ + \left( a_0(x) y_1^{(n)}(x) + a_1(x) y_1^{(n - 1)}(x) + \ldots + a_n(x) y_1(x) \right) = \\ = 0 + b(x) = b(x),
	\end{multline*}
	тобто $y(x) = y_0(x) + y_1(x)$ --- розв'язок неоднорідного диференціального рівняння.
\end{proof}

\begin{property}[Принцип суперпозиції]
	Якщо $y_i(x)$, $i = \overline{1, n}$ --- розв'язки лінійних неоднорідних диференціальних рівнянь
	\begin{equation*}
		a_0(x) y^{(n)}(x) + a_1(x) y^{(n - 1)}(x) + \ldots + a_n(x) y(x) = b_i(x), \quad i = \overline{1, n}
	\end{equation*}
	то $y(x) = \sum_{i = 1}^n C_i y_i(x)$ з довільними сталими $C_i$ буде розв'язком лінійного неоднорідного рівняння
	\begin{equation*}
		a_0(x) y^{(n)}(x) + a_1(x) y^{(n - 1)}(x) + \ldots + a_n(x) y(x) = \sum_{i = 1}^n C_i  b_i(x)
	\end{equation*}
\end{property}

\begin{proof}
	Дійсно, нехай $y_i(x)$, $i = \overline{1, n}$ --- розв'язки відповідних неоднорідних рівнянь, тобто
	\begin{equation*}
		a_0(x) y_i^{(n)}(x) + a_1(x) y_i^{(n - 1)}(x) + \ldots + a_n(x) y_i(x) = b_i(x), \quad i = \overline{1, n}
	\end{equation*}

	Склавши лінійну комбінацію з рівнянь і їхніх правих частин з коефіцієнтами $C_i$ одержимо
	\begin{equation*}
		\sum_{i = 1}^n C_i \left( a_0(x) y_i^{(n)}(x) + a_1(x) y_i^{(n - 1)}(x) + \ldots + a_n(x) y_i(x) \right) = \sum_{i = 1}^n C_i b_i(x),
	\end{equation*}
	або, перегрупувавши, запишемо
	\begin{multline*}
		a_0(x) \left( \sum_{i = 1}^n C_i y_i^{(n)}(x) \right) + a_1(x) \left( \sum_{i = 1}^n C_i y_i^{(n - 1)}(x)\right) + \ldots \\ \ldots + a_n(x) \left( \sum_{i = 1}^n C_i y_i(x) \right) = \sum_{i = 1}^n C_i b_i(x),
	\end{multline*}
	що і було потрібно довести.
\end{proof}

\begin{property}
	Якщо комплексна функція $y(x) = u(x) + i v(x)$ з дійсними елементами є розв'язком лінійного неоднорідного рівняння з комплексною правою частиною $b(x) = f(x) + i p(x)$, то дійсна частина $u(x)$ є розв'язком рівняння з правою частиною $f(x)$, а уявна $v(x)$ є розв'язком рівняння з правою частиною $p(x)$.
\end{property}

\begin{proof}
	Дійсно, як випливає з умови,
	\begin{multline*}
	 	a_0(x) (u + i v)^{(n)}(x) + a_1(x) (u + i v)^{(n - 1)}(x) + \ldots + a_n(x) (u + i v)(x) = \\ = f(x) + i p(x).
	\end{multline*}
	
	Розкривши дужки, одержимо
	\begin{multline*}
		\left( a_0(x) u^{(n)}(x) + a_1(x) u^{(n - 1)}(x) + \ldots + a_n(x) u(x) \right) + \\ + i \left( a_0(x) v^{(n)}(x) + a_1(x) v^{(n - 1)}(x) + \ldots + a_n(x) v(x) \right) = \\ = f(x) + i p(x).
	\end{multline*}

	А комплексні вирази рівні між собою тоді і тільки тоді, коли дорівнюють окремо дійсні та уявні частини, тобто
	\begin{align*}
		a_0(x) u^{(n)}(x) + a_1(x) u^{(n - 1)}(x) + \ldots + a_n(x) u(x) &= f(x), \\ 
		a_0(x) v^{(n)}(x) + a_1(x) v^{(n - 1)}(x) + \ldots + a_n(x) v(x) &= p(x),
	\end{align*}
	що і було потрібно довести.
\end{proof}

\begin{theorem}
	Загальний розв'язок лінійного неоднорідного диференціального рівняння складається з загального розв'язку лінійного однорідного рівняння і частинного розв'язку неоднорідного рівняння.
\end{theorem}
\begin{proof}
	Нехай $y_{\text{homo}}(x) = \sum_{i = 1}^n C_i y_i(x)$ --- загальний розв'язок однорідного\footnote{Homogeneous equation --- однорідне рівняння.} рівняння, а $y_{\text{hetero}}(x)$ --- частинний розв'язок неоднорідного\footnote{Heterogeneous equation --- неоднорідне рівняння.} рівняння. \parvskip

	Тоді, як випливає з першої властивості, $y(x) = \sum_{i = 1}^n C_i y_i(x) + y_{\text{hetero}}(x)$, буде розв'язком неоднорідного рівняння. Покажемо, що цей розв'язок  загальний, тобто вибором коефіцієнтів $C_i$ можна розв'язати довільну задачу Коші
	\begin{equation*}
		y(x_0) = y_0, \quad y'(x_0) = y_0', \quad \ldots, \quad y^{(n - 1)}(x_0) = y_0^{(n - 1)}.
	\end{equation*}

	Дійсно, оскільки $y_{\text{homo}}$ загальний роз\-в'яз\-ок однорідного рівняння, то система функцій $y_i$, $i = \overline{1, n}$ лінійно незалежна, тому визначник Вронського $W[y_1, y_2, \ldots, y_n] \ne 0$. Звідси, неоднорідна система лінійних алгебраїчних рівнянь 
	\begin{equation*}
		\left\{ \begin{array}{rl}
			C_1 y_1(x_0) + C_2 y_2(x_0) + \ldots + C_n y_n(x_0) &= y_0 - y_{\text{hetero}}(x_0), \\
			C_1 y_1'(x_0) + C_2 y_2'(x_0) + \ldots + C_n y_n'(x_0) &= y_0' - y_{\text{hetero}}'(x_0), \\
			\hdotsfor{2} \\
			C_1 y_1^{(n - 1)}(x_0) + C_2 y_2^{(n - 1)}(x_0) + \ldots + C_n y_n^{(n - 1)}(x_0) &= y_0 - y_{\text{hetero}}^{(n - 1)}(x_0),
		\end{array} \right.
	\end{equation*}
	має єдиний роз\-в'яз\-ок для довільних наперед обраних $y_0, y_0', \ldots, y_0^{(n - 1)}$. Нехай роз\-в'яз\-ком системи буде $C_1^0, C_2^0, \ldots, C_n^0$. Тоді, як випливає з вигляду системи, функція $y(x) = \sum_{i = 1}^n C_i^0 y_i(x) + y_{\text{hetero}}$ є роз\-в'яз\-ком поставленої задачі Коші.
\end{proof}

Як випливає з теореми для знаходження загального розв'язку лінійного неоднорідного рівняння треба шукати загальний розв'язок однорідного рівняння, тобто будь-які $n$ лінійно незалежні розв'язки і якийсь частинний розв'язок неоднорідного рівняння. Розглянемо три методи побудови частинного розв'язку лінійного неоднорідного рівняння.

		\subsubsection{Метод варіації довільної сталої побудови частинного роз\-в'яз\-ку лінійного неоднорідного диференціального рівняння}
		Метод варіації довільної сталої полягає в тому, що розв’язок неоднорідного рівняння шукається в такому ж вигляді, як і розв’язок однорідного, але сталі $C_i$, $i = \overline{1, n}$ вважаються невідомими функціями. Нехай загальний розв’язок лінійного однорідного рівняння
\begin{equation*}
	%\label{eq:3.3.16}
	a_0(x) \cdot y^{(n)}(x) + a_1(x) \cdot y^{(n - 1)}(x) + \ldots + a_n(x) \cdot y(x) = 0.
\end{equation*}
записано у вигляді $y(x) = C_1 \cdot y_1(x) + C_2 \cdot y_2(x) + \ldots + C_n \cdot y_n(x)$. \parvskip

Розв’язок лінійного неоднорідного рівняння
\begin{equation*}
	%\label{eq:3.3.17}
	a_0(x) \cdot y^{(n)}(x) + a_1(x) \cdot y^{(n - 1)}(x) + \ldots + a_n(x) \cdot y(x) = b(x).
\end{equation*}
шукаємо у вигляді $y(x) = C_1(x) \cdot y_1(x) + C_2(x) \cdot y_2(x) + \ldots + C_n(x) \cdot y_n(x)$, де $C_i(x)$, $i = \overline{1, n}$ --- невідомі функції. Оскільки підбором $n$ функцій необхідно задовольнити одному рівнянню, тобто одній умові, то $n - 1$ умову можна накласти довільно. Розглянемо першу похідну від записаного розв’язку
\begin{equation*}
	%\label{eq:3.3.18}
	y'(x) = \sum_{i = 1}^n C_i(x) \cdot y_i'(x) + \sum_{i = 1}^n C_i'(x) \cdot y_i(x).
\end{equation*}
і зажадаємо, щоб $\sum_{i = 1}^n C_i'(x) \cdot y_i(x) = 0$. Розглянемо другу похідну
\begin{equation*}
	%\label{eq:3.3.19}
	y'(x) = \sum_{i = 1}^n C_i(x) \cdot y_i''(x) + \sum_{i = 1}^n C_i'(x) \cdot y_i'(x).
\end{equation*}
і зажадаємо, щоб $\sum_{i = 1}^n C_i'(x) \cdot y_i'(x) = 0$. Продовжимо процес взяття похідних до $(n - 1)$-ої 
\begin{equation*}
	%\label{eq:3.3.20}
	y^{(n - 1)}(x) = \sum_{i = 1}^n C_i(x) \cdot y_i^{(n - 1)}(x) + \sum_{i = 1}^n C_i'(x) \cdot y_i^{(n - 2)}(x).
\end{equation*}
і зажадаємо, щоб $\sum_{i = 1}^n C_i'(x) \cdot y_i^{(n - 2)}(x)$. На цьому $(n - 1)$ умова вичерпалася. І для $n$-ої похідної справедливо
\begin{equation*}
	%\label{eq:3.3.21}
	y^{(n)}(x) = \sum_{i = 1}^n C_i(x) \cdot y_i^{(n)}(x) + \sum_{i = 1}^n C_i'(x) \cdot y_i^{(n - 1)}(x).
\end{equation*}

Підставимо взяту функцію та її похідні в неоднорідне диференціальне рівняння
\begin{multline*}
 	%\label{eq:3.3.22}
 	a_0(x) \cdot \left( \sum_{i = 1}^n C_i(x) \cdot y_i^{(n)}(x) \right) + a_0(x) \cdot \left( \sum_{i = 1}^n C_i'(x) \cdot y_i^{(n - 1)}(x) \right) + \\ + a_1(x) \cdot \left( \sum_{i = 1}^n C_i(x) \cdot y_i^{(n - 1)}(x) \right) + \ldots + a_n(x) \cdot \left( \sum_{i = 1}^n C_i(x) \cdot y_i(x) \right) = b(x).
\end{multline*} 
Оскільки $y(x) = \sum_{i = 1}^n C_i(x) \cdot y_i(x)$ --- розв’язок однорідного диференціального рівняння, то після скорочення одержимо $n$-у умову
\begin{equation*}
	%\label{eq:3.3.23}
	\left( \sum_{i = 1}^n C_i'(x) \cdot y_i^{(n - 1)}(x) \right) = \frac{b(x)}{a_0(x)}.
\end{equation*}
Додаючи перші $(n -1 )$ умови, одержимо систему
\begin{equation*}
	%\label{eq:3.3.24}
	\left\{ \begin{aligned}
		C_1'(x) \cdot y_1(x) + C_2'(x) \cdot y_2(x) + \ldots + C_n'(x) \cdot y_n(x) &= 0, \\
		C_1'(x) \cdot y_1'(x) + C_2'(x) \cdot y_2'(x) + \ldots + C_n'(x) \cdot y_n'(x) &= 0, \\
		\ldots \ldots \ldots \ldots \ldots \ldots \ldots \ldots \ldots \ldots \ldots \ldots \ldots \ldots \ldots \ldots & . \ldots . \\
		C_1'(x) \cdot y_1^{(n - 2)}(x) + C_2'(x) \cdot y_2^{(n - 2)}(x) + \ldots + C_n'(x) \cdot y_n^{(n - 2)}(x) &= 0, \\
		C_1'(x) \cdot y_1^{(n - 1)}(x) + C_2'(x) \cdot y_2^{(n - 1)}(x) + \ldots + C_n'(x) \cdot y_n^{(n - 1)}(x) &= \frac{b(x)}{a_0(x)}.
	\end{aligned} \right.
\end{equation*}
 
Оскільки визначником системи є визначник Вронського і він відмінний від нуля, то система має єдиний роз\-в'яз\-ок
\begin{equation*}
	%\label{eq:3.3.25}
	\begin{aligned}
		C_1(x) &= \int \frac{\begin{vmatrix} 0 & y_2(x) & \cdots & y_{n - 1}(x) & y_n'(x) \\ 0 & y_2'(x) & \cdots & y_{n - 1}'(x) & y_n'(x) \\ \vdots & \vdots & \ddots & \vdots & \vdots \\ 0 & y_2^{(n - 2)}(x) & \cdots & y_{n - 1}^{(n - 2)}(x) & y_n^{(n - 2)}(x) \\ \frac{b(x)}{a_0(x)} & y_2^{(n - 1)}(x) & \cdots & y_{n - 1}^{(n - 1)}(x) & y_n^{(n - 1)}(x) \end{vmatrix}}{W[y_1, y_2, \ldots, y_n]} \diff x, \\
		\ldots \ldots & \ldots \ldots \ldots \ldots \ldots \ldots \ldots \ldots \ldots \ldots \ldots \ldots \ldots \ldots \ldots \ldots \\
		C_n(x) &= \int \frac{\begin{vmatrix} y_1(x) & y_2(x) & \cdots & y_{n - 1}(x) & 0 \\ y_1'(x) & y_2'(x) & \cdots & y_{n - 1}'(x) & 0 \\ \vdots & \vdots & \ddots & \vdots & \vdots \\ y_1^{(n - 2)} & y_2^{(n - 2)}(x) & \cdots & y_{n - 1}^{(n - 2)}(x) & 0 \\ y_1^{(n - 1)}(x) & y_2^{(n - 1)}(x) & \cdots & y_{n - 1}^{(n - 1)}(x) & \frac{b(x)}{a_0(x)} \end{vmatrix}}{W[y_1, y_2, \ldots, y_n]} \diff x.
	\end{aligned}
\end{equation*}

І загальний розв’язок лінійного неоднорідного диференціального рівняння запишеться у вигляді
\begin{equation*}
	%\label{eq:3.3.26}
	y(x) = \bar C_1 \cdot y_1(x) + \bar C_2 \cdot y_2(x) + \ldots + \bar C_n \cdot y_n(x) + y_{\text{hetero}}(x),
\end{equation*}
де $\bar C_i$ --- довільні сталі, а
\begin{equation*}
	%\label{eq:3.3.27}
	y_{\text{hetero}}(x) = C_1(x) \cdot y_1(x) + C_2(x) \cdot y_2(x) + \ldots + C_n(x) \cdot y_n(x).
\end{equation*}

Якщо розглядати диференціальне рівняння другого порядку
\begin{equation*}
	%\label{eq:3.3.27}
	a_0(x) \cdot y''(x) + a_1(x) \cdot y'(x) + a_2(x) \cdot y(x) = b(x),
\end{equation*}
і загальний розв’язок однорідного рівняння має вигляд
\begin{equation*}
	%\label{eq:3.3.28}
	y_{\text{homo}}(x) = C_1 \cdot y_1(x) + C_2 \cdot y_2(x),
\end{equation*}
то частинний розв’язок неоднорідного має вигляд 
\begin{equation*}
	%\label{eq:3.3.29}
	y_{\text{hetero}}(x) = C_1(x) \cdot y_1(x) + C_2(x) \cdot y_2(x).
\end{equation*}
І для знаходження функцій $C_1(x), C_2(x)$ маємо систему
\begin{equation*}
	%\label{eq:3.3.30}
	\left\{ \begin{aligned}
		C_1'(x) \cdot y_1(x) + C_2'(x) \cdot y_2(x) &= 0, \\
		C_1'(x) \cdot y_1'(x) + C_2'(x) \cdot y_2'(x) &= \frac{b(x)}{a_0(x)}.
	\end{aligned} \right.
\end{equation*}

Звідси
\begin{equation*}
	%\label{eq:3.3.31}
	C_1(x) = \int \frac{\begin{vmatrix} 0 & y_2(x) \\ \frac{b(x)}{a_0(x)} & y_2'(x) \end{vmatrix}}{\begin{vmatrix} y_1(x) & y_2(x) \\ y_1'(x) & y_2'(x) \end{vmatrix}} \diff x, \quad C_2(x) = \int \frac{\begin{vmatrix} y_1(x) & 0 \\ y_1'(x) & \frac{b(x)}{a_0(x)} \end{vmatrix}}{\begin{vmatrix} y_1(x) & y_2(x) \\ y_1'(x) & y_2'(x) \end{vmatrix}} \diff x
\end{equation*}

І одержуємо $y_{\text{hetero}}(x) = C_1(x) \cdot y_1(x) + C_2(x) \cdot y_2(x)$ з обчисленими функціями $C_1(x)$ і $C_2(x)$.


		\subsubsection{Метод Коші}
		Нехай $y(x) = K(x, s)$ --- розв’язок однорідного диференціального рівняння, що задовольняє умовам
\begin{equation*}
	K(s, s) = K_x'(s, s) = \ldots = K_{x^{n - 2}}^{(n - 2)}(s, s) = 0, \quad K_{x^{n - 1}}^{(n - 1)}(s, s) = 1.
\end{equation*}

Тоді функція
\begin{equation*}
	y(x) = \int_{x_0}^x K(x, s) \cdot \frac{b(s)}{a_0(s)} \diff s
\end{equation*}
буде розв’язком неоднорідного рівняння, що задовольняє початковим умовам
\begin{equation*}
	y(x_0) = y'(x_0) = \ldots = y^{(n - 1)}(x_0) = 0.
\end{equation*}

Дійсно, розглянемо похідні від функції $y(x)$:
\begin{equation*}
	y'(x) = \int_{x_0}^x K_x'(x, s) \cdot \frac{b(s)}{a_0(s)} \diff s + K(x, x) \cdot \frac{b(x)}{a_0(x)}.
\end{equation*}

І, оскільки $K(x, x) = 0$, то
\begin{equation*}
	y'(x) = \int_{x_0}^x K_x'(x, s) \cdot \frac{b(s)}{a_0(s)} \diff s.
\end{equation*}

Аналогічно
\begin{equation*}
	y''(x) = \int_{x_0}^x K_{x^2}''(x, s) \cdot \frac{b(s)}{a_0(s)} \diff s + K_x'(x, x) \cdot \frac{b(x)}{a_0(x)} = \int_{x_0}^x K_{x^2}''(x, s) \cdot \frac{b(s)}{a_0(s)} \diff s,
\end{equation*}
і так далі до
\begin{align*}
	y^{(n - 1)}(x) &= \int_{x_0}^x K_{x^{n - 1}}^{(n - 1)}(x, s) \cdot \frac{b(s)}{a_0(s)} \diff s + K_{x^{n - 2}}^{(n - 2)}(x, x) \cdot \frac{b(x)}{a_0(x)} = \\ &= \int_{x_0}^x K_{x^{n - 1}}^{(n - 1)}(x, s) \cdot \frac{b(s)}{a_0(s)} \diff s, \\
	y^{(n)}(x) &= \int_{x_0}^x K_{x^n}^{(n)}(x, s) \cdot \frac{b(s)}{a_0(s)} \diff s + K_{x^{n - 1}}^{(n - 1)}(x, x) \cdot \frac{b(x)}{a_0(x)}.
\end{align*}

І, оскільки $K_{x^{n - 1}}^{(n - 1)}(x, x) = 1$, то
\begin{equation*}
	y^{(n)}(x) = \int_{x_0}^x K_{x^n}^{(n)}(x, s) \cdot \frac{b(s)}{a_0(s)} \diff s + \frac{b(x)}{a_0(x)}.
\end{equation*}

Підставивши функцію $y(x)$ і її похідні у вихідне диференціальне рівняння, одержимо
\begin{multline*}
	a_0(x) \cdot \left( \int_{x_0}^x K_{x^n}^{(n)} (x, s) \cdot \frac{b(s)}{a_0(s)} \diff s + \frac{b(x)}{a_0(x)} \right) + \\ + a_1(x) \cdot \left( \int_{x_0}^x K_{x^{n - 1}}^{(n - 1)} (x, s) \cdot \frac{b(s)}{a_0(s)} \diff s \right) + \ldots + a_n(x) \int_{x_0}^x K_x' (x, s)  \cdot \frac{b(s)}{a_0(s)} \diff s = \\ = \int_{x_0}^x \left( a_0(x) \cdot K_{x^n}^{(n)}(x, s) + a_1(x) \cdot K_{x^{n - 1}}^{(n - 1)}(x, s) + \ldots + a_n(x) \cdot K(x, s) \right).
\end{multline*}

Оскільки $K(x, s)$ -- є розв’язком лінійного однорідного рівняння і, отже,
\begin{equation*}
	a_0(x) \cdot K_{x^n}^{(n)}(x, s) + a_1(x) \cdot K_{x^{n - 1}}^{(n - 1)}(x, s) + \ldots + a_n(x) \cdot K(x, s) = 0.
\end{equation*}
 
У такий спосіб показано, що $y(x) = \int_{x_0}^x K(x, s) \cdot \frac{b(s)}{a_0(s)} \diff s$ -- є розв’язком лінійного неоднорідного рівняння. \parvskip

Підставляючи $x = x_0$ в значення $y(x), y'(x), \ldots, y^{(n)}(x)$ одержимо, що
\begin{equation*}
	y(x_0) = y'(x_0) = \ldots = y^{(n - 1)}(x_0) = 0.
\end{equation*}

Для знаходження функції $K(x, s)$ (інтегрального ядра) можна використати такий спосіб. Якщо $y_1(x), y_2(x), \ldots, y_n(x)$ лінійно незалежні роз\-в'яз\-ки однорідного рівняння, то загальний роз\-в'яз\-ок однорідного рівняння має вигляд 
\begin{equation*}
	y_{\text{homo}}(x) = C_1 \cdot y_1(x) + C_2 \cdot y_2(x) + \ldots + C_n \cdot y_n(x).
\end{equation*}
Оскільки $K(x, s)$ є розв’язком однорідного рівняння, то його слід шукати у вигляді
\begin{equation*}
	K(x, s) = C_1(s) \cdot y_1(x) + C_2(s) \cdot y_2(x) + \ldots + C_n(s) \cdot y_n(x).
\end{equation*}
Відповідні початкові умови мають вигляд
\begin{align*}
	K(s, s) = 0 &\Rightarrow C_1(s) \cdot y_1(s) + C_2(s) \cdot y_2(s) + \ldots + C_n(s) \cdot y_n(s) = 0, \\
	K_x'(s, s) = 0 &\Rightarrow C_1(s) \cdot y_1'(s) + C_2(s) \cdot y_2'(s) + \ldots + C_n(s) \cdot y_n'(s) = 0,
\end{align*}
і так далі до
\begin{multline*}
	K_{x^{n - 2}}^{(n - 2)}(s, s) = 0 \Rightarrow \\ \Rightarrow C_1(s) \cdot y_1^{(n - 2)}(s) + C_2(s) \cdot y_2^{(n - 2)}(s) + \ldots + C_n(s) \cdot y_n^{(n - 2)}(s) = 0,
\end{multline*}
і
\begin{multline*}
	K_{x^{n - 1}}^{(n - 1)}(s, s) = 1 \Rightarrow \\ \Rightarrow C_1(s) \cdot y_1^{(n - 1)}(s) + C_2(s) \cdot y_2^{(n - 1)}(s) + \ldots + C_n(s) \cdot y_n^{(n - 1)}(s) = 0. 
\end{multline*}
Звідси
\begin{align*}
	C_1(s) &= \int \frac{\begin{vmatrix} 0 & y_2(s) & \cdots & y_n(s) \\ \vdots & \vdots & \ddots & \vdots \\ 0 & y_2^{(n - 2)}(s) & \cdots & y_n^{(n - 2)}(s) \\ 1 & y_2^{(n - 1)}(s) & \cdots & y_n^{(n - 1)}(s) \end{vmatrix}}{W[y_1, y_2, \ldots, y_n](s)} \diff s, \\
	C_2(s) &= \int \frac{\begin{vmatrix} y_1(s) & 0 & \cdots & y_n(s) \\ \vdots & \vdots & \ddots & \vdots \\ y_2^{(n - 2)} & 0 & \cdots & y_n^{(n - 2)}(s) \\ y_2^{(n - 1)}(s) & 1 & \cdots & y_n^{(n - 1)}(s) \end{vmatrix}}{W[y_1, y_2, \ldots, y_n](s)} \diff s,
\end{align*}
і так далі до
\begin{align*}
	C_n(s) &= \int \frac{\begin{vmatrix} y_1(s) & y_2(s) & \cdots & 0 \\ \vdots & \vdots & \ddots & \vdots \\ y_1^{(n - 2)}(s) & y_2^{(n - 2)}(s) & \cdots & 0 \\ y_1^{(n - 1)}(s) & y_2^{(n - 1)}(s) & \cdots & 1 \end{vmatrix}}{W[y_1, y_2, \ldots, y_n](s)} \diff s.
\end{align*}
 
І ядро $K(x, s)$ має вигляд
\begin{equation*}
	K(x, s) = C_1(s) \cdot y_1(x) + C_2(s) \cdot y_2(x) + \ldots + C_n(s) \cdot y_n(x)
\end{equation*}
з одержаними функціями $C_1(s), C_2(s), \ldots, C_n(s)$. \parvskip

Якщо розглядати диференціальне рівняння другого порядку 
\begin{equation*}
	a_0(x) \cdot y''(x) + a_1(x) \cdot y'(x) + a_2(x) \cdot y(x) = b(x),
\end{equation*}
то функція  має вигляд
\begin{equation*}
	K(x, s) = C_1(s) \cdot y_1(x) + C_2(s) \cdot y_2(x),
\end{equation*}
де
\begin{equation*}
	C_1(s) = \frac{\begin{vmatrix} 0 & y_2(s) \\ 1 & y_2'(s) \end{vmatrix}}{\begin{vmatrix} y_1(s) & y_2(s) \\ y_1'(s) & y_2'(s) \end{vmatrix}}, \quad C_1(s) = \frac{\begin{vmatrix} y_1(s) & 0 \\ y_1'(s) & 1 \end{vmatrix}}{\begin{vmatrix} y_1(s) & y_2(s) \\ y_1'(s) & y_2'(s) \end{vmatrix}}.
\end{equation*}
Звідси
\begin{equation*}
	K(x, s) = \frac{\begin{vmatrix} 0 & y_2(s) \\ 1 & y_2'(s) \end{vmatrix} y_1(x) + \begin{vmatrix} y_1(s) & 0 \\ y_1'(s) & 1 \end{vmatrix} y_2(x) }{W[y_1, y_2](s)} = \frac{y_1(s) \cdot y_2(x) - y_1(x) \cdot y_2(s)}{W[y_1, y_2](s)}
\end{equation*}

		\subsubsection{Метод невизначених коефіцієнтів}
		Якщо лінійне диференціальне рівняння є рівнянням з сталими коефіцієнтами, а функція $b(x)$ спеціального виду, то частинний розв’язок можна знайти за допомогою методу невизначених коефіцієнтів.

\begin{enumerate}
	\item Нехай $b(x)$ має вид многочлена, тобто
	\begin{equation*}
		b(x) = A_0 \cdot x^s + A_1 \cdot x^{s - 1} + \ldots + A_{s - 1} \cdot x + A_s.
	\end{equation*}

	\begin{enumerate}
		\item Розглянемо випадок, коли характеристичне рівняння не має нульового кореня, тобто $\lambda \ne 0$. Частинний розв’язок неоднорідного рівняння шукаємо вигляді:
		\begin{equation*}
			y_{\text{part}} = B_0 \cdot x^s + B_1 \cdot x^{s - 1} + \ldots + B_{s - 1} + B_s,
		\end{equation*}
		де $B_0, \ldots, B_s$ -- невідомі сталі. Тоді
		\begin{align*}
			y_{\text{part}}' &= s \cdot B_0 \cdot x^{s - 1} + (s - 1) \cdot B_1 \cdot x^{s - 2} + \ldots + 1 \cdot B_{s - 1}, \\
			y_{\text{part}}'' &= s \cdot (s - 1) \cdot B_0 \cdot x^{s - 2} + (s - 1) \cdot (s - 2) \cdot B_1 \cdot x^{s - 3} + \ldots \\ & \quad \ldots + 2 \cdot 1 \cdot B_{s - 2},
		\end{align*}
		і так далі. \\

		Підставляючи у вихідне диференціальне рівняння, одержимо
		\begin{multline*}
			a_0 \left( s! B_s \right) + \ldots \\ + a_{n - 2} \left( s (s - 1) B_0 x^{s - 2} + (s - 1) (s - 2) B_1 x^{s - 3} + \ldots + 2 B_{s - 1} \right) + \\ + a_{n - 1} \left( s B_0 x^{s - 1} + (s - 1) B_1 x^{s - 2} + \ldots + B_{s - 1} \right) + \\ + a_n \left( B_0 x^s + B_1 x^{s - 1} + \ldots + B_{s - 1} + B_s \right) = \\ = A_0 x^s + A_1 x^{s - 1} + \ldots + A_{s - 1} x + A_s.
		\end{multline*}

		Прирівнявши коефіцієнти при однакових степенях $x$ запишемо: 
		\begin{table}[H]
			\centering
			\begin{tabular}{c|l}
				$x^s$ & $a_n \cdot B_0 = A_0$ \\
				$x^{s - 1}$ & $a_n \cdot B_1 + s \cdot a_{n - 1} \cdot B_0 = A_1$ \\
				$x^{s - 2}$ & $a_n \cdot B_2 + (s - 1) \cdot a_{n - 1} \cdot B_1 + s \cdot (s - 1) \cdot a_{n - 2} \cdot B_0 = A_2$
			\end{tabular}
		\end{table}
		і так далі. \\

		Оскільки характеристичне рівняння не має нульового кореня, то $a_n \ne 0$. Звідси одержимо $B_0 = \frac{A_0}{a_n}$, $B_1 = \frac{A_1 - s \cdot a_{n - 1} \cdot B_0}{a_n}$, і так далі.

		\item Розглянемо випадок, коли характеристичне рівняння має нульовий корінь кратності $r$. Тоді диференціальне рівняння має вигляд
		\begin{equation*}
			a_0 \cdot y^{(n)} + a_1 \cdot y^{(n - 1)} + \ldots + a_{n - r} \cdot y^{(r)} = A_0 \cdot x^s + A_1 \cdot x^{s - 1} + \ldots + A_s.
		\end{equation*}

		Зробивши заміну $y^{(r)} = z$ одержимо диференціальне рівняння 
		\begin{equation*}
			a_0 \cdot z^{(n - r)} + a_1 \cdot z^{(n - r - 1)} + \ldots + a_{n - r} \cdot z = A_0 \cdot x^s + A_1 \cdot x^{s - 1} + \ldots + A_s,
		\end{equation*}
		характеристичне рівняння якого вже не має нульового кореня, тобто повернемося до попереднього випадку. Звідси частинний розв’язок шукається у вигляді
		\begin{equation*}
			z_{\text{part}} = \bar B_0 \cdot x^s + \bar B_1 \cdot x^{s - 1} + \ldots + \bar B_s.
		\end{equation*}

 		Проінтегрувавши його $r$-разів, одержимо, що частиний роз\-в'яз\-ок вихідного однорідного рівняння має вигляд
 		\begin{equation*}
			y_{\text{part}} = \left(B_0 \cdot x^s + B_1 \cdot x^{s - 1} + \ldots + B_s\right) \cdot x^r.
		\end{equation*}
 	\end{enumerate}
	\item Нехай $b(x)$ має вигляд $b(x) = e^{px} \cdot \left( A_0 \cdot x^s + A_1 \cdot x^{s - 1} + \ldots + A_s \right)$.
	\begin{enumerate}
		\item Розглянемо випадок, коли $p$ не є коренем характеристичного рівняння. Зробимо заміну
		\begin{align*}
			y &= e^{p x} \cdot z, \\
			y' &= p \cdot e^{p x} \cdot z + e^{p x} \cdot z = e^{p x} \cdot (p z + z'), \\
			y'' &= p \cdot e^{p x} \cdot (p z + z') + e^{p x} \cdot (p z' + z'') = e^{p x} \cdot (p^2 z + 2 p z' + z''),
		\end{align*}
 		і так далі до
 		\begin{equation*}
 			y^{(n)} = e^{p x} \cdot \left( p^n \cdot z + n \cdot p^{n - 1} \cdot z' + \ldots + z^{(n)} \right).
 		\end{equation*}
		
		Підставивши отримані вирази у вихідне диференціальне рівняння, одержимо
		\begin{multline*}
			e^{p x} \cdot \left( B_0 \cdot z^{(n)} + B_1 \cdot z^{(n - 1)} + \ldots B_n \cdot z \right) = \\ = e^{p z} \cdot \left( A_0 \cdot x^s + A_1 \cdot x^{s - 1} + \ldots + A_s \right).
		\end{multline*}
		де $B_i$ -- сталі коефіцієнти, що виражаються через $a_i$ і $p$. Скоротивши на $e^{p x}$, одержимо рівняння 
 		\begin{equation*}
			B_0 \cdot z^{(n)} + B_1 \cdot z^{(n - 1)} + \ldots B_n \cdot z = A_0 \cdot x^s + A_1 \cdot x^{s - 1} + \ldots + A_s.
		\end{equation*}
		
		Причому, оскільки $p$ не є коренем характеристичного рівняння, то після заміни $y = e^{px} \cdot z$, отримане диференціальне рівняння не буде мати коренем характеристичного рівняння $\mu = 0$. Таким чином, повернулися до випадку 1.a). Частинний розв’язок неоднорідного рівняння шукаємо у вигляді
		\begin{equation*}
			z_{\text{part}} = B_0 \cdot x^s + B_1 \cdot x^{s - 1} + \ldots + B_{s - 1} + B_s,
		\end{equation*}

		А частинний розв’язок вихідного неоднорідного диференціального рівняння у вигляді:
		\begin{equation*}
			y_{\text{part}} = e^{px} \cdot \left( B_0 \cdot x^s + B_1 \cdot x^{s - 1} + \ldots + B_{s - 1} + B_s \right),
		\end{equation*}

		\item Розглянемо випадок, коли $p$ -- корінь характеристичного рівняння кратності $r$. Це значить, що після, заміни $y = e^{px} \cdot z$ і скорочення на $e^{p x}$, вийде диференціальне рівняння, що має коренем характеристичного рівняння, число $\mu = 0$ кратності $r$, тобто
		\begin{equation*}
			B_0 \cdot z^{(n)} + B_1 \cdot z^{(n - 1)} + \ldots B_{n - r} \cdot z^{(r)} = A_0 \cdot x^s + A_1 \cdot x^{s - 1} + \ldots + A_s.
		\end{equation*}

		Як випливає з пункту 1.б) частинний розв’язок шукається у вигляді
		\begin{equation*}
			z_{\text{part}} = \left( B_0 \cdot x^s + B_1 \cdot x^{s - 1} + \ldots + B_s \right) \cdot x^r,
		\end{equation*}
		а частинний розв’язок вихідного неоднорідного диференціального рівняння у вигляді
		\begin{equation*}
			y_{\text{part}} = e^{p x} \cdot \left( B_0 \cdot x^s + B_1 \cdot x^{s - 1} + \ldots + B_s \right) \cdot x^r,
		\end{equation*}
 	\end{enumerate}
	\item Нехай $b(x)$ має вигляд:
	\begin{equation*}
		b(x) = e^{px} \cdot \left( P_s(x) \cdot \cos (qx) + Q_\ell (x) \cdot \sin(q x) \right),
	\end{equation*}
	де $P_s(x)$, $Q_\ell(x)$ -- многочлени степеня $s$ і $\ell$, відповідно, і, наприклад, $\ell \le s$. Використовуючи формулу Ейлера, перетворимо вираз до вигляду:
	\begin{equation*}
		b(x) = e^{(p + i q) \cdot x} \cdot R_s(x) + e^{(p - i q) \cdot x} \cdot T_s(x),
	\end{equation*}
	де $R_s(x)$, $T_s(x)$ -- многочлени степеня не вище, ніж $s$. Використовуючи властивості 2, 3 розв’язків неоднорідних диференціальних рівнянь, а також випадки 2.а), 2.б) знаходження частинного розв’язку лінійних неоднорідних рівнянь, одержимо, що частинний розв’язок шукається у виглядах:
 	\begin{enumerate}
 		\item 
 		\begin{multline*}
 			y_{\text{part}} = e^{p x} \cdot \left( \left( A_0 \cdot x^s + A_1 \cdot x^{s - 1} + \ldots + A_s \right) \cdot \cos (qx) \right. + \\ + \left. \left( B_0 \cdot x^s + B_1 \cdot x^{s - 1} + \ldots + B_s \right) \cdot \sin (q x) \right),
 		\end{multline*}
		якщо $p \pm i q$ не є коренем характеристичного рівняння;
		\item
		\begin{multline*}
 			y_{\text{part}} = e^{p x} \cdot \left( \left( A_0 \cdot x^s + A_1 \cdot x^{s - 1} + \ldots + A_s \right) \cdot \cos (qx) \right. + \\ + \left. \left( B_0 \cdot x^s + B_1 \cdot x^{s - 1} + \ldots + B_s \right) \cdot \sin (q x) \right) \cdot x^r,
 		\end{multline*}
 		якщо $p \pm i q$ є коренем характеристичного рівняння кратності $r$.
 	\end{enumerate}
\end{enumerate}

		\subsubsection{Вправи для самостійної роботи}
		\begin{example}
	Знайти загальний розв'язок рівняння $y'' - 2 y' + y = \frac{e^x}{x}$.
\end{example}

\begin{solution}
	Загальний розв'язок складається з суми загального роз\-в'яз\-ку однорідного та частинного роз\-в'яз\-ку неоднорідного рівнянь. \\

	Розглянемо однорідне рівняння
	\begin{equation*}
		y'' - 2 y' + y = 0.
	\end{equation*}
	
	Його характеристичне рівняння має вигляд
	\begin{equation*}
		\lambda^2 - 2 \lambda + 1 = 0.
	\end{equation*}

	Його коренями будуть $\lambda_1 = 1$, $\lambda_2 = 1$. І загальний роз\-в'яз\-ок однорідного має вигляд $y_{\text{homo}}(x) = C_1 \cdot e^x + C_2 \cdot x \cdot e^x$.  \\

	Частинний розв’язок неоднорідного рівняння шукаємо методом варіації довільної сталої у вигляді $y_{\text{part}}(x) = C_1(x) \cdot e^x + C_2(x) \cdot x \cdot e^x$. Для знаходження функцій $C_1(x)$, $C_2(x)$ отримаємо систему 
	\begin{equation*}
		\left\{
			\begin{aligned}
				C_1'(x) \cdot e^x + C_2'(x) \cdot x \cdot e^x &= 0, \\
				C_1'(x) \cdot e^x + C_2'(x) \cdot \left( x \cdot e^x + e^x \right) &= \frac{e^x}{x}.
			\end{aligned}
		\right.
	\end{equation*}
	Звідси
	\begin{align*}
		C_1(x) &= \int \frac{\begin{vmatrix} 0 & x \cdot e^x \\ \frac{e^x}{x} & x \cdot e^x + e^x \end{vmatrix}}{\begin{vmatrix} e^x & x \cdot e^x \\ e^x & x \cdot e^x + e^x \end{vmatrix}} \diff x = \int \frac{e^{2x}}{e^{2x}} \diff x = x + \bar C_1, \\
		C_2(x) &= \int \frac{\begin{vmatrix} e^x & 0 \\ e^x & \frac{e^x}{x} \end{vmatrix}}{\begin{vmatrix} e^x & x \cdot e^x \\ e^x & x \cdot e^x + e^x \end{vmatrix}} \diff x = \int \frac{e^{2x}}{x \cdot e^{2x}} \diff x = \ln |x| + \bar C_2.
	\end{align*}

	Поклавши (для зручності) $\bar C_1 = 0$, $\bar C_2 = 0$, одержимо
	\begin{equation*}
		y_{\text{part}}(x) = x \cdot e^x + \ln |x| \cdot x \cdot e^x.
	\end{equation*}
	Загальний розв’язок має вигляд
	\begin{equation*}
		y_{\text{hetero}}(x) = C_1 \cdot e^x + C_2 \cdot x \cdot e^x + \ln |x| \cdot x \cdot e^x.
	\end{equation*}
\end{solution}

\begin{example}
	Знайти загальний розв’язок рівняння \[y'' + 3 y' + 2 y = \frac{1}{e^x + 1}.\]
\end{example}
\begin{solution}
	Загальний розв’язок складається з суми загального роз\-в'яз\-ку однорідного та частинного роз\-в'яз\-ку неоднорідного. Розглянемо однорідне рівняння
	\begin{equation*}
		y'' + 3 y ' + 2 y = 0.
	\end{equation*}
	Його характеристичне рівняння має вигляд
	\begin{equation*}
		\lambda^2 + 3 \lambda + 2 = 0.
	\end{equation*}
	Його коренями будуть $\lambda_1 = - 1$, $\lambda_2 = -2$. І загальний розв’язок однорідного має вигляд $y_{\text{homo}}(x) = C_1 \cdot e^{-x} + C_2 \cdot e^{-2x}$. \\

	Частинний розв’язок неоднорідного рівняння шукаємо методом Коші. Враховуючи вигляд загального роз\-в'яз\-ку однорядного рівняння функцію $K(x, s)$ шукаємо у вигляді
	\begin{equation*}
		K(x, s) = C_1(s) \cdot e^{-x} + C_2(s) \cdot e^{-2x}.
	\end{equation*}

	Початкові умови дають наступне
	\begin{align*}
		K(s, s) = 0 &\implies C_1(s) \cdot e^{-x} + C_2(s) \cdot e^{-2s} = 0, \\
		K_x'(s, s) = 1 &\implies C_1(s) \cdot e^{-x} - 2 C_2(s) \cdot e^{-2s} = 1, \\
	\end{align*}

	Звідси
	\begin{align*}
		C_1(s) &= \frac{\begin{vmatrix} e^{-s} & 0 \\ -e^{-s} & 1 \end{vmatrix}}{\begin{vmatrix} e^{-s} & e^{-2s} \\ -e^{-s} & -2e^{-2s} \end{vmatrix}} = \frac{e^{-s}}{-e^{-3s}} = -e^{2s}.
	\end{align*}

	Таким чином $K(x, s) = e^{s - x} - e^{2(s - x)}$. І частинний роз\-в'яз\-ок, що задовольняє нульовим початковим умовам, має вигляд
	\begin{align*}
		y_{\text{part}}(x) &= \int \frac{e^{s - x} - e^{2(s - x)}}{e^s + 1} \diff s = e^{-x} \int_{x_0}^x \frac{e^s}{e^s + 1} \diff s - e^{-2x} \frac{e^{2 s}}{e^s + 1} \diff s = \\ &= e^{-x} \cdot \left. \ln |e^s + 1| \right|_{s = x_0}^{s = x} - e^{-2x} \cdot \int_{x_0}^x \frac{e^s + 1 - 1}{e^s + 1} \diff (e^s) = \\ &= e^{-x} \cdot \left( \ln |e^x + 1| - \ln |e^{x_0} - 1| \right) + \\ & \quad + e^{-2x} \cdot \left( e^x - e^{x_0} - \ln |e^x + 1| + \ln |e^{x_0} + 1| \right).
	\end{align*}
	Враховуючи, що початкові дані не задані, остаточно отримаємо
	\begin{equation*}
		y_{\text{hetero}}(x) = C_1 \cdot e^{-x} + C_2 \cdot e^{-2x} + e^{-x} \cdot \ln |e^x + 1| + e^{-2x} \cdot \ln |e^x + 1|.
	\end{equation*}
\end{solution}

Розв’язати лінійні неоднорідні рівняння
\begin{multicols}{2}
\begin{problem}
	\[y'' + y = \frac{1}{\sin x};\]
\end{problem}
\begin{problem}
	\[y'' + 4 y = 2 \tan (x);\]
\end{problem}
\begin{problem}
	\[y'' + 2 y' + y = 3 \cdot e^{-x \cdot \sqrt{x + 1}};\]
\end{problem}
\begin{problem}
	\[y'' + y = 2 \sec^3(x);\]
\end{problem}
\begin{problem}
	\[y'' - y= \frac{x^2 - 2}{x^3}.\]
\end{problem}
\end{multicols}

Якщо рівняння зі сталими коефіцієнтами, а функція $b(x)$ спеціального вигляду, то зручніше використовувати метод невизначених коефіцієнтів.

\begin{example}
	Розв’язати лінійне неоднорідне рівняння \[y'' + 2 y' + y = x^2 + 1.\]
\end{example}
\begin{solution}
	Спочатку розв’язуємо однорідне рівняння
	\begin{equation*}
		y'' + 2 y' + y = 0.	
	\end{equation*}

	Його характеристичне рівняння має вигляд
	\begin{equation*}
		\lambda^2 + 2 \lambda + 1 = 0.
	\end{equation*}

	Його коренями будуть $\lambda_1 = -1$, $\lambda_2 = -1$. І загальним роз\-в'яз\-ком однорідного рівняння буде $y_{\text{homo}}(x) = C_1 \cdot e^{-x} + C_2 \cdot x \cdot e^{-x}$. Оскільки справа стоїть многочлени другого ступеня і характеристичне рівняння не містить нульових коренів, то частинний роз\-в'яз\-ок має вигляд
	\begin{equation*}
		y_{\text{part}}(x) = a x^2 + b x + c.
	\end{equation*}

	Звідси
	\begin{equation*}
		y_{\text{part}}'(x) = 2 a x + b.	
	\end{equation*}

	Підставляємо одержані вирази в диференціальне рівняння
	\begin{equation*}
		2 a + 2 (2 a x + b) + (a x^2 + b x + c) = x^2 + 1
	\end{equation*}

	Прирівнюємо коефіцієнти при однакових степенях
	\begin{table}[H]
		\centering
		\begin{tabular}{c|l}
			$x^2$ & $a = 1$ \\
			$x$ & $4 a + b = 0$ \\
			$1$ & $a + 2 b + c = 1$
		\end{tabular}
	\end{table}

	Звідси $a = 1$, $b = - 4$, $c = 7$. \\

	Таким чином загальний розв’язок має вигляд
	\begin{equation*}
		y_{\text{hetero}}(x) = C_1 \cdot e^{-x} + C_2 \cdot x \cdot e^{-x} + x^2 - 4 x + 7.
	\end{equation*}
\end{solution}

\begin{example}
	Розв’язати лінійне неоднорідне рівняння \[y''' + y'' = x + 1.\]
\end{example}

\begin{solution}
	Розв’язуємо однорідне рівняння
	\begin{equation*}
		y''' + y'' = 0.
	\end{equation*}

	Його характеристичне рівняння має вигляд
	\begin{equation*}
		\lambda^3 + \lambda^2 = 0
	\end{equation*}
	Його коренями будуть $\lambda_1 = \lambda_2 = 0$, $\lambda_3 = 1$. І загальним роз\-в'яз\-ком однорідного рівняння буде
	\begin{equation*}
		y_{\text{homo}}(x) = C_1 + C_2 \cdot x + C_3 \cdot e^{-x}.
	\end{equation*}

	Оскільки справа стоїть многочлен другого порядку, а характеристичне рівняння має нульовий корінь кратності два, то частинний розв’язок має вигляд 
	\begin{equation*}
		y_{\text{part}}(x) = x^2 \cdot (a x + b),
	\end{equation*}
	або
	\begin{equation*}
		y_{\text{part}}(x) = a x^3 + b x^2.
	\end{equation*}

	Звідси
	\begin{align*}
		y_{\text{part}}'(x) &= 3 a x^2 + 2 b x, \\
		y_{\text{part}}''(x) &= 6 a x + 2 b.
	\end{align*}

	Підставляємо одержані вирази в диференціальне рівняння
	\begin{equation*}
		6 a + (6 a x + 2 b) = x + 1.
	\end{equation*}

	Прирівнюємо коефіцієнти при однакових ступенях
	\begin{table}[H]
		\centering
		\begin{tabular}{c|l}
			$x$ & $6 a = 1$ \\
			$1$ & $6 a + 2 b = 1$
		\end{tabular}
	\end{table}

	Звідси $a = \frac16$, $b = 0$. \\

	Таким чином загальний розв’язок має вигляд
	\begin{equation*}
		y_{\text{hetero}}(x) = C_1 + C_2 \cdot x + C_3 \cdot e^{-x} + \frac{x^3}{6}
	\end{equation*}
\end{solution}

\begin{example}
	Розв’язати лінійне неоднорідне рівняння $y'' + y = e^x \cdot x$.
\end{example}
\begin{solution}
	Розв’язуємо лінійне однорідне рівняння
	\begin{equation*}
		y'' + y = 0.
	\end{equation*}
	
	Характеристичне рівняння має вигляд
	\begin{equation*}
		\lambda^2 + 1 = 0.
	\end{equation*}

	Його коренями будуть $\lambda_{1, 2} = \pm i$. І загальним роз\-в'яз\-ком однорідного рівняння буде
	\begin{equation*}
		y_{\text{homo}}(x) = C_1 \cdot \cos (x) + C_2 \cdot \sin (x).
	\end{equation*}

	Оскільки справа стоїть многочлен першого порядку, помножений на експоненту, то частинний роз\-в'яз\-ок має вигляд
	\begin{equation*}
		y_{\text{part}}(x) = e^x \cdot (a x + b).
	\end{equation*}

	Звідси
	\begin{align*}
		y_{\text{part}}'(x) &= e^x \cdot (a x + a + b), \\
		y_{\text{part}}'(x) &= e^x \cdot (a x + 2 a + b).
	\end{align*}

	Підставляємо одержані вирази у диференціальне рівняння
	\begin{equation*}
		e^x \cdot (a x + 2 a + b) + e^x \cdot (a x + b) = e^x \cdot x.
	\end{equation*}

	Прирівнюємо коефіцієнти при однакових членах
	\begin{table}[H]
		\centering
		\begin{tabular}{c|l}
			$x \cdot e^x$ & $2 a = 1$ \\
			$e^x$ & $2 a + 2 b = 0$
		\end{tabular}
	\end{table}

	Звідси $a = \frac12$, $b = - \frac12$. \\

	Таким чином загальний розв’язок має вигляд
	\begin{equation*}
		y_{\text{hetero}}(x) = C_1 \cdot \cos (x) + C_2 \cdot \sin (x) + \frac{e^x \cdot (x - 1)}{2}.
	\end{equation*}
\end{solution}

\begin{example}
	Розв’язати лінійне неоднорідне рівняння \[ y'' - 2 y' + y = e^x \cdot x.\]
\end{example}
\begin{solution}
	Розв’язуємо однорідне рівняння
	\begin{equation*}
		y'' - 2 y' + y = 0.
	\end{equation*}
	
	Характеристичне рівняння має вигляд
	\begin{equation*}
		\lambda^2 - 2 \lambda + 1 = 0.
	\end{equation*}
	
	Його коренями будуть $\lambda_1 = 1$, $\lambda_2 = 1$. І загальним роз\-в'яз\-ком однорідного рівняння буде
	\begin{equation*}
		y_{\text{homo}}(x) = C_1 \cdot e^x + C_2 \cdot x \cdot e^x.
	\end{equation*}

	Оскільки справа стоїть многочлен першого порядку, а показник при експоненті є двократним коренем характеристичного рівняння, частинний розв’язок має вигляд
	\begin{equation*}
		y_{\text{part}}(x) = x^2 \cdot e^x \cdot (a x + b),
	\end{equation*}
	або
	\begin{equation*}
		y_{\text{part}}(x) = e^x \cdot (a x^3 + b x^2),
	\end{equation*}

	Звідси
	\begin{align*}
		y_{\text{part}}'(x) &= e^x \cdot (a x^3 + (3 a + b) x^2 + 2 b x), \\
		y_{\text{part}}''(x) &= e^x \cdot (a x^3 + (6 a + b) x^2 + (6 a + 4 b) x + 2 b).
	\end{align*}

	Підставляємо одержані вирази в диференціальне рівняння
	\begin{multline*}
		e^x \cdot (a x^3 + (6 a + b) x^2 + (6 a + 4 b) x + 2 b) - 2 e^x \cdot (a x^3 + (3 a + b) x^2 + 2 b x) + \\ + e^x \cdot (a x^3 + b x^2) = e^x \cdot x.
	\end{multline*}

	Прирівнюємо коефіцієнти при однакових членах
	\begin{table}[H]
		\centering
		\begin{tabular}{c|l}
			$x \cdot e^x$ & $6 a + 4 b + 2 b = 1$ \\
			$e^x$ & $2 b = 0$
		\end{tabular}
	\end{table}

	Звідси $a = \frac16$, $b = 0$. \\

	Таким чином загальний розв’язок має вигляд
	\begin{equation}
		y_{\text{hetero}}(x) = C_1 \cdot e^x + C_2 \cdot x \cdot e^x + \frac{x^3 \cdot e^x}{6}.
	\end{equation}
\end{solution}

\begin{example}
	Розв’язати лінійне неоднорідне рівняння \[ y'' - y = x \cdot \cos(x) + \sin (x).\]
\end{example}
\begin{solution}
	Розв’язуємо однорідне рівняння
	\begin{equation*}
		y'' - y = 0.
	\end{equation*}

	Характеристичне рівняння має вигляд
	\begin{equation*}
		\lambda^2 - 1 = 0.
	\end{equation*}

	Його коренями будуть $\lambda_1 = 1$, $\lambda_2 = -1$. І загальним роз\-в'яз\-ком однорідного рівняння буде
	\begin{equation*}
		y_{\text{homo}}(x) = C_1 \cdot e^x + C_2 \cdot e^{-x}.
	\end{equation*}

	Частинний розв’язок неоднорідного має вигляд
	\begin{equation*}
		y_{\text{part}}(x) = (a x + b) \cdot \cos (x) + (c x + d) \cdot \sin(x).
	\end{equation*}

	Звідси
	\begin{align*}
		y_{\text{part}}'(x) &= (c x + a + d) \cdot \cos (x) + (-a x - b + c) \cdot \sin(x), \\
		y_{\text{part}}''(x) &= (-a x - b + 2 c) \cdot \cos (x) + (- c x - 2 a - d) \cdot \sin(x)
	\end{align*}

	Підставляємо одержані вирази в диференціальне рівняння
	\begin{multline*}
		(-a x - b + 2 c) \cdot \cos (x) + (- c x - 2 a - d) \cdot \sin(x) - \\
		- (a x + b) \cdot \cos (x) - (c x + d) \cdot \sin(x) = x \cdot \cos(x) + \sin (x).
	\end{multline*}
		 
	Прирівнюємо коефіцієнти при однакових виразах
	\begin{table}[H]
		\centering
		\begin{tabular}{c|l}
			$x \cdot \cos (x)$ & $- 2 a = 1$ \\
			$x \cdot \sin (x)$ & $- 2 c = 0$ \\
			$\cos (x)$ & $- b + 2c - b = 0$ \\
			$\sin (x)$ & $- 2 a - d - d = 1$
		\end{tabular}
	\end{table}

	Звідси $a = - \frac12$, $b = c = d = 0$. \\

	Таким чином загальний розв’язок має вигляд
	\begin{equation*}
		y_{\text{hetero}}(x) = C_1 \cdot e^x + C_2 \cdot x \cdot e^{x} - \frac{\cos (x)}{2}.
	\end{equation*}
\end{solution}

\begin{example}
	Розв’язати диференціальне рівняння \[y'' + 2 y' + 2 y = e^{-x} \cdot \sin (x).\]
\end{example}
\begin{solution}
	Розв’язуємо однорідне рівняння
	\begin{equation*}
		y'' + 2 y' + 2y = 0.
	\end{equation*}

	Характеристичне рівняння $\lambda^2 + 2 \lambda + 2 = 0$ має корені $\lambda_{1,2} = -1\pm i$. І загальним розв’язком однорідного рівняння буде
	\begin{equation*}
		y_{\text{homo}}(x) = C_1 \cdot e^{-x} \cdot \cos(x) + C_2 \cdot e^{-x} \cdot \sin(x).
	\end{equation*}

	Оскільки $\lambda_1 = 1 + i$ корінь кратності один, то частинний роз\-в'яз\-ок неоднорідного має вигляд
	\begin{equation*}
		y_{\text{part}}(x) = x \cdot e^{-x} \cdot (a \cdot \cos(x) + b \cdot \sin(x)).
	\end{equation*}

	Звідси
	\begin{align*}
		y_{\text{part}}'(x) &= e^{-x} \cdot ((b - a x) \cdot \sin(x) + (a - (a - b) \cdot x) \cdot \cos(x)) \\
		y_{\text{part}}'(x) &= -2 e^{-x} \cdot ((a + b - a x) \cdot \sin(x) + ((a - b) + b \cdot x) \cdot \cos(x))
	\end{align*}

	Підставляємо одержані вирази в диференціальне рівняння
	\begin{multline*}
		-2 e^{-x} \cdot ((a + b - a x) \cdot \sin(x) + ((a - b) + b \cdot x) \cdot \cos(x)) + \\ + 2 e^{-x} \cdot ((b - a x) \cdot \sin(x) + (a - (a - b) \cdot x) \cdot \cos(x)) + \\ + 2 x \cdot e^{-x} \cdot (a \cdot \cos(x) + b \cdot \sin(x)) = e^{-x} \cdot \sin(x). 
	\end{multline*}

	Прирівнюємо коефіцієнти при однакових членах
	\begin{table}[H]
		\centering
		\begin{tabular}{c|l}
			$e^{-x} \cdot \cos(x)$ & $2a + 2b = 0$ \\
			$e^{-x} \cdot \sin(x)$ & $-2 a - 2 b + c = 1$
		\end{tabular}
	\end{table}

	Звідси $a = -1$, $b = 1$. \\

	Таким чином загальний розв’язок має вигляд
	\begin{equation*}
		y_{\text{hetero}}(x) = C_1 \cdot e^{-x} \cdot \cos(x) + C_2 \cdot e^{-x} \cdot \sin(x) + x \cdot e^{-x} \cdot \left( \sin(x) - \cos(x) \right).
	\end{equation*}
\end{solution}

Знайти загальний розв’язок рівнянь:

\begin{multicols}{2}
\begin{problem}
	\[y'''-4y''+5y'-2y=2x+3;\]
\end{problem}
\begin{problem}
	\[y'''-3y'+2y=e^{-x}(4x^2+4x-10);\]
\end{problem}
\begin{problem}
	\[y^{(4)}+8y''+16y=\cos(x);\]
\end{problem}
\begin{problem}
	\[y^{(5)}+y'''=x^2-1;\]
\end{problem}
\begin{problem}
	\[y^{(4)}-y=x\cdot e^x+\cos(x);\]
\end{problem}
\begin{problem}
	\[y^{(4)}+2y''+y=x^2\cdot\cos(x);\]
\end{problem}
\begin{problem}
	\[y^{(4)}-y=5e^x\cdot\sin(x)+x^4;\]
\end{problem}
\begin{problem}
	\[y^{(4)}+5y''+4y=\sin(x)\cdot\cos(2x);\]
\end{problem}
\begin{problem}
	\[y'''-4y''+3y'=x^3 \cdot e^{2x};\]
\end{problem}
\begin{problem}
	\[y^{(4)}+y''=7x-3\cos(x);\]
\end{problem}
\begin{problem}
	\[y'''-y''-y'+y=3e^x+5x\cdot\sin(x);\]
\end{problem}
\begin{problem}
	\[y'''-2y''+4y'-8y=e^{2x}\sin(2x)+2x^2;\]
\end{problem}
\begin{problem}
	\[y'''+y'=\sin(x)+x\cdot\cos(x);\]
\end{problem}
\begin{problem}
	\[y'''-y=x^3-1;\]
\end{problem}
\begin{problem}
	\[y'''+y''=x^2+1+3x\cdot e^x;\]
\end{problem}
\begin{problem}
	\[y'''+y''+y'+y=x\cdot e^x;\]
\end{problem}
\begin{problem}
	\[y'''-9y'=-9(e^{3x}-2\sin3x+\cos3x);\]
\end{problem}
\begin{problem}
	\[y'''-y'=10\sin(x)+6\cos(x)+4e^x;\]
\end{problem}
\begin{problem}
	\[y'''-6y''+9y'=4x\cdot e^x;\]
\end{problem}
\begin{problem}
	\[y'''+2y''-3y'=(8x+6)\cdot e^x;\]
\end{problem}
\begin{problem}
	\[y^{(4)}+y''=x^2+x;\]
\end{problem}
\begin{problem}
	\[y'''-3y'+2y=(2x^2-x) e^x+\cos(x);\]
\end{problem}
\begin{problem}
	\[y^{(4)}-y=5e^x\cdot\cos(x)+3;\]
\end{problem}
\begin{problem}
	\[y^{(5)}-y'''=x^2+\cos(x);\]
\end{problem}
\begin{problem}
	\[y^{(4)}-2y''+y'=e^x;\]
\end{problem}
\begin{problem}
	\[y^{(4)}-2y'''+y''=x^3;\]
\end{problem}
\begin{problem}
	\[y^{(4)}+y'''=\cos(3x).\]
\end{problem}
\end{multicols}

Знайти частинний розв’язок диференціальних рівнянь:
\begin{problem}
	\[y'''-2y''+y'=4(\sin(x)+\cos(x)),\quad y(0)=1,y'(0)=0,y''(0)=-1;\]
\end{problem}
\begin{problem}
	\[y'''+2y''+y'=-2e^{-2x},\quad y(0)=2,y'(0)=y''(0)=1;\]
\end{problem}
\begin{problem}
	\[y'''-3y'=3(2-x^2),\quad y(0)=y'(0)=y''(0)=1;\]
\end{problem}
\begin{problem}
	\[y'''+2y''+y'=5e^x,\quad y(0)=y'(0)=y''(0)=0;\]
\end{problem}
\begin{problem}
	\[y'''-y'=3(2-x^2),\quad y(0)=y'(0)=y''(0)=1;\]
\end{problem}
\begin{problem}
	\[y'''+2y''+2y'+y=x,\quad y(0)=y'(0)=y''(0)=0.\]
\end{problem}
%
\end{document}