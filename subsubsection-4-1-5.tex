\begin{definition}
	Комбінацією, що інтегрується, називається диференціальне рівняння, отримане шляхом перетворень із системи, диференціальних рівнянь, але яке вже можна легко інтегрувати.
\end{definition}
\begin{equation*}
	\diff \Phi(t, x_1, x_2, \ldots, x_n) = 0.
\end{equation*}

Одна комбінація, що інтегрується, дає можливість одержати одне кінцеве рівняння
\begin{equation*}
	\Phi(t, x_1, x_2, \ldots, x_n) = C,
\end{equation*}
яке є першим інтегралом системи. \parvskip

Геометрично перший інтеграл являє собою $n$-вимірну поверхню в $(n + 1)$-вимірному просторі, що цілком складається з інтегральних кривих. \parvskip

Якщо знайдено $k$ комбінацій, що інтегруються, то одержуємо $k$ перших інтегралів
\begin{equation*}
	\left\{
		\begin{array}{rl}
			\Phi_1(t, x_1, x_2, \ldots, x_n) &= C_1, \\
			\Phi_2(t, x_1, x_2, \ldots, x_n) &= C_2, \\
			\hdotsfor{2} \\
			\Phi_n(t, x_1, x_2, \ldots, x_n) &= C_n.
		\end{array}
	\right.
\end{equation*}
 
І, якщо інтеграли незалежні, то хоча б один з визначників \[\frac{D(\Phi_1, \Phi_2, \ldots, \Phi_k)}{D(x_{i_1}, x_{i_2}, \ldots, x_{i_k})} \ne 0.\] Звідси з системи можна виразити $k$ невідомих функцій $x_{i_1}, x_{i_2}, \ldots, x_{i_k}$ через інші і підставивши їх у вихідну систему, понизити порядок до $(n - k)$ рівнянь. Якщо $n = k$ і всі інтеграли незалежні, то одержимо загальний інтеграл системи. \parvskip

Особливо поширеним засобом знаходження комбінацій, що інтегруються, є використання систем у симетричному вигляді. \parvskip

Систему диференціальних рівнянь, що записана в нормальній формі
\begin{equation*}
	\left\{
		\begin{array}{rl}
			\dot x_1 &= f_1 (x_1, x_2, \ldots, x_n, t), \\
			\dot x_2 &= f_2 (x_1, x_2, \ldots, x_n, t), \\
			\hdotsfor{2} \\
			\dot x_n &= f_n (x_1, x_2, \ldots, x_n, t).
		\end{array}
	\right.
\end{equation*}
можна переписати у вигляді
\begin{equation*}
	\frac{\diff x_1}{f_1(x_1, x_2, \ldots, x_n, t)} = \frac{\diff x_2}{f_2 (x_1, x_2, \ldots, x_n, t)} = \ldots = \frac{\diff x_n}{f_n (x_1, x_2, \ldots, x_n, t)} = \frac{\diff t}{1}.
\end{equation*}

При такій формі запису всі змінні $x_1, x_2, \ldots, x_n, t$ рівнозначні. \parvskip

Система диференціальних рівнянь, що записана у вигляді
\begin{equation*}
	\frac{\diff x_1}{X_1(x_1, x_2, \ldots, x_n)} = \frac{\diff x_2}{X_2 (x_1, x_2, \ldots, x_n)} = \ldots = \frac{\diff x_n}{X_n (x_1, x_2, \ldots, x_n)}.
\end{equation*}
називається системою у симетричному вигляді. \parvskip

При знаходженні комбінацій, що інтегруються, найбільш часто використовується властивість ``пропорційності''. А саме, для систем в симетричному вигляді справедлива рівність
\begin{multline*}
	\frac{\diff x_1}{X_1(x_1, x_2, \ldots, x_n)} = \frac{\diff x_2}{X_2 (x_1, x_2, \ldots, x_n)} = \ldots = \frac{\diff x_n}{X_n (x_1, x_2, \ldots, x_n)} = \\
	= \frac{k_1 \diff x_1 + k_2 \diff x_2 + \ldots k_n \diff x_n}{(k_1 X_1  + k_2 X_2  + \ldots + k_n X_n) (x_1, x_2, \ldots, x_n)}.
\end{multline*}
