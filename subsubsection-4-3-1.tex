Розглянемо один з методів побудови розв’язку систем з сталими коефіцієнтами. \\

Розв’язок системи шукаємо у вигляді вектора \[x(t) = (\alpha_1 \cdot e^{\lambda t}, \alpha_2 \cdot e^{\lambda t}, \ldots, \alpha_n \cdot e^{\lambda t})^T. \]

Підставивши в систему диференціальних рівнянь, одержимо
\begin{equation*}
	\left\{
		\begin{array}{rl}
			\alpha_1 \cdot \lambda \cdot e^{\lambda t} &= a_{11} \cdot \alpha_1 \cdot e^{\lambda t} + a_{12} \cdot \alpha_2  \cdot e^{\lambda t} + \ldots + a_{1n} \cdot \alpha_n \cdot e^{\lambda t}, \\
			\alpha_2 \cdot \lambda \cdot e^{\lambda t} &= a_{21} \cdot \alpha_1 \cdot e^{\lambda t} + a_{22} \cdot \alpha_2 \cdot e^{\lambda t} + \ldots + a_{2n} \cdot \alpha_n \cdot e^{\lambda t}, \\
			\hdotsfor{2} \\
			\alpha_n \cdot \lambda \cdot e^{\lambda t} &= a_{n1} \cdot \alpha_1 \cdot e^{\lambda t} + a_{n2} \cdot \alpha_2 \cdot e^{\lambda t} + \ldots + a_{nn} \cdot \alpha_n \cdot e^{\lambda t}.
		\end{array}
	\right.
\end{equation*}
 
Скоротивши на $e^{\lambda t}$, і перенісши всі члени вправо, запишемо
\begin{equation*}
	\left\{
		\begin{array}{rl}
			(a_{11} - \lambda) \cdot \alpha_1 + a_{12} \cdot \alpha_2 + \ldots + a_{1n} \cdot \alpha_n &= 0, \\
			a_{21} \cdot \alpha_1 + (a_{22} - \lambda) \cdot \alpha_2 + \ldots + a_{2n} \cdot \alpha_n &= 0, \\
			\hdotsfor{2} \\
			a_{n1} \cdot \alpha_1 + a_{n2} \cdot \alpha_2 + \ldots + (a_{nn} - \lambda) \cdot \alpha_n &= 0.
		\end{array}
	\right.
\end{equation*}
 
Отримана однорідна система лінійних алгебраїчних рівнянь має розв’язок тоді і тільки тоді, коли її визначник дорівнює нулю, тобто
\begin{equation*}
	\begin{vmatrix}
		a_{11} - \lambda & a_{12} & \ldots & a_{1n} \\
		a_{21} & a_{22} - \lambda & \ldots & a_{2n} \\
		\vdots & \vdots & \ddots & \vdots \\
		a_{n1} & a_{n2} & \ldots & a_{nn} - \lambda
	\end{vmatrix} = 0.
\end{equation*}

Це рівняння, може бути записаним у векторно-матричній формі
\begin{equation*}
	\det(A - \lambda E) = 0.
\end{equation*}
і воно називається характеристичним рівнянням. Розкриємо його
\begin{equation*}
	\lambda^n + p_1 \cdot \lambda^{n - 1} + \ldots + p_{n - 1} \cdot \lambda + p_n = 0.
\end{equation*}

Алгебраїчне рівняння $n$-го ступеня має $n$ коренів. Розглянемо різні випадки:
\begin{enumerate}
\item Всі корені характеристичного рівняння $\lambda_1, \lambda_2, \ldots, \lambda_n$ (власні числа матриці $A$) дійсні і різні. Підставляючи їх по черзі в систему алгебраїчних рівнянь
\begin{equation*}
	\left\{
		\begin{array}{rl}
			(a_{11} - \lambda_i) \cdot \alpha_1 + a_{12} \cdot \alpha_2 + \ldots + a_{1n} \cdot \alpha_n &= 0, \\
			a_{21} \cdot \alpha_1 + (a_{22} - \lambda_i) \cdot \alpha_2 + \ldots + a_{2n} \cdot \alpha_n &= 0, \\
			\hdotsfor{2} \\
			a_{n1} \cdot \alpha_1 + a_{n2} \cdot \alpha_2 + \ldots + (a_{nn} - \lambda_i) \cdot \alpha_n &= 0.
		\end{array}
	\right.
\end{equation*}

одержуємо відповідні ненульові розв’язки системи
\begin{equation*}
	\alpha^1 = \begin{pmatrix} \alpha_1^1 \\ \alpha_2^1 \\ \vdots \\ \alpha_n^1 \end{pmatrix}, \quad
	\alpha^2 = \begin{pmatrix} \alpha_1^2 \\ \alpha_2^2 \\ \vdots \\ \alpha_n^2 \end{pmatrix}, \quad
	\ldots, \quad
	\alpha^n = \begin{pmatrix} \alpha_1^n \\ \alpha_2^n \\ \vdots \\ \alpha_n^n \end{pmatrix}
\end{equation*}
що являють собою власні вектори, які відповідають власним числам $\lambda_i$, $i = \overline{1, n}$. \\

У такий спосіб одержимо $n$ розв’язків
\begin{equation*}
	x_1(t) = \begin{pmatrix} \alpha_1^1 \cdot e^{\lambda_1 x} \\ \alpha_2^1 \cdot e^{\lambda_1 x} \\ \vdots \\ \alpha_n^1 \cdot e^{\lambda_1 x} \end{pmatrix}, 
	x_2(t) = \begin{pmatrix} \alpha_1^2 \cdot e^{\lambda_2 x} \\ \alpha_2^2 \cdot e^{\lambda_2 x} \\ \vdots \\ \alpha_n^2 \cdot e^{\lambda_2 x} \end{pmatrix},
	\ldots, 
	x_n(t) = \begin{pmatrix} \alpha_1^n \cdot e^{\lambda_n x} \\ \alpha_2^n \cdot e^{\lambda_n x} \\ \vdots \\ \alpha_n^n \cdot e^{\lambda_n x} \end{pmatrix}
\end{equation*}

Причому оскільки $\lambda_1, \lambda_2, \ldots, \lambda_n$ --- різні а $\alpha^1, \alpha^2, \ldots, \alpha^n$ -- відповідні їм власні вектори, то розв’язки $x_1(t), x_2(t), \ldots, x_n(t)$ --- лінійно незалежні, і загальний розв’язок системи має вигляд
\begin{equation*}
	x(t) = \sum_{i = 1}^n C_i \cdot x_i(t).
\end{equation*}

Або у векторно-матричній формі запису
\begin{equation*}
	\begin{pmatrix} x_1 \\ x_2 \\ \vdots \\ x_n \end{pmatrix} = 
	\begin{pmatrix}
		\alpha_1^1 \cdot e^{\lambda_1 t} & \alpha_1^2 \cdot e^{\lambda_2 t} & \ldots & \alpha_1^n \cdot e^{\lambda_n t} \\
		\alpha_2^1 \cdot e^{\lambda_1 t} & \alpha_2^2 \cdot e^{\lambda_2 t} & \ldots & \alpha_2^n \cdot e^{\lambda_n t} \\
		\vdots & \vdots & \ddots & \vdots \\
		\alpha_n^1 \cdot e^{\lambda_1 t} & \alpha_n^2 \cdot e^{\lambda_2 t} & \ldots & \alpha_n^n \cdot e^{\lambda_n t}
	\end{pmatrix}
	\cdot
	\begin{pmatrix} C_1 \\ C_2 \\ \vdots \\ C_n \end{pmatrix},
\end{equation*}
де $C_1, C_2, \ldots, C_n$ --- довільні сталі.

\item Нехай $\lambda_{1,2} = p \pm i \cdot q$ --- пара комплексно спряжених коренів. Візьмемо один з них, наприклад $\lambda = p + i \cdot q$. Комплексному власному числу відповідає комплексний власний вектор
\begin{equation*}
	\begin{pmatrix} \alpha_1 \\ \alpha_2 \\ \vdots \\ \alpha_n \end{pmatrix} =
	\begin{pmatrix} r_1 + i \cdot s_1 \\ r_2 + i \cdot s_2 \\ \vdots \\ r_n + i \cdot s_n \end{pmatrix}
\end{equation*}
і, відповідно, розв’язок
\begin{equation*}
	\begin{pmatrix} x_1 \\ x_2 \\ \vdots \\ x_n \end{pmatrix} =
	\begin{pmatrix} (r_1 + i \cdot s_1) \cdot e^{(p + i q) t} \\ (r_2 + i \cdot s_2) \cdot e^{(p + i q) t} \\ \vdots \\ (r_n + i \cdot s_n) \cdot e^{(p + i q) t} \end{pmatrix}
\end{equation*}

Використовуючи залежність $e^{(p + i q) t} = e^{pt} \cdot (\cos (qt) + i \cdot \sin(qt))$, перетворимо розв’язок до вигляду:
\begin{multline*}
	\begin{pmatrix} x_1 \\ x_2 \\ \vdots \\ x_n \end{pmatrix} =
	\begin{pmatrix} (r_1 + i \cdot s_1) \cdot e^{p t} \cdot (\cos (qt) + i \cdot \sin(qt)) \\ (r_2 + i \cdot s_2) \cdot e^{p t} \cdot (\cos (qt) + i \cdot \sin(qt)) \\ \vdots \\ (r_n + i \cdot s_n) \cdot e^{p t} \cdot (\cos (qt) + i \cdot \sin(qt)) \end{pmatrix} = \\
	= \begin{pmatrix} e^{p t} \cdot (r_1 \cdot \cos (qt) - s_1 \cdot \sin(qt)) \\ e^{pt} \cdot (r_2 \cdot \cos (qt) - s_2 \cdot \sin(qt)) \\ \vdots \\ e^{pt} \cdot (r_n \cdot \cos (qt) - s_n \cdot \sin(qt)) \end{pmatrix} + \\ + i \cdot \begin{pmatrix} e^{p t} \cdot (s_1 \cdot \cos (qt) + r_1 \cdot \sin(qt)) \\ e^{pt} \cdot (s_2 \cdot \cos (qt) + r_2 \cdot \sin(qt)) \\ \vdots \\ e^{pt} \cdot (s_n \cdot \cos (qt) + r_n \cdot \sin(qt)) \end{pmatrix} = \\
	= u(t) + i \cdot v(t).
\end{multline*}

І, як випливає з властивості 4 розв’язків однорідних систем, якщо комплексна функція $u(t) + i \cdot v(t)$ дійсного аргументу є розв’язком однорідної системи, то окремо дійсна і уявна частини також будуть розв’язками, тобто комплексним власним числам $\lambda_{1,2} = p \pm i \cdot q$  відповідають лінійно незалежні розв’язки
\begin{align*}
	u(t) &= \begin{pmatrix} e^{p t} \cdot (r_1 \cdot \cos (qt) - s_1 \cdot \sin(qt)) \\ e^{pt} \cdot (r_2 \cdot \cos (qt) - s_2 \cdot \sin(qt)) \\ \vdots \\ e^{pt} \cdot (r_n \cdot \cos (qt) - s_n \cdot \sin(qt)) \end{pmatrix}, \\
	v(t) &= \begin{pmatrix} e^{p t} \cdot (s_1 \cdot \cos (qt) + r_1 \cdot \sin(qt)) \\ e^{pt} \cdot (s_2 \cdot \cos (qt) + r_2 \cdot \sin(qt)) \\ \vdots \\ e^{pt} \cdot (s_n \cdot \cos (qt) + r_n \cdot \sin(qt)) \end{pmatrix}
\end{align*}

\item Якщо характеристичне рівняння має кратний корінь $\lambda$ кратності $\gamma$, тобто $\lambda_1 = \lambda_2 = \ldots = \lambda_\gamma = \lambda$, то розв’язок системи рівнянь має вигляд
\begin{equation*}
	\begin{pmatrix} x_1 \\ x_2 \\ \vdots \\ x_n \end{pmatrix} = \begin{pmatrix} \left(\alpha_1^1 + \alpha_1^2 \cdot t + \ldots + \alpha_1^\gamma \cdot t^{\gamma - 1}\right) \cdot e^{\lambda t} \\ \left(\alpha_2^1 + \alpha_2^2 \cdot t + \ldots + \alpha_2^\gamma \cdot t^{\gamma - 1}\right) \cdot e^{\lambda t} \\ \vdots \\ \left(\alpha_n^1 + \alpha_n^2 \cdot t + \ldots + \alpha_n^\gamma \cdot t^{\gamma - 1}\right) \cdot e^{\lambda t} \end{pmatrix}
\end{equation*}

Підставивши його у вихідне диференціальне рівняння і прирівнявши коефіцієнти при однакових степенях, одержимо $\gamma \cdot n$ рівнянь, що містять $\gamma \cdot n$ невідомих. Тому що корінь характеристичного рівняння $\lambda$ має кратність $\gamma$, то ранг отриманої системи $\gamma n - \gamma = \gamma \cdot (n - 1)$. Уводячи $\gamma$ довільних сталих $C_1, C_2, \ldots, C_\gamma$ і розв’язуючи систему, одержимо
\begin{equation*}
	\alpha_i^j = \alpha_i^j(C_1, C_2, \ldots, C_\gamma), \quad i = \overline{1, n}, \quad j = \overline{1, \gamma}.
\end{equation*}
\end{enumerate}