% cd ..\..\Users\NikitaSkybytskyi\Desktop\differential-equations
% pdflatex chapter-2.tex && cls && pdflatex chapter-2.tex && cls && pdflatex chapter-2.tex && del chapter-2.toc, chapter-2.log, chapter-2.aux, chapter-2.out && start chapter-2.pdf

\documentclass[a4paper, 12pt]{article}
\usepackage[utf8]{inputenc}
\usepackage[T2A,T1]{fontenc}
\usepackage[english, ukrainian]{babel}
\usepackage{amsmath, amssymb, natbib, float, multirow, multicol, xcolor, hyperref}

\allowdisplaybreaks
\setlength\parindent{0pt}

\title{Диференціальні рівняння}
\author{Скибицький Нікіта}
\date{\today}

\hypersetup{unicode=true, colorlinks=true, linktoc=all, linkcolor=red}

\usepackage{amsthm}
\newtheorem{theorem}{Теорема}[section]
\newtheorem{lemma}{Лема}[section]
\theoremstyle{definition}
\newtheorem*{definition}{Визначення}
\newtheorem{problem}{Задача}[subsection]
\newtheorem*{example*}{Приклад}
\newtheorem{example}[problem]{Приклад}
\newtheorem{property}{Властивість}
\newtheorem*{solution}{Розв'язок}
\newtheorem*{remark}{Зауваження}

\renewcommand{\phi}{\varphi}
\renewcommand{\epsilon}{\varepsilon}
\newcommand{\RR}{\mathbb{R}}
\newcommand{\NN}{\mathbb{N}}

\DeclareMathOperator{\trace}{tr}

\newcommand{\todo}{\texttt{[TO DO]}}

\newcommand*\diff{\mathop{}\!\mathrm{d}}
\newcommand*\rfrac[2]{{}^{#1}\!/_{\!#2}}

\numberwithin{equation}{section}% reset equation counter for sections
\numberwithin{equation}{subsection}% Omit `.0` in equation numbers for non-existent subsections.
\renewcommand*{\theequation}{%
	\ifnum\value{subsection}=0%
		\thesection%
	\else%
		\thesubsection%
	\fi%
	.\arabic{equation}%
}

\makeatletter
\def\old@comma{,}
\catcode`\,=13
\def,{%
	\ifmmode%
		\old@comma\discretionary{}{}{}%
	\else%
		\old@comma%
	\fi%
}
\makeatother

\newcommand{\parvskip}{\vspace{1em}}

\begin{document}

\setcounter{section}{1}
\section{Нелінійні диференціальні рівняння вищих порядків}
\subsection{Загальні визначення. Існування та єдиність роз\-в’я\-з\-ків рівнянь}

Диференціальне рівняння $n$-го порядку має вигляд
\begin{equation*}
	%\label{eq:2.1.1}
	F \left( x, y, y', \ldots, y^{(n)} \right) = 0.
\end{equation*}
Якщо диференціальне рівняння розв’язане відносно старшої похідної, то воно має вигляд
\begin{equation*}
	%\label{eq:2.1.2}
	y^{(n)} = f \left( x, y, y', \ldots, y^{(n-1)} \right) = 0.
\end{equation*}
Іноді його називають диференціальним рівнянням у нормальній формі. Для диференціального рівняння, розв’язаного відносно похідної, задача Коші ставиться таким чином. Потрібно знайти функцію $y = y(x)$, $n$ разів неперервно диференційовану і таку, що при підстановці в останнє рівняння обертає його в тотожність і задовольняє початковим умовам 
\begin{equation*}
	%\label{eq:2.1.3}
	y(x_0) = y_0, y'(x_0) = y_0', \ldots, y^{(n - 1)} (x_0) = y_0^{(n-1)}.
\end{equation*}
Для диференціального рівняння, не розв’язаного відносно похідної, задача Коші полягає в знаходженні розв’язку $y = y(x)$, що задовольняє початковим даним 
\begin{equation*}
	%\label{eq:2.1.4}
	y(x_0) = y_0, y'(x_0) = y_0', \ldots, y^{(n - 1)} (x_0) = y_0^{(n-1)}, y^{(n)} (x_0) = y_0^{(n)},
\end{equation*}
де значення $x_0, y_0, y_0', \ldots, y_0^{(n-1)}$ довільні, а $y_0^{(n)}$ один з коренів алгебраїчного рівняння 
\begin{equation*}
	%\label{eq:2.1.5}
	F \left( x_0, y_0, y_0', \ldots, y_0^{(n)} \right) = 0.
\end{equation*}

\begin{theorem}[існування та єдиності розв’язку задачі Коші рівняння, розв’язаного відносно похідної]
	Нехай у деякому замкненому околі точки $\left(x_0, y_0, y_0', \ldots, y_0^{(n-1)}\right)$ функція $f\left(x,y,y',\ldots,y^{(n-1)}\right)$ задовольняє умовам:
	\begin{enumerate}
		\item вона визначена і неперервна по всім змінним;
		\item задовольняє умові Ліпшиця по всім змінним, починаючи з другої.
	\end{enumerate}
	Тоді при $x_0 - h \le x \le x_0 + h$, де $h$ -- досить мала величина, існує і єдиний розв’язок $y=y(x)$ рівняння
		\begin{equation*}
    	%\label{eq:2.1.2}
    	y^{(n)} = f \left( x, y, y', \ldots, y^{(n-1)} \right) = 0,
    \end{equation*}
    що задовольняє початковим умовам 
	\begin{equation*}
    	%\label{eq:2.1.3}
    	y(x_0) = y_0, y'(x_0) = y_0', \ldots, y^{(n - 1)} (x_0) = y_0^{(n-1)}.
    \end{equation*}
\end{theorem}
  
\begin{theorem}[існування та єдиності розв’язку задачі Коші рівняння, не розв’язаного відносно похідної]
	Нехай у деяком замкненому околі точки $\left(x_0, y_0, y_0', \ldots, y_0^{(n-1)}, y_0^{(n)}\right)$ функція $F\left(x,y,y',\ldots,y^{(n-1)},y^{(n)}\right)$ задовольняє умовам:
	\begin{enumerate}
		\item вона визначена і неперервна по всім змінним;
		\item її частинні похідні по всім змінним з другої пердеостанньої обмежені:
		\begin{equation*}
			%\label{eq:2.1.6}
			\left|\frac{\partial F}{\partial y}\right| < M_0, \quad \left|\frac{\partial F}{\partial y'}\right| < M_1, \quad \ldots, \quad \left|\frac{\partial F}{\partial y^{(n-1)}}\right| < M_{n-1}.
		\end{equation*}
		\item її частинна похідна по останній змінній не обертаєтсья на нуль: \[\left|\frac{\partial F}{\partial y^{(n)}}\right|\ne0.\]
	\end{enumerate}
	Тоді при $x_0 - h \le x \le x_0 + h$, де $h$ -- досить мала величина, існує і єдиний розв’язок $y=y(x)$ рівняння
    \begin{equation*}
	    %\label{eq:2.1.1}
    	F \left( x, y, y', \ldots, y^{(n)} \right) = 0.
    \end{equation*}
    що задовольняє початковим умовам
    \begin{equation*}
    	%\label{eq:2.1.4}
    	y(x_0) = y_0, y'(x_0) = y_0', \ldots, y^{(n - 1)} (x_0) = y_0^{(n-1)}, y^{(n)} (x_0) = y_0^{(n)}.
    \end{equation*}
\end{theorem}
\begin{definition}
	Загальним розв’язком диференціального рівняння $n$-го порядку називається $n$ разів неперервно диференційована функція $y=y(x,C_1,C_2,\ldots,C_n)$, що обертає при підстановці рівняння в тотожність, у якій вибором сталих $C_1, C_2, \ldots, C_n$ можна одержати розв’язок довільної задачі Коші в області існування та єдиності розв’язків.
\end{definition}



	\subsection{Загальні визначення. Існування та єдиність роз\-в'яз\-ків рівнянь}
	Диференціальне рівняння $n$-го порядку має вигляд
\begin{equation*}
	F \left( x, y, y', \ldots, y^{(n)} \right) = 0.
\end{equation*}

Якщо диференціальне рівняння розв'язане відносно старшої похідної, то воно має вигляд
\begin{equation*}
	y^{(n)} = f \left( x, y, y', \ldots, y^{(n-1)} \right) = 0.
\end{equation*}

Іноді його називають диференціальним рівнянням у нормальній формі. Для диференціального рівняння, розв'язаного відносно похідної, задача Коші ставиться таким чином. Потрібно знайти функцію $y = y(x)$, $n$ разів неперервно диференційовану і таку, що при підстановці в останнє рівняння обертає його в тотожність і задовольняє початковим умовам 
\begin{equation*}
	y(x_0) = y_0, \quad y'(x_0) = y_0', \quad \ldots, \quad y^{(n - 1)} (x_0) = y_0^{(n-1)}.
\end{equation*}

Для диференціального рівняння, не розв'язаного відносно похідної, задача Коші полягає в знаходженні розв'язку $y = y(x)$, що задовольняє початковим даним 
\begin{equation*}
	y(x_0) = y_0, \quad y'(x_0) = y_0', \quad \ldots, \quad y^{(n - 1)} (x_0) = y_0^{(n-1)}, \quad y^{(n)} (x_0) = y_0^{(n)},
\end{equation*}
де значення $x_0, y_0, y_0', \ldots, y_0^{(n-1)}$ довільні, а $y_0^{(n)}$ один з коренів алгебраїчного рівняння 
\begin{equation*}
	F \left( x_0, y_0, y_0', \ldots, y_0^{(n)} \right) = 0.
\end{equation*}

\begin{theorem}[існування та єдиності розв'язку задачі Коші рівняння, розв'язаного відносно похідної]
	Нехай у деякому замкненому околі точки $\left(x_0, y_0, y_0', \ldots, y_0^{(n-1)}\right)$ функція $f\left(x,y,y',\ldots,y^{(n-1)}\right)$ задовольняє умовам:
	\begin{enumerate}
		\item вона визначена і неперервна по всім змінним;
		\item задовольняє умові Ліпшиця по всім змінним, починаючи з другої.
	\end{enumerate}
	
	Тоді при $x_0 - h \le x \le x_0 + h$, де $h$ --- досить мала величина, існує і єдиний розв'язок $y=y(x)$ рівняння
		\begin{equation*}
		y^{(n)} = f \left( x, y, y', \ldots, y^{(n-1)} \right) = 0,
	\end{equation*}
	що задовольняє початковим умовам 
	\begin{equation*}
		y(x_0) = y_0, \quad y'(x_0) = y_0', \quad \ldots, \quad y^{(n - 1)} (x_0) = y_0^{(n-1)}.
	\end{equation*}
\end{theorem}

\begin{theorem}[існування та єдиності розв'язку задачі Коші рівняння, не розв'язаного відносно похідної]
	Нехай у деяком замкненому околі точки $\left(x_0, y_0, y_0', \ldots, y_0^{(n-1)}, y_0^{(n)}\right)$ функція $F\left(x,y,y',\ldots,y^{(n-1)},y^{(n)}\right)$ задовольняє умовам:
	\begin{enumerate}
		\item вона визначена і неперервна по всім змінним;
		\item її частинні похідні по всім змінним з другої до пердеостанньої обмежені:
		\begin{equation*}
			\left|\frac{\partial F}{\partial y}\right| < M_0, \quad \left|\frac{\partial F}{\partial y'}\right| < M_1, \quad \ldots, \quad \left|\frac{\partial F}{\partial y^{(n-1)}}\right| < M_{n-1}.
		\end{equation*}
		\item її частинна похідна по останній змінній не обертаєтсья на нуль: \[\left|\frac{\partial F}{\partial y^{(n)}}\right|\ne0.\]
	\end{enumerate}
	
	Тоді при $x_0 - h \le x \le x_0 + h$, де $h$ --- досить мала величина, існує і єдиний розв'язок $y=y(x)$ рівняння
	\begin{equation*}
		F \left( x, y, y', \ldots, y^{(n)} \right) = 0.
	\end{equation*}
	що задовольняє початковим умовам
	\begin{equation*}
		y(x_0) = y_0, \quad y'(x_0) = y_0', \quad \ldots, \quad y^{(n - 1)} (x_0) = y_0^{(n-1)}, \quad y^{(n)} (x_0) = y_0^{(n)}.
	\end{equation*}
\end{theorem}

\begin{definition}
	Загальним розв'язком диференціального рівняння $n$-го порядку називається $n$ разів неперервно диференційована функція вигляду $y = y(x,C_1, C_2, \ldots, C_n)$, що обертає при підстановці рівняння в тотожність, у якій вибором сталих $C_1, C_2, \ldots, C_n$ можна одержати розв'язок довільної задачі Коші в області існування та єдиності розв'язків.
\end{definition}



	\subsection{Диференціальні рівняння вищих порядків, що інтегруються в квадратурах}
	Розглянемо деякі типи диференціальних рівнянь, що інтегруються в квадратурах.

\begin{enumerate}
\item Рівняння вигляду
\begin{equation*}
	%\label{eq:2.2.1}
	y^{(n)} = f(x).
\end{equation*}
Проінтегрувавши його $n$ разів одержимо загальний розв’язок у вигляді
\begin{equation*}
	%\label{eq:2.2.2}
	y = \underset{n}{\underbrace{\int \cdots \int}} f(x) \,\underset{n}{\underbrace{\diff x \cdots \diff x}} + C_1 x^{n-1} + C_2 x^{n-2} + \ldots + C_{n-1} x + C_n.
\end{equation*}
Якщо задані умови Коші
\begin{equation*}
	%\label{eq:2.1.3}
	y(x_0) = y_0, y'(x_0) = y_0', \ldots, y^{(n - 1)} (x_0) = y_0^{(n-1)},
\end{equation*}
то розв’язок має вигляд
\begin{multline*}
	%\label{eq:2.2.3}
	y = \underset{n}{\underbrace{\int_{x_0}^x \cdots \int_{x_0}^x}} f(t) \, \underset{n}{\underbrace{\diff t \cdots \diff t}} + \frac{y_0}{(n-1)!} \cdot (x-x_0)^{n-1} + \\ 
	+ \frac{y_0'}{(n-2)!} \cdot (x-x_0)^{n-2} + \ldots + y_0^{(n-2)} \cdot (x-x_0) + y_0^{(n-1)}.
\end{multline*}
\item Рівняння вигляду
\begin{equation*}
	%\label{eq:2.2.4}
	F\left(x, y^{(n)}\right) = 0.
\end{equation*}
Нехай це рівняння вдалося записати в параметричному вигляді
\begin{equation*}
	%\label{eq:2.2.5}
	\left\{
		\begin{aligned}
			x &= \phi(t), \\
			y^{(n)} &= \psi (t).
		\end{aligned}
	\right.
\end{equation*}
Використовуючи основне співвідношення $\diff y^{(n-1)} = y^{(n)} \cdot \diff x$, одержимо
\begin{equation*}
	%\label{eq:2.2.6}
	\diff y^{(n-1)} = \psi(t) \cdot \phi(t) \cdot \diff t
\end{equation*}
Проінтегрувавши його, маємо 
  \begin{equation*}
	%\label{eq:2.2.7}
	y^{(n-1)} = \int \psi(t) \cdot \phi(t) \cdot \diff t + C_1 = \psi_1(t, C_1).
\end{equation*}
І одержимо параметричний запис рівняння $(n-1)$-го порядку:
\begin{equation*}
	%\label{eq:2.2.8}
	\left\{
		\begin{aligned}
			x &= \phi(t), \\
			y^{(n-1)} &= \psi_1(t, C_1).
		\end{aligned}
	\right.
\end{equation*}
Проробивши зазначений процес ще $(n-1)$ раз, одержимо загальний розв’язок рівняння в параметричному вигляді
\begin{equation*}
	%\label{eq:2.2.9}
	\left\{
		\begin{aligned}
			x &= \phi(t), \\
			y &= \psi_n(t, C_1, \ldots, C_n).
		\end{aligned}
	\right.
\end{equation*}
 
\item Рівняння вигляду
\begin{equation*}
	%\label{eq:2.2.10}
	F \left( y^{(n-1)}, y^{(n)} \right) = 0.
\end{equation*}
Нехай це рівняння вдалося записати в параметричному вигляді 
\begin{equation*}
	%\label{eq:2.2.11}
	\left\{
		\begin{aligned}
			y^{(n-1)} &= \phi(t), \\
			y^{(n)} &= \psi(t).
		\end{aligned}
	\right.
\end{equation*}
Використовуючи основне співвідношення $\diff y^{(n-1)} = y^{(n)} \cdot \diff x$, одержуємо
\begin{equation*}
	%\label{eq:2.2.12}
	\diff x = \frac{\diff y^{(n-1)}}{y^{(n)}} = \frac{\phi'(t)}{\psi(t)} \cdot \diff t.
\end{equation*}
Проінтегрувавши, маємо
\begin{equation*}
	%\label{eq:2.2.13}
	x = \int \frac{\phi'(t)}{\psi(t)} \cdot \diff t + C_1 = \psi_1(t, C_1).
\end{equation*}
І одержали параметричний запис майже з попереднього пункту. \parvskip

Використовуючи попередній пункт, запишемо загальний розв’язок у параметричному вигляді:
\begin{equation*}
	%\label{eq:2.2.14}
	\left\{
		\begin{aligned}
			x &= \psi(t, C_1), \\
			y &= \phi_n(t, C_2, \ldots, C_n).
		\end{aligned}
	\right.
\end{equation*}
 
\item Нехай рівняння вигляду
\begin{equation*}
	%\label{eq:2.2.15}
	F \left( y^{(n-2)}, y^{(n)} \right) = 0
\end{equation*}
можна розв'язати відносно старшої похідної
\begin{equation*}
	%\label{eq:2.2.16}
	y^{(n)} = f \left( y^{(n-2)} \right).
\end{equation*}
Домножимо його на $2 y^{(n-1)} \cdot \diff x$ й одержимо
\begin{equation*}
	%\label{eq:2.2.17}
	2 y^{(n-1)} \cdot y^{(n)} \cdot \diff x= 2 f \left( y^{(n-2)} \right) \cdot y^{(n-1)} \cdot \diff x.
\end{equation*}
Перепишемо його у вигляді
\begin{equation*}
	%\label{eq:2.2.18}
	\diff \left( y^{(n-1)} \right)^2 = 2 f \left( y^{(n-2)} \right) \cdot \diff y^{(n-2)}.
\end{equation*}
Проінтегрувавши, маємо
\begin{equation*}
	%\label{eq:2.2.19}
	\left( y^{(n-1)} \right)^2 = 2 \int f \left( y^{(n-2)} \right) \cdot \diff y^{(n-2)} + C_1,
\end{equation*}
тобто 
\begin{equation*}
	%\label{eq:2.2.20}
	y^{(n-1)}  = \pm \sqrt{2 \int f \left( y^{(n-2)} \right) \cdot \diff y^{(n-2)} + C_1},
\end{equation*}
або
\begin{equation*}
	%\label{eq:2.2.21}
	y^{(n-1)}  = \pm \psi_1 \left( y^{(n-2)}, C_1 \right).
\end{equation*}
Таким чином одержали повернулися до третього випадку.

\end{enumerate}

	\subsection{Найпростіші випадки зниження порядку в диференціальних рівняннях вищих порядків}
	Розглянемо деякі типи диференціальних рівнянь вищого порядку, що допускають зниження порядку.
\begin{enumerate}
\item Рівняння не містить шуканої функції і її похідних до $(k-1)$-го порядку включно:
\begin{equation}
	\label{eq:2.3.1}
	F \left( x, y^{(k)}, y^{(k + 1)}, \ldots, y^{(n)} \right) = 0.
\end{equation}
Зробивши заміну:
\begin{equation}
	\label{eq:2.3.2}
	y^{(k)} = z, \quad y^{(k + 1)} = z', \quad \ldots, \quad y^{(n)} = z^{(n - k)},
\end{equation}
одержимо рівняння $(n-k)$-го порядку
\begin{equation}
	\label{eq:2.3.3}
	F \left( x, z, z', \ldots, z^{(n - k)} \right) = 0.
\end{equation}

\item Рівняння не містить явно незалежної змінної
\begin{equation}
	\label{eq:2.3.4}
	F \left( y, y', \ldots, y^{(n)} \right) = 0.
\end{equation}

Будемо вважати, що $y$ -- нова незалежна змінна, а $y', \ldots, y^{(n)}$ -- функції від $y$. Тоді
\begin{align}
	\label{eq:2.3.5}
	y_x^\prime &= p(y), \\
	\label{eq:2.3.6}
	y_{x^2}^{\prime\prime} &= \frac{\diff}{\diff x} \cdot y_x^\prime = \frac{\diff}{\diff x} \cdot p(y) \cdot \frac{\diff y}{\diff x} = p_y^{\prime} \cdot  p(y), \\
	\label{eq:2.3.7}
	y_{x^3}^{\prime\prime\prime} &= \frac{\diff}{\diff x} \cdot y_{x^2}^{\prime\prime} = \frac{\diff}{\diff x} \cdot (p_y^\prime p) \cdot \frac{\diff y}{\diff x} = \left( p_{y^2}^{\prime\prime} \cdot p + \left( p_y^\prime \right)^2 \right) \cdot p,
\end{align}
і так далі до $y_{x^n}^{(n)}$. Після підстановки одержимо 
\begin{equation}
	\label{eq:2.3.8}
	F \left( y, p, p_y^{\prime} \cdot  p(y), \left( p_{y^2}^{\prime\prime} \cdot p + \left( p_y^\prime \right)^2 \right) \cdot p, \ldots, p^{(n - 1)} \right) = 0,
\end{equation}
диференціальне рівняння $(n-1)$-го порядку.
\item Нехай функція $F$ диференціального рівняння
\begin{equation}
	\label{eq:2.3.9}
	F \left( x, y, y', \ldots, y^{(n)} \right) = 0.
\end{equation}
є однорідної щодо аргументів  $y, y', \ldots, y^{(n)}$. \\

Робимо заміну $y = e^{\int u \diff x}$, де $u=u(x)$ -- нова невідома функція. Одержимо
\begin{align}
	\label{eq:2.3.10}
	y' &= e^{\int u \diff x} u, \\
	\label{eq:2.3.11}
	y^{\prime\prime} &= e^{\int u \diff x}  u^2 + e^{\int u \diff x} u' = e^{\int u \diff x} \left(u^2 + u'\right), \\
	\label{eq:2.3.12}
	y^{\prime\prime\prime} &= e^{\int u \diff x} u \left( u^2 + u' \right) + e^{\int u \diff x}  \left(2 u u' + u''\right) = \\ 
	&= e^{\int u \diff x} \left( u^3 + 3 u u' + u'' \right), \nonumber
\end{align}
і так далі до $y^{(n)}$. Після підстановки одержимо
\begin{equation}
	\label{eq:2.3.13}
	F \left( x, e^{\int u \diff x}, e^{\int u \diff x} u, e^{\int u \diff x} \left(u^2 + u'\right), e^{\int u \diff x} \left( u^3 + 3 u u' + u'' \right), \ldots \right) = 0.
\end{equation}

Оскільки \eqref{eq:2.3.9} (а отже і \eqref{eq:2.3.13}) однорідне відносно $e^{\int u\diff x}$, то цей член можна винести і на нього скоротити. Одержимо
\begin{equation}
	\label{eq:2.3.14}
	F \left( x, 1, u, u^2 + u', u^3 + 3 u u' + u'', \ldots \right) = 0,
\end{equation} 
диференціальне рівняння $(n-1)$-го порядку. 
\item Нехай ліва частина рівняння
\begin{equation}
	\label{eq:2.3.15}
	F \left( x, y, y', \ldots, y^{(n)} \right) = 0.
\end{equation}
є похідної деякого диференціального вираза ступеня $(n-1)$, тобто
\begin{equation}
	\label{eq:2.3.16}
	\frac{\diff}{\diff x} \cdot \Phi\left( x, y, y', \ldots, y^{(n-1)} \right) = F \left( x, y, y', \ldots, y^{(n)} \right).
\end{equation}
У цьому випадку легко обчислюється так званий перший інтеграл
\begin{equation}
	\label{eq:2.3.17}
	\Phi\left( x, y, y', \ldots, y^{(n-1)} \right) = C.
\end{equation}

\item Нехай диференціальне рівняння
\begin{equation}
	\label{eq:2.3.18}
	F \left( x, y, y', \ldots, y^{(n)} \right) = 0,
\end{equation}
розписано у вигляді диференціалів
\begin{equation}
	\label{eq:2.3.19}
	F \left( x, y, \diff y, \diff^2 y , \ldots, \diff^n y \right) = 0,
\end{equation}
і $F$ -- функція однорідна по всім змінним. Зробимо заміну $x = e^t$, $y = u \cdot e^t$, де $u$, $t$ -- нові змінні. Тоді одержуємо
\begin{align}
	\label{eq:2.3.20}
	\diff x &= e^t \diff t, \\
	\label{eq:2.3.21}
	y_x^\prime &=  \frac{y_t^\prime}{x_y^\prime} = \frac{u_t^\prime e^t + u e^t}{e^t} = u_t^\prime + u, \\
	\label{eq:2.3.22}
	y_{x^2}^{\prime\prime} &= \frac{\diff}{\diff x} \cdot y_x^\prime = \frac{\diff}{\diff t} \left( u_t^\prime + u \right) \cdot \frac{\diff t}{\diff x} = \frac{u_{t^2}^{\prime\prime} + u_t^\prime}{e^t}, \\
	\label{eq:2.3.23}
	y_{x^3}^{\prime\prime\prime} &= \frac{\diff}{\diff x} \cdot y_{x^2}^{\prime\prime} = \frac{\diff}{\diff t} \left( \frac{u_{t^2}^{\prime\prime} + u_t^\prime }{e^t} \right) \cdot \frac{\diff t}{\diff x} = \\
	&= \frac{\left( u_{t^3}^{\prime\prime\prime} + u_{t^2}^{\prime\prime} \right) e^t - \left( u_{t^2}^{\prime\prime} + u_t^\prime \right) e^t}{e^{3t}} = \frac{u_{t^3}^{\prime\prime\prime} - u_t^\prime}{e^{2t}}, \nonumber
\end{align}
і так далі до $y^{(n)}$. Підставивши, одержимо
\begin{multline}
	\label{eq:2.3.24}
	\Phi \left(x, y, \diff y, \diff^2 y, \ldots, \diff^n y\right) = \\
	= \Phi\left( e^t, u e^t, e^t \diff t, (u_t^\prime + u)e^t \diff t, \left(u_{t^2}^{\prime\prime} + u_t^\prime\right) e^t \diff t, \ldots\right) = 0.
\end{multline} 
Скоротивши на $e^t$ одержимо
\begin{equation}
	\label{eq:2.3.25}
	\Phi\left( 1, u, \diff t, u_t^\prime + u, u_{t^2}^{\prime\prime} + u_t^\prime, \ldots\right) = 0.
\end{equation} 
Тобто одержимо диференціальне рівняння вигляду \eqref{eq:2.3.4} і повертаємося до другого випадку.
\end{enumerate}

	\subsection{Вправи для самостійної роботи}
	Розглянемо приклади.
\begin{example}
	Розв’язати рівняння: $y'' = x + \sin x$.
\end{example}
\begin{solution}
	Інтегруємо два рази
	\begin{align*}
		y' &= \int (x + \sin x) \diff x + C_1 = \frac{x^2}{2} - \cos x + C_1; \\
		y &= \int \left( \frac{x^2}{2} - \cos x + C_1 \right) \diff x = \frac{x^3}{6} - \sin x + C_1 x+ C_2.
	\end{align*}
\end{solution}
\begin{example}
	Розв’язати рівняння $(y'')^3 - 2 y'' - x = 0$.
\end{example}
\begin{solution}
	Запишемо рівняння у параметричній формі \[ y'' = y, \quad x = t^3 - 2 t. \] Використовуючи співвідношення $\diff y' = y'' \diff x$, одержуємо \[ \diff y' = t (3t^2 - 2) \diff t,\] або \[ \diff y' = (3t^2 - 2t) \diff t.\] Звідси понижуємо порядок рівняння на одиницю \[ y' = \frac{3t^3}{4}-t^2+C_1, \quad x = t^3 - 2t.\] Знов використовуючи співвідношення $\diff y = y' \diff x$, одержуємо \[ \diff y = \left( \frac{3t^3}{4}-t^2+C_1 \right) \cdot (3t^2 - 2) \diff t,\] або \[ \diff y = \left( \frac{9t^5}{4} - 3 t^4 - \frac{3t^3}{2} + (2 + 3 C_1) t^2 - 2 C_1 \right) \diff t.\]	Звідси загальний розв’язок у параметричній формі має вигляд \[ x = t^3 - 2 t, \quad y = \frac{3t^6}{8} - \frac{3 t^5}{5} - \frac{3t^4}{8} + \frac{(2 + 3 C_1) t^3}{3} - 2 C_1 t + C_2.\]
\end{solution}
\begin{example}
	Розв’язати рівняння: $(y'')^3 + x y'' = y'$.
\end{example}
\begin{solution}
	Запишемо рівняння у параметричній формі \[ y'' = t, \quad y''' = e^{-t}. \] Використовуючи співвідношення $\diff y' = y'' \diff x$, одержуємо \[ \diff t = e^{-t} \diff x. \] Звідси $\diff x = e^t \diff t$ і $x = e^t + C_1$. Запишемо рівняння другого порядку \[ x = e^r + C_1, \quad y'' = t.\] Запишемо рівняння у параметричній формі \[ y'' = t, \quad x = t^3 - 2 t.\] Використовуючи співвідношення $\diff y' = y'' \diff x$, одержуємо \[ \diff y' = t e^t \diff t. \] Звідси \[y' = \int t e^t \diff t = e^t (t - 1) + C_2.\] Одержали диференціальне рівняння першого порядку \[ x = e^t + C_1, \quad y' = e^t (t - 1) + C_2.\] Використовуючи співвідношення $\diff y = y' \diff x$, запишемо \[ \diff y = (e^t (t - 1) + C_2) e^t \diff t.\] Звідси \[ y = \frac{e^{2t}(t-1)}{2} - \frac{e^{2t}}{4} + C_2 e^t + C_3.\] Остаточно загальний розв’язок має вигляд \[ x = e^t + C_1, \quad y = \frac{e^{2t}(2t - 3)}{4}+C_2e^t + C_3.\] Якщо вилучити параметр $t$, то одержимо загальний розв’язок \[ y = \frac{(x-C_1)^2}{4} \cdot \left( 2 \ln |x - C_1| - 3 \right) + C_2 (x - C_1) + C_3.\]
\end{solution}
\begin{example}
	Розв’язати рівняння: $3 \sqrt[3]{y} y'' = 1$.
\end{example}
\begin{solution}
	Запишемо рівняння у вигляді \[ y'' = \frac{1}{3\sqrt[3]{y}}. \] Помножимо обидві частини на $2 y' \diff x$. Одержимо \[ 2 y'' y' \diff x = \frac{2y'\diff x}{3\sqrt[3]{y}}, \] або \[ \diff (y')^2 = \frac{2\diff y}{3\sqrt[3]{y}}. \] Проінтегруємо і одержимо \[ (y')^2 = \sqrt[3]{y^2} + C_1.\] Звідси $y' = \pm \sqrt{y^{2/3} + C_1}$. Нехай початкові умови такі, що $C_1 = {\bar C_1}^2 > 0$. Тобто рівняння має вигляд \[ \frac{\diff y}{\diff x} = \pm \sqrt{y^{2/3} + {\bar C_1}^2}. \] Розділимо змінні \[ \int \frac{\diff y}{\sqrt{y^{2/3} + {\bar C_1}^2}} = \pm \int \diff x + C_2. \] Робимо заміну $\sqrt{y^{2/3} + {\bar C_1}^2} = t$. Тоді \[ y = \left(t^2 - {\bar C_1}^2\right)^{3/2}, \quad \diff y = 3 \left(t^2 - {\bar C_1}^2\right)^{1/2} t \diff t,\]	і інтеграл має вигляд \[ \int \frac{\diff y}{\sqrt{y^{2/3} + {\bar C_1}^2}}=3\int \sqrt{t^2 - {\bar C_1}^2} \diff t = 3.\]
\end{solution}
Розв’язати рівняння:
\begin{multicols}{2}
\begin{problem}
	\[ y'' \cdot x \cdot \ln x = y';\]
\end{problem}
\begin{problem}
	\[y''' = x + \cos x;\]
\end{problem}
\end{multicols}
\begin{problem}
	$2 x y'' = y'$ при $x_0 = 0$, $y_0 = 0$, $y_0' = 0$, $y_0'' = 0$;
\end{problem}
\begin{problem}
	$x y'' + y' = x + 1$ при $x_0 = 0$, $y_0 = 0$, $y_0' = 0$, $y_0'' = 0$;
\end{problem}
\begin{problem}
	$\tan x \cdot y'' = y' + \frac{1}{\sin x} = 0$ при $x_0 = 0$, $y_0 = 2$, $y_0' = 1$, $y_0'' = 1$;
\end{problem}
\begin{multicols}{2}
\begin{problem}
	\[(y'')^4+y''-x=0;\]
\end{problem}
\begin{problem}
	\[y''+\ln y''-x=0;\]
\end{problem}
\begin{problem}
	\[y''-a\cdot(1+(y')^2)^{3/2}=0;\]
\end{problem}
\begin{problem}
	\[y'''-(y'')^3;\]
\end{problem}
\begin{problem}
	\[y'''-y''=0;\]
\end{problem}
\begin{problem}
	\[y''+2y''\cdot\ln y'-1=0;\]
\end{problem}
\begin{problem}
	\[(y''')^2+(y'')^2-1=0;\]
\end{problem}
\begin{problem}
	\[y''\cdot y^3-1=0;\]
\end{problem}
\begin{problem}
	\[y^3\cdot y''-y^4+0;\]
\end{problem}
\begin{problem}
	\[4\sqrt{y}\cdot y''=1;\]
\end{problem}
\begin{problem}
	\[3y''=y^{-5/3};\]
\end{problem}
\begin{problem}
	\[(y'')^2+(y')^2-(y')^4=0;\]
\end{problem}
\end{multicols}
\begin{example}
	Розв’язати рівняння: $(y'')^3 + x \cdot y'' = y'$.
\end{example}
\begin{solution}
	Позначимо $y'=z$, $y''=z'$. Одержимо рівняння $(z')^3+xz'=z$, тобто рівняння Клеро, що легко інтегрується введенням параметра. \\

	Нехай $z' = p$. Тоді $z=xp+p^3$. Продиференцюємо це співвідношення: \[\diff z + x \diff p + p \diff x + 3 p^2 \diff p.\] Підставивши $\diff z = p \diff z$, отримаємо $(x+3p^2)\diff p = 0$. Це рівняння розділяється на два:
	\begin{enumerate}
		\item $x + 3p^2 = 0$. Звідси маємо $x = -3p^2$, $z=-2p^2$. Повертаємось до вихідних змінних $x=-3p^2$, $y'=-2p^3$. Використовуємо основне співвідношення $\diff y = y' \diff x$. Одержуємо \[ \diff y = 12 p^4 \diff p \implies y = \frac{12p^5}{5} + C_1.\] Таким чином перша гілка дає розв’язок \[ x = -3p^2, \quad y=\frac{12p^5}{5} + C_1.\]
		\item $\diff p = 0$. Звідси маємо $z = C_1 x + C_1^3$. Повертаємось до вихідних змінних $y'=C_1x+C_1^3$. Проінтегруємо і отримаємо другу гілку розв’язків \[y=\frac{C_1x^2}{2}+C_1^3x+C_2.\]
	\end{enumerate}
\end{solution}
\begin{example}
	Розв’язати рівняння: $y^4 - y^3 \cdot y'' = 1$.
\end{example}
\begin{solution}
	Відсутній аргумент $x$, отже, його порядок знижується заміною: \[y'=p,\quad y''=p\cdot\frac{\diff p}{\diff y}.\] Звідси одержуємо \[ y^4 - y^3 \cdot p \cdot \frac{\diff p}{\diff y} = 1.\] Розділимо змінні: \[ \frac{y^4-1}{y^3} \cdot \diff y=p \cdot \diff p.\] Проінтегруємо \[ \frac{y^2}{2}+\frac{1}{2y^2}=\frac{p^2}{2}-\frac{C_1}{2}.\] Звідси одержали \[p^2=y^2+C_1+y^{-2}.\] Повертаємось до вихідних змінних \[(y')^2=y^2+C_1+y^{-2}.\] Розв’яжемо рівняння відносно похідної \[ y' = \pm \sqrt{y^2 + C_1 + y^{-2}}.\] Розділимо змінні \[ \pm\frac{\diff y}{\sqrt{y^2+C_1+y^{-2}}}=\diff x.\] Візьмемо інтеграл \begin{multline*} \pm \int \frac{\diff y}{\sqrt{y^2+C_1+y^{-2}}} = \pm \int \frac{y \diff y}{\sqrt{y^4+C_1y^2+1}} = \\ =\pm \int \frac{\diff \left(y^2 + \frac{C_1}{2}\right)}{\sqrt{\left(y^2+\frac{C_1}{2}\right)^2+\left(1-\frac{C_1}{4}\right)}} = \pm \frac12 \ln \left| y^2 + \frac{C_1}{2} + \sqrt{y^4+C_1y^2+1}\right|. \end{multline*} Таким чином загальний розв’язок має вигляд: \[x = \pm \frac12 \ln \left| y^2 + C_1/2 + \sqrt{y^4+C_1y^2+1}\right|.\]
\end{solution}
\begin{example}
	Розв’язати рівняння: $y \cdot y'' = (y')^2$.
\end{example}
\begin{solution}
	Оскільки рівняння однорідне по змінним $y, y', y''$, то робимо заміну \[ y = e^{\int u \diff x}, \quad y' = e^{\int u \diff x} u, \quad y'' = e^{\int u \diff x} (u^2 +u). \] Рівняння буде мати вигляд \[ e^{\int u \diff x} \cdot e^{\int u \diff x} (u^2 + u) = \left( e^{\int u \diff x} u\right)^2.\] Скоротимо на $e^{\int u \diff x}$. Маємо $u^2 + u' = u^2$, або $u'=0$. Звідси $u'=C_1$ і одержимо загальний розв’язок \[ y = e^{\int C_1 x \diff x} = e^{c_1 x^2 + \ln |c_2|}=c_2 e^{c_1x^2}.\]
\end{solution}
\begin{example}
	Розв’язати рівняння: $y \cdot y'' - (y')^2 = y^2$.
\end{example} 
\begin{solution}
	Розділимо рівняння на $y^2$: \[ \frac{y \cdot y'' - (y')^2}{y^2} = 1,\] і перепишемо у вигляді: \[ \frac{\diff}{\diff x} \left( \frac{y'}{y}\right)=1.\] Проінтегрувавши, одержимо загальний розв’зок \[\frac{y'}{y}=x+C_1\implies \ln |y| = \frac{x^2}{2}+C_1x+\ln |C_2|\implies y=C_2 e^{x^2/2+C_1x}.\]
\end{solution}
Розв’язати рівняння:
\begin{multicols}{2}
\begin{problem}
	\[x \cdot y'' = y' \cdot \ln (y'/x);\]
\end{problem}
\begin{problem}
	\[ 2 y \cdot y'' - 3 (y')^2 = 4 y^2;\]
\end{problem}
\begin{problem}
	\[2x \cdot y'' = y';\]
\end{problem}
\begin{problem}
	\[x \cdot y'' + y' = x + 1;\]
\end{problem}
\begin{problem}
	\[\tan x \cdot y'' - y' + \frac{1}{\sin x}=0;\]
\end{problem}
\begin{problem}
	\[x^2\cdot y''+x\cdot y'=1;\]
\end{problem}
\begin{problem}
	\[y'' \cdot \cot (2x) + 2y' = 0;\]
\end{problem}
\begin{problem}
	\[x^3 \cdot y'' + x^2 \cdot y = 0;\]
\end{problem}
\begin{problem}
	\[\tan x \cdot y'' = 2 y';\]
\end{problem}
\begin{problem}
	\[y \cdot y'' - (y')^2 - y^2 \cdot \ln y = 0;\]
\end{problem}
\begin{problem}
	\[x^4\cdot y''+x^3\cdot y'=1;\]
\end{problem}
\begin{problem}
	\[x\cdot y\cdot y''-x \cdot (y')^2 - 2 y \cdot y' = 0;\]
\end{problem}
\begin{problem}
	\[x\cdot y\cdot y''-x \cdot (y')^2 - y \cdot y'+\frac{x \cdot (y')^2}{\sqrt{1-x^2}}=0;\]
\end{problem}
\begin{problem}
	\[x^2\cdot y'''-x \cdot (y'')^2=0;\]
\end{problem}
\begin{problem}
	\[x^5\cdot y''+x^4 \cdot y'=1.\]
\end{problem}
\end{multicols}%
\end{document}