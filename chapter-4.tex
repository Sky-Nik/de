% cd ..\..\Users\NikitaSkybytskyi\Desktop\differential-equations
% pdflatex chapter-3.tex && cls && pdflatex chapter-3.tex && cls && pdflatex chapter-3.tex && del chapter-3.toc, chapter-3.log, chapter-3.aux, chapter-3.out && start chapter-3.pdf

\documentclass[a4paper, 12pt]{article}
\usepackage[utf8]{inputenc}
\usepackage[T2A,T1]{fontenc}
\usepackage[english, ukrainian]{babel}
\usepackage{amsmath, amssymb, natbib, float, multirow, multicol, xcolor, hyperref}

\allowdisplaybreaks
\setlength\parindent{0pt}

\title{Диференціальні рівняння}
\author{Скибицький Нікіта}
\date{\today}

\hypersetup{unicode=true, colorlinks=true, linktoc=all, linkcolor=red}

\usepackage{amsthm}
\newtheorem{theorem}{Теорема}[section]
\newtheorem{lemma}{Лема}[section]
\theoremstyle{definition}
\newtheorem*{definition}{Визначення}
\newtheorem{problem}{Задача}[subsection]
\newtheorem*{example*}{Приклад}
\newtheorem{example}[problem]{Приклад}
\newtheorem{property}{Властивість}
\newtheorem*{solution}{Розв'язок}
\newtheorem*{remark}{Зауваження}

\renewcommand{\phi}{\varphi}
\renewcommand{\epsilon}{\varepsilon}
\newcommand{\RR}{\mathbb{R}}
\newcommand{\NN}{\mathbb{N}}

\DeclareMathOperator{\trace}{tr}
\DeclareMathOperator{\rang}{rang}

\newcommand{\todo}{\texttt{[TO DO]}}

\newcommand*\diff{\mathop{}\!\mathrm{d}}
\newcommand*\rfrac[2]{{}^{#1}\!/_{\!#2}}

\numberwithin{equation}{section}% reset equation counter for sections
\numberwithin{equation}{subsection}% Omit `.0` in equation numbers for non-existent subsections.
\renewcommand*{\theequation}{%
	\ifnum\value{subsection}=0%
		\thesection%
	\else%
		\thesubsection%
	\fi%
	.\arabic{equation}%
}

\makeatletter
\def\old@comma{,}
\catcode`\,=13
\def,{%
	\ifmmode%
		\old@comma\discretionary{}{}{}%
	\else%
		\old@comma%
	\fi%
}
\makeatother

\newcommand{\parvskip}{\medskip}

\begin{document}

\setcounter{section}{3}
\section{Системи диференціальних рівнянь}
\subsection{Загальна теорія}

Співвідношення вигляду
\begin{equation*}
	\left\{
		\begin{aligned}
			F_1(\dot x_1, \dot x_2, \ldots, \dot x_n, x_1, x_2, \ldots, x_n, t) &= 0, \\
			F_2(\dot x_1, \dot x_2, \ldots, \dot x_n, x_1, x_2, \ldots, x_n, t) &= 0, \\
			\ldots \ldots \ldots \ldots \ldots \ldots \ldots \ldots \ldots \ldots \ldots & \ldots . \\
			F_n(\dot x_1, \dot x_2, \ldots, \dot x_n, x_1, x_2, \ldots, x_n, t) &= 0
		\end{aligned}
	\right.
\end{equation*}
називається системою $n$ звичайних диференціальних рівнянь першого порядку. \\

Якщо система розв'язана відносно похідних і має вигляд
\begin{equation*}
	\left\{
		\begin{aligned}
			\dot x_1 &= f_1(x_1, x_2, \ldots, x_n, t) &= 0, \\
			\dot x_2 &= f_2(x_1, x_2, \ldots, x_n, t) &= 0, \\
			\ldots & \ldots \ldots \ldots \ldots \ldots \ldots \ldots \ldots &\ldots . \\
			\dot x_n &= f_n(x_1, x_2, \ldots, x_n, t) &= 0
		\end{aligned}
	\right.
\end{equation*}
то вона називається системою в нормальній формі.

\begin{definition}
	Розв'язком системи диференціальних рівнянь на\-зи\-ва\-є\-ть\-ся набір $n$ неперервно диференційованих функцій $x_1(t), \ldots, x_n(t)$ що тотожно задовольняють кожному з рівнянь системи.
\end{definition}

У загальному випадку розв'язок системи залежить від $n$ довільних сталих і має вигляд  $x_1(t, C_1, \ldots, C_n), \ldots, x_n(t, C_1, \ldots, C_n)$ і задача Коші для системи звичайних диференціальних рівнянь першого порядку ставиться в такий спосіб. Потрібно знайти розв'язок, що задовольняє початковим умовам (умовам Коші):
\begin{equation*}
	x_1(t_0) = x_1^0, \quad x_2(t_0) = x_2^0, \quad \ldots, \quad x_n(t_0) = x_n^0
\end{equation*}

\begin{definition}
	Розв'язок $x_1(t, C_1, \ldots, C_n)$, $\ldots$, $x_n(t, C_1, \ldots, C_n)$ на\-зи\-ва\-є\-ть\-ся загальним, якщо за рахунок вибору сталих $C_1, \ldots, C_n$ можна роз\-в'яз\-а\-ти довільну задачу Коші.
\end{definition}

Для систем звичайних диференціальних рівнянь досить важливим є поняття інтеграла системи. В залежності від гладкості (тобто диференційованості) можна розглядати два визначення інтеграла.

\begin{definition}
	\begin{enumerate}
		\item Функція $F(x_1, x_2, \ldots, x_n, t)$ стала уздовж розв'язків системи, називається інтегралом системи.
		\item Функція $F(x_1, x_2, \ldots, x_n, t)$ повна похідна, якої в силу системи тотожно дорівнює нулю, називається інтегралом системи.
	\end{enumerate}
\end{definition}

Для лінійних рівнянь існує поняття лінійної залежності і незалежності розв'язків. Для нелінійних рівнянь (систем рівнянь) аналогічним поняттям є функціональна незалежність.

\begin{definition}
	Інтеграли 
	\begin{equation*}
		F_1(x_1, x_2, \ldots, x_n, t), \quad F_2(x_1, x_2, \ldots, x_n, t), \quad \ldots, \quad F_n(x_1, x_2, \ldots, x_n, t)
	\end{equation*}
	називаються функціонально незалежними, якщо не існує функції $n$ змінних $\Phi(z_1, z_2, \ldots, z_n)$ такої, що
	\begin{equation*}
		\Phi(F_1(x_1, x_2, \ldots, x_n, t), F_2(x_1, x_2, \ldots, x_n, t), \ldots, F_n(x_1, x_2, \ldots, x_n, t)) = 0.
	\end{equation*}
\end{definition}

\begin{theorem}
	Для того щоб інтеграли 
	\begin{equation*}
		F_1(x_1, x_2, \ldots, x_n, t), \quad F_2(x_1, x_2, \ldots, x_n, t), \quad \ldots, \quad F_n(x_1, x_2, \ldots, x_n, t)
	\end{equation*}
	системи звичайних диференціальних рівнянь були функціонально незалежними, необхідно і достатньо, щоб визначник Якобі був відмінний від тотожного нуля, тобто 
	\begin{equation*}
		\frac{D(F_1, F_2, \ldots, F_n)}{D(x_1, x_2, \ldots, x_n)} \ne 0.
	\end{equation*}
\end{theorem}

\begin{definition}
	Якщо $F(x_1, x_2, \ldots, x_n, t)$ -- інтеграл системи диференціальних рівнянь, то рівність $F(x_1, x_2, \ldots, x_n, t) = C$ називається першим інтегралом.
\end{definition}

\begin{definition}
	Сукупність $n$ функціонально незалежних інтегралів називається загальним інтегралом системи диференціальних рівнянь.
\end{definition}

Власне кажучи загальний інтеграл -- це загальний роз\-в'яз\-ок системи диференціальних рівнянь у неявному вигляді.

\begin{theorem}[існування та єдиності розв'язку задачі Коші]
	Щоб система диференціальних рівнянь, розв'язаних відносно похідної, мала єдиний роз\-в'яз\-ок, що задовольняє умовам Коші: 
	\begin{equation*}
		x_1(t_0) = x_1^0, \quad x_2(t_0) = x_2^0, \quad \ldots, \quad x_n(t_0) = x_n^0
	\end{equation*}
	досить, щоб:
	\begin{enumerate}
		\item функції $f_1, f_2, \ldots, f_n$ були неперервними по змінним $x_1, x_2, \ldots, x_n, t$ в околі точки $\left(x_1^0, x_2^0, \ldots, x_n^0, t_0\right)$;
		\item функції $f_1, f_2, \ldots, f_n$ задовольняли умові Ліпшиця по аргументах $x_1, x_2, \ldots, x_n$ у тому ж околі.
	\end{enumerate}
\end{theorem}

\begin{remark}
	Умову Ліпшиця можна замінити більш грубою умовою, але такою, що перевіряється легше, існування обмежених частинних похідних, тобто 
	\begin{equation*}
		\left| \frac{\partial f_i}{\partial x_j} \right| \le M, \quad i,j = 1, 2, \ldots, n.
	\end{equation*}
\end{remark}

	\subsection{Загальна теорія}
	Співвідношення вигляду
\begin{equation*}
	\left\{
		\begin{aligned}
			F_1(\dot x_1, \dot x_2, \ldots, \dot x_n, x_1, x_2, \ldots, x_n, t) &= 0, \\
			F_2(\dot x_1, \dot x_2, \ldots, \dot x_n, x_1, x_2, \ldots, x_n, t) &= 0, \\
			\ldots \ldots \ldots \ldots \ldots \ldots \ldots \ldots \ldots \ldots \ldots & \ldots . \\
			F_n(\dot x_1, \dot x_2, \ldots, \dot x_n, x_1, x_2, \ldots, x_n, t) &= 0
		\end{aligned}
	\right.
\end{equation*}
називається системою $n$ звичайних диференціальних рівнянь першого порядку. \parvskip

Якщо система розв'язана відносно похідних і має вигляд
\begin{equation*}
	\left\{
		\begin{aligned}
			\dot x_1 &= f_1(x_1, x_2, \ldots, x_n, t) &= 0, \\
			\dot x_2 &= f_2(x_1, x_2, \ldots, x_n, t) &= 0, \\
			\ldots & \ldots \ldots \ldots \ldots \ldots \ldots \ldots \ldots &\ldots . \\
			\dot x_n &= f_n(x_1, x_2, \ldots, x_n, t) &= 0
		\end{aligned}
	\right.
\end{equation*}
то вона називається системою в нормальній формі.

\begin{definition}
	Розв'язком системи диференціальних рівнянь на\-зи\-ва\-є\-ть\-ся набір $n$ неперервно диференційованих функцій $x_1(t), \ldots, x_n(t)$ що тотожно задовольняють кожному з рівнянь системи.
\end{definition}

У загальному випадку розв'язок системи залежить від $n$ довільних сталих і має вигляд  $x_1(t, C_1, \ldots, C_n), \ldots, x_n(t, C_1, \ldots, C_n)$ і задача Коші для системи звичайних диференціальних рівнянь першого порядку ставиться в такий спосіб. Потрібно знайти розв'язок, що задовольняє початковим умовам (умовам Коші):
\begin{equation*}
	x_1(t_0) = x_1^0, \quad x_2(t_0) = x_2^0, \quad \ldots, \quad x_n(t_0) = x_n^0
\end{equation*}

\begin{definition}
	Розв'язок $x_1(t, C_1, \ldots, C_n)$, $\ldots$, $x_n(t, C_1, \ldots, C_n)$ на\-зи\-ва\-є\-ть\-ся загальним, якщо за рахунок вибору сталих $C_1, \ldots, C_n$ можна роз\-в'яз\-а\-ти довільну задачу Коші.
\end{definition}

Для систем звичайних диференціальних рівнянь досить важливим є поняття інтеграла системи. В залежності від гладкості (тобто диференційованості) можна розглядати два визначення інтеграла.

\begin{definition}
	\begin{enumerate}
		\item Функція $F(x_1, x_2, \ldots, x_n, t)$ стала уздовж розв'язків системи, називається інтегралом системи.
		\item Функція $F(x_1, x_2, \ldots, x_n, t)$ повна похідна, якої в силу системи тотожно дорівнює нулю, називається інтегралом системи.
	\end{enumerate}
\end{definition}

Для лінійних рівнянь існує поняття лінійної залежності і незалежності розв'язків. Для нелінійних рівнянь (систем рівнянь) аналогічним поняттям є функціональна незалежність.

\begin{definition}
	Інтеграли 
	\begin{equation*}
		F_1(x_1, x_2, \ldots, x_n, t), \quad F_2(x_1, x_2, \ldots, x_n, t), \quad \ldots, \quad F_n(x_1, x_2, \ldots, x_n, t)
	\end{equation*}
	називаються функціонально незалежними, якщо не існує функції $n$ змінних $\Phi(z_1, z_2, \ldots, z_n)$ такої, що
	\begin{equation*}
		\Phi(F_1(x_1, x_2, \ldots, x_n, t), F_2(x_1, x_2, \ldots, x_n, t), \ldots, F_n(x_1, x_2, \ldots, x_n, t)) = 0.
	\end{equation*}
\end{definition}

\begin{theorem}
	Для того щоб інтеграли 
	\begin{equation*}
		F_1(x_1, x_2, \ldots, x_n, t), \quad F_2(x_1, x_2, \ldots, x_n, t), \quad \ldots, \quad F_n(x_1, x_2, \ldots, x_n, t)
	\end{equation*}
	системи звичайних диференціальних рівнянь були функціонально незалежними, необхідно і достатньо, щоб визначник Якобі був відмінний від тотожного нуля, тобто 
	\begin{equation*}
		\frac{D(F_1, F_2, \ldots, F_n)}{D(x_1, x_2, \ldots, x_n)} \ne 0.
	\end{equation*}
\end{theorem}

\begin{definition}
	Якщо $F(x_1, x_2, \ldots, x_n, t)$ --- інтеграл системи диференціальних рівнянь, то рівність $F(x_1, x_2, \ldots, x_n, t) = C$ називається першим інтегралом.
\end{definition}

\begin{definition}
	Сукупність $n$ функціонально незалежних інтегралів називається загальним інтегралом системи диференціальних рівнянь.
\end{definition}

Власне кажучи загальний інтеграл --- це загальний роз\-в'яз\-ок системи диференціальних рівнянь у неявному вигляді.

\begin{theorem}[існування та єдиності розв'язку задачі Коші]
	Щоб система диференціальних рівнянь, розв'язаних відносно похідної, мала єдиний роз\-в'яз\-ок, що задовольняє умовам Коші: 
	\begin{equation*}
		x_1(t_0) = x_1^0, \quad x_2(t_0) = x_2^0, \quad \ldots, \quad x_n(t_0) = x_n^0
	\end{equation*}
	досить, щоб:
	\begin{enumerate}
		\item функції $f_1, f_2, \ldots, f_n$ були неперервними по змінним $x_1, x_2, \ldots, x_n, t$ в околі точки $\left(x_1^0, x_2^0, \ldots, x_n^0, t_0\right)$;
		\item функції $f_1, f_2, \ldots, f_n$ задовольняли умові Ліпшиця по аргументах $x_1, x_2, \ldots, x_n$ у тому ж околі.
	\end{enumerate}
\end{theorem}

\begin{remark}
	Умову Ліпшиця можна замінити більш грубою умовою, але такою, що перевіряється легше, існування обмежених частинних похідних, тобто 
	\begin{equation*}
		\left| \frac{\partial f_i}{\partial x_j} \right| \le M, \quad i,j = 1, 2, \ldots, n.
	\end{equation*}
\end{remark}

		\subsubsection{Геометрична інтерпретація розв'язків}
		Назвемо $(n+1)$-вимірний простір змінних $x_1, x_2, \ldots, x_n, t$ розширеним фазовим простором $\RR^{n + 1}$. Тоді розв’язок $x_1 = x_1(t), x_2 = x_2(t), \ldots, x_n = x_n(t)$ визначає в просторі $\RR^{n + 1}$ деяку криву, що називається інтегральною кривою. Загальний розв’язок (чи загальний інтеграл) визначає сім’ю інтегральних кривих, що всюди щільно заповнюють деяку область $D \subseteq \RR^{n + 1}$  (область існування та єдиності розв’язків). Задача Коші ставиться як виділення із сім’ї інтегральних кривих, окремої кривої, що проходить через задану початкову точку $M \left(x_1^0, x_2^0, \ldots, x_n^0, t_0\right) \in D$.

		\subsubsection{Механічна інтерпретація розв'язків}
		В евклідовому просторі $\RR^n$ змінних $x_1(t), x_2(t), \ldots, x_n(t)$ розв’язок $x_1 = x_1(t), x_2 = x_2(t), \ldots, x_n = x_n(t)$ визначає закон руху по деякій траєкторії в залежності від часу $t$. При такій інтерпретації функції $f_1, f_2, \ldots, f_n$ є складовими швидкості руху, простір зміни перемінних називається фазовим простором, система динамічної, а крива, по якій відбувається рух $x_1 = x_1(t), x_2 = x_2(t), \ldots, x_n = x_n(t)$ -- фазовою траєкторією. Фазова траєкторія є проекцією інтегральної кривої на фазовий простір.


		\subsubsection{Зведення одного диференціального рівняння вищого порядку до системи рівнянь першого порядку}
		Нехай маємо диференціальне рівняння
\begin{equation*}
 	\frac{\diff^n y}{\diff x^n} = f \left( x, y, \frac{\diff y}{\diff x}, \frac{\diff^2 y}{\diff x^2}, \ldots, \frac{\diff^{n - 1} y}{\diff x^{n - 1}} \right).
\end{equation*}

Розглянемо заміну змінних
\begin{equation*}
	x \mapsto t, \quad y \mapsto x_1, \quad, \frac{\diff y}{\diff x} \mapsto x_2, \quad \ldots, \quad \frac{\diff^{n - 1} y}{\diff x^{n - 1}} \mapsto x_n.
\end{equation*}

Тоді одержимо систему рівнянь
\begin{equation*}
	\left\{
		\begin{aligned}
			\dot x_1 &= x_2, \\
			\dot x_2 &= x_3, \\
			\ldots & \ldots \ldots \\
			\dot x_{n - 1} &= x_n, \\
			\dot x_n &= f(t, x_1, x_2, \ldots, x_n).
		\end{aligned}
	\right.
\end{equation*}

		\subsubsection{Зведення системи диференціальних рівнянь до одного рівняння вищого порядку}
		Нехай маємо систему диференціальних рівнянь
\begin{equation*}
	\left\{
		\begin{aligned}
			\dot x_1 &= f_1 (x_1, x_2, \ldots, x_n, t), \\
			\dot x_2 &= f_2 (x_1, x_2, \ldots, x_n, t), \\
			\ldots & \ldots \ldots \ldots \ldots \ldots \ldots \ldots, \\
			\dot x_n &= f_n (x_1, x_2, \ldots, x_n, t).
		\end{aligned}
	\right.
\end{equation*}
і заданий її розв'язок $x_1 = x_1(t), x_2 = x_2(t), \ldots, x_n = x_n(t)$. Якщо цей розв'язок підставити в перше рівняння, то вийде тотожність і її можна диференціювати
\begin{equation*}
	\frac{\diff^2 x_1}{\diff t^2} = \frac{\partial f_1}{\partial t} + \sum_{i = 1}^n \frac{\partial f_1}{\partial x_i} \cdot \frac{\diff x_i(t)}{\diff t}.
\end{equation*}

Підставивши замість $\frac{\diff x_i(t)}{\diff t}$ їх значення, одержимо
\begin{equation*}
	\frac{\diff^2 x_1}{\diff t^2} = \frac{\partial f_1}{\partial t} + \sum_{i = 1}^n \frac{\partial f_1}{\partial x_i} \cdot f_i = F_2(t, x_1, x_2, \ldots, x_n).
\end{equation*}

Знову диференціюємо це рівняння й одержимо
\begin{equation*}
	\frac{\diff^3 x_1}{\diff t^3} = \frac{\partial F_2}{\partial t} + \sum_{i = 1}^n \frac{\partial F_2}{\partial x_i} \cdot \frac{\diff x_i(t)}{\diff t} = \frac{\partial F_2}{\partial t} + \sum_{i = 1}^n \frac{\partial F_2}{\partial x_i} \cdot f_i = F_3(t, x_1, x_2, \ldots, x_n).
\end{equation*}

Продовжуючи процес далі, одержимо
\begin{align*}
	\frac{\diff^{n - 1} x_1}{\diff t^{n - 1}} &= F_{n - 1}(t, x_1, x_2, \ldots, x_n), \\
	\frac{\diff^n x_1}{\diff t^n} &= F_n(t, x_1, x_2, \ldots, x_n).
\end{align*} 
 
Таким чином, маємо систему
\begin{equation*}
	\left\{
		\begin{aligned}
			\frac{\diff x_1}{\diff t} &= f_1 (x_1, x_2, \ldots, x_n, t), \\
			\frac{\diff^2 x_1}{\diff t^2} &= F_2(t, x_1, x_2, \ldots, x_n), \\
			\ldots \ldots & \ldots \ldots \ldots \ldots \ldots \ldots \ldots \ldots, \\
			\frac{\diff^{n - 1} x_1}{\diff t^{n - 1}} &= F_{n - 1}(t, x_1, x_2, \ldots, x_n), \\
			\frac{\diff^n x_1}{\diff t^n} &= F_n(t, x_1, x_2, \ldots, x_n).
		\end{aligned}
	\right.
\end{equation*}

Припустимо, що \[\frac{D(f_1, F_2, \ldots, F_{n - 1})}{D(x_2, x_3, \ldots, x_n)} \ne 0.\] Тоді систему перших $(n - 1)$ рівнянь 
\begin{equation*}
	\left\{
		\begin{aligned}
			\frac{\diff x_1}{\diff t} &= f_1 (x_1, x_2, \ldots, x_n, t), \\
			\frac{\diff^2 x_1}{\diff t^2} &= F_2(t, x_1, x_2, \ldots, x_n), \\
			\ldots \ldots & \ldots \ldots \ldots \ldots \ldots \ldots \ldots \ldots, \\
			\frac{\diff^{n - 1} x_1}{\diff t^{n - 1}} &= F_{n - 1}(t, x_1, x_2, \ldots, x_n).
		\end{aligned}
	\right.
\end{equation*}
можна розв'язати відносно останніх $(n - 1)$ змінних $x_2, x_3, \ldots, x_n$ і одержати
\begin{equation*}
	\left\{
		\begin{aligned}
			x_2 &= \phi_2 \left( t, x_1, \frac{\diff x_1}{\diff t}, \ldots, \frac{\diff^{n - 1} x_1}{\diff t^{n - 1}} \right), \\
			x_3 &= \phi_3 \left( t, x_1, \frac{\diff x_1}{\diff t}, \ldots, \frac{\diff^{n - 1} x_1}{\diff t^{n - 1}} \right), \\
			\ldots & \ldots \ldots \ldots \ldots \ldots \ldots \ldots \ldots \ldots \ldots, \\
			x_n &= \phi_n \left( t, x_1, \frac{\diff x_1}{\diff t}, \ldots, \frac{\diff^{n - 1} x_1}{\diff t^{n - 1}} \right), \\
		\end{aligned}
	\right.
\end{equation*}

Підставивши одержані вирази в останнє рівняння, запишемо
\begin{multline*}
	\frac{\diff^n x_1}{\diff t^n} = F_n \left( t, x_1, \phi_2 \left( t, x_1, \frac{\diff x_1}{\diff t}, \ldots, \frac{\diff^{n - 1} x_1}{\diff t^{n - 1}} \right), \ldots, \right. \\ \left. \phi_n \left( t, x_1, \frac{\diff x_1}{\diff t}, \ldots, \frac{\diff^{n - 1} x_1}{\diff t^{n - 1}} \right) \right).	
\end{multline*}

Або, після перетворень
\begin{equation*}
	\frac{\diff^n x_1}{\diff t^n} = \Phi \left( t, x_1, \frac{\diff x_1}{\diff t}, \ldots, \frac{\diff^{n - 1} x_1}{\diff t^{n - 1}} \right),
\end{equation*}
одержимо одне диференціальне рівняння $n$-го порядку. \parvskip

У загальному випадку, одержимо, що система диференціальних рівнянь першого порядку
\begin{equation*}
	\left\{
		\begin{aligned}
			\dot x_1 &= f_1 (x_1, x_2, \ldots, x_n, t), \\
			\dot x_2 &= f_2 (x_1, x_2, \ldots, x_n, t), \\
			\ldots & \ldots \ldots \ldots \ldots \ldots \ldots \ldots, \\
			\dot x_n &= f_n (x_1, x_2, \ldots, x_n, t).
		\end{aligned}
	\right.
\end{equation*}
зводиться до одного рівняння $n$-го порядку
\begin{equation*}
	\frac{\diff^n x_1}{\diff t^n} = \Phi \left( t, x_1, \frac{\diff x_1}{\diff t}, \ldots, \frac{\diff^{n - 1} x_1}{\diff t^{n - 1}} \right),
\end{equation*}
і системи $(n - 1)$ рівнянь зв'язку
\begin{equation*}
	\left\{
		\begin{aligned}
			x_2 &= \phi_2 \left( t, x_1, \frac{\diff x_1}{\diff t}, \ldots, \frac{\diff^{n - 1} x_1}{\diff t^{n - 1}} \right), \\
			x_3 &= \phi_3 \left( t, x_1, \frac{\diff x_1}{\diff t}, \ldots, \frac{\diff^{n - 1} x_1}{\diff t^{n - 1}} \right), \\
			\ldots & \ldots \ldots \ldots \ldots \ldots \ldots \ldots \ldots \ldots \ldots, \\
			x_n &= \phi_n \left( t, x_1, \frac{\diff x_1}{\diff t}, \ldots, \frac{\diff^{n - 1} x_1}{\diff t^{n - 1}} \right), \\
		\end{aligned}
	\right.
\end{equation*}
 
\begin{remark}
	Було зроблене припущення, що \[\frac{D(f_1, F_2, \ldots, F_{n - 1})}{D(x_2, x_3, \ldots, x_n)} \ne 0.\] Якщо ця умова не виконана, то можна зводити до рівняння щодо інших змінних, наприклад відносно $x_2$.
\end{remark}


		\subsubsection{Комбінації, що інтегруються}
		\begin{definition}
	Комбінацією, що інтегрується, називається диференціальне рівняння, отримане шляхом перетворень із системи, диференціальних рівнянь, але яке вже можна легко інтегрувати.
\end{definition}
\begin{equation*}
	\diff \Phi(t, x_1, x_2, \ldots, x_n) = 0.
\end{equation*}

Одна комбінація, що інтегрується, дає можливість одержати одне кінцеве рівняння
\begin{equation*}
	\Phi(t, x_1, x_2, \ldots, x_n) = C,
\end{equation*}
яке є першим інтегралом системи. \\

Геометрично перший інтеграл являє собою $n$-вимірну поверхню в $(n + 1)$-вимірному просторі, що цілком складається з інтегральних кривих. \\

Якщо знайдено $k$ комбінацій, що інтегруються, то одержуємо $k$ перших інтегралів
\begin{equation*}
	\left\{
		\begin{aligned}
			\Phi_1(t, x_1, x_2, \ldots, x_n) &= C_1, \\
			\Phi_2(t, x_1, x_2, \ldots, x_n) &= C_2, \\
			\ldots \ldots \ldots \ldots \ldots \ldots \ldots & \ldots \ldots, \\
			\Phi_n(t, x_1, x_2, \ldots, x_n) &= C_n.
		\end{aligned}
	\right.
\end{equation*}
 
І, якщо інтеграли незалежні, то хоча б один з визначників \[\frac{D(\Phi_1, \Phi_2, \ldots, \Phi_k)}{D(x_{i_1}, x_{i_2}, \ldots, x_{i_k})} \ne 0.\] Звідси з системи можна виразити $k$ невідомих функцій $x_{i_1}, x_{i_2}, \ldots, x_{i_k}$ через інші і підставивши їх у вихідну систему, понизити порядок до $(n - k)$ рівнянь. Якщо $n = k$ і всі інтеграли незалежні, то одержимо загальний інтеграл системи. \\

Особливо поширеним засобом знаходження комбінацій, що інтегруються, є використання систем у симетричному вигляді. \\

Систему диференціальних рівнянь, що записана в нормальній формі
\begin{equation*}
	\left\{
		\begin{aligned}
			\dot x_1 &= f_1 (x_1, x_2, \ldots, x_n, t), \\
			\dot x_2 &= f_2 (x_1, x_2, \ldots, x_n, t), \\
			\ldots & \ldots \ldots \ldots \ldots \ldots \ldots \ldots, \\
			\dot x_n &= f_n (x_1, x_2, \ldots, x_n, t).
		\end{aligned}
	\right.
\end{equation*}
можна переписати у вигляді
\begin{equation*}
	\frac{\diff x_1}{f_1(x_1, x_2, \ldots, x_n, t)} = \frac{\diff x_2}{f_2 (x_1, x_2, \ldots, x_n, t)} = \ldots = \frac{\diff x_n}{f_n (x_1, x_2, \ldots, x_n, t)} = \frac{\diff t}{1}.
\end{equation*}

При такій формі запису всі змінні $x_1, x_2, \ldots, x_n, t$ рівнозначні. \\

Система диференціальних рівнянь, що записана у вигляді
\begin{equation*}
	\frac{\diff x_1}{X_1(x_1, x_2, \ldots, x_n)} = \frac{\diff x_2}{X_2 (x_1, x_2, \ldots, x_n)} = \ldots = \frac{\diff x_n}{X_n (x_1, x_2, \ldots, x_n)}.
\end{equation*}
називається системою у симетричному вигляді. \\

При знаходженні комбінацій, що інтегруються, найбільш часто використовується властивість ``пропорційності''. А саме, для систем в симетричному вигляді справедлива рівність
\begin{multline*}
	\frac{\diff x_1}{X_1(x_1, x_2, \ldots, x_n)} = \frac{\diff x_2}{X_2 (x_1, x_2, \ldots, x_n)} = \ldots = \frac{\diff x_n}{X_n (x_1, x_2, \ldots, x_n)} = \\
	= \frac{k_1 \cdot \diff x_1 + k_2 \cdot \diff x_2 + \ldots k_n \cdot \diff x_n}{(k_1 \cdot X_1  + k_2 \cdot X_2  + \ldots + k_n \cdot X_n) (x_1, x_2, \ldots, x_n)}.
\end{multline*}


		\subsubsection{Вправи для самостійної роботи}
		\begin{example}
	Розв'язати систему диференціальних рівнянь зведенням до одного рівняння вищого порядку: \[ \frac{\diff y}{\diff x} = \frac{x}{z}, \quad \frac{\diff z}{\diff x} = -\frac{x}{y}.\]
\end{example}

\begin{solution}
	Диференціюємо перше рівняння по змінній $x$:
	\[ y'' = \frac{1}{z} - \frac{x}{z^2} \cdot z' = \frac{1}{z} + \frac{x}{z^2} \cdot \frac{x}{y}. \]
	
	Таким чином, одержали допоміжну систему:
	\[ y' = \frac{x}{z}, \quad y'' = \frac{1}{z} + \frac{x}{z^2} \cdot \frac{x}{y}. \]

	З першого рівняння отримаємо $z = \frac{x}{y'}$. Підставляємо одержане значення в другу систему \[ y'' = \frac{y'}{x} + \frac{(y')^2}{y}.\]

	Маємо однорідне (по $y, y', y''$) диференціальне рівняння другого порядку. Робимо заміну \begin{align*} y &= \exp\left\{\int u \diff x\right\}, \\ y' &= u \cdot \exp\left\{\int u \diff x\right\}, \\ y'' &= (u^2 + u') \cdot \exp\left\{\int u \diff x\right\}. \end{align*}

	Після підстановки та скорочення на $e^{\int u \diff x}$ одержуємо \[ u^2 + u' = \frac{u}{x} + u^2, \] або \[ \frac{\diff u}{\diff x} = \frac{u}{x}. \]

	Далі \[ \frac{\diff u}{u} = \frac{\diff x}{x} \implies u = \frac{C_1 \cdot x}{2}. \]

	Звідси \[ y = \exp\left\{\int \frac{C_1 \cdot x}{2} \diff x\right\} = e^{C_1 x^2 + \ln C_2}, \] або \[ y_2 = C_2 \cdot e^{C_1 x^2}. \]

	Змінна $z$ знаходиться з умови $z = \frac{x}{y'}$, або \[ z = \frac{x}{2 \cdot C_2 \cdot C_1 \cdot x \cdot e^{C_1 x^2}} = \frac{1}{2 \cdot C_1 \cdot C_2} \cdot e^{- C_1 x^2}. \]
\end{solution}

\begin{example}
	Розв'язати систему в симетричному вигляді за допомогою інтегрованих комбінацій \[\frac{\diff x}{y} = \frac{\diff y}{x} = \frac{\diff z}{z}.\]
\end{example}

\begin{solution}
	Використовуючи властивості ``пропорційності'', маємо \[ \frac{\diff x}{y} = \frac{\diff y}{x} = \frac{\diff z}{z} = \frac{\diff (x - y)}{-(x - y)} = \frac{\diff (x + y + z)}{(x + y + z)}. \]

	\begin{enumerate}
		\item Візьмемо \[\frac{\diff z}{z} = \frac{\diff (x - y)}{-(x - y)}.\] Звідси \[\ln |z| + \ln |x - y| = \ln C_1,\] і перший інтеграл має вигляд $z \cdot (x - y) = C_1$.
		\item Візьмемо \[\frac{\diff z}{z} = \frac{\diff (x + y + z)}{(x + y + z)}.\] Звідси \[\ln |z| + \ln |x + y + z| = \ln C_2,\] і ще один інтеграл має вигляд $\frac{x + y + z}{z} = C_2$.
	\end{enumerate}

	Умовою функціональної незалежності одержаних інтегралів є \[\frac{D(C_1, C_2)}{D(x, y)} \ne 0.\] Перевіряємо: \[ \frac{D(C_1, C_2)}{D(x, y)} = \begin{vmatrix} \frac{\partial C_1}{\partial x} & \frac{\partial C_1}{\partial y} \\ \frac{\partial C_2}{\partial x} & \frac{\partial C_2}{\partial y} \end{vmatrix} = \begin{vmatrix} z & -z \\ \frac{1}{z} & \frac{1}{z} \end{vmatrix} = 2 \ne 0. \]
	
	Таким чином \[ z \cdot (x - y) = C_1, \quad \frac{x + y + z}{z} = C_2 \] є загальним інтегралом системи.
\end{solution}

Розв'язати системи диференціальних рівнянь, зведенням до одного рівняння вищого порядку

\begin{multicols}{2}
\begin{problem}
	\[ \frac{\diff y}{\diff x} = \frac{y^2}{z - x}, \quad \frac{\diff z}{\diff x} = y + 1; \]
\end{problem}
\begin{problem}
	\[ \frac{\diff y}{\diff x} = \frac{z}{x}, \quad \frac{\diff z}{\diff x} = \frac{z (y + 2 z - 1)}{x (y - 1)}; \]
\end{problem}
\begin{problem}
	\[ \frac{\diff y}{\diff x} = y^2 \cdot z, \quad \frac{\diff z}{\diff x} = \frac{z}{x} - y \cdot z^2; \]
\end{problem}
\begin{problem}
	\[ \frac{\diff y}{\diff x} = \frac{y^2 - z^2 + 1}{2 z}, \quad \frac{\diff z}{\diff x} = z + y. \]
\end{problem}
\end{multicols}

Розв’язати системи диференціальних рівнянь за допомогою інтегрованих комбінацій.

\begin{multicols}{2}
\begin{problem}
	\[ \frac{\diff x}{y + z} = \frac{\diff y}{x + z} = \frac{\diff z}{x + y}; \]
\end{problem}
\begin{problem}
	\[ \frac{\diff x}{y - x} = \frac{\diff y}{x + y + z} = \frac{\diff z}{x - y}; \]
\end{problem}
\begin{problem}
	\[ \frac{\diff x}{z} = \frac{\diff y}{x \cdot z} = \frac{\diff z}{y}; \]
\end{problem}
\begin{problem}
	\[ \frac{\diff x}{z^2 - y^2} = \frac{\diff y}{z} = \frac{\diff z}{- y}; \]
\end{problem}
\begin{problem}
	\[ \frac{\diff x}{x} = \frac{\diff y}{y} = \frac{\diff z}{x \cdot y + z}; \]
\end{problem}
\begin{problem}
	\[ \frac{\diff x}{x + y^2 + z^2} = \frac{\diff y}{y} = \frac{\diff z}{z}; \]
\end{problem}
\begin{problem}
	\[ \frac{\diff x}{x (y + z)} = \frac{\diff y}{z (z - y)} = \frac{\diff z}{y  (y - z)}; \]
\end{problem}
\begin{problem}
	\[ \frac{\diff x}{-x^2} = \frac{\diff y}{x \cdot y - 2 z^2} = \frac{\diff z}{x \cdot z}; \]
\end{problem}
\begin{problem}
	\[ \frac{\diff x}{x (z - y)} = \frac{\diff y}{y (y - x)} = \frac{\diff z}{y^2 - x \cdot z}; \]
\end{problem}
\begin{problem}
	\[ \frac{\diff x}{x (y^2 - z^2)} = \frac{\diff y}{- y (z^2 + x^2)} = \frac{\diff z}{z (x^2 + y^2)}. \]
\end{problem}
\end{multicols}

	\subsection{Системи лінійних диференціальних рівнянь. Загальні положення}
	Система диференціальних рівнянь, що записана у вигляді
\begin{equation*}
	\left\{
		\begin{array}{rl}
			\dot x_1 &= a_{11}(t) \cdot x_1 + a_{12}(t) \cdot x_2 + \ldots + a_{1n}(t) \cdot x_n + f_1(t), \\
			\dot x_2 &= a_{21}(t) \cdot x_1 + a_{22}(t) \cdot x_2 + \ldots + a_{2n}(t) \cdot x_n + f_2(t), \\
			\hdotsfor{2}, \\
			\dot x_n &= a_{n1}(t) \cdot x_1 + a_{n2}(t) \cdot x_2 + \ldots + a_{nn}(t) \cdot x_n + f_n(t),
		\end{array}
	\right.
\end{equation*}
називається лінійною неоднорідною системою диференціальних рівнянь. Система 
\begin{equation*}
	\left\{
		\begin{array}{rl}
			\dot x_1 &= a_{11}(t) \cdot x_1 + a_{12}(t) \cdot x_2 + \ldots + a_{1n}(t) \cdot x_n, \\
			\dot x_2 &= a_{21}(t) \cdot x_1 + a_{22}(t) \cdot x_2 + \ldots + a_{2n}(t) \cdot x_n, \\
			\hdotsfor{2}, \\
			\dot x_n &= a_{n1}(t) \cdot x_1 + a_{n2}(t) \cdot x_2 + \ldots + a_{nn}(t) \cdot x_n,
		\end{array}
	\right.
\end{equation*}
називається лінійною однорідною системою диференціальних рівнянь. Якщо ввести векторні позначення
\begin{equation*}
	x = \begin{pmatrix} x_1 \\ x_2 \\ \vdots \\ x_n \end{pmatrix}, \quad 
	f(t) = \begin{pmatrix} f_1(t) \\ f_2(t) \\ \vdots \\ f_n(t) \end{pmatrix}, \quad
	A(t) = \begin{pmatrix} a_{11}(t) & a_{12}(t) & \ldots & a_{1n}(t) \\ a_{21}(t) & a_{22}(t) & \ldots & a_{2n}(t) \\ \vdots & \vdots & \ddots & \vdots \\ a_{n1}(t) & a_{n2}(t) & \ldots & a_{nn}(t) \end{pmatrix},
\end{equation*}
то лінійну неоднорідну систему можна переписати у вигляді
\begin{equation*}
	\dot x = A(t) \cdot x + f(t),
\end{equation*}
а лінійну однорідну систему у вигляді
\begin{equation*}
	\dot x = A(t) \cdot x.
\end{equation*}

Якщо функції $a_{ij}(t)$, $f_i(t)$, $i,j=\overline{1,n}$ неперервні в околі точки \[(x_0, t_0) = (x_1^0, x_2^0, \ldots, x_n^0, t_0),\] то виконані умови теореми існування та єдиності розв'язку задачі Коші, і існує єдиний розв'язок
\begin{equation*}
	x_1 = x_1(t), \quad x_2 = x_2(t), \quad \ldots, \quad x_n = x_n(t),
\end{equation*}
системи рівнянь, що задовольняє початковим даним
 \begin{equation*}
	x_1(t_0) = x_1^0, \quad x_2(t_0) = x_2^0, \quad \ldots, \quad x_n(t_0) = x_n^0.
\end{equation*}


		\subsubsection{Властивості розв'язків лінійних однорідних систем}
		\setcounter{property}{0}
\begin{property}
	Якщо вектор $x(t) = (x_1(t), x_2(t), \ldots, x_n(t))^T$ є розв'язком лінійної однорідної системи, то і \[ C \cdot x(t) = (C \cdot x_1(t), C \cdot x_2(t), \ldots, C \cdot x_n(t))^T,\] де $C$ --- стала скалярна величина, також є розв'язком цієї системи.
\end{property}

\begin{proof}
	Дійсно, за умовою 
	\begin{equation*}
		\dot x(t) - A(t) \cdot x(t) \equiv 0.
	\end{equation*}

	Але тоді і
	\begin{multline*}
		\frac{\diff}{\diff t} (C \cdot x(t)) - A(t) \cdot (C \cdot x(t)) = \\ = C \cdot (\dot x(t) - A(t) \cdot x(t)) \equiv 0
	\end{multline*}
	оскільки дорівнює нулю вираз в дужках. Тобто $C \cdot x(t)$ є розв'язком однорідної системи.
\end{proof}

\begin{property}
	Якщо дві векторні функції $x_1 = (x_{11}(t), x_{21}(t), \ldots, x_{n1}(t))^T$, $x_2 = (x_{12}(t), x_{22}(t), \ldots, x_{n2}(t))^T$ є розв'язками однорідної системи, то і їхня сума також буде розв’язком однорідної системи.
\end{property}

\begin{proof}
	Дійсно, за умовою
	\begin{align*}
		\dot x_1(t) - A(t) \cdot x_1(t) &\equiv 0, \\
		\dot x_2(t) - A(t) \cdot x_2(t) &\equiv 0.
	\end{align*}

	Але тоді і
	\begin{equation*}
		\frac{\diff}{\diff t} (x_1(t) + x_2(t)) - A(t) \cdot (x_1(t) + x_2(t)) = (\dot x_1(t) - A(t) \cdot x_1(t)) + (\dot x_2(t) - A(t) \cdot x_2(t)) \equiv 0
	\end{equation*}
	тому що дорівнюють нулю вирази в дужках, тобто $x_1(t) + x_2(t)$ є розв'язком однорідної системи.
\end{proof}

\begin{property}
	Якщо вектори $x_1 = (x_{11}(t), x_{21}(t), \ldots, x_{n1}(t))^T$, $\ldots$, $x_n = (x_{1n}(t), x_{2n}(t), \ldots, x_{nn}(t))^T$ є розв'язками однорідної системи, та і їхня лінійна комбінація з довільними коефіцієнтами також буде розв'язком однорідної системи. 
\end{property}

\begin{proof}
	Дійсно, за умовою
	\begin{equation*}
		\dot x_i(t) - A(t) \cdot x_i(t) \equiv 0, \quad i = \overline{1, n}.
	\end{equation*}
	
	Але тоді і
	\begin{multline*}
		\frac{\diff}{\diff t} \left( \sum_{i=1}^n C_i \cdot x_i(t) \right) - A(t) \cdot \left( \sum_{i = 1}^n C_i \cdot x_i(t) \right) = \\ = \sum_{i = 1}^n C_i \cdot \left( \dot x_i(t) - A(t) \cdot x_i(t) \right) \equiv 0
	\end{multline*}
	тому що дорівнює нулю кожний з доданків, тобто $\sum_{i=1}^n C_i \cdot x_i(t)$ є роз\-в'яз\-ком однорідної системи.
\end{proof}

\begin{property}
	Якщо комплексний вектор з дійсними елементами $u(t) + i \cdot v(t) = (u_1(t), \ldots, u_n(t))^T + i \cdot (v_1(t), \ldots, v_n(t))^T$ є розв’язком однорідної системи, то окремо дійсна та уявна частини є розв'язками системи.
\end{property}

\begin{proof}
	Дійсно за умовою
	\begin{equation*}
		\frac{\diff}{\diff t} (u(t) + i \cdot v(t)) - A(t) \cdot (u(t) + i \cdot v(t)) \equiv 0.
	\end{equation*}

	Розкривши дужки і зробивши перетворення, одержимо
	\begin{equation*}
		(\dot u(t) - A(t) \cdot u(t)) + i \cdot (\dot v(t) - A(t) \cdot v(t)) \equiv 0.
	\end{equation*}
	 
	А комплексний вираз дорівнює нулю тоді і тільки тоді, коли дорівнюють нулю дійсна і уявна частини, тобто
	\begin{align*}
		\dot u(t) - A(t) \cdot u(t) &\equiv 0, \\
		\dot v(t) - A(t) \cdot v(t) &\equiv 0.
	\end{align*}
	що і було потрібно довести.
\end{proof}

\begin{definition}
	Вектори \[ x_1 = \begin{pmatrix} x_{11}(t) \\ x_{21}(t) \\ \vdots \\ x_{n1}(t) \end{pmatrix}, \quad x_2 = \begin{pmatrix} x_{12}(t) \\ x_{22}(t) \\ \vdots \\ x_{n2}(t) \end{pmatrix}, \quad \ldots, \quad x_n = \begin{pmatrix} x_{1n}(t) \\ x_{2n}(t) \\ \vdots \\ x_{nn}(t) \end{pmatrix} \] називаються лінійно залежними на відрізку $t \in [a, b]$, якщо існують не всі рівні нулю сталі $C_1, C_2, \ldots, C_n$, такі, що 
	\begin{equation*}
		C_1 \cdot x_1(t) + C_2 \cdot x_2(t) + \ldots + C_n \cdot x_n(t) \equiv 0
	\end{equation*}
	при $t \in [a, b]$. \\

	Якщо тотожність справедлива лише при $C_i = 0$, $i = \overline{1, n}$, то вектори лінійно незалежні.
\end{definition}

\begin{definition}
	Визначник, що складається з векторів $x_1(t), x_2(t), \ldots, x_n(t)$, тобто
	\begin{equation*}
		W[x_1, x_2, \ldots, x_n](t) = \begin{vmatrix} x_{11}(t) & x_{12}(t) & \ldots & x_{1n}(t) \\ x_{21}(t) & x_{22}(t) & \ldots & x_{2n}(t) \\ \vdots & \vdots & \ddots & \vdots \\ x_{n1}(t) & x_{n2}(t) & \ldots & x_{nn}(t) \end{vmatrix}.
	\end{equation*}
	називається визначником Вронського.
\end{definition}

\begin{theorem}
	Якщо векторні функції $x_1(t), x_2(t), \ldots, x_n(t)$ лінійно залежні, то визначник Вронського тотожно дорівнює нулю.
\end{theorem}

\begin{proof}
	За умовою існують не всі рівні нулю $C_1, C_2, \ldots, C_n$, такі, що $C_1 \cdot x_1(t) + C_2 \cdot x_2(t) + \ldots + C_n \cdot x_n(t) \equiv 0$ при $t \in [a, b]$. \\

	Або, розписавши покоординатно, одержимо
	\begin{equation*}
		\left\{
			\begin{array}{rl}
				C_1 \cdot x_{11}(t) + C_2 \cdot x_{12}(t) + \ldots + C_n \cdot x_{1n}(t) &\equiv 0, \\
				C_1 \cdot x_{21}(t) + C_2 \cdot x_{22}(t) + \ldots + C_n \cdot x_{2n}(t) &\equiv 0, \\
				\hdotsfor{2}, \\
				C_1 \cdot x_{n1}(t) + C_2 \cdot x_{n2}(t) + \ldots + C_n \cdot x_{nn}(t) &\equiv 0.
			\end{array}
		\right.
	\end{equation*}

	А однорідна система має ненульовий розв'язок $C_1, C_2, \ldots, C_n$ тоді і тільки тоді, коли визначник дорівнює нулю, тобто
	\begin{equation*}
		W[x_1, x_2, \ldots, x_n](t) = \begin{vmatrix} x_{11}(t) & x_{12}(t) & \ldots & x_{1n}(t) \\ x_{21}(t) & x_{22}(t) & \ldots & x_{2n}(t) \\ \vdots & \vdots & \ddots & \vdots \\ x_{n1}(t) & x_{n2}(t) & \ldots & x_{nn}(t) \end{vmatrix} \equiv 0, \quad t \in [a, b].
	\end{equation*}
\end{proof}

\begin{theorem}
	Якщо розв'язки $x_1(t), x_2(t), \ldots, x_n(t)$ лінійної однорідної системи лінійно незалежні, то визначник Вронського не дорівнює нулю в жодній точці $t \in [a, b]$. 
\end{theorem}

\begin{proof}
	Нехай, від супротивного, існує точка $t_0 \in [a, b]$ і \[W[x_1, x_2, \ldots, x_n](t_0) = 0.\]

	Тоді виконується система однорідних алгебраїчних рівнянь \[C_1 \cdot x_1(t_0) + C_2 \cdot x_2(t_0) + \ldots + C_n \cdot x_n(t_0) = 0. \]

	Або, розписавши покоординатно, одержимо
	\begin{equation*}
		\left\{
			\begin{array}{rl}
				C_1 \cdot x_{11}(t_0) + C_2 \cdot x_{12}(t_0) + \ldots + C_n \cdot x_{1n}(t_0) &= 0, \\
				C_1 \cdot x_{21}(t_0) + C_2 \cdot x_{22}(t_0) + \ldots + C_n \cdot x_{2n}(t_0) &= 0, \\
				\hdotsfor{2}, \\
				C_1 \cdot x_{n1}(t_0) + C_2 \cdot x_{n2}(t_0) + \ldots + C_n \cdot x_{nn}(t_0) &= 0.
			\end{array}
		\right.
	\end{equation*}
 	має ненульовий розв'язок $C_1^0, C_2^0, \ldots, C_n^0$. Розглянемо лінійну комбінацію розв'язків з отриманими коефіцієнтами
 	\begin{equation*}
 		x(t) = C_1^0 \cdot x_1(t) + C_2^0 \cdot x_2(t) + \ldots + C_n^0 \cdot x_n(t).
 	\end{equation*}

	Відповідно до властивості 4, ця комбінація буде розв'язком. Крім того, як випливає із системи алгебраїчних рівнянь, для отриманих $C_1^0, C_2^0, \ldots, C_n^0$: $x(t_0) = 0$, $t_0 \in [a, b]$. Але розв'язком, що задовольняють таким умовам, є $x \equiv 0$. І в силу теореми існування та єдиності ці два розв’язки збігаються, тобто $x(t) \equiv 0$ при $t \in [a, b]$, або 
 	\begin{equation*}
 		C_1^0 \cdot x_1(t) + C_2^0 \cdot x_2(t) + \ldots + C_n^0 \cdot x_n(t) \equiv 0,
 	\end{equation*}
	або розв'язки $x_1(t), x_2(t), \ldots, x_n(t)$ лінійно залежні, що суперечить умові теореми.  \\

	Таким чином, $W[x_1, x_2, \ldots, x_n](t) \ne 0$ у жодній точці $t \in [a, b]$, що і було потрібно довести.
\end{proof}

\begin{theorem}
	Для того щоб розв'язки $x_1(t), \ldots, x_n(t)$ були лінійно незалежні, необхідно і достатно, щоб $W[x_1, \ldots, x_n](t) \ne 0$ у жодній точці $t \in [a, b]$.
\end{theorem}

\begin{proof}
	Випливає з попередніх двох теорем.
\end{proof}

\begin{theorem}
	Загальний розв'язок лінійної однорідної системи представляється у вигляді лінійної комбінації $n$ лінійно незалежних роз\-в'яз\-ків.
\end{theorem}

\begin{proof}
	Як випливає з властивості 3, лінійна комбінація розв'язків також буде розв'язком. Покажемо, що цей розв'язок загальний, тобто завдяки вибору коефіцієнтів $C_1, \ldots, C_n$ можна розв'язати будь-яку задачу Коші $x(t_0) = x_0$ або в координатній формі:
	\begin{equation*}
		x_1(t_0) = x_1^0, \quad x_2(t_0) = x_2^0, \quad \ldots, \quad x_n(t_0) = x_n^0.
	\end{equation*}

	Оскільки розв'язки $x_1(t), \ldots, x_n(t)$  лінійно незалежні, то визначник Вронського відмінний від нуля. Отже, система алгебраїчних рівнянь
	\begin{equation*}
		\left\{
			\begin{array}{rl}
				C_1 \cdot x_{11}(t_0) + C_2 \cdot x_{12}(t_0) + \ldots + C_n \cdot x_{1n}(t_0) &= x_1^0, \\
				C_1 \cdot x_{21}(t_0) + C_2 \cdot x_{22}(t_0) + \ldots + C_n \cdot x_{2n}(t_0) &= x_2^0, \\
				\hdotsfor{2}, \\
				C_1 \cdot x_{n1}(t_0) + C_2 \cdot x_{n2}(t_0) + \ldots + C_n \cdot x_{nn}(t_0) &= x_n^0.
			\end{array}
		\right.
	\end{equation*}
	має єдиний розв'язок $C_1^0, C_2^0, \ldots, C_n^0$. \\

	Тоді лінійна комбінація
	\begin{equation*}
		x(t) = C_1^0 \cdot x_1(t) + C_2^0 \cdot x_2(t) + \ldots + C_n^0 \cdot x_n(t)
	\end{equation*}
	є розв'язком поставленої задачі Коші. Теорема доведена.
\end{proof}

\begin{remark}
	Максимальне число незалежних розв'язків дорівнює кількості рівнянь $n$.
\end{remark}

\begin{proof}
	Це випливає з теореми про загальний розв'язок системи однорідних рівнянь, тому що будь-який інший розв'язок може бути представлений у вигляді лінійної комбінації $n$ лінійно незалежних розв'язків.
\end{proof}

\begin{definition}
	Матриця, складена з будь-яких $n$ лінійно незалежних роз\-в'яз\-ків, називається фундаментальною матрицею розв'язків системи.
\end{definition}

Якщо лінійно незалежними розв'язками будуть \[ x_1 = \begin{pmatrix} x_{11}(t) \\ x_{21}(t) \\ \vdots \\ x_{n1}(t) \end{pmatrix}, \quad x_2 = \begin{pmatrix} x_{12}(t) \\ x_{22}(t) \\ \vdots \\ x_{n2}(t) \end{pmatrix}, \quad \ldots, \quad x_n = \begin{pmatrix} x_{1n}(t) \\ x_{2n}(t) \\ \vdots \\ x_{nn}(t) \end{pmatrix} \] то матриця
\begin{equation*}
	X(t) = \begin{pmatrix} x_{11} (t) & x_{12} (t) & \ldots & x_{1n} (t) \\ x_{21} (t) & x_{22} (t) & \ldots & x_{2n} (t) \\ \vdots & \vdots & \ddots & \vdots \\ x_{n1} (t) & x_{n2} (t) & \ldots & x_{nn} (t) \end{pmatrix}
\end{equation*}
буде фундаментальною матрицею розв'язків. \\

Як випливає з попередньої теореми загальний розв'язок може бути представлений у вигляді
\begin{equation*}
	x_{\text{homo}} = \sum_{i = 1}^n C_i \cdot x_i(t),
\end{equation*}
де $C_i$ --- довільні сталі. Якщо ввести вектор $C = (C_1, C_2, \ldots, C_n)^T$, то загальний розв'язок можна записати у вигляді $x_{\text{homo}} = X(t) \cdot C$.


		\subsubsection{Формула Якобі}
		Нехай $x_1(t), \ldots, x_n(t)$ --- лінійно незалежні розв'язки однорідної системи, $W[x_1, \ldots, x_n]$ --- визначник Вронського. Обчислимо похідну визначника Вронського
\begin{multline*}
	\frac{\diff}{\diff t} W[x_1, \ldots, x_n] = \frac{\diff}{\diff t} \begin{vmatrix} x_{11}(t) & x_{12}(t) & \ldots & x_{1n}(t) \\ x_{21}(t) & x_{22}(t) & \ldots & x_{2n}(t) \\ \vdots & \vdots & \ddots & \vdots \\ x_{n1}(t) & x_{n2}(t) & \ldots & x_{nn}(t) \end{vmatrix} = \\
	= \begin{vmatrix} x_{11}'(t) & x_{12}(t) & \ldots & x_{1n}(t) \\ x_{21}'(t) & x_{22}(t) & \ldots & x_{2n}(t) \\ \vdots & \vdots & \ddots & \vdots \\ x_{n1}'(t) & x_{n2}(t) & \ldots & x_{nn}(t) \end{vmatrix} + \begin{vmatrix} x_{11}(t) & x_{12}'(t) & \ldots & x_{1n}(t) \\ x_{21}(t) & x_{22}'(t) & \ldots & x_{2n}(t) \\ \vdots & \vdots & \ddots & \vdots \\ x_{n1}(t) & x_{n2}'(t) & \ldots & x_{nn}(t) \end{vmatrix} + \ldots \\
	\ldots + \begin{vmatrix} x_{11}(t) & x_{12}(t) & \ldots & x_{1n}'(t) \\ x_{21}(t) & x_{22}(t) & \ldots & x_{2n}'(t) \\ \vdots & \vdots & \ddots & \vdots \\ x_{n1}(t) & x_{n2}(t) & \ldots & x_{nn}'(t) \end{vmatrix}.
\end{multline*}

Оскільки для похідних виконується співвідношення
\begin{equation*}
\begin{array}{rl}
	& \left\{
		\begin{array}{rl}
			x_{11}'(t) &= \alpha_{11} \cdot x_{11} (t) + \alpha_{12} \cdot x_{12}(t) + \ldots + \alpha_{1n} \cdot x_{1n}(t), \\
			x_{12}'(t) &= \alpha_{11} \cdot x_{21} (t) + \alpha_{12} \cdot x_{22}(t) + \ldots + \alpha_{1n} \cdot x_{2n}(t), \\
			\hdotsfor{2}, \\
			x_{1n}'(t) &= \alpha_{11} \cdot x_{n1} (t) + \alpha_{12} \cdot x_{n2}(t) + \ldots + \alpha_{1n} \cdot x_{nn}(t),
		\end{array}
	\right. \\
	& \left\{
		\begin{array}{rl}
			x_{21}'(t) &= \alpha_{21} \cdot x_{11} (t) + \alpha_{22} \cdot x_{12}(t) + \ldots + \alpha_{2n} \cdot x_{1n}(t), \\
			x_{22}'(t) &= \alpha_{21} \cdot x_{21} (t) + \alpha_{22} \cdot x_{22}(t) + \ldots + \alpha_{2n} \cdot x_{2n}(t), \\
			\hdotsfor{2}, \\
			x_{2n}'(t) &= \alpha_{21} \cdot x_{n1} (t) + \alpha_{22} \cdot x_{n2}(t) + \ldots + \alpha_{2n} \cdot x_{nn}(t),
		\end{array}
	\right. \\
	\hdotsfor{2}, \\
	& \left\{
		\begin{array}{rl}
			x_{n1}'(t) &= \alpha_{n1} \cdot x_{11} (t) + \alpha_{n2} \cdot x_{12}(t) + \ldots + \alpha_{nn} \cdot x_{1n}(t), \\
			x_{2n}'(t) &= \alpha_{n1} \cdot x_{21} (t) + \alpha_{n2} \cdot x_{22}(t) + \ldots + \alpha_{nn} \cdot x_{2n}(t), \\
			\hdotsfor{2}, \\
			x_{nn}'(t) &= \alpha_{n1} \cdot x_{n1} (t) + \alpha_{n2} \cdot x_{n2}(t) + \ldots + \alpha_{nn} \cdot x_{nn}(t),
		\end{array}
	\right.
\end{array}
\end{equation*}
то після підстановки одержимо
\begin{multline*}
	\frac{\diff}{\diff t} W[x_1, \ldots, x_n] = \\
	= \begin{vmatrix} \alpha_{11} \cdot x_{11} (t) + \alpha_{12} \cdot x_{12}(t) + \ldots + \alpha_{1n} \cdot x_{1n}(t) & x_{12}(t) & \ldots & x_{1n}(t) \\  \alpha_{11} \cdot x_{21} (t) + \alpha_{12} \cdot x_{22}(t) + \ldots + \alpha_{1n} \cdot x_{2n}(t) & x_{22}(t) & \ldots & x_{2n}(t) \\ \vdots & \vdots & \ddots & \vdots \\ \alpha_{11} \cdot x_{n1} (t) + \alpha_{12} \cdot x_{n2}(t) + \ldots + \alpha_{1n} \cdot x_{nn}(t) & x_{n2}(t) & \ldots & x_{nn}(t) \end{vmatrix} + \\
	+ \begin{vmatrix} x_{11}(t) & \alpha_{21} \cdot x_{11} (t) + \alpha_{22} \cdot x_{12}(t) + \ldots + \alpha_{2n} \cdot x_{1n}(t) & \ldots & x_{1n}(t) \\ x_{21}(t) & \alpha_{21} \cdot x_{21} (t) + \alpha_{22} \cdot x_{22}(t) + \ldots + \alpha_{2n} \cdot x_{2n}(t) & \ldots & x_{2n}(t) \\ \vdots & \vdots & \ddots & \vdots \\ x_{n1}(t) & \alpha_{21} \cdot x_{n1} (t) + \alpha_{22} \cdot x_{n2}(t) + \ldots + \alpha_{2n} \cdot x_{nn}(t) & \ldots & x_{nn}(t) \end{vmatrix} + \ldots \\
	\ldots + \begin{vmatrix} x_{11}(t) & x_{12}(t) & \ldots & \alpha_{n1} \cdot x_{11} (t) + \alpha_{n2} \cdot x_{12}(t) + \ldots + \alpha_{nn} \cdot x_{1n}(t) \\ x_{21}(t) & x_{22}(t) & \ldots & \alpha_{n1} \cdot x_{21} (t) + \alpha_{n2} \cdot x_{22}(t) + \ldots + \alpha_{nn} \cdot x_{2n}(t) \\ \vdots & \vdots & \ddots & \vdots \\ x_{n1}(t) & x_{n2}(t) & \ldots & \alpha_{n1} \cdot x_{n1} (t) + \alpha_{n2} \cdot x_{n2}(t) + \ldots + \alpha_{nn} \cdot x_{nn}(t) \end{vmatrix}.
\end{multline*}

Розкривши кожний з визначників, і з огляду на те, що визначники з однаковими стовпцями дорівнюють нулю, одержимо
\begin{multline*}
	\frac{\diff}{\diff t} W[x_1, \ldots, x_n] = a_{11} \cdot \begin{vmatrix} x_{11}(t) & x_{12}(t) & \ldots & x_{1n}(t) \\ x_{21}(t) & x_{22}(t) & \ldots & x_{2n}(t) \\ \vdots & \vdots & \ddots & \vdots \\ x_{n1}(t) & x_{n2}(t) & \ldots & x_{nn}(t) \end{vmatrix} + \\
	+ a_{22} \cdot \begin{vmatrix} x_{11}(t) & x_{12}(t) & \ldots & x_{1n}(t) \\ x_{21}(t) & x_{22}(t) & \ldots & x_{2n}(t) \\ \vdots & \vdots & \ddots & \vdots \\ x_{n1}(t) & x_{n2}(t) & \ldots & x_{nn}(t) \end{vmatrix} + \ldots + a_{nn} \cdot \begin{vmatrix} x_{11}(t) & x_{12}(t) & \ldots & x_{1n}(t) \\ x_{21}(t) & x_{22}(t) & \ldots & x_{2n}(t) \\ \vdots & \vdots & \ddots & \vdots \\ x_{n1}(t) & x_{n2}(t) & \ldots & x_{nn}(t) \end{vmatrix} = \\
	= (a_{11} + a_{22} + \ldots + a_{nn}) \cdot \begin{vmatrix} x_{11}(t) & x_{12}(t) & \ldots & x_{1n}(t) \\ x_{21}(t) & x_{22}(t) & \ldots & x_{2n}(t) \\ \vdots & \vdots & \ddots & \vdots \\ x_{n1}(t) & x_{n2}(t) & \ldots & x_{nn}(t) \end{vmatrix} = \\
	= \trace A \cdot \begin{vmatrix} x_{11}(t) & x_{12}(t) & \ldots & x_{1n}(t) \\ x_{21}(t) & x_{22}(t) & \ldots & x_{2n}(t) \\ \vdots & \vdots & \ddots & \vdots \\ x_{n1}(t) & x_{n2}(t) & \ldots & x_{nn}(t) \end{vmatrix} = \trace A \cdot W[x_1, \ldots, x_n].
\end{multline*}

Або
\begin{equation*}
	\frac{\diff}{\diff t} W[x_1, \ldots, x_n] = \trace A \cdot W[x_1, \ldots, x_n].
\end{equation*}

Розділивши змінні, одержимо
\begin{equation*}
	\frac{\diff W[x_1, \ldots, x_n]}{W[x_1, \ldots, x_n]}  = \trace A \cdot \diff t.
\end{equation*}

Проінтегруємо в межах $t_0 \le s \le t$,
\begin{equation*}
	\ln W[x_1, \ldots, x_n](t) - \ln W[x_1, \ldots, x_n](t_0) = \int_{t_0}^t \trace A \cdot \diff t,
\end{equation*}
або 
\begin{equation*}
	W[x_1, \ldots, x_n](t) = W[x_1, \ldots, x_n](t_0) \cdot \exp\left\{\int_{t_0}^t \trace A \cdot \diff t\right\}.
\end{equation*}

Взагалі кажучи, доведення проводилося в припущенні, що система рівнянь може залежати від часу, тобто
\begin{equation*}
	W[x_1, \ldots, x_n](t) = W[x_1, \ldots, x_n](t_0) \cdot \exp\left\{\int_{t_0}^t \trace A(t) \cdot \diff t\right\}.
\end{equation*}

Отримана формула називається формулою Якобі.


	\subsection{Системи лінійних однорідних диференціальних рівнянь з сталими коефіцієнтами}
	Система диференціальних рівнянь вигляду
\begin{equation*}
	\left\{
		\begin{array}{rl}
			\dot x_1 &= a_{11} x_1 + a_{12} x_2 + \ldots + a_{1n} x_n, \\
			\dot x_1 &= a_{21} x_1 + a_{22} x_2 + \ldots + a_{2n} x_n, \\
			\hdotsfor{2} \\
			\dot x_1 &= a_{n1} x_1 + a_{n2} x_2 + \ldots + a_{nn} x_n,
		\end{array}
	\right.
\end{equation*}
де $a_{ij}$, $i,j = \overline{1, n}$ --- сталі величини, називається лінійною однорідною системою з сталими коефіцієнтами. У матричному вигляді вона записується
\begin{equation*}
	\dot x = A x.
\end{equation*}


		\subsubsection{Розв'язування систем однорідних рівнянь з сталими коефіцієнтами методом Ейлера}
		Розглянемо один з методів побудови розв'язку систем з сталими коефіцієнтами. \parvskip

Розв'язок системи шукаємо у вигляді вектора \[x(t) = (\alpha_1 e^{\lambda t}, \alpha_2 e^{\lambda t}, \ldots, \alpha_n e^{\lambda t})^T. \]

Підставивши в систему диференціальних рівнянь, одержимо
\begin{equation*}
	\left\{
		\begin{array}{rl}
			\alpha_1 \lambda e^{\lambda t} &= a_{11} \alpha_1 e^{\lambda t} + a_{12} \alpha_2  e^{\lambda t} + \ldots + a_{1n} \alpha_n e^{\lambda t}, \\
			\alpha_2 \lambda e^{\lambda t} &= a_{21} \alpha_1 e^{\lambda t} + a_{22} \alpha_2 e^{\lambda t} + \ldots + a_{2n} \alpha_n e^{\lambda t}, \\
			\hdotsfor{2} \\
			\alpha_n \lambda e^{\lambda t} &= a_{n1} \alpha_1 e^{\lambda t} + a_{n2} \alpha_2 e^{\lambda t} + \ldots + a_{nn} \alpha_n e^{\lambda t}.
		\end{array}
	\right.
\end{equation*}
 
Скоротивши на $e^{\lambda t}$, і перенісши всі члени вправо, запишемо
\begin{equation*}
	\left\{
		\begin{array}{rl}
			(a_{11} - \lambda) \alpha_1 + a_{12} \alpha_2 + \ldots + a_{1n} \alpha_n &= 0, \\
			a_{21} \alpha_1 + (a_{22} - \lambda) \alpha_2 + \ldots + a_{2n} \alpha_n &= 0, \\
			\hdotsfor{2} \\
			a_{n1} \alpha_1 + a_{n2} \alpha_2 + \ldots + (a_{nn} - \lambda) \alpha_n &= 0.
		\end{array}
	\right.
\end{equation*}
 
Отримана однорідна система лінійних алгебраїчних рівнянь має роз\-в'яз\-ок тоді і тільки тоді, коли її визначник дорівнює нулю, тобто
\begin{equation*}
	\begin{vmatrix}
		a_{11} - \lambda & a_{12} & \cdots & a_{1n} \\
		a_{21} & a_{22} - \lambda & \cdots & a_{2n} \\
		\vdots & \vdots & \ddots & \vdots \\
		a_{n1} & a_{n2} & \cdots & a_{nn} - \lambda
	\end{vmatrix} = 0.
\end{equation*}

Це рівняння, може бути записаним у векторно-матричній формі
\begin{equation*}
	\det(A - \lambda E) = 0.
\end{equation*}
і воно називається характеристичним рівнянням. Розкриємо його
\begin{equation*}
	\lambda^n + p_1 \lambda^{n - 1} + \ldots + p_{n - 1} \lambda + p_n = 0.
\end{equation*}

Алгебраїчне рівняння $n$-го ступеня має $n$ коренів. Розглянемо різні випадки:
\begin{enumerate}
\item Всі корені характеристичного рівняння $\lambda_1, \lambda_2, \ldots, \lambda_n$ (власні числа матриці $A$) дійсні і різні. Підставляючи їх по черзі в систему алгебраїчних рівнянь
\begin{equation*}
	\left\{
		\begin{array}{rl}
			(a_{11} - \lambda_i) \alpha_1 + a_{12} \alpha_2 + \ldots + a_{1n} \alpha_n &= 0, \\
			a_{21} \alpha_1 + (a_{22} - \lambda_i) \alpha_2 + \ldots + a_{2n} \alpha_n &= 0, \\
			\hdotsfor{2} \\
			a_{n1} \alpha_1 + a_{n2} \alpha_2 + \ldots + (a_{nn} - \lambda_i) \alpha_n &= 0.
		\end{array}
	\right.
\end{equation*}

одержуємо відповідні ненульові розв'язки системи
\begin{equation*}
	\alpha^1 = \begin{pmatrix} \alpha_1^1 \\ \alpha_2^1 \\ \vdots \\ \alpha_n^1 \end{pmatrix}, \quad
	\alpha^2 = \begin{pmatrix} \alpha_1^2 \\ \alpha_2^2 \\ \vdots \\ \alpha_n^2 \end{pmatrix}, \quad
	\ldots, \quad
	\alpha^n = \begin{pmatrix} \alpha_1^n \\ \alpha_2^n \\ \vdots \\ \alpha_n^n \end{pmatrix}
\end{equation*}
що являють собою власні вектори, які відповідають власним числам $\lambda_i$, $i = \overline{1, n}$. \parvskip

У такий спосіб одержимо $n$ розв'язків
\begin{equation*}
	x_1(t) = \begin{pmatrix} \alpha_1^1 e^{\lambda_1 x} \\ \alpha_2^1 e^{\lambda_1 x} \\ \vdots \\ \alpha_n^1 e^{\lambda_1 x} \end{pmatrix}, 
	x_2(t) = \begin{pmatrix} \alpha_1^2 e^{\lambda_2 x} \\ \alpha_2^2 e^{\lambda_2 x} \\ \vdots \\ \alpha_n^2 e^{\lambda_2 x} \end{pmatrix},
	\ldots, 
	x_n(t) = \begin{pmatrix} \alpha_1^n e^{\lambda_n x} \\ \alpha_2^n e^{\lambda_n x} \\ \vdots \\ \alpha_n^n e^{\lambda_n x} \end{pmatrix}
\end{equation*}

Причому оскільки $\lambda_1, \lambda_2, \ldots, \lambda_n$ --- різні а $\alpha^1, \alpha^2, \ldots, \alpha^n$ -- відповідні їм власні вектори, то розв'язки $x_1(t), x_2(t), \ldots, x_n(t)$ --- лінійно незалежні, і загальний розв'язок системи має вигляд
\begin{equation*}
	x(t) = \sum_{i = 1}^n C_i x_i(t).
\end{equation*}

Або у векторно-матричній формі запису
\begin{equation*}
	\begin{pmatrix} x_1 \\ x_2 \\ \vdots \\ x_n \end{pmatrix} = 
	\begin{pmatrix}
		\alpha_1^1 e^{\lambda_1 t} & \alpha_1^2 e^{\lambda_2 t} & \cdots & \alpha_1^n e^{\lambda_n t} \\
		\alpha_2^1 e^{\lambda_1 t} & \alpha_2^2 e^{\lambda_2 t} & \cdots & \alpha_2^n e^{\lambda_n t} \\
		\vdots & \vdots & \ddots & \vdots \\
		\alpha_n^1 e^{\lambda_1 t} & \alpha_n^2 e^{\lambda_2 t} & \cdots & \alpha_n^n e^{\lambda_n t}
	\end{pmatrix}
	\cdot
	\begin{pmatrix} C_1 \\ C_2 \\ \vdots \\ C_n \end{pmatrix},
\end{equation*}
де $C_1, C_2, \ldots, C_n$ --- довільні сталі.

\item Нехай $\lambda_{1,2} = p \pm i q$ --- пара комплексно спряжених коренів. Візьмемо один з них, наприклад $\lambda = p + i q$. Комплексному власному числу відповідає комплексний власний вектор
\begin{equation*}
	\begin{pmatrix} \alpha_1 \\ \alpha_2 \\ \vdots \\ \alpha_n \end{pmatrix} =
	\begin{pmatrix} r_1 + i s_1 \\ r_2 + i s_2 \\ \vdots \\ r_n + i s_n \end{pmatrix}
\end{equation*}
і, відповідно, розв'язок
\begin{equation*}
	\begin{pmatrix} x_1 \\ x_2 \\ \vdots \\ x_n \end{pmatrix} =
	\begin{pmatrix} (r_1 + i s_1) e^{(p + i q) t} \\ (r_2 + i s_2) e^{(p + i q) t} \\ \vdots \\ (r_n + i s_n) e^{(p + i q) t} \end{pmatrix}
\end{equation*}

Використовуючи залежність $e^{(p + i q) t} = e^{pt} (\cos qt + i \sin qt)$, перетворимо розв'язок до вигляду:
\begin{multline*}
	\begin{pmatrix} x_1 \\ x_2 \\ \vdots \\ x_n \end{pmatrix} =
	\begin{pmatrix} (r_1 + i s_1) e^{p t} (\cos qt + i \sin qt) \\ (r_2 + i s_2) e^{p t} (\cos qt + i \sin qt) \\ \vdots \\ (r_n + i s_n) e^{p t} (\cos qt + i \sin qt) \end{pmatrix} = \\
	= \begin{pmatrix} e^{p t} (r_1 \cos qt - s_1 \sin qt) \\ e^{pt} (r_2 \cos qt - s_2 \sin qt) \\ \vdots \\ e^{pt} (r_n \cos qt - s_n \sin qt) \end{pmatrix} + i \begin{pmatrix} e^{p t} (s_1 \cos qt + r_1 \sin qt) \\ e^{pt} (s_2 \cos qt + r_2 \sin qt) \\ \vdots \\ e^{pt} (s_n \cos qt + r_n \sin qt) \end{pmatrix} = \\
	= u(t) + i v(t).
\end{multline*}

І, як випливає з властивості 4 розв'язків однорідних систем, якщо комплексна функція $u(t) + i v(t)$ дійсного аргументу є розв'язком однорідної системи, то окремо дійсна і уявна частини також будуть розв'язками, тобто комплексним власним числам $\lambda_{1,2} = p \pm i q$  відповідають лінійно незалежні розв'язки
\begin{align*}
	u(t) &= \begin{pmatrix} e^{p t} (r_1 \cos qt - s_1 \sin qt) \\ e^{pt} (r_2 \cos qt - s_2 \sin qt) \\ \vdots \\ e^{pt} (r_n \cos qt - s_n \sin qt) \end{pmatrix}, \\
	v(t) &= \begin{pmatrix} e^{p t} (s_1 \cos qt + r_1 \sin qt) \\ e^{pt} (s_2 \cos qt + r_2 \sin qt) \\ \vdots \\ e^{pt} (s_n \cos qt + r_n \sin qt) \end{pmatrix}
\end{align*}

\item Якщо характеристичне рівняння має кратний корінь $\lambda$ кратності $\gamma$, тобто $\lambda_1 = \lambda_2 = \ldots = \lambda_\gamma = \lambda$, то розв'язок системи рівнянь має вигляд
\begin{equation*}
	\begin{pmatrix} x_1 \\ x_2 \\ \vdots \\ x_n \end{pmatrix} = \begin{pmatrix} \left(\alpha_1^1 + \alpha_1^2 t + \ldots + \alpha_1^\gamma t^{\gamma - 1}\right) e^{\lambda t} \\ \left(\alpha_2^1 + \alpha_2^2 t + \ldots + \alpha_2^\gamma t^{\gamma - 1}\right) e^{\lambda t} \\ \vdots \\ \left(\alpha_n^1 + \alpha_n^2 t + \ldots + \alpha_n^\gamma t^{\gamma - 1}\right) e^{\lambda t} \end{pmatrix}
\end{equation*}

Підставивши його у вихідне диференціальне рівняння і прирівнявши коефіцієнти при однакових степенях, одержимо $\gamma n$ рівнянь, що містять $\gamma n$ невідомих. Тому що корінь характеристичного рівняння $\lambda$ має кратність $\gamma$, то ранг отриманої системи $\gamma n - \gamma = \gamma (n - 1)$. Уводячи $\gamma$ довільних сталих $C_1, C_2, \ldots, C_\gamma$ і розв'язуючи систему, одержимо
\begin{equation*}
	\alpha_i^j = \alpha_i^j(C_1, C_2, \ldots, C_\gamma), \quad i = \overline{1, n}, \quad j = \overline{1, \gamma}.
\end{equation*}
\end{enumerate}

		\subsubsection{Розв'язок систем однорідних рівнянь зі сталими коефіцієнтами матричним методом}
		Досить універсальним методом розв’язку лінійних однорідних систем з сталими коефіцієнтами є матричний метод. Він полягає в наступному. Розглядається лінійна система з сталими коефіцієнтами, що записана у векторно-матричному вигляді
\begin{equation*}
	\dot x(t) = A x, \quad x \in \RR^n.	
\end{equation*}

Робиться невироджене перетворення $x = S y$, $y \in \RR^n$, $\det S \ne 0$, де вектор $y(t)$ --- нова невідома векторна функція. Тоді рівняння прийме вигляд
\begin{equation*}
	S \dot y = A S y,
\end{equation*}
або
\begin{equation*}
	\dot y = S^{-1} A S y.
\end{equation*}

Для довільної матриці $A$ завжди існує неособлива матриця $S$, що приводить її до жорданової форми, тобто $S^{-1} A S = \Lambda$, де $\Lambda$ --- жорданова форма матриці $A$. І система диференціальних рівнянь прийме вигляд
\begin{equation*}
	\dot y = \Lambda y, \quad y \in \RR^n.
\end{equation*}

Складемо характеристичне рівняння матриці $A$
\begin{equation*}
	\det (D - \lambda E) = 0,
\end{equation*}
або
\begin{equation*}
	\lambda^n + p_1 \lambda^{n - 1} + \ldots + p_{n - 1} \lambda + p_n = 0.
\end{equation*}

Алгебраїчне рівняння $n$-го ступеня має $n$ коренів. Розглянемо різні випадки:
\begin{enumerate}
\item Нехай $\lambda_1, \lambda_2, \ldots, \lambda_n$ --- дійсні різні числа. Тоді матриця $\Lambda$ має вигляд
\begin{equation*}
	\Lambda = 
	\begin{pmatrix}
		\lambda_1 & 0 & \cdots & 0 \\
		0 & \lambda_2 & \cdots & 0 \\
		\vdots & \vdots & \ddots & \vdots \\
		0 & 0 & \cdots & \lambda_n.
	\end{pmatrix}
\end{equation*}

І перетворена система диференціальних рівнянь розпадається на $n$ незалежних рівнянь
\begin{equation*}
	\dot y_1 = \lambda_1 y_1, \quad \dot y_2 = \lambda_2 y_2, \quad \ldots, \quad \dot y_n = \lambda_n y_n.
\end{equation*}

Розв’язуючи кожне окремо, отримаємо
\begin{equation*}
	y_1 = C_1 e^{\lambda_1 t}, \quad y_2 = C_2 e^{\lambda_2 t}, \quad \ldots, \quad y_n = C_n e^{\lambda_n t}.
\end{equation*}

Або в матричному вигляді
\begin{equation*}
	y = e^{\Lambda t} C,
\end{equation*}
де
\begin{equation*}
	e^{\Lambda t} = 
	\begin{pmatrix}
		e^{\lambda_1 t} & 0 & \cdots & 0 \\
		0 & e^{\lambda_2 t} & \cdots & 0 \\
		\vdots & \vdots & \ddots & \vdots \\
		0 & 0 & \cdots & e^{\lambda_n t}
	\end{pmatrix}, \quad
	C = \begin{pmatrix} C_1 \\ C_2 \\ \vdots \\ C_n \end{pmatrix}.
\end{equation*}

Звідси розв’язок вихідного рівняння має вигляд $x = S e^{\Lambda t} C$. Для знаходження матриці $S$ треба розв’язати матричне рівняння
\begin{equation*}
	S^{-1} A S = \Lambda
\end{equation*}
або
\begin{equation*}
	A S = S \Lambda
\end{equation*}
де $\Lambda$ --- жорданова форма матриці $A$. Якщо матрицю $S$ записати у вигляді
\begin{equation*}
	S = 
	\begin{pmatrix} 
		\alpha_1^1 & \alpha_1^2 & \cdots & \alpha_1^n \\
		\alpha_2^1 & \alpha_2^2 & \cdots & \alpha_2^n \\
		\vdots & \vdots & \ddots & \vdots \\
		\alpha_n^1 & \alpha_n^2 & \cdots & \alpha_n^n
	\end{pmatrix},
\end{equation*}
то для кожного з стовпчиків $s_i = (\alpha_1^i, \alpha_2^i, \ldots, \alpha_n^i)^T$, матричне рівняння перетвориться до
\begin{equation*}
	A s_i = \lambda_i s_i, \quad i = \overline{1, n}.
\end{equation*}

Таким чином, у випадку різних дійсних власних чисел матриця $S$ являє собою набір $n$ власних векторів, що відповідають різним власним числам.

\item Нехай $\lambda_{1,2} = p \pm i q$ --- комплексний корінь. Тоді відповідна клітка Жордана має вигляд
\begin{equation*}
	\Lambda_{1,2} = \begin{pmatrix} p & q \\ -q & p \end{pmatrix},
\end{equation*}
а перетворена система диференціальних рівнянь
\begin{equation*}
	\left\{
		\begin{aligned}
			\dot y_1 &= p y_1 + q y_2, \\
			\dot y_2 &= - q y_1 + p y_2.
		\end{aligned}
	\right.
\end{equation*}

Неважко перевірити, що розв’язок отриманої системи диференціальних рівнянь має вигляд
\begin{align*}
	y_1 &= c_1 e^{pt} \cos  qt + c_2 e^{pt} \sin qt, \\
	y_2 &= c_2 e^{pt} \cos  qt - c_1 e^{pt} \sin qt.
\end{align*}

Або в матричному вигляді
\begin{equation*}
	\begin{pmatrix} y_1 \\ y_2 \end{pmatrix} =
	\begin{pmatrix}
		e^{pt} \cos qt & e^{pt} \sin qt \\
		- e^{pt} \sin qt & e^{pt} \cos qt
	\end{pmatrix}
	\begin{pmatrix} c_1 \\ c_2 \end{pmatrix}.
\end{equation*}

Таким чином, комплексно-спряженим власним числам $\lambda_{1,2}$ відповідає розв’язок  
\begin{equation*}
	y = e^{\Lambda t} C,
\end{equation*}
де
\begin{equation*}
	e^{\Lambda t} =
	\begin{pmatrix}
		e^{pt} \cos qt & e^{pt} \sin qt \\
		- e^{pt} \sin qt & e^{pt} \cos qt
	\end{pmatrix} 
\end{equation*}

\item Нехай $\lambda$ --- кратний корінь, кратності $m \le n$, тобто $\lambda_1 = \lambda_2 = \ldots = \lambda_m = \lambda$ і йому відповідають $r \le m$ лінійно незалежних векторів. Тоді клітка Жордана, що відповідає цьому власному числу, має вид
\begin{equation*}
	\Lambda = 
	\begin{pmatrix}
		\Lambda_1 & \textbf{0} \\
		\textbf{0} & \Lambda_2, 
	\end{pmatrix}
\end{equation*}
де
\begin{align*}
	\Lambda_1 &= 
	\begin{pmatrix} 
		\lambda & 0 & \cdots & 0 & 0 \\
		0 & \lambda & \cdots & 0 & 0 \\
		\vdots & \vdots & \ddots & \vdots & \vdots \\
		0 & 0 & \cdots & \lambda & 0 \\
		0 & 0 & \cdots & 0 & \lambda
	\end{pmatrix} \in \RR^{r \times r}, \\
	\Lambda_2 &= 
	\begin{pmatrix} 
		\lambda & 1 & \cdots & 0 & 0 \\
		0 & \lambda & \ddots & 0 & 0 \\
		\vdots & \vdots & \ddots & \ddots & \vdots \\
		0 & 0 & \cdots & \lambda & 1 \\
		0 & 0 & \cdots & 0 & \lambda
	\end{pmatrix} \in \RR^{(m - r) \times (m - r)}.
\end{align*}
 
І перетворена підсистема, що відповідає власному числу $\lambda$, розпадається не дві підсистеми
\begin{align*}
	\dot y_1 &= \Lambda_1 y_1, \quad y_1 \in \RR^r, \\
	\dot y_2 &= \Lambda_2 y_2, \quad y_2 \in \RR^{m - r},
\end{align*}

Розв’язок першої знаходиться з використанням зазначеного в першому пункті підходу. Розглянемо другу підсистему. Запишемо її в координатному вигляді
 
Розв’язок останнього рівняння цієї підсистеми має вигляд
\begin{equation*}
	y_{2, m - r} = c_{2, m - r} e^{\lambda t}
\end{equation*}

Підставимо його в передостаннє рівняння. Одержуємо
\begin{equation*}
	\dot y_{2, m - r - 1} = \lambda y_{2, m - r - 1} + c_{2, m - r} e^{\lambda t}.
\end{equation*}

Загальний розв’язок лінійного неоднорідного рівняння має вигляд суми загального розв’язку однорідного і частинного розв’язку неоднорідних рівнянь, тобто
\begin{equation*}
	y_{2, m - r - 1} = y_{2, m - r - 1, \text{homo}} + y_{2, m - r - 1, \text{hetero}}.
\end{equation*}

Загальний розв’язок однорідного має вигляд
\begin{equation*}
	\dot y_{2, m - r - 1, \text{homo}} = c_{2, m - r - 1} e^{\lambda t}.
\end{equation*}

Частинний розв’язок неоднорідного шукаємо методом невизначених коефіцієнтів у вигляді
\begin{equation*}
	y_{2, m - r - 1, \text{hetero}} = A t e^{\lambda t},
\end{equation*}
де $A$ --- невідома стала. Підставивши в неоднорідне рівняння, одержимо
\begin{equation*}
	A e^{\lambda t} + A \lambda t e^{\lambda t} = A \lambda t e^{\lambda t} + c_{2, m - r} e^{\lambda t}.
\end{equation*}

Звідси $A = c_{2, m - r}$ і загальний розв’язок неоднорідного рівняння має вигляд
\begin{equation*}
	y_{2, m - r - 1} = c_{2, m - r - 1} e^{\lambda t} + c_{2, m - r} t e^{\lambda t}.
\end{equation*}

Піднявшись ще на один крок нагору одержимо
\begin{equation*}
	y_{2, m - r - 1} = c_{2, m - r - 2} e^{\lambda t} + c_{2, m - r - 1} t e^{\lambda t} + c_{2, m - r } \frac{t^2}{2!} e^{\lambda t}.
\end{equation*}

Продовжуючи процес далі, маємо
\begin{equation*}
	y_{2, 1} = c_{2, 1} e^{\lambda t} + c_{2, 2} t e^{\lambda t} + \ldots + c_{2, m - r} \frac{t^{m - r - 1}}{(m - r - 1)!} e^{\lambda t}.
\end{equation*}

Або у векторно-матричному вигляді
\begin{equation*}
	y_2(t) = 
	\begin{pmatrix}
		e^{\lambda t} & t e^{\lambda t} & \cdots & \dfrac{t^{m - r - 2}}{(m - r - 2)!} & \dfrac{t^{m - r - 1}}{(m - r - 1)!} \\
		0 & e^{\lambda t} & \cdots & \dfrac{t^{m - r - 3}}{(m - r - 3)!} & \dfrac{t^{m - r - 2}}{(m - r - 2)!} \\
		\vdots & \vdots & \ddots & \vdots & \vdots \\
		0 & 0 & \cdots & e^{\lambda t} & t e^{\lambda t} \\
		0 & 0 & \cdots & 0 & e^{\lambda t}
	\end{pmatrix}
	\begin{pmatrix} c_{2,1} \\ c_{2,2} \\ \vdots \\ c_{2,m-r-1} \\ c_{2,m-r} \end{pmatrix}.
\end{equation*}

Додавши першу підсистему, одержимо
\begin{equation*}
	y = \begin{pmatrix} e^{\Lambda_1 t} & \textbf{0} \\ \textbf{0} & e^{\Lambda_2 t} \end{pmatrix} C,
\end{equation*}
де
\begin{align*}
	e^{\Lambda_1 t} &= 
	\begin{pmatrix} 
		e^{\lambda t} & 0 & \cdots & 0 & 0 \\
		0 & e^{\lambda t} & \cdots & 0 & 0 \\
		\vdots & \vdots & \ddots & \vdots & \vdots \\
		0 & 0 & \cdots & e^{\lambda t} & 0 \\
		0 & 0 & \cdots & 0 & e^{\lambda t}
	\end{pmatrix}, \\
	e^{\Lambda_2 t} &= 
	\begin{pmatrix}
		e^{\lambda t} & t e^{\lambda t} & \cdots & \dfrac{t^{m - r - 2}}{(m - r - 2)!} & \dfrac{t^{m - r - 1}}{(m - r - 1)!} \\
		0 & e^{\lambda t} & \cdots & \dfrac{t^{m - r - 3}}{(m - r - 3)!} & \dfrac{t^{m - r - 2}}{(m - r - 2)!} \\
		\vdots & \vdots & \ddots & \vdots & \vdots \\
		0 & 0 & \cdots & e^{\lambda t} & t e^{\lambda t} \\
		0 & 0 & \cdots & 0 & e^{\lambda t}
	\end{pmatrix}, \\
	C &= \begin{pmatrix} c_{1,1} & \cdots & c_{1,r} & c_{2,1} & \cdots & c_{2,m-r} \end{pmatrix}^T.
\end{align*}

Для останніх двох випадків матриця   знаходиться як розв’язок матричного рівняння
\begin{equation*}
	A S = S \Lambda	
\end{equation*}
\end{enumerate}


		\subsubsection{Вправи для самостійної роботи}
		При розв'язуванні систем методом Ейлера складають характеристичне рівняння, і в залежності від його коренів для кожного $\lambda_i$,    $i = \overline{1, n}$ знаходять відповідний лінійно незалежний розв'язок.

\begin{example}
    Розв'язати систему:
    \[ \left\{ \begin{aligned}
        \dot x &= 2 x + 3 y, \\
        \dot y &= 3 x + 4 y.
    \end{aligned} \right. \]
\end{example}

\begin{solution}
    Характеристичне рівняння має вигляд
    \[ \begin{vmatrix}
        2 - \lambda & 1 \\
        3 & 4 - \lambda 
    \end{vmatrix} = 0, \]
    або $\lambda^2 - 6 \lambda + 5 = 0$. \parvskip
    
    Коренями будуть $\lambda_1 = 1$, $\lambda_2 = 5$.
    
    \begin{enumerate}
        \item Знайдемо власний вектор, що відповідає $\lambda_1 = 1$. Підставивши в систему
        \[ \left\{ \begin{aligned}
            (2 - \lambda) \alpha_1 + \alpha_2 &= 0, \\
            3 \alpha_1 + (4 - \lambda) \alpha_2 &= 0, 
        \end{aligned} \right. \]
        одержимо 
        \[ \left\{ \begin{aligned}
            \alpha_1 + \alpha_2 &= 0, \\
            3 \alpha_1 + 3 \alpha_2 &= 0.
        \end{aligned} \right. \]
        
        Звідси $\alpha_1 = 1$, $\alpha_2 = -1$.
        
        \item
        Знайдемо власний вектор, що відповідає $\lambda_2 = 5$. Підставивши в систему, одержимо
        \[ \left\{ \begin{aligned}
            - 3 \alpha_1 + \alpha_2 &= 0, \\
            3 \alpha_1 - \alpha_2 &= 0.
        \end{aligned} \right. \]
        
        Звідси $\alpha_1 = 1$, $\alpha_2 = 3$.
    \end{enumerate}
    
    Таким чином, одержимо розв'язок системи у вигляді
    \[ \begin{pmatrix} x \\ y \end{pmatrix} = c_1 e^t \begin{pmatrix} 1 \\ -1 \end{pmatrix} + C_2 e^{5t} \begin{pmatrix} 1 \\ 3 \end{pmatrix} = \begin{pmatrix} e^t & e^{5t} \\ - e^t & 3 e^{5t} \end{pmatrix} \begin{pmatrix} c_1 \\ c_2 \end{pmatrix}. \]
\end{solution}

\begin{example}
    Розв'язати систему:
    \[ \left\{ \begin{aligned}
        \dot x &= x + y, \\
        \dot y &= -2 x + 3 y.
    \end{aligned} \right. \]
\end{example}

\begin{solution}
    Характеристичне рівняння має вигляд
    \[ \begin{vmatrix}
        1 - \lambda & 1 \\
        -2 & 3 - \lambda 
    \end{vmatrix} = 0, \]
    або $\lambda^2 - 4 \lambda + 5 = 0$. \parvskip
    
    Коренями будуть $\lambda_{1,2} = 2 \pm i$. \parvskip
    
    Візьмемо $\lambda_1 = 2 + i$. Підставивши в систему
    \[ \left\{ \begin{aligned}
        (1 - \lambda) \alpha_1 + \alpha_2 &= 0, \\
        -2 \alpha_1 + (3 - \lambda) \alpha_2 &= 0, 
    \end{aligned} \right. \]
    одержимо 
    \[ \left\{ \begin{aligned}
        (-1 - i) \alpha_1 + \alpha_2 &= 0, \\
        -2 \alpha_1 + (1 - i) \alpha_2 &= 0.
    \end{aligned} \right. \]
    
    Звідси $\alpha_1 = 1$, $\alpha_2 = 1 + i$. \parvskip
    
    Запишемо вектор розв'язку
    \begin{multline*} \begin{pmatrix} x \\ y \end{pmatrix} = \begin{pmatrix} e^{(2 + i) t} \\ (1 + i) e^{(2 + i) t} \end{pmatrix} = \begin{pmatrix} e^{2 t} (\cos t + i \sin t) \\ e^{2 t} (1 + i) (\cos t + i \sin t) \end{pmatrix} = \\ = \begin{pmatrix} e^{2 t} \cos t \\ e^{2 t} (\cos t - \sin t) \end{pmatrix} + i \begin{pmatrix} e^{2 t} \sin t \\ e^{2 t} (\cos t + \sin t) \end{pmatrix}. \end{multline*}
    
    Оскільки комплексно-спряженому розв'язку відповідають два лінійно незалежних розв'язки, то загальний розв'язок має вигляд
    \begin{multline*} \begin{pmatrix} x \\ y \end{pmatrix} = c_1 \begin{pmatrix} e^{2 t} \cos t \\ e^{2 t} \cos t - \sin t \end{pmatrix} + c_2 \begin{pmatrix} e^{2 t} \sin t \\ e^{2 t} (\cos t + \sin t) \end{pmatrix} = \\ = \begin{pmatrix} e^{2 t} \cos t & e^{2 t} \sin t \\ e^{2 t} (\cos t - \sin t) & e^{2 t} (\cos t + \sin t) \end{pmatrix} \begin{pmatrix} c_1 \\ c_2 \end{pmatrix}. \end{multline*}
\end{solution}

\begin{example}
    Розв'язати систему:
    \[ \left\{ \begin{aligned}
        \dot x &= 2 x + y, \\
        \dot y &= -x + 4 y.
    \end{aligned} \right. \]
\end{example}
 
\begin{solution}
    Характеристичне рівняння має вигляд
    \[ \begin{vmatrix}
        2 - \lambda & 1 \\
        -1 & 4 - \lambda 
    \end{vmatrix} = 0, \]
    або $\lambda^2 - 6 \lambda + 9 = 0$. \parvskip
    
    Коренями будуть $\lambda_1 = \lambda_2 = 3$. Оскільки
    \[ \rang \left. \begin{pmatrix} 
        2 - \lambda & 1 \\
        -1 & 4 - \lambda 
    \end{pmatrix} \right|_{\lambda = 3} 
    = 
    \rang \begin{pmatrix} 
        -1 & 1 \\
        -1 & 1
    \end{pmatrix} = 1, \]
    то матриця має один власний вектор. Тому розв'язок шукаємо у вигляді
    \[ x = (a_1^1 + a_1^2 t) e^{3t}, \quad y = (a_2^1 + a_2^2 t) e^{3t}. \]
    
    Підставимо в систему
    \[ \left\{ \begin{aligned}
        3 e^{3t} (a_1^1 + a_1^2 t) + a_1^2 e^{3t} &= 2 (a_1^1 + a_1^2 t) e^{3t} + (a_2^1 + a_2^2) e^{3t}, \\
        3 e^{3t} (a_2^1 + a_2^2 t) + a_2^2 e^{3t} &= - (a_1^1 + a_1^2 t) e^{3t} + 4 (a_2^1 + a_2^2) e^{3t}.
    \end{aligned} \right. \]
    
    Прирівнявши коефіцієнти при однакових членах, одержимо дві системи
    \[ \left\{ \begin{aligned} 
        3 a_1^2 &= 2 a_1^2 + a_2^2, \\
        3 a_2^2 &= -a_1^2 + 4 a_2^2,
    \end{aligned} \right. 
    \qquad
    \left\{ \begin{aligned} 
        3 a_1^1 + a_1^2 &= 2 a_1^1 + a_2^1, \\
        3 a_2^1 + a_2^2 &= -a_1^1 + 4 a_2^1.
    \end{aligned} \right.\]

    Або
    \[ \left\{ \begin{aligned} 
        -a_1^2 + a_2^2 &= 0, \\
        -a_1^2 + a_2^2 &= 0,
    \end{aligned} \right. 
    \qquad
    \left\{ \begin{aligned} 
        -a_1^1 + a_2^1 &= a_1^2, \\
        -a_1^1 + a_2^1 &= a_1^2.
    \end{aligned} \right.\]

    З першої системи одержуємо $a_1^2 = a_2^2 = c_1$. Підставивши в другу, одержимо $-a_1^1 + a_2^1 = c_1$. Поклавши $a_1^1 = c_2$, одержимо $c_2^1 = c_1 + c_2$. Таким чином,
    \begin{multline*} \begin{pmatrix} x \\ y \end{pmatrix} = \begin{pmatrix} (c_2 + c_1 t) e^{3 t} \\ (c_1 + c_2 + c_1 t) e^{3 t} \end{pmatrix} = c_1 \begin{pmatrix} t e^{3 t} \\ (1 + t) e^{3 t} \end{pmatrix} + c_2 \begin{pmatrix} e^{3 t} \\ e^{3 t} \end{pmatrix} = \\ = \begin{pmatrix} t e^{3 t} & e^{3 t} \\ (1 + t) e^{3 t} & e^{3 t} \end{pmatrix} \begin{pmatrix} c_1 \\ c_2 \end{pmatrix}. \end{multline*}
\end{solution}

Розв'яжемо ці ж системи матричним методом.

\setcounter{problem}{0}
\begin{example}
    Розв'язати систему:
    \[ \left\{ \begin{aligned}
        \dot x &= 2 x + 3 y, \\
        \dot y &= 3 x + 4 y.
    \end{aligned} \right. \]
\end{example}

\begin{solution}
    Характеристичне рівняння має вигляд
    \[ \begin{vmatrix}
        2 - \lambda & 1 \\
        3 & 4 - \lambda 
    \end{vmatrix} = 0, \]
    або $\lambda^2 - 6 \lambda + 5 = 0$. \parvskip
    
    Його коренями будуть $\lambda_1 = 1$, $\lambda_2 = 5$. Тому 
    \[ \Lambda = \begin{pmatrix} 1 & 0 \\ 0 & 5 \end{pmatrix} \quad e^{\Lambda t} = \begin{pmatrix} e^t & 0 \\ 0 & e^{5 t} \end{pmatrix}. \]
    
    Розв'язуємо матричне рівняння $A S = S \Lambda$, або 
    \[ \begin{pmatrix} 2 & 1 \\ 3 & 4 \end{pmatrix} \begin{pmatrix} a_1^1 & a_1^2 \\ a_2^1 & a_2^2 \end{pmatrix} = \begin{pmatrix} a_1^1 & a_1^2 \\ a_2^1 & a_2^2 \end{pmatrix} \begin{pmatrix} 1 & 0 \\ 0 & 5 \end{pmatrix}. \]
    
    Воно розпадається на два 
    \[ \left\{ \begin{aligned} 
        2 a_1^1 + a_2^1 &= a_1^1, \\
        3 a_1^1 + 4 a_2^1 &= a_2^1,
    \end{aligned} \right. 
    \qquad
    \left\{ \begin{aligned} 
        2 a_1^2 + a_2^2 &= 5 a_1^2, \\
        3 a_1^2 + 4 a_2^2 &= 5 a_2^2,
    \end{aligned} \right.\]
    
    Після перенесення всіх членів уліво, одержимо
    \[ \left\{ \begin{aligned} 
        a_1^1 + a_2^1 &= 0, \\
        3  a_1^1 + 3 a_2^1 &= 0,
    \end{aligned} \right. 
    \qquad
    \left\{ \begin{aligned} 
        -3 a_1^2 + a_2^2 &= 0, \\
        3 a_1^2 - a_2^2 &= 0,
    \end{aligned} \right.\]
    
    Звідси $a_1^1 = 1$, $a_2^1 = - 1$, $a_1^2 = 1$, $a_2^2 = 3$. \parvskip
    
    Таким чином, загальний розв'язок має вигляд
    \[ S = \begin{pmatrix} 1 & 1 \\ -1 & 3 \end{pmatrix}, \quad \begin{pmatrix} x \\ y \end{pmatrix} = \begin{pmatrix} e^t & e^{5t} \\ -e^t & 3 e^{5 t} \end{pmatrix} \begin{pmatrix} c_1 \\ c_2 \end{pmatrix}. \]
\end{solution}

\begin{example}
    Розв'язати систему:
    \[ \left\{ \begin{aligned}
        \dot x &= x + y, \\
        \dot y &= -2 x + 3 y.
    \end{aligned} \right. \]
\end{example}
 
\begin{solution}
    Характеристичне рівняння має вигляд
    \[ \begin{vmatrix}
        1 - \lambda & 1 \\
        -2 & 3 - \lambda 
    \end{vmatrix} = 0, \]
    або $\lambda^2 - 4 \lambda + 5 = 0$. \parvskip
    
    Коренями будуть $\lambda_{1,2} = 2 \pm i$. Тому 
    \[ \Lambda = \begin{pmatrix} 2 & 1 \\ -1 & 2 \end{pmatrix} \quad e^{\Lambda t} = \begin{pmatrix} e^{2 t} \cos t & e^{2 t} \sin t \\ - e^{2 t} \sin t & e^{2 t} \cos t \end{pmatrix}. \]
    
    Матричне рівняння має вигляд $A S = S \Lambda$, чи
    \[ \begin{pmatrix} 1 & 1 \\ -2 & 3 \end{pmatrix} \begin{pmatrix} a_1^1 & a_1^2 \\ a_2^1 & a_2^2 \end{pmatrix} = \begin{pmatrix} a_1^1 & a_1^2 \\ a_2^1 & a_2^2 \end{pmatrix} \begin{pmatrix} 2 & 1 \\ -1 & 2 \end{pmatrix}. \]
    
    Розпишемо його поелементно
    \[ \left\{ \begin{aligned} 
        a_1^1 + a_2^1 &= 2 a_1^1 - a_1^2, \\
        -2 a_1^1 + 3 a_2^1 &= 2 a_2^1 - a_2^2, \\
        a_1^2 + a_2^2 &= a_1^1 + 2 a_1^2, \\
        -2 a_1^2 + 3 a_2^2 &= a_2^1 + 2 a_2^2.
    \end{aligned} \right.\]
    
    На відміну від попереднього пункту (і це істотно ускладнює обчислення) система не розщеплюється  на дві незалежні підсистеми. Після перенесення всіх членів в одну сторону, одержимо систему
    \[ \left\{ \begin{aligned} 
        - a_1^1 - a_1^2 + a_2^1 &= 0, \\
        -2 a_1^1 + a_2^1 + a_2^2 &= 0, \\
        -a_1^1 - a_1^2 + a_2^2 &= 0, \\
        -2 a_1^2 + a_2^1 + a_2^2 &= 0.
    \end{aligned} \right.\]
     
    Помножимо перше рівняння на $-2$ і, склавши з другим, підставимо на місце другого. Далі, помножимо перше рівняння на $-1$ і, склавши з третім, поставимо його на місце третього. Одержуємо систему
    \[ \left\{ \begin{aligned} 
        - a_1^1 + a_1^2 + a_2^1 &= 0, \\
        -2 a_1^2 - a_2^1 + a_2^2 &= 0, \\
        -2 a_1^2 - a_2^1 + a_2^2 &= 0, \\
        -2 a_1^2 - a_2^1 + a_2^2 &= 0.
    \end{aligned} \right.\]
     
    Останні два рівняння можна відкинути. Залишається
    \[ \left\{ \begin{aligned} 
        - a_1^1 + a_1^2 + a_2^1 &= 0, \\
        -2 a_1^2 - a_2^1 + a_2^2 &= 0.
    \end{aligned} \right.\]
    
    Покладаємо $a_1^2 = a_2^2 = 1$. Тоді $a_2^1 = -1$, $a_1^1 = 0$. Таким чином,
    \begin{multline*} S = \begin{pmatrix} 0 & 1 \\ -1 & 1 \end{pmatrix}, \quad \begin{pmatrix} x \\ y \end{pmatrix} = \begin{pmatrix} 0 & 1 \\ -1 & 1 \end{pmatrix} \begin{pmatrix} e^{2 t} \cos t & e^{2 t} \sin t \\ - e^{2 t} \sin t & e^{2 t} \cos t \end{pmatrix} \begin{pmatrix} c_1 \\ c_2 \end{pmatrix} = \\ = \begin{pmatrix} - e^{2 t} \sin t & e^{2 t} \cos t \\ - e^{2 t} (\cos t + \sin t) & e^{2 t} (\cos t - \sin t) \end{pmatrix} \begin{pmatrix} c_1 \\ c_2 \end{pmatrix}. \end{multline*}
\end{solution}

\begin{example}
    Розв'язати систему:
    \[ \left\{ \begin{aligned}
        \dot x &= 2 x + y, \\
        \dot y &= -x + 4 y.
    \end{aligned} \right. \]
\end{example}

\begin{solution}
    Характеристичне рівняння має вигляд
    \[ \begin{vmatrix}
        2 - \lambda & 1 \\
        -1 & 4 - \lambda 
    \end{vmatrix} = 0, \]
    або $\lambda^2 - 6 \lambda + 9 = 0$. \parvskip
    
    Коренями будуть $\lambda_1 = \lambda_2 = 3$. Оскільки
    \[ \rang \left. \begin{pmatrix} 
        2 - \lambda & 1 \\
        -1 & 4 - \lambda 
    \end{pmatrix} \right|_{\lambda = 3} 
    = 
    \rang \begin{pmatrix} 
        -1 & 1 \\
        -1 & 1
    \end{pmatrix} = 1, \]
    то матриця має один власний вектор і клітка Жордана має вигляд
    \[ \Lambda = \begin{pmatrix} 3 & 1 \\ 0 & 3 \end{pmatrix} \quad e^{\Lambda t} = \begin{pmatrix} e^{3 t} & t e^{3 t} \\ 0 t & e^{3 t} \cos t \end{pmatrix}. \]
    
    Матричне рівняння має вигляд $A S = S \Lambda$, чи
    \[ \begin{pmatrix} 2 & 1 \\ -1 & 4 \end{pmatrix} \begin{pmatrix} a_1^1 & a_1^2 \\ a_2^1 & a_2^2 \end{pmatrix} = \begin{pmatrix} a_1^1 & a_1^2 \\ a_2^1 & a_2^2 \end{pmatrix} \begin{pmatrix} 3 & 1 \\ 0 & 3 \end{pmatrix}. \]
    
    Розпишемо його поелементно
    \[ \left\{ \begin{aligned} 
        2 a_1^1 + a_2^1 &= 3 a_1^1, \\
        - a_1^1 + 4 a_2^1 &= 3 a_2^1, \\
    \end{aligned} \right. 
    \qquad
    \left\{ \begin{aligned} 
        2 a_1^2 + a_2^2 &= a_1^1 + 3 a_1^2, \\
        - a_1^2 + 4 a_2^2 &= a_2^1 + 3 a_2^2.
    \end{aligned} \right.\]
    
    На відміну від комплексних коренів, можна розв'язати спочатку першу підсистему, а потім другу. Перша має вид
    \[ \left\{ \begin{aligned} 
        - a_1^1 + a_2^1 &= 0, \\
        - a_1^1 + a_2^1 &= 0, \\
    \end{aligned} \right. \]
    
    Звідси $a_1^1 = a_2^1 = 1$. \parvskip
    
    Підставивши в другу, одержимо
    \[ \left\{ \begin{aligned} 
        - a_1^2 + a_2^2 &= 1, \\
        - a_1^2 + a_2^2 &= 1.
    \end{aligned} \right.\]
    
    Звідси $a_2^2 = 1$, $a_1^2 = 0$. Таким чином одержали
    \begin{multline*} S = \begin{pmatrix} 1 & 0 \\ 1 & 1 \end{pmatrix}, \quad \begin{pmatrix} x \\ y \end{pmatrix} = \begin{pmatrix} 1 & 0 \\ 1 & 1 \end{pmatrix} \begin{pmatrix} e^{3 t} & t e^{3 t} \\ -0 & e^{3 t} \end{pmatrix} \begin{pmatrix} c_1 \\ c_2 \end{pmatrix} = \\ = \begin{pmatrix} e^{3 t} & t e^{3 t} \\ e^{3 t} & (t + 1) e^{3 t} \end{pmatrix} \begin{pmatrix} c_1 \\ c_2 \end{pmatrix}. \end{multline*}
\end{solution}

\begin{remark}
    Якщо власні числа дійсні різні, то обидва методи еквівалентні. Якщо власні числа комплексні, переважніше метод Ейлера, якщо кратні, то матричний метод.
\end{remark}

Розв'язати лінійні однорідні системи методом Ейлера чи матричним методом.
\begin{multicols}{2}
    \begin{problem}
        \[ \left\{ \begin{aligned} 
            \dot x &= x - y, \\
            \dot y &= -4 x + y.
        \end{aligned} \right. \]
    \end{problem}
    
    \begin{problem}
        \[ \left\{ \begin{aligned} 
            \dot x &= -x + 8 y, \\
            \dot y &= x + y.
        \end{aligned} \right. \]
    \end{problem}
    
    \begin{problem}
        \[ \left\{ \begin{aligned} 
            \dot x &= x -3 y, \\
            \dot y &= 3 x + y.
        \end{aligned} \right. \]
    \end{problem}
    
    \begin{problem}
        \[ \left\{ \begin{aligned} 
            \dot x &= - x - 5 y, \\
            \dot y &= x + y.
        \end{aligned} \right. \]
    \end{problem}
    
    \begin{problem}
        \[ \left\{ \begin{aligned} 
            \dot x &= 3 x - y, \\
            \dot y &= 4 x - y.
        \end{aligned} \right. \]
    \end{problem}
    
    \begin{problem}
        \[ \left\{ \begin{aligned} 
            \dot x &= -3 x + 2 y, \\
            \dot y &= -2 x + y.
        \end{aligned} \right. \]
    \end{problem}
    
    \begin{problem}
        \[ \left\{ \begin{aligned} 
            \dot x &= 5 x + 3 y, \\
            \dot y &= -3 x - y.
        \end{aligned} \right. \]
    \end{problem}
\end{multicols}

Розв'язати лінійні однорідні системи методом Ейлера чи матричним методом (після системи вкзані власні числа для спрощення обчислень).
\begin{multicols}{2}
    \begin{problem}
        \[ \left\{ \begin{aligned} 
            \dot x &= x - y + z, \\
            \dot y &= x + y - z, \\
            \dot z &= 2 x - y.
        \end{aligned} \right. \]
        ($\lambda_1 = 1$, $\lambda_2 = 2$, $\lambda_3 = -1$)
    \end{problem}
    
    \begin{problem}
        \[ \left\{ \begin{aligned} 
            \dot x &= x - 2 y - z, \\
            \dot y &= -x + y + z, \\
            \dot z &= x - z.
        \end{aligned} \right. \]
        ($\lambda_1 = 0$, $\lambda_2 = 2$, $\lambda_3 = -1$)
    \end{problem}
    
    \begin{problem}
        \[ \left\{ \begin{aligned} 
            \dot x &= 2 x - y + z, \\
            \dot y &= x + 2 y - z, \\
            \dot z &= x - y + 2 z.
        \end{aligned} \right. \]
        ($\lambda_1 = 1$, $\lambda_2 = 2$, $\lambda_3 = 3$)
    \end{problem}
    
    \begin{problem}
        \[ \left\{ \begin{aligned} 
            \dot x &= 3 x - y + z, \\
            \dot y &= x + y + z, \\
            \dot z &= 4 x - y + 4 z.
        \end{aligned} \right. \]
        ($\lambda_1 = 1$, $\lambda_2 = 2$, $\lambda_3 = 5$)
    \end{problem}
    
    \begin{problem}
        \[ \left\{ \begin{aligned} 
            \dot x &= -3 x - 4 y - 2 z, \\
            \dot y &= x + z, \\
            \dot z &= 6 z - 6 y + 5 z.
        \end{aligned} \right. \]
        ($\lambda_1 = 1$, $\lambda_2 = 2$, $\lambda_3 = -1$)
    \end{problem}
    
    \begin{problem} 
        \[ \left\{ \begin{aligned} 
            \dot x &= x - y - z, \\
            \dot y &= x + y, \\
            \dot z &= 3 x + z.
        \end{aligned} \right. \]
        ($\lambda_1 = 1$, $\lambda_{2, 3} = 1 +\pm 3 i$)
    \end{problem}
    
    \begin{problem}
        \[ \left\{ \begin{aligned} 
            \dot x &= 2 x + y, \\
            \dot y &= x + 3 y - z, \\
            \dot z &= -x + y - z.
        \end{aligned} \right. \]
        ($\lambda_1 = 2$, $\lambda_{2, 3} = 3 \pm i$)
    \end{problem}
    
    \begin{problem}
        \[ \left\{ \begin{aligned} 
            \dot x &= 2 x - y + 2 z, \\
            \dot y &= x + z, \\
            \dot z &= -2 x - y + 2 z.
        \end{aligned} \right. \]
        ($\lambda_1 = 2$, $\lambda_{2, 3} = \pm i$)
    \end{problem}
    
    \begin{problem}
        \[ \left\{ \begin{aligned} 
            \dot x &= 4 x - y - z, \\
            \dot y &= x + 2 y - z, \\
            \dot z &= x - y + 2 z.
        \end{aligned} \right. \]
        ($\lambda_1 = 2$, $\lambda_2 = \lambda_3 = 3$)
    \end{problem}
    
    \begin{problem}
        \[ \left\{ \begin{aligned} 
            \dot x &= 2 x - y - z, \\
            \dot y &= 3 x - 2 y - 3 z, \\
            \dot z &= -x + y + 2 z.
        \end{aligned} \right. \]
        ($\lambda_1 = 0$, $\lambda_2 = \lambda_3 = 1$)
    \end{problem}
    
    \begin{problem}
        \[ \left\{ \begin{aligned} 
            \dot x &= - 2 x + y - 2 z, \\
            \dot y &= x - 2 y + 2 z, \\
            \dot z &= 3 x - 3 y + 5 z.
        \end{aligned} \right. \]
        ($\lambda_1 = 3$, $\lambda_2 = \lambda_3 = -1$)
    \end{problem}
    
    \begin{problem}
        \[ \left\{ \begin{aligned} 
            \dot x &= 3 x - 2 y - z, \\
            \dot y &= 3 x - 4 y + 2 z, \\
            \dot z &= 2 x - 4 y.
        \end{aligned} \right. \]
        ($\lambda_1 = \lambda_2 = 2$, $\lambda_3 = -5$)
    \end{problem}
    
    \begin{problem}
        \[ \left\{ \begin{aligned} 
            \dot x &= x - y + z, \\
            \dot y &= x + y - z, \\
            \dot z &= - y + 2 z.
        \end{aligned} \right. \]
        ($\lambda_1 = \lambda_2 = 1$, $\lambda_3 = 2$)
    \end{problem}
    
    \begin{problem}
        \[ \left\{ \begin{aligned} 
            \dot x &= - x + y - 2 z, \\
            \dot y &= 4 x + y, \\
            \dot z &= - y + 2 z.
        \end{aligned} \right. \]
        ($\lambda_1 = 1$, $\lambda_2 = \lambda_3 = -1$)
    \end{problem}
    
    \begin{problem}
        \[ \left\{ \begin{aligned} 
            \dot x &= 2 x + y, \\
            \dot y &= 2 y + 4 z, \\
            \dot z &= x - z.
        \end{aligned} \right. \]
        ($\lambda_1 = \lambda_2 = 0$, $\lambda_3 = 3$)
    \end{problem}
    
    \begin{problem}
        \[ \left\{ \begin{aligned} 
            \dot x &= 2 x - y - z, \\
            \dot y &= 2 x - y - z, \\
            \dot z &= - x + z.
        \end{aligned} \right. \]
        ($\lambda_1 = \lambda_2 = \lambda_3 = 2$)
    \end{problem}
\end{multicols}


% 	\subsection{Лінійні неоднорідні системи \todo}
% 	Система диференціальних рівнянь, що записана у вигляді 
\begin{equation*}
	\left\{
		\begin{array}{rl}
			\dot x_1 &= a_{11}(t) x_1 + a_{12}(t) x_2 + \ldots + a_{1n}(t) x_n + f_1(t), \\
			\dot x_2 &= a_{21}(t) x_1 + a_{22}(t) x_2 + \ldots + a_{2n}(t) x_n + f_2(t), \\
			\hdotsfor{2} \\
			\dot x_n &= a_{n1}(t) x_1 + a_{n2}(t) x_2 + \ldots + a_{nn}(t) x_n + f_2(t),
		\end{array}
	\right.
\end{equation*} 
чи у векторно-матричному вигляді
\begin{equation*}
	\dot x = A(t) x + f(t)
\end{equation*}
називається системою лінійних неоднорідних диференціальних рівнянь.


% 		\subsubsection{Властивості розв'язків лінійних неоднорідних систем \todo}
% 		\setcounter{property}{0}
\begin{property}
	Якщо вектор 
	\begin{equation*}
		x(t) = \begin{pmatrix} x_1(t) & \cdots & x_n(t) \end{pmatrix}^T
	\end{equation*}
	є розв'язком лінійної неоднорідної системи, a 
	\begin{equation*}
		y(t) = \begin{pmatrix} y_1(t) & \cdots & y_n(t) \end{pmatrix}^T
	\end{equation*}
	розв'язком відповідної лінійної однорідної системи, то сума $x(t) + y(t)$ є розв'язком лінійної неоднорідної системи.
\end{property}

\begin{proof}
	Дійсно, за умовою
	\begin{equation*}
		\dot x(t) - A(t) x(t) \equiv f(t)
	\end{equation*}
	і
	\begin{equation*}
		\dot y(t) - A(t) y(t) \equiv 0.
	\end{equation*}

	Але тоді і
	\begin{multline*}
		\frac{\diff}{\diff t} (x(t) + y(t)) - A(t) (x(t) + y(t)) = \left( \frac{\diff}{\diff t} x(t) - A(t) x(t) \right) + \\
		+ \left( \frac{\diff}{\diff t} y(t) - A(t) y(t) \right) \equiv f(t) + 0 \equiv f(t),
	\end{multline*}
	тобто $x(t) + y(t)$ є розв'язком неоднорідної системи.
\end{proof}

\begin{property}[Принцип суперпозиції]
	Якщо вектори 
	\begin{equation*}
		x_i(t) = \begin{pmatrix} x_{1i}(t) & \cdots & x_{ni}(t) \end{pmatrix}^T, \quad i = \overline{1, n}
	\end{equation*}
	є розв'язками лінійних неоднорідних систем
	\begin{equation*}
		\dot x(t) - A(t) x(t) \equiv f_i(t) \quad i = \overline{1, n}
	\end{equation*}
	де 
	\begin{equation*}
		f_i(t) = \begin{pmatrix} f_{1i}(t) & \cdots & f_{ni}(t) \end{pmatrix}^T, \quad i = \overline{1, n},
	\end{equation*}
	то вектор $x(t) = \sum_{i = 1}^n C_i x_i(t)$, де $C_i$ --- довільні сталі буде розв'язком лінійної неоднорідної системи
	\begin{equation*}
		\dot x(t) - A(t) x(t) \equiv \sum_{i = 1}^n C_i f_i(t) \quad i = \overline{1, n}.
	\end{equation*}
\end{property}

\begin{proof}
	Дійсно, за умовою виконуються $n$ тотожностей
	\begin{equation*}
		\dot x_i(t) - A(t) x_i(t) \equiv f_i(t) \quad i = \overline{1, n}.
	\end{equation*}

	Склавши лінійну комбінацію з лівих і правих частин, одержимо
	\begin{multline*}
		\frac{\diff}{\diff t} \left( \sum_{i = 1}^n C_i x_i(t) \right) - A(t) \left( \sum_{i = 1}^n C_i x_i(t) \right) = \\
		= \sum_{i = 1}^n C_i (\dot x_i(t) - A(t) x_i(t) ) \equiv \sum_{i = 1}^n C_i f_i(t),
	\end{multline*}
	тобто лінійна комбінація $x(t) = \sum_{i = 1}^n C_i x_i(t)$ буде розв'язком системи
	\begin{equation*}
		\dot x(t) - A(t) x(t) \equiv \sum_{i = 1}^n C_i f_i(t) \quad i = \overline{1, n}.
	\end{equation*}
\end{proof}

\begin{property}
	Якщо комплексний вектор з дійсними елементами 
	\begin{equation*}
		x(t) = u(t) + i v(t) = \begin{pmatrix} u_1(t) & \cdots & u_n(t) \end{pmatrix}^T + i \begin{pmatrix} v_1(t) & \cdots & v_n(t) \end{pmatrix}^T
	\end{equation*}
	є розв'язком неоднорідної системи $\dot x = A(t) x + f(t)$, де 
	\begin{equation*}
		f(t) = p(t) + i q(t) = \begin{pmatrix} p_1(t) & \cdots & p_n(t) \end{pmatrix}^T + i \begin{pmatrix} q_1(t) & \cdots & q_n(t) \end{pmatrix}^T,
	\end{equation*}
	то окремо дійсна і уявна частини є розв'язками систем $\dot x = A(t) x + p(t)$ і $\dot x = A(t) x + q(t)$ відповідно.
\end{property}

\begin{proof}
	Дійсно, за умовою
	\begin{equation*}
		\frac{\diff}{\diff t} (u(t) + i v(t)) - A(t) (u(t) + i v(t) \equiv p(t) + i q(t).
	\end{equation*}

	Розкривши дужки і перетворивши, одержимо
	\begin{equation*}
		(\dot u(t) - A(t) u(t)) + i (\dot v(t) - A(t) v(t)) \equiv p(t) + i q(t).
	\end{equation*}
	
	Але комплексні вирази рівні між собою тоді і тільки тоді, коли рівні дійсні та уявні частини, що і було потрібно довести.
\end{proof}

\begin{theorem}[про загальний розв'язок лінійної неоднорідної системи]
	Загальний розв'язок лінійної неоднорідної системи складається із суми загального розв'язку однорідної системи і якого-небудь частинного розв'язку неоднорідної системи.
\end{theorem}

\begin{proof}
	Нехай $x(t) = \sum_{i = 1}^n C_i x_i(t)$ --- загальний розв'язок однорідної системи і $y(t)$ --- частинний розв'язок неоднорідної. Тоді, як випливає з властивості 1, їхня сума $x(t) + y(t)$ буде розв'язком неоднорідної системи. \parvskip

	Покажемо, що цей розв'язок загальний, тобто підбором сталих $C_i$, $i = \overline{1, n}$  можна розв'язати довільну задачу Коші
	\begin{equation*}
		x_1(t_0) = x_1^0, \quad x_2(t_0) = x_2^0, \quad \ldots, \quad x_n(t_0) = x_n^0.
	\end{equation*}

	Оскільки $x(t) = \sum_{i = 1}^n C_i x_i(t)$ --- загальний розв'язок однорідного рівняння, то вектори $x_1(t), \ldots, x_n(t)$ лінійно незалежні, $W[x_1, \ldots, x_n](t) \ne 0$ і система алгебраїчних рівнянь
	\begin{equation*}
		\left\{
			\begin{array}{rl}
				C_1 x_{11}(t_0) + C_2 x_{12}(t_0) + \ldots + C_n x_{1n}(t_0) &= x_1^0 - y_1(t_0), \\
				C_1 x_{21}(t_0) + C_2 x_{22}(t_0) + \ldots + C_n x_{2n}(t_0) &= x_2^0 - y_2(t_0), \\
				\hdotsfor{2} \\
				C_1 x_{n1}(t_0) + C_2 x_{n2}(t_0) + \ldots + C_n x_{nn}(t_0) &= x_n^0 - y_n(t_0)
			\end{array}
		\right.
	\end{equation*}
 	має єдиний розв'язок $C_i^0$, $i = \overline{1, n}$. І лінійна комбінація $z(t) = y(t) + \sum_{i = 1}^n C_i^0 x_i(t)$ з отриманими сталими є розв'язком поставленої задачі Коші.
\end{proof}

% 		\subsubsection{Побудова частинного розв'язку неоднорідної системи методом варіації довільних сталих \todo}
% 		Як випливає з останньої теореми, для побудови загального розв'язку неоднорідної системи потрібно розв'язати однорідну і яким-небудь засобом знайти частинний розв'язок неоднорідної системи. Розглянемо метод, який називається методом варіації довільної сталої. \parvskip

Нехай маємо систему
\begin{equation*}
 	\dot x = A(t) x + f(t)
\end{equation*}
і $x(t) = \sum_{i = 1}^n C_i x_i(t)$ --- загальний розв'язок однорідної системи. Розв'язок неоднорідної будемо шукати в такому ж вигляді, але вважати $C_i$ не сталими, а невідомими функціями, тобто $C_i = C_i(t)$ і 
\begin{equation*}
	x_{\text{hetero}}(t) = \sum_{i = 1}^n C_i(t) x_i(t),
\end{equation*}
чи в матричній формі
\begin{equation*}
	x_{\text{hetero}}(t) = X(t) C(t),
\end{equation*}
де $X(t)$ --- фундаментальна матриця розв'язків, $C(t)$ --- вектор з невідомих функцій. Підставивши в систему, одержимо
\begin{equation*}
	\frac{\diff}{\diff t} X(t) C(t) + X(t) \frac{\diff C(t)}{\diff t} = A(t) X(t) C(t) + f(t),
\end{equation*}
чи
\begin{equation*}
	\left( \frac{\diff}{\diff t} X(t) - A(t) X(t) \right) C(t) + X(t) \frac{\diff C(t)}{\diff t} = f(t).
\end{equation*}

Оскільки $X(t)$ --- фундаментальна матриця, тобто матриця складена з розв'язків, то
\begin{equation*}
	\frac{\diff}{\diff t} X(t) - A(t) X(t) \equiv 0
\end{equation*}
і залишається система рівнянь
\begin{equation*}
	X(t) C'(t) = f(t).
\end{equation*}

Розписавши покоординатно, одержимо
\begin{equation*}
	\left\{
		\begin{array}{rl}
			C_1' x_{11}(t) + C_2' x_{12}(t) + \ldots + C_n' x_{1n}(t) &= f_1(t), \\
			C_1' x_{21}(t) + C_2' x_{22}(t) + \ldots + C_n' x_{2n}(t) &= f_2(t), \\
			\hdotsfor{2} \\
			C_1' x_{n1}(t) + C_2' x_{n2}(t) + \ldots + C_n' x_{nn}(t) &= f_n(t).
		\end{array}
	\right.
\end{equation*}

Оскільки визначником системи є визначник Вронського і він не дорівнює нулю, то система має єдиний розв'язок і функції  визначаються в такий спосіб
\begin{equation*}
	\begin{array}{rl}
		C_1(t) &= \displaystyle \int \frac{\begin{vmatrix} f_1(t) & x_{12}(t) & \cdots & x_{1n}(t) \\ f_2(t) & x_{22}(t) & \cdots & x_{2n}(t) \\ \vdots & \vdots & \ddots & \vdots \\ f_n(t) & x_{n2}(t) & \cdots & x_{nn}(t) \end{vmatrix}}{W[x_1, \ldots, x_n](t)} \diff t, \\
		C_2(t) &= \displaystyle \int \frac{\begin{vmatrix} x_{11}(t) & f_1(t) & \cdots & x_{1n}(t) \\ x_{21}(t) & f_2(t) & \cdots & x_{2n}(t) \\ \vdots & \vdots & \ddots & \vdots \\ x_{n1}(t) & f_n(t) & \cdots & x_{nn}(t) \end{vmatrix}}{W[x_1, \ldots, x_n](t)} \diff t, \\
		\hdotsfor{2} \\
		C_n(t) &= \displaystyle \int \frac{\begin{vmatrix} x_{11}(t) & x_{12}(t) & \cdots & f_1(t) \\ x_{21}(t) & x_{22}(t) & \cdots & f_2(t) \\ \vdots & \vdots & \ddots & \vdots \\ x_{n1}(t) & x_{n2}(t) & \cdots & f_n(t) \end{vmatrix}}{W[x_1, \ldots, x_n](t)} \diff t.
	\end{array}
\end{equation*}

Звідси частинний розв'язок неоднорідної системи має вигляд
\begin{equation*}
	x_{\text{hetero}}(t) = \sum_{i = 1}^n C_i(t) x_i(t).
\end{equation*}

Для лінійної неоднорідної системи на площині 
\begin{equation*}
	\left\{
		\begin{aligned}
			\dot x_1 &= a_{11} x_1 + a_{12}(t) x_2 + f_1(t), \\
			\dot x_2 &= a_{21} x_1 + a_{22}(t) x_2 + f_2(t) 
		\end{aligned}
	\right.
\end{equation*}
метод варіації довільної сталої реалізується таким чином. \parvskip

Нехай
\begin{equation*}
	X(t) = \begin{pmatrix}
		x_{11}(t) & x_{12}(t) \\
		x_{21}(t) & x_{22}(t).
	\end{pmatrix}
\end{equation*}

Фундаментальна матриця розв'язків однорідної системи. Тоді частинний розв'язок неоднорідної шукається з системи
\begin{equation*}
	\left\{
		\begin{aligned}
			C_1' x_{11}(t) + C_2' x_{12}(t) &= f_1(t), \\
			C_2' x_{21}(t) + C_2' x_{22}(t) &= f_2(t).
		\end{aligned}
	\right.
\end{equation*}

Звідси
\begin{equation*}
	C_1(t) = \int \frac{\begin{vmatrix} f_1(t) & x_{12}(t) \\ f_2(t) & x_{22}(t) \end{vmatrix}}{\begin{vmatrix} x_{11}(t) & x_{12}(t) \\ x_{21}(t) & x_{22}(t) \end{vmatrix}}, \qquad C_2(t) = \int \frac{\begin{vmatrix} x_{11}(t) & f_1(t) \\ x_{21}(t) & f_2(t) \end{vmatrix}}{\begin{vmatrix} x_{11}(t) & x_{12}(t) \\ x_{21}(t) & x_{22}(t) \end{vmatrix}}
\end{equation*}
   
І загальний розв'язок має вигляд
\begin{equation*}
	\begin{pmatrix} x_1(t) \\ x_2(t) \end{pmatrix} = \begin{pmatrix} x_{11}(t) & x_{12}(t) \\ x_{21}(t) & x_{22}(t) \end{pmatrix} \begin{pmatrix} C_1 \\ C_2 \end{pmatrix} + \begin{pmatrix} x_{11}(t) & x_{12}(t) \\ x_{21}(t) & x_{22}(t) \end{pmatrix} \begin{pmatrix} C_1(t) \\ C_2(t) \end{pmatrix},
\end{equation*}
де $C_1, C_2$ --- довільні сталі.


% 		\subsubsection{Формула Коші \todo}
% 		Нехай $X(t, t_0)$ --- фундаментальна система, нормована при $t = t_0$ тобто $X(t_0, t_0) = E$, де $E$ --- одинична матриця. Загальний розв'язок однорідної системи має вигляд
\begin{equation*}
	x(t) = X(t, t_0) C.
\end{equation*}

Вважаючи $C$ невідомою вектором-функцією і повторюючи викладення методу варіації довільної постійний, одержимо
\begin{equation*}
	X(t, t_0) C'(t) = f(t).
\end{equation*}

Звідси
\begin{equation*}
	\frac{\diff C(t)}{\diff t} = X^{-1}(t, t_0) f(t).
\end{equation*}

Проінтегруємо отриманий вираз
\begin{equation*}
	C(t) = C + \int_{t_0}^t X^{-1}(\tau, t_0) f(\tau) \diff \tau.
\end{equation*}

Тут $C$ --- вектор із сталих, що отриманий при інтегруванні системи. Підставивши у вихідний вираз, одержимо:
\begin{multline*}
	x(t) = X(t, t_0) \left( C + \int_{t_0}^t X^{-1}(\tau, t_0) f(\tau) \diff \tau \right) = \\
	= X(t, t_0) C + \int_{t_0}^t X(t, t_0) X^{-1}(\tau, t_0) f(\tau) \diff \tau 
\end{multline*}
  
Якщо $X(t, t_0)$ --- фундаментальна матриця, нормована при $t = t_0$, то $X(t, t_0) = X(t) X^{-1} (t_0)$. Звідси
\begin{equation*}
	X(t, t_0) X^{-1}(\tau, t_0) = X(t) X^{-1}(t_0) \left( X(\tau) X^{-1}(t_0) \right)^{-1} = X(t) X^{-1} (\tau) = X(t, \tau).
\end{equation*}
 
Підставивши початкові значення $x(t_0 = x_0)$ і з огляду на те, що фундаметнальна матриця нормована, тобто $X(t_0, t_0) = E$, одержимо
\begin{equation*}
	x(t) = X(t, t_0) x_0 + \int_{t_0}^t X(t, \tau) f(\tau) \diff \tau.
\end{equation*}

Отримана формула називається формулою Коші загального розв'язку неоднорідного рівняння. \parvskip

Частинний розв'язок неоднорідного рівняння, що задовольняє нульовій початковій умові, має вид
\begin{equation*}
	x_{\text{hetero}}(t) = \int_{t_0}^t X(t, \tau) f(\tau) \diff \tau.
\end{equation*}

Якщо система з сталою матрицею $A$, то
\begin{equation*}
	X(t, t_0) = X(t - t_0), \qquad X(t, \tau) = X(t - \tau).
\end{equation*}

І формула Коші має вигляд
\begin{equation*}
	x(t) = X(t - t_0) x_0 + \int_{t_0}^t X(t - \tau) f(\tau) \diff \tau.
\end{equation*}


% 		\subsubsection{Метод невизначених коефіцієнтів \todo}
% 		Якщо система лінійних диференціальних рівнянь з сталими коефіцієнтами, а векторна функція $f(t)$ спеціального виду, то частинний розв'язок можна знайти методом невизначених коефіцієнтів. Доведення існування частинного розв'язку зазначеного виду аналогічно доведенню для лінійних рівнянь вищих порядків.

\begin{enumerate}
	\item Нехай кожна з компонент вектора $f(x)$ є многочленом степеню не більш ніж $s$, тобто
	\begin{equation*}
		\begin{pmatrix} f_1(t) \\ f_2(t) \\ \vdots \\ f_n(t) \end{pmatrix} =
		\begin{pmatrix} A_0^1 t^s + A_1^1 t^{s - 1} + \ldots + A_{s - 1}^1 t + A_s^1 \\ A_0^2 t^s + A_1^2 t^{s - 1} + \ldots + A_{s - 1}^2 t + A_s^2 \\ \vdots \\ A_0^n t^s + A_1^n t^{s - 1} + \ldots + A_{s - 1}^n t + A_s^n \end{pmatrix}.
	\end{equation*}

	\begin{enumerate}
		\item Якщо характеристичне рівняння не має нульового кореня, тобто $\lambda_i \ne 0$, $i = \overline{1, n}$, то частинний розв'язок шукається в такому ж вигляді, тобто
		\begin{equation*}
			\begin{pmatrix} x_1(t) \\ x_2(t) \\ \vdots \\ x_n(t) \end{pmatrix} =
			\begin{pmatrix} B_0^1 t^s + B_1^1 t^{s - 1} + \ldots + B_{s - 1}^1 t + B_s^1 \\ B_0^2 t^s + B_1^2 t^{s - 1} + \ldots + B_{s - 1}^2 t + B_s^2 \\ \vdots \\ B_0^n t^s + B_1^n t^{s - 1} + \ldots + B_{s - 1}^n t + B_s^n \end{pmatrix}.
		\end{equation*}

		\item Якщо характеристичне рівняння має нульовий корінь кратності $r$, тобто $\lambda_1 = \lambda_2 = \ldots = \lambda_r = 0$, то частинний розв'язок шукається у вигляді многочлена степеню $s + r$, тобто
		\begin{equation*}
			\begin{pmatrix} x_1(t) \\ x_2(t) \\ \vdots \\ x_n(t) \end{pmatrix} =
			\begin{pmatrix} B_0^1 t^{s + r} + B_1^1 t^{s + r - 1} + \ldots + B_{s + r - 1}^1 t + B_{s + r}^1 \\ B_0^2 t^{s + r} + B_1^2 t^{s + r - 1} + \ldots + B_{s + r - 1}^2 t + B_{s + r}^2 \\ \vdots \\ B_0^n t^{s + r} + B_1^n t^{s + r - 1} + \ldots + B_{s + r - 1}^n t + B_{s + r}^n \end{pmatrix}.
		\end{equation*}

		Причому перші $(s + 1) n$ коефіцієнти $B_i^j$, $i = \overline{0, s}$, $j = \overline{1, n}$ знаходяться точно, а інші $r n$ --- з точністю до сталих інтегрування $C_1, \ldots, C_n$, що входять у загальний розв'язок однорідних систем.
	\end{enumerate}

	\item Нехай $f(t)$ має вид
	\begin{equation*}
		\begin{pmatrix} f_1(t) \\ f_2(t) \\ \vdots \\ f_n(t) \end{pmatrix} =
		\begin{pmatrix} e^{pt} (A_0^1 t^s + A_1^1 t^{s - 1} + \ldots + A_{s - 1}^1 t + A_s^1) \\ e^{pt} (A_0^2 t^s + A_1^2 t^{s - 1} + \ldots + A_{s - 1}^2 t + A_s^2) \\ \vdots \\ e^{pt} (A_0^n t^s + A_1^n t^{s - 1} + \ldots + A_{s - 1}^n t + A_s^n) \end{pmatrix}.
	\end{equation*}

	\begin{enumerate}
		\item Якщо характеристичне рівняння не має коренем значення $p$, тобто $\lambda_i \ne p$, $i = \overline{1, n}$, то частинний розв'язок шукається в такому ж вигляді, тобто
		\begin{equation*}
			\begin{pmatrix} x_1(t) \\ x_2(t) \\ \vdots \\ x_n(t) \end{pmatrix} =
			\begin{pmatrix} e^{pt} (B_0^1 t^s + B_1^1 t^{s - 1} + \ldots + B_{s - 1}^1 t + B_s^1) \\ e^{pt} (B_0^2 t^s + B_1^2 t^{s - 1} + \ldots + B_{s - 1}^2 t + B_s^2) \\ \vdots \\ e^{pt} (B_0^n t^s + B_1^n t^{s - 1} + \ldots + B_{s - 1}^n t + B_s^n) \end{pmatrix}.
		\end{equation*}

		\item Якщо $p$ є коренем характеристичного рівняння кратності $r$, тобто $\lambda_1 = \lambda_2 = \ldots = \lambda_r = p$, то частинний розв'язок шукається у вигляді
		\begin{equation*}
			\begin{pmatrix} x_1(t) \\ x_2(t) \\ \vdots \\ x_n(t) \end{pmatrix} =
			\begin{pmatrix} e^{pt} (B_0^1 t^{s + r} + B_1^1 t^{s + r - 1} + \ldots + B_{s + r - 1}^1 t + B_{s + r}^1) \\ e^{pt} (B_0^2 t^{s + r} + B_1^2 t^{s + r - 1} + \ldots + B_{s + r - 1}^2 t + B_{s + r}^2) \\ \vdots \\ e^{pt} (B_0^n t^{s + r} + B_1^n t^{s + r - 1} + \ldots + B_{s + r - 1}^n t + B_{s + r}^n) \end{pmatrix}.
		\end{equation*}

		І, як у попередньому пункті, перші $(s + 1) n$ коефіцієнти $B_i^j$, $i = \overline{0, s}$, $j = \overline{1, n}$, а інші з точністю до сталих інтегрування $C_1, \ldots, C_n$.
	\end{enumerate}
	
	\item Нехай $f(t)$ має вигляд:
	\begin{multline*}
		\begin{pmatrix} f_1(t) \\ f_2(t) \\ \vdots \\ f_n(t) \end{pmatrix} =
		\begin{pmatrix} e^{pt} (A_0^1 t^s + A_1^1 t^{s - 1} + \ldots + A_{s - 1}^1 t + A_s^1) \cos qt \\ e^{pt} (A_0^2 t^s + A_1^2 t^{s - 1} + \ldots + A_{s - 1}^2 t + A_s^2) \cos qt \\ \vdots \\ e^{pt} (A_0^n t^s + A_1^n t^{s - 1} + \ldots + A_{s - 1}^n t + A_s^n) \cos qt \end{pmatrix} + \\
		+ \begin{pmatrix} e^{pt} (B_0^1 t^s + B_1^1 t^{s - 1} + \ldots + B_{s - 1}^1 t + B_s^1) \sin qt \\ e^{pt} (B_0^2 t^s + B_1^2 t^{s - 1} + \ldots + B_{s - 1}^2 t + B_s^2) \sin qt \\ \vdots \\ e^{pt} (B_0^n t^s + B_1^n t^{s - 1} + \ldots + B_{s - 1}^n t + B_s^n) \sin qt \end{pmatrix}.
	\end{multline*}

 
	\begin{enumerate}
		\item Якщо характеристичне рівняння не має коренем значення $p \pm i q$, то частинний розв'язок шукається в такому ж вигляді, тобто
		\begin{multline*}
			\begin{pmatrix} x_1(t) \\ x_2(t) \\ \vdots \\ x_n(t) \end{pmatrix} =
			\begin{pmatrix} e^{pt} (C_0^1 t^s + C_1^1 t^{s - 1} + \ldots + C_{s - 1}^1 t + C_s^1) \cos qt \\ e^{pt} (C_0^2 t^s + C_1^2 t^{s - 1} + \ldots + C_{s - 1}^2 t + C_s^2) \cos qt \\ \vdots \\ e^{pt} (C_0^n t^s + C_1^n t^{s - 1} + \ldots + C_{s - 1}^n t + C_s^n) \cos qt \end{pmatrix} + \\
			+ \begin{pmatrix} e^{pt} (D_0^1 t^s + D_1^1 t^{s - 1} + \ldots + D_{s - 1}^1 t + D_s^1) \sin qt \\ e^{pt} (D_0^2 t^s + D_1^2 t^{s - 1} + \ldots + D_{s - 1}^2 t + D_s^2) \sin qt \\ \vdots \\ e^{pt} (D_0^n t^s + D_1^n t^{s - 1} + \ldots + D_{s - 1}^n t + D_s^n) \sin qt \end{pmatrix}.
		\end{multline*}
 
		\item Якщо $p \pm iq$ є коренем характеристичного рівняння кратності $r$, то частинний розв'язок має вигляд
		\begin{multline*}
			\begin{pmatrix} x_1(t) \\ x_2(t) \\ \vdots \\ x_n(t) \end{pmatrix} = \\
			= \begin{pmatrix} e^{pt} (C_0^1 t^{s + r} + C_1^1 t^{s + r - 1} + \ldots + C_{s + r - 1}^1 t + C_{s + r}^1) \cos qt \\ e^{pt} (C_0^2 t^{s + r} + C_1^2 t^{s + r - 1} + \ldots + C_{s + r - 1}^2 t + C_{s + r}^2) \cos qt \\ \vdots \\ e^{pt} (C_0^n t^{s + r} + C_1^n t^{s + r - 1} + \ldots + C_{s + r - 1}^n t + C_{s + r}^n) \cos qt \end{pmatrix} + \\
			+ \begin{pmatrix} e^{pt} (D_0^1 t^{s + r} + D_1^1 t^{s + r - 1} + \ldots + D_{s + r - 1}^1 t + D_{s + r}^1) \sin qt \\ e^{pt} (D_0^2 t^{s + r} + D_1^2 t^{s + r - 1} + \ldots + D_{s + r - 1}^2 t + D_{s + r}^2) \sin qt \\ \vdots \\ e^{pt} (D_0^n t^{s + r} + D_1^n t^{s + r - 1} + \ldots + D_{s + r - 1}^n t + D_{s + r}^n) \sin qt \end{pmatrix}.
		\end{multline*}
	\end{enumerate} 
\end{enumerate}

% 		\subsubsection{Вправи для самостійної роботи \todo}
% 		\begin{example}
	Розв'язати систему неоднорідних рівнянь методом варіації довільної сталої
	\begin{equation*}
		\left\{
			\begin{aligned}
				\dot x &= - 4 x - 2 y + \frac{2}{e^t - 1}, \\
				\dot y &= 6 x + 3 y - \frac{3}{e^t - 1}.
			\end{aligned}
		\right.
	\end{equation*}
\end{example}

\begin{solution}
	Розв'язуємо спочатку однорідну систему. Її характеристичне рівняння має вигляд
	\begin{equation*}
		\det (A - \lambda E) = \begin{vmatrix} - 4 - \lambda & -2 \\ 6 & 3 - \lambda \end{vmatrix} = \lambda^2 + \lambda = 0 \implies \lambda_1 = 0, \lambda_2 = -1.
	\end{equation*}

	Розв'язуємо (наприклад) матричним методом. Маємо
	\begin{equation*}
		\Lambda = \begin{pmatrix}
			0 & 0 \\ 0 & -1
		\end{pmatrix}, \quad 
		e^{\Lambda t} = \begin{pmatrix}
			1 & 0 \\ 0 & e^{-t}
		\end{pmatrix}
	\end{equation*}

	Матричне рівняння $A S = S \Lambda$ має вигляд
	\begin{equation*}
		\begin{pmatrix} -4 & -2 \\ 6 & 3 \end{pmatrix} \begin{pmatrix} a_1^1 & a_1^2 \\ a_2^1 & a_2^2 \end{pmatrix} = \begin{pmatrix} a_1^1 & a_1^2 \\ a_2^1 & a_2^2 \end{pmatrix} \begin{pmatrix} 0 & 0 \\ 0 & -1 \end{pmatrix}.
	\end{equation*}

	Звідси маємо дві системи рівнянь
	\begin{equation*}
		\left\{
			\begin{aligned}
				- 4 a_1^1 - 2 a_2^1 &= 0, \\
				6 a_1^1 + 3 a_2^1 &= 0,
			\end{aligned}
		\right.
		\qquad
		\left\{
			\begin{aligned}
				- 4 a_1^2 - 2 a_2^2 &= -a_1^2, \\
				6 a_1^2 + 3 a_2^2 &= -a_2^2,
			\end{aligned}
		\right.
	\end{equation*}

	Їх розв'язками будуть
	\begin{equation*}
		a_1^1 = 1, \quad a_2^1 = -2, \quad a_1^2 = -2, \quad a_2^2 = 3.
	\end{equation*}

	І розв'язок однорідної системи має вигляд
	\begin{equation*}
		\begin{pmatrix} x_1(t) \\ x_2(t) \end{pmatrix} = \begin{pmatrix} 1 & -2 \\ -2 & 3 \end{pmatrix} \begin{pmatrix} 1 & 0 \\ 0 & e^t \end{pmatrix} \begin{pmatrix} C_1 \\ C_2 \end{pmatrix} = \begin{pmatrix} 1 & -2e^{-t} \\ -2 & 3e^{-t} \end{pmatrix} \begin{pmatrix} C_1 \\ C_2 \end{pmatrix}.
	\end{equation*}

	Частинний розв'язок неоднорідної системи має вигляд
	\begin{equation*}
		\begin{pmatrix} x_1(t) \\ x_2(t) \end{pmatrix} = \begin{pmatrix} 1 & -2e^{-t} \\ -2 & 3e^{-t} \end{pmatrix} \begin{pmatrix} C_1(t) \\ C_2(t) \end{pmatrix}.
	\end{equation*}

	Функції $C_1(t), C_2(t)$ задовольняють системі рівнянь
	\begin{equation*}
		\left\{
			\begin{aligned}
				C_1'(t) - 2 C_2(r) e^{-t} = \frac{2}{e^t - 1}, \\
				-2 C_1'(t) + 3 C_2(r) e^{-t} = - \frac{3}{e^t - 1}.
			\end{aligned}
		\right.
	\end{equation*}

	Звідси
	\begin{align*}
		C_1(t) &= \int \frac{\begin{vmatrix} \rfrac{2}{e^t - 1} & - 2 e^{-t} \\ \rfrac{-3}{e^t - 1} & 3 e^{-t} \end{vmatrix}}{\begin{vmatrix} 1 & - 2 e^{-t} \\ -2 & 3 e^{-t} \end{vmatrix}} \diff t = 0 + \bar C_1, \\
		C_2(t) &= \int \frac{\begin{vmatrix} 1 & \rfrac{2}{e^t - 1} \\ - 2 & \rfrac{-3}{e^t - 1} \end{vmatrix}}{\begin{vmatrix} 1 & - 2 e^{-t} \\ -2 & 3 e^{-t} \end{vmatrix}} \diff t =\int \frac{\frac{1}{e^t - 1}}{- e^{-t}} \diff t = - \int \frac{e^t}{e^t - 1} \diff t = \\
		&= - \ln |e^t - 1| + \bar C_2.
	\end{align*}

	Поклавши $\bar C_1 = \bar C_2 = 0$, одержуємо $C_1(t) \equiv 0$, $C_2(t) = - \ln |e^t - 1|$. Таким чином, частинний розв'язок має вигляд
	\begin{equation*}
		\begin{pmatrix} x_1(t) \\ x_2(t) \end{pmatrix} = \begin{pmatrix} 1 & -2e^{-t} \\ -2 & 3e^{-t} \end{pmatrix} \begin{pmatrix} 0 \\ - \ln |e^t - 1| \end{pmatrix} = \begin{pmatrix} 2 e^{-t} \ln |e^t - 1| \\ - 3 e^{-t} \ln |e^t - 1| \end{pmatrix}
	\end{equation*}

	А загальний розв'язок 
	\begin{equation*}
		\begin{pmatrix} x_1(t) \\ x_2(t) \end{pmatrix} = \begin{pmatrix} 1 & -2e^{-t} \\ -2 & 3e^{-t} \end{pmatrix} \begin{pmatrix} C_1 \\ C_2 \end{pmatrix} + \begin{pmatrix} 2 e^{-t} \ln |e^t - 1| \\ - 3 e^{-t} \ln |e^t - 1| \end{pmatrix}
	\end{equation*}
\end{solution}

\begin{example}
	Розв'язати систему неоднорідних рівнянь за допомогою формули Коші
	\begin{equation*}
		\left\{
			\begin{aligned}
				\dot x &= - x + 2 y, \\
				\dot y &= 3 x + 4 y + \frac{e^{3 t}}{e^{2 t} + 1}.
			\end{aligned}
		\right.
	\end{equation*}
\end{example}

\begin{solution}
	Розв'язуємо спочатку однорідну систему. Характеристичне рівняння має вигляд
	\begin{equation*}
		\det (A - \lambda E) = \begin{vmatrix} - 1 - \lambda & 2 \\ 3 & 4 - \lambda \end{vmatrix} = \lambda^2 - 3 \lambda + 2 = 0 \implies \lambda_1 = 1, \lambda_2 = -1.
	\end{equation*}

	Розв'язуємо матричним методом. Маємо
	\begin{equation*}
		\Lambda = \begin{pmatrix}
			1 & 0 \\ 0 & 2
		\end{pmatrix}, \quad 
		e^{\Lambda t} = \begin{pmatrix}
			e^t & 0 \\ 0 & e^{2t}
		\end{pmatrix}
	\end{equation*}

	Матричне рівняння $A S = S \Lambda$ має вигляд
	\begin{equation*}
		\begin{pmatrix} -1 & 2 \\ 3 & 4 \end{pmatrix} \begin{pmatrix} a_1^1 & a_1^2 \\ a_2^1 & a_2^2 \end{pmatrix} = \begin{pmatrix} a_1^1 & a_1^2 \\ a_2^1 & a_2^2 \end{pmatrix} \begin{pmatrix} 1 & 0 \\ 0 & 2 \end{pmatrix}.
	\end{equation*}
 
	Одержуємо дві системи 
	\begin{equation*}
		\left\{
			\begin{aligned}
				- a_1^1 + 2 a_2^1 &= a_1^1, \\
				3 a_1^1 + 4 a_2^1 &= a_2^1,
			\end{aligned}
		\right.
		\qquad
		\left\{
			\begin{aligned}
				- a_1^2 + 2 a_2^2 &= 2 a_1^2, \\
				3 a_1^2 + 4 a_2^2 &= 2 a_2^2,
			\end{aligned}
		\right.
	\end{equation*}

	Їх розв'язками будуть
	\begin{equation*}
		a_1^1 = 1, \quad a_2^1 = 1, \quad a_1^2 = 2, \quad a_2^2 = 3.
	\end{equation*}

	І розв'язок однорідної системи має вигляд
	\begin{equation*}
		\begin{pmatrix} x_1(t) \\ x_2(t) \end{pmatrix} = \begin{pmatrix} 1 & 2 \\ 1 & 3 \end{pmatrix} \begin{pmatrix} e^t & 0 \\ 0 & e^{2 t} \end{pmatrix} \begin{pmatrix} C_1 \\ C_2 \end{pmatrix} = \begin{pmatrix} e^t & 2 e^{2 t} \\ e^t & 3 e^{2t} \end{pmatrix} \begin{pmatrix} C_1 \\ C_2 \end{pmatrix}.
	\end{equation*}

	Фундаментальна матриця лінійної однорідної системи, нормована в точці $t = 0$, має вигляд
	\begin{equation*}
		X(t) = \begin{pmatrix} e^t & 2 e^{2 t} \\ e^t & 3 e^{2t} \end{pmatrix} \begin{pmatrix} 1 & 2 \\ 1 & 3 \end{pmatrix}^{-1} = \begin{pmatrix} (3 - 2 e^t) e^t & -2 (1 - e^t) e^t \\ 3 (1 - e^t) e^t & (-2 + 3 e^t) e^t \end{pmatrix}.
	\end{equation*}

	Використовуючи формулу Коші, одержуємо частинний розв'язок, який задовольняє нульовим початковим умовам
	\begin{align*}
		\begin{pmatrix} x_1(t) \\ x_2(t) \end{pmatrix} &= \int_0^t \begin{pmatrix} (3 - 2 e^s) e^s & -2 (1 - e^s) e^s \\ 3 (1 - e^s) e^s & (-2 + 3 e^s) e^s \end{pmatrix} \begin{pmatrix} C_1 \\ C_2 \end{pmatrix} \diff s = \\
		&= \begin{pmatrix} \displaystyle \int_0^t \frac{-2 (1 - e^{t - s}) e^{t + 2 s}}{e^{2 s}} \diff s \\ \\ \displaystyle  \int_0^t \frac{(-2 +3 e^{t - s}) e^{t + 2 s}}{e^{2 s}} \diff s \end{pmatrix} = \\
		&= \begin{pmatrix} - 2 e^t \displaystyle \int_0^t \frac{e^{2 s}}{e^{2 s}} \diff s + 2 e^{2 t} \displaystyle \int_0^t \frac{e^{2 s}}{e^{2 s}} \diff s \\ \\ - 2 e^t \displaystyle \int_0^t \frac{e^{2 s}}{e^{2 s}} \diff s + 3 e^{2 t} \displaystyle \int_0^t \frac{e^{2 s}}{e^{2 s}} \diff s \end{pmatrix} = \\
		&= \left. \begin{pmatrix} -e^t \ln |e^{2 s} + 1| + 2 e^{2 t} \arctan e^s \\ -e^t \ln |e^{2 s} + 1| + 3 e^{2 t} \arctan e^s \end{pmatrix} \right|_{s = 0}^{s = t} = \\
		&= \begin{pmatrix} -e^t (\ln |e^{2 t} + 1| - \ln 2) + 2 e^{2 t} \left( \arctan e^t - \frac\pi4 \right) \\ -e^t (\ln |e^{2 t} + 1| - \ln 2) + 3 e^{2 t} \left( \arctan e^t - \frac\pi4 \right) \end{pmatrix}.
	\end{align*}

	І загальний розв'язок системи у формі Коші має вигляд
	\begin{multline*}
		\begin{pmatrix} x_1(t) \\ x_2(t) \end{pmatrix} = \begin{pmatrix} (3 - 2 e^t) e^t & -2 (1 - e^t) e^t \\ 3 (1 - e^t) e^t & (-2 + 3 e^t) e^t \end{pmatrix} \begin{pmatrix} x_1(0) \\ x_2(0) \end{pmatrix} + \\
		+ \begin{pmatrix} -e^t (\ln |e^{2 t} + 1| - \ln 2) + 2 e^{2 t} \left( \arctan e^t - \frac\pi4 \right) \\ -e^t (\ln |e^{2 t} + 1| - \ln 2) + 3 e^{2 t} \left( \arctan e^t - \frac\pi4 \right) \end{pmatrix}.
	\end{multline*}
\end{solution}

\begin{remark}
	Якщо шукати розв'язок не в формі Коші, то він має більш простіший вигляд
	\begin{equation*}
		\begin{pmatrix} x_1(t) \\ x_2(t) \end{pmatrix} = \begin{pmatrix} e^t & 2 e^t \\ e^t & 3 e^t \end{pmatrix} \begin{pmatrix} C_1 \\ C_2 \end{pmatrix} + \begin{pmatrix} -e^t \ln |e^{2 t} + 1| + 2 e^{2 t} \arctan e^t \\ -e^t \ln |e^{2 t} + 1| + 3 e^{2 t} \arctan e^t \end{pmatrix}.
	\end{equation*}
\end{remark}

\begin{example}
	Знайти загальний розв'язок системи лінійних неоднорідних рівнянь за допомогою методу невизначених коефіцієнтів:
	\begin{equation*}
		\left\{
			\begin{aligned}
				\dot x_1 &= x_2, \\
				\dot x_2 &= x_1 + t.
			\end{aligned}
		\right.
	\end{equation*}
\end{example}

\begin{solution}
	Складаємо характеристичне рівняння
	\begin{equation*}
		\det (A - \lambda E) = \begin{vmatrix} - \lambda & 1 \\ 1 & - \lambda \end{vmatrix} = \lambda^2 - 1 = 0 \implies \lambda_1 = 1, \lambda_2 = -1.
	\end{equation*}

	Оскільки рівняння не містить нульових коренів, частинний розв'язок шукаємо у вигляді
	\begin{equation*}
		\begin{pmatrix} x_1(t) \\ x_2(t) \end{pmatrix} = \begin{pmatrix} a t + b \\ c t + d \end{pmatrix}.
	\end{equation*}

	Підставивши в систему, отримаємо
	\begin{equation*}
		\left\{
			\begin{aligned}
				a &= c t + d, \\
				c &= a t + b + t.
			\end{aligned}
		\right.
	\end{equation*}

	Прирівнявши коефіцієнти при членах з однаковими степенями, отримаємо
	\begin{equation*}
		0 = c, \quad 0 = a + 1, \quad a = d, \quad c = b.
	\end{equation*}

	Звідси $a = -1$, $b = c = 0$, $d = -1$. І частинний розв'язок має вигляд
	\begin{equation*}
		\begin{pmatrix} x_1(t) \\ x_2(t) \end{pmatrix} = \begin{pmatrix} - t \\ - 1 \end{pmatrix}.
	\end{equation*}
\end{solution}

\begin{example}
	Знайти загальний розв'язок системи лінійних неоднорідних рівнянь за допомогою методу невизначених коефіцієнтів:
	\begin{equation*}
		\left\{
			\begin{aligned}
				\dot x_1 &= x_1 + 2 x_2, \\
				\dot x_2 &= 2 x_1 + 4 x_2 + t.
			\end{aligned}
		\right.
	\end{equation*}
\end{example}

\begin{solution}
	Складаємо характеристичне рівняння
	\begin{equation*}
		\det (A - \lambda E) = \begin{vmatrix} 1 - \lambda & 2 \\ 2 & 4 - \lambda \end{vmatrix} = \lambda^2 - 5 \lambda = 0 \implies \lambda_1 = 0, \lambda_2 = 5.
	\end{equation*}

	Оскільки є один нульовий корінь, то частинний розв'язок шукаємо у вигляді
	\begin{equation*}
		\begin{pmatrix} x_1(t) \\ x_2(t) \end{pmatrix} = \begin{pmatrix} a t^2 + b t + c \\ d t^2 + e t + f \end{pmatrix}.
	\end{equation*}

	Підставляємо в неоднорідну систему
	\begin{equation*}
		\left\{
			\begin{aligned}
				2 a t + b &= a t^2 + b t + c + 2 (d t^2 + e t + f), \\
				2 d t + e &= 2 (a t^2 + b t + c) + 4 (d t^2 + e t + f) + t.
			\end{aligned}
		\right.
	\end{equation*}

	Прирівнюємо коефіцієнти при членах з однаковими степенями.
	\begin{equation*}
		\left\{
			\begin{aligned}
				0 &= a + 2 d, \\
				0 &= 2 a + 4 d, 
			\end{aligned}
		\right. \qquad \left\{
			\begin{aligned}
				2 a &= b + 2 e, \\
				2 d &= 2 b + 4 e + 1, 
			\end{aligned}
		\right. \qquad \left\{
			\begin{aligned}
				b &= c + 2 f, \\
				e &= 2 c + 4 f.
			\end{aligned}
		\right.
	\end{equation*}

	Помноживши перше рівняння у другій підсистемі на мінус два і склавши з другим рівнянням, одержуємо $-4 a + 2 d = 1$. Разом з першим рівнянням першої системи маємо
	\begin{equation*}
		\left\{
			\begin{aligned}
				a + 2 d &= 0, \\
				- 4 a + 2 d &= 1.
			\end{aligned}
		\right.
	\end{equation*}

	Звідси $a = - 1 / 5, d = - 1 / 10$. І перше рівняння другої підсистеми має вигляд*
	\begin{equation*}
		b - 2 e = - 2 / 5.
	\end{equation*}

	Помноживши перше рівняння останньої підсистеми на два і віднявши друге рівняння, маємо
	\begin{equation*}
		2 b - e = 0.
	\end{equation*}

	З одержаних двох рівнянь дістаємо $b = - 2 / 25, e = - 4 / 25$. І остання підсистема дає співвідношення $c = - 2 / 25 - 2 f$. Таким чином частинний розв'язок має вигляд
	\begin{equation*}
		\begin{pmatrix} x_1(t) \\ x_2(t) \end{pmatrix} = \begin{pmatrix} - t^2 / 5 - 2 t / 25 - 2 / 25 - 2 f \\ - t^2 / 10 - 4 t / 25 + f \end{pmatrix}.
	\end{equation*}

	Стала $f$ входить в загальний розв'язок однорідної системи і точно не визначається. Поклавши $f = 0$, одержуємо
	\begin{equation*}
		\begin{pmatrix} x_1(t) \\ x_2(t) \end{pmatrix} = \begin{pmatrix} - t^2 / 5 - 2 t / 25 - 2 / 25 \\ - t^2 / 10 - 4 t / 25 \end{pmatrix}.
	\end{equation*}
\end{solution}

\begin{example}
	Знайти частинний розв'язок системи за допомогою методу невизначених коефіцієнтів:
	\begin{equation*}
		\left\{
			\begin{aligned}
				\dot x_1 &= x_2 + e^t, \\
				\dot x_2 &= - x_1 + t e^t.
			\end{aligned}
		\right.
	\end{equation*}
\end{example}

\begin{solution}
	Складаємо характеристичне рівняння однорідної системи
	\begin{equation*}
		\det (A - \lambda E) = \begin{vmatrix} - \lambda & 1 \\ -1 & - \lambda \end{vmatrix} = \lambda^2 + 1 = 0 \implies \lambda_{1, 2} = \pm i.
	\end{equation*}

	Оскільки одиниця не є коренем, то частинний розв'язок шукаємо у вигляді
	\begin{equation*}
		\begin{pmatrix} x_1(t) \\ x_2(t) \end{pmatrix} = \begin{pmatrix} (a t + b) e^t \\ (c t + d) e^t \end{pmatrix}.
	\end{equation*}

	Підставляємо в неоднорідну систему, одержуємо
	\begin{equation*}
		\left\{
			\begin{aligned}
				a e^t + (a t + b) e^t &= (c t + d) e^t + e^t, \\
				c e^t + (c t + d) e^t &= - (a t + b) e^t + t e^t.
			\end{aligned}
		\right.
	\end{equation*}

	Прирівнюємо коефіцієнти при однакових членах, одержуємо
	\begin{equation*}
		\left\{
			\begin{aligned}
				a &= c, \\
				c &= - a + 1, 
			\end{aligned}
		\right. \qquad \left\{
			\begin{aligned}
				a + b &= d + 1, \\
				e &= 2 c + 4 f.
			\end{aligned}
		\right.
	\end{equation*}

	Розв'язавши, одержуємо: $b = 0, a = c = d = 1 / 2$. Таким чином частинний розв'язок має вигляд
	\begin{equation*}
		\begin{pmatrix} x_1(t) \\ x_2(t) \end{pmatrix} = \begin{pmatrix} t e^t / 2 \\ (t + 1) e^t / 2 \end{pmatrix}.
	\end{equation*}
\end{solution}

\begin{example}
	Знайти частинний розв'язок системи за допомогою методу невизначених коефіцієнтів:
	\begin{equation*}
		\left\{
			\begin{aligned}
				\dot x_1 &= x_2 + e^t, \\
				\dot x_2 &= x_1 + t e^t.
			\end{aligned}
		\right.
	\end{equation*}
\end{example}

\begin{solution}
	Складаємо характеристичне рівняння
	\begin{equation*}
		\det (A - \lambda E) = \begin{vmatrix} - \lambda & 1 \\ 1 & - \lambda \end{vmatrix} = \lambda^2 - 1 = 0 \implies \lambda_1 = 1, \lambda_2 = -1.
	\end{equation*}

	Оскільки характеристичне рівняння має коренем одиницю кратності \allowbreak один, то частинний роз\-в'я\-зок шукаємо у вигляді
	\begin{equation*}
		\begin{pmatrix} x_1(t) \\ x_2(t) \end{pmatrix} = \begin{pmatrix} (a t^2 + b t + c) e^t \\ (d t^2 + e t + f) e^t \end{pmatrix}.
	\end{equation*}

	Підставляємо в неоднорідну систему, одержуємо
	\begin{equation*}
		\left\{
			\begin{aligned}
				(2 a t + b) e^t + (a t^2 + b t + c) e^t &= (d t^2 + e t + f) e^t + e^t, \\
				(2 d t + e) e^t + (d t^2 + e t + f) e^t &= (a t^2 + b t + c) e^t + t e^t.
			\end{aligned}
		\right.
	\end{equation*}

	Прирівнюємо коефіцієнти при однакових членах, одержуємо
	\begin{equation*}
		\left\{
			\begin{aligned}
				a &= d, \\
				d &= a,
			\end{aligned}
		\right. \qquad \left\{
			\begin{aligned}
				2 a + b &= e, \\
				2 d + e &= b + 1,
			\end{aligned}
		\right. \qquad \left\{
			\begin{aligned}
				b + c &= f + 1, \\
				e + f &= c.
			\end{aligned}
		\right.
	\end{equation*}

	З першої підсистеми одержуємо $a = d$. Підставляємо в другу
	\begin{equation*}
		\left\{
			\begin{aligned}
				2 a + b - e&= 0, \\
				2 a + e - b &= 1,
			\end{aligned}
		\right.
	\end{equation*}

	Склавши два рівняння, одержуємо: $a = 1 / 4$, $b - e = - 1 / 2$. Склавши два рівняння останньої підсистеми, маємо $b + e = 1$. Звідси  $b = 1 / 4, e = 3 / 4$, $f - c = 3 / 4$. Таким чином частинний розв'язок має вигляд
	\begin{equation*}
		\begin{pmatrix} x_1(t) \\ x_2(t) \end{pmatrix} = \begin{pmatrix} (t^2 / 4 + t / 4 + c) e^t \\ (t^2 / 4 + 3 t / 4 - 3 / 4 + c) e^t \end{pmatrix}.
	\end{equation*}

	Поклавши $c = 0$, одержуємо
	\begin{equation*}
		\begin{pmatrix} x_1(t) \\ x_2(t) \end{pmatrix} = \begin{pmatrix} (t^2 + t) e^t / 4 \\ (t^2 + 3 t - 3) e^t / 4 \end{pmatrix}.
	\end{equation*}
\end{solution}

Знайти загальний розв'язок неоднорідної системи.
\begin{multicols}{2}
	\begin{problem}
		\[ \left\{ \begin{aligned}
			\dot x &= y + \tan^2 t - 1, \\
			\dot y &= - x + \tan t.
		\end{aligned} \right. \]
	\end{problem}

	\begin{problem}
		\[ \left\{ \begin{aligned}
			\dot x &= 3 x - y, \\
			\dot y &= 2 x - y + 15 e^t \sqrt{t}.
		\end{aligned} \right. \]
	\end{problem}

	\begin{problem}
		\[ \left\{ \begin{aligned}
			\dot x &= y - 5 \cos t, \\
			\dot y &= 2 x + y.
		\end{aligned} \right. \]
	\end{problem}
	
	\begin{problem}
		\[ \left\{ \begin{aligned}
			\dot x &= 2 x - 4 y + 4 e^{-2t}, \\
			\dot y &= 2 x - 2 y.
		\end{aligned} \right. \]
	\end{problem}

	\begin{problem}
		\[ \left\{ \begin{aligned}
			\dot x &= - x + 2 y + 1, \\
			\dot y &= 2 x - 2 y.
		\end{aligned} \right. \]
	\end{problem}

	\begin{problem}
		\[ \left\{ \begin{aligned}
			\dot x &= 2 x + y + e^t, \\
			\dot y &= - 2 x + 2 t.
		\end{aligned} \right. \]
	\end{problem}
	
	\begin{problem}
		\[ \left\{ \begin{aligned}
			\dot x &= 3 x - 4 y, \\
			\dot y &= x - 3 y + 3 e^t.
		\end{aligned} \right. \]
	\end{problem}
	
	\begin{problem}
		\[ \left\{ \begin{aligned}
			\dot x &= x + 2 y + 16 t e^t, \\
			\dot y &= 2 x - 2 y.
		\end{aligned} \right. \]
	\end{problem}

	\begin{problem}
		\[ \left\{ \begin{aligned}
			\dot x &= 2 x - 3 y, \\
			\dot y &= x - 2 y + 2 \sin t.
		\end{aligned} \right. \]
	\end{problem}

	\begin{problem}
		\[ \left\{ \begin{aligned}
			\dot x &= 2 x - y, \\
			\dot y &= x + 2 e^t.
		\end{aligned} \right. \]
	\end{problem}

	\begin{problem}
		\[ \left\{ \begin{aligned}
			\dot x &= 2 x + y + 2 e^t, \\
			\dot y &= x + 2 y - 3 e^{4 t}.
		\end{aligned} \right. \]
	\end{problem}
	
	\begin{problem}
		\[ \left\{ \begin{aligned}
			\dot x &= x - y + 1 / \cos t, \\
			\dot y &= 2 x - y.
		\end{aligned} \right. \]
	\end{problem}
	
	\begin{problem}
		\[ \left\{ \begin{aligned}
			\dot x &= y + 2 e^t, \\
			\dot y &= x + t^2.
		\end{aligned} \right. \]
	\end{problem}

	\begin{problem}
		\[ \left\{ \begin{aligned}
			\dot x &= 3 x + 2 y + 4 e^{5 t}, \\
			\dot y &= x + 2 y.
		\end{aligned} \right. \]
	\end{problem}
	
	\begin{problem}
		\[ \left\{ \begin{aligned}
			\dot x &= 4 x + y - e^{2 t}, \\
			\dot y &= - 2 x + t.
		\end{aligned} \right. \]
	\end{problem}
	
	\begin{problem}
		\[ \left\{ \begin{aligned}
			\dot x &= 5 x - 3 y + 2 e^{3 t}, \\
			\dot y &= x + y + 5 e^{-t}.
		\end{aligned} \right. \]
	\end{problem}

	\begin{problem}
		\[ \left\{ \begin{aligned}
			\dot x &= x + 2 y, \\
			\dot y &= x - 5 \sin t.
		\end{aligned} \right. \]
	\end{problem}

	\begin{problem}
		\[ \left\{ \begin{aligned}
			\dot x &= 2 x - y, \\
			\dot y &= - 2 x + y + 18 t.
		\end{aligned} \right. \]
	\end{problem}
\end{multicols}

\begin{multicols}{2}
	\begin{problem}
		\[ \left\{ \begin{aligned}
			\dot x &= 2 x + 4 t - 8, \\
			\dot y &= 3 x + 4 y.
		\end{aligned} \right. \]
	\end{problem}
	
	\begin{problem}
		\[ \left\{ \begin{aligned}
			\dot x &= x - y + 2 \sin t, \\
			\dot y &= 2 x - y.
		\end{aligned} \right. \]
	\end{problem}
	
	\begin{problem}
		\[ \left\{ \begin{aligned}
			\dot x &= 4 x - 3 y + \sin t, \\
			\dot y &= 2 x - y - 2 \cos t.
		\end{aligned} \right. \]
	\end{problem}

	\begin{problem}
		\[ \left\{ \begin{aligned}
			\dot x &= 2 x - y, \\
			\dot y &= - x + 2 y - 5 e^t \sin t.
		\end{aligned} \right. \]
	\end{problem}
\end{multicols}
%
\end{document}