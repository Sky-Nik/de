Назвемо $(n+1)$-вимірний простір змінних $x_1, x_2, \ldots, x_n, t$ розширеним фазовим простором $\RR^{n + 1}$. Тоді розв’язок $x_1 = x_1(t), x_2 = x_2(t), \ldots, x_n = x_n(t)$ визначає в просторі $\RR^{n + 1}$ деяку криву, що називається інтегральною кривою. Загальний розв’язок (чи загальний інтеграл) визначає сім’ю інтегральних кривих, що всюди щільно заповнюють деяку область $D \subseteq \RR^{n + 1}$  (область існування та єдиності розв’язків). Задача Коші ставиться як виділення із сім’ї інтегральних кривих, окремої кривої, що проходить через задану початкову точку $M \left(x_1^0, x_2^0, \ldots, x_n^0, t_0\right) \in D$.