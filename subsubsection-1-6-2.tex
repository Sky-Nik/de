\setcounter{problem}{0}
\begin{example}
	Побудувати послідовні наближення $y_0(x)$, $y_1(x)$, $y_2(x)$ для рівняння $y' = x - y^2$, $y(0) = 0$.
\end{example}

\begin{solution}
	Візьмемо початкову функцію $y_0(0) \equiv 0$. Підставивши в ітераційну залежність \[ y_{n+1} (x) = y(x_0) + \int_{x_0}^x f(s, y_n(s)) \diff s \] 

	отримаємо 
	\begin{align*} 
		y_1(x) &= \int_0^x s \diff s = \frac{x^2}{2}, \\
		y_2(x) &= \int_0^x (s - y_1^2(s)) \diff s = \int_0^x \left(s - \frac{s^4}{4}\right) \diff s = \frac{x^2}{2} - \frac{x^5}{20}.
	\end{align*}
\end{solution}

Побудувати послідовні наближення $y_0(x)$, $y_1(x)$, $y_2(x)$ для рівнянь
\begin{problem}
	\[y' = y^2 + 3x^2 - 1, \quad y(0) = 1;\]
\end{problem}

\begin{problem}
	\[y'=y+e^{y-1},\quad y(0)=1;\]
\end{problem}

\begin{problem}
	\[y'=1+x\sin y, \quad y(\pi)=2\pi.\]
\end{problem}

\begin{example}
	Вказати на проміжок з $a=1$, $b=1$, на якому гарантується існування та єдиність розв'язку диференціального рівняння $y'=y^3+x$ за умови $y(0)=1$.
\end{example}

\begin{solution}
	Як випливає з теореми про існування та єдиність розв'язку, проміжок, на якому гарантується існування та єдиність розв'язку задачі Коші дорівнює $h = \min \left\{ a, b/M, 1/L\right\}$, де \[ M = \max_{(x,y)\in D} |f(x,y)|, \quad L = \max_{(x,y)\in D} \left|\frac{\partial f(x,y)}{\partial y}\right|.\]

	Для цієї задачі отримаємо $D = \{ (x,y): |x| \le 1, |y| \le 1\}$, $M=2$, $L=3$. Тому $h = 1/3$.
\end{solution}	

Вказати проміжки, де гарантується існування та єдиність розв'язку задачі Коші рівняння
\begin{problem}
	\[y'=y+e^y,\quad x_0=0, \quad y_0=0,\quad a=1,\quad b=2;\]
\end{problem}

\begin{problem}
	\[y'=2xy+y^3,\quad x_0=1, \quad y_0=1,\quad a=2,\quad b=1;\]
\end{problem}

\begin{problem}
	\[y'=2+\sqrt[3]{y-2x},\quad x_0=0, \quad y_0=1,\quad a=1,\quad b=1.\]
\end{problem}

\begin{example}
	Знайти особливий розв'язок рівняння $y' - \sqrt{y}$.
\end{example}

\begin{solution}
	Особливий розв'язок слід шукати там, де $\partial f(x,y)/\partial y=\pm\infty$. \parvskip
	
	Оскільки \[\frac{\partial f(x,y)}{\partial y}=\frac{1}{2\sqrt{y}},\] то отримаємо $\bar y(x)=0$ --- крива, що підозріла на особливу. Перевірка показує, що це дійсно інтегральна крива. Щоб до кінця переконатися, що ця крива особлива, розв'язуємо рівняння \[y'=\sqrt{y}\implies \frac{\diff y}{\sqrt{y}}=\diff x\implies 2\sqrt{y}=x+C\implies y(x)=\frac{(x+c)^2}{4}.\]

	Легко переконатися, що $\bar y(x)=0$ є кривою, що огинає сім'ю інтегральних кривих $y(x)=(x+c)^2/4$. 
\end{solution}

\begin{example}
	Знайти особливий розв'язок рівняння $y = x + y' - \ln y'$.
\end{example}

\begin{solution}
	Складаємо рівняння $p$-дискриминантної кривої \[ y = x + p - \ln p, \quad 0 = 1 - \frac{1}{p}.\]

	 Із другого рівняння $p=1$. Підставивши в перше, отримаємо, що крива, що є підозрілою як особлива, має вигляд $\bar y(x)=x+1$. \parvskip

	Підставивши у рівняння, отримаємо $x + 1=x+1-\ln 1$, тобто впевнились, що $\bar y(x)=x+1$ є інтегральною кривою. \parvskip

	Роз\-в'яж\-е\-мо рівняння методом введення параметру. Його загальний роз\-в'яз\-ок має вигляд \[y=Ce^x-\ln C.\]

	 Можна переконатися, що $\bar y(x)=x+1$ є кривою, що огинає сім'ю інтегральних кривих. \parvskip

	Щоб перевірити це аналітично, запишемо умову дотику кривої $y=x+1$ та $y=Ce^x-\ln C$ в точці $(x_0,y_0)$. Вона має вигляд: \[ \bar y(x_0) = y(x_0, C), \quad \bar{y}'(x_0) = y'(x_0,C).\] 

	Тобто \[x_0+1=Ce^{x_0}-\ln C, \quad 1 = Ce^{x_0}.\]

	З другого рівняння отримаємо $C = e^{-x_0}$. Підставивши у перше рівняння, маємо $x_0 + 1 = 1 - \ln e^{-x_0}$, тобто $x_0+1=x_0+1$ --- тотожність. Таким чином при кожному $x_0$ відбувається дотик інтегральних кривих та $\bar y(x)=x+1$, що огинає сім'ю інтегральних кривих.
\end{solution}

Знайти особливі розв'язки та зробити рисунок.
\begin{multicols}{2}
	\begin{problem}
		\[8(y')^3-27y=0;\]
	\end{problem}
	
	\begin{problem}
		\[(y'+1)^3-27(x+y)^2=0;\]
	\end{problem}

	\begin{problem}
		\[y^2((y')^2+1)=1;\]
	\end{problem}

	\begin{problem}
		\[(y')^2-4y^3=0.\]
	\end{problem}
\end{multicols}