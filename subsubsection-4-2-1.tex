\setcounter{property}{0}
\begin{property}
	Якщо вектор $x(t) = (x_1(t), x_2(t), \ldots, x_n(t))^T$ є розв'язком лінійної однорідної системи, то і \[ C \cdot x(t) = (C \cdot x_1(t), C \cdot x_2(t), \ldots, C \cdot x_n(t))^T,\] де $C$ --- стала скалярна величина, також є розв'язком цієї системи.
\end{property}

\begin{proof}
	Дійсно, за умовою 
	\begin{equation*}
		\dot x(t) - A(t) \cdot x(t) \equiv 0.
	\end{equation*}

	Але тоді і
	\begin{multline*}
		\frac{\diff}{\diff t} (C \cdot x(t)) - A(t) \cdot (C \cdot x(t)) = \\ = C \cdot (\dot x(t) - A(t) \cdot x(t)) \equiv 0
	\end{multline*}
	оскільки дорівнює нулю вираз в дужках. Тобто $C \cdot x(t)$ є розв'язком однорідної системи.
\end{proof}

\begin{property}
	Якщо дві векторні функції $x_1 = (x_{11}(t), x_{21}(t), \ldots, x_{n1}(t))^T$, $x_2 = (x_{12}(t), x_{22}(t), \ldots, x_{n2}(t))^T$ є розв'язками однорідної системи, то і їхня сума також буде розв’язком однорідної системи.
\end{property}

\begin{proof}
	Дійсно, за умовою
	\begin{align*}
		\dot x_1(t) - A(t) \cdot x_1(t) &\equiv 0, \\
		\dot x_2(t) - A(t) \cdot x_2(t) &\equiv 0.
	\end{align*}

	Але тоді і
	\begin{equation*}
		\frac{\diff}{\diff t} (x_1(t) + x_2(t)) - A(t) \cdot (x_1(t) + x_2(t)) = (\dot x_1(t) - A(t) \cdot x_1(t)) + (\dot x_2(t) - A(t) \cdot x_2(t)) \equiv 0
	\end{equation*}
	тому що дорівнюють нулю вирази в дужках, тобто $x_1(t) + x_2(t)$ є розв'язком однорідної системи.
\end{proof}

\begin{property}
	Якщо вектори $x_1 = (x_{11}(t), x_{21}(t), \ldots, x_{n1}(t))^T$, $\ldots$, $x_n = (x_{1n}(t), x_{2n}(t), \ldots, x_{nn}(t))^T$ є розв'язками однорідної системи, та і їхня лінійна комбінація з довільними коефіцієнтами також буде розв'язком однорідної системи. 
\end{property}

\begin{proof}
	Дійсно, за умовою
	\begin{equation*}
		\dot x_i(t) - A(t) \cdot x_i(t) \equiv 0, \quad i = \overline{1, n}.
	\end{equation*}
	
	Але тоді і
	\begin{multline*}
		\frac{\diff}{\diff t} \left( \sum_{i=1}^n C_i \cdot x_i(t) \right) - A(t) \cdot \left( \sum_{i = 1}^n C_i \cdot x_i(t) \right) = \\ = \sum_{i = 1}^n C_i \cdot \left( \dot x_i(t) - A(t) \cdot x_i(t) \right) \equiv 0
	\end{multline*}
	тому що дорівнює нулю кожний з доданків, тобто $\sum_{i=1}^n C_i \cdot x_i(t)$ є роз\-в'яз\-ком однорідної системи.
\end{proof}

\begin{property}
	Якщо комплексний вектор з дійсними елементами $u(t) + i \cdot v(t) = (u_1(t), \ldots, u_n(t))^T + i \cdot (v_1(t), \ldots, v_n(t))^T$ є розв’язком однорідної системи, то окремо дійсна та уявна частини є розв'язками системи.
\end{property}

\begin{proof}
	Дійсно за умовою
	\begin{equation*}
		\frac{\diff}{\diff t} (u(t) + i \cdot v(t)) - A(t) \cdot (u(t) + i \cdot v(t)) \equiv 0.
	\end{equation*}

	Розкривши дужки і зробивши перетворення, одержимо
	\begin{equation*}
		(\dot u(t) - A(t) \cdot u(t)) + i \cdot (\dot v(t) - A(t) \cdot v(t)) \equiv 0.
	\end{equation*}
	 
	А комплексний вираз дорівнює нулю тоді і тільки тоді, коли дорівнюють нулю дійсна і уявна частини, тобто
	\begin{align*}
		\dot u(t) - A(t) \cdot u(t) &\equiv 0, \\
		\dot v(t) - A(t) \cdot v(t) &\equiv 0.
	\end{align*}
	що і було потрібно довести.
\end{proof}

\begin{definition}
	Вектори \[ x_1 = \begin{pmatrix} x_{11}(t) \\ x_{21}(t) \\ \vdots \\ x_{n1}(t) \end{pmatrix}, \quad x_2 = \begin{pmatrix} x_{12}(t) \\ x_{22}(t) \\ \vdots \\ x_{n2}(t) \end{pmatrix}, \quad \ldots, \quad x_n = \begin{pmatrix} x_{1n}(t) \\ x_{2n}(t) \\ \vdots \\ x_{nn}(t) \end{pmatrix} \] називаються лінійно залежними на відрізку $t \in [a, b]$, якщо існують не всі рівні нулю сталі $C_1, C_2, \ldots, C_n$, такі, що 
	\begin{equation*}
		C_1 \cdot x_1(t) + C_2 \cdot x_2(t) + \ldots + C_n \cdot x_n(t) \equiv 0
	\end{equation*}
	при $t \in [a, b]$. \\

	Якщо тотожність справедлива лише при $C_i = 0$, $i = \overline{1, n}$, то вектори лінійно незалежні.
\end{definition}

\begin{definition}
	Визначник, що складається з векторів $x_1(t), x_2(t), \ldots, x_n(t)$, тобто
	\begin{equation*}
		W[x_1, x_2, \ldots, x_n](t) = \begin{vmatrix} x_{11}(t) & x_{12}(t) & \ldots & x_{1n}(t) \\ x_{21}(t) & x_{22}(t) & \ldots & x_{2n}(t) \\ \vdots & \vdots & \ddots & \vdots \\ x_{n1}(t) & x_{n2}(t) & \ldots & x_{nn}(t) \end{vmatrix}.
	\end{equation*}
	називається визначником Вронського.
\end{definition}

\begin{theorem}
	Якщо векторні функції $x_1(t), x_2(t), \ldots, x_n(t)$ лінійно залежні, то визначник Вронського тотожно дорівнює нулю.
\end{theorem}

\begin{proof}
	За умовою існують не всі рівні нулю $C_1, C_2, \ldots, C_n$, такі, що $C_1 \cdot x_1(t) + C_2 \cdot x_2(t) + \ldots + C_n \cdot x_n(t) \equiv 0$ при $t \in [a, b]$. \\

	Або, розписавши покоординатно, одержимо
	\begin{equation*}
		\left\{
			\begin{array}{rl}
				C_1 \cdot x_{11}(t) + C_2 \cdot x_{12}(t) + \ldots + C_n \cdot x_{1n}(t) &\equiv 0, \\
				C_1 \cdot x_{21}(t) + C_2 \cdot x_{22}(t) + \ldots + C_n \cdot x_{2n}(t) &\equiv 0, \\
				\hdotsfor{2}, \\
				C_1 \cdot x_{n1}(t) + C_2 \cdot x_{n2}(t) + \ldots + C_n \cdot x_{nn}(t) &\equiv 0.
			\end{array}
		\right.
	\end{equation*}

	А однорідна система має ненульовий розв'язок $C_1, C_2, \ldots, C_n$ тоді і тільки тоді, коли визначник дорівнює нулю, тобто
	\begin{equation*}
		W[x_1, x_2, \ldots, x_n](t) = \begin{vmatrix} x_{11}(t) & x_{12}(t) & \ldots & x_{1n}(t) \\ x_{21}(t) & x_{22}(t) & \ldots & x_{2n}(t) \\ \vdots & \vdots & \ddots & \vdots \\ x_{n1}(t) & x_{n2}(t) & \ldots & x_{nn}(t) \end{vmatrix} \equiv 0, \quad t \in [a, b].
	\end{equation*}
\end{proof}

\begin{theorem}
	Якщо розв'язки $x_1(t), x_2(t), \ldots, x_n(t)$ лінійної однорідної системи лінійно незалежні, то визначник Вронського не дорівнює нулю в жодній точці $t \in [a, b]$. 
\end{theorem}

\begin{proof}
	Нехай, від супротивного, існує точка $t_0 \in [a, b]$ і \[W[x_1, x_2, \ldots, x_n](t_0) = 0.\]

	Тоді виконується система однорідних алгебраїчних рівнянь \[C_1 \cdot x_1(t_0) + C_2 \cdot x_2(t_0) + \ldots + C_n \cdot x_n(t_0) = 0. \]

	Або, розписавши покоординатно, одержимо
	\begin{equation*}
		\left\{
			\begin{array}{rl}
				C_1 \cdot x_{11}(t_0) + C_2 \cdot x_{12}(t_0) + \ldots + C_n \cdot x_{1n}(t_0) &= 0, \\
				C_1 \cdot x_{21}(t_0) + C_2 \cdot x_{22}(t_0) + \ldots + C_n \cdot x_{2n}(t_0) &= 0, \\
				\hdotsfor{2}, \\
				C_1 \cdot x_{n1}(t_0) + C_2 \cdot x_{n2}(t_0) + \ldots + C_n \cdot x_{nn}(t_0) &= 0.
			\end{array}
		\right.
	\end{equation*}
 	має ненульовий розв'язок $C_1^0, C_2^0, \ldots, C_n^0$. Розглянемо лінійну комбінацію розв'язків з отриманими коефіцієнтами
 	\begin{equation*}
 		x(t) = C_1^0 \cdot x_1(t) + C_2^0 \cdot x_2(t) + \ldots + C_n^0 \cdot x_n(t).
 	\end{equation*}

	Відповідно до властивості 4, ця комбінація буде розв'язком. Крім того, як випливає із системи алгебраїчних рівнянь, для отриманих $C_1^0, C_2^0, \ldots, C_n^0$: $x(t_0) = 0$, $t_0 \in [a, b]$. Але розв'язком, що задовольняють таким умовам, є $x \equiv 0$. І в силу теореми існування та єдиності ці два розв’язки збігаються, тобто $x(t) \equiv 0$ при $t \in [a, b]$, або 
 	\begin{equation*}
 		C_1^0 \cdot x_1(t) + C_2^0 \cdot x_2(t) + \ldots + C_n^0 \cdot x_n(t) \equiv 0,
 	\end{equation*}
	або розв'язки $x_1(t), x_2(t), \ldots, x_n(t)$ лінійно залежні, що суперечить умові теореми.  \\

	Таким чином, $W[x_1, x_2, \ldots, x_n](t) \ne 0$ у жодній точці $t \in [a, b]$, що і було потрібно довести.
\end{proof}

\begin{theorem}
	Для того щоб розв'язки $x_1(t), \ldots, x_n(t)$ були лінійно незалежні, необхідно і достатно, щоб $W[x_1, \ldots, x_n](t) \ne 0$ у жодній точці $t \in [a, b]$.
\end{theorem}

\begin{proof}
	Випливає з попередніх двох теорем.
\end{proof}

\begin{theorem}
	Загальний розв'язок лінійної однорідної системи представляється у вигляді лінійної комбінації $n$ лінійно незалежних роз\-в'яз\-ків.
\end{theorem}

\begin{proof}
	Як випливає з властивості 3, лінійна комбінація розв'язків також буде розв'язком. Покажемо, що цей розв'язок загальний, тобто завдяки вибору коефіцієнтів $C_1, \ldots, C_n$ можна розв'язати будь-яку задачу Коші $x(t_0) = x_0$ або в координатній формі:
	\begin{equation*}
		x_1(t_0) = x_1^0, \quad x_2(t_0) = x_2^0, \quad \ldots, \quad x_n(t_0) = x_n^0.
	\end{equation*}

	Оскільки розв'язки $x_1(t), \ldots, x_n(t)$  лінійно незалежні, то визначник Вронського відмінний від нуля. Отже, система алгебраїчних рівнянь
	\begin{equation*}
		\left\{
			\begin{array}{rl}
				C_1 \cdot x_{11}(t_0) + C_2 \cdot x_{12}(t_0) + \ldots + C_n \cdot x_{1n}(t_0) &= x_1^0, \\
				C_1 \cdot x_{21}(t_0) + C_2 \cdot x_{22}(t_0) + \ldots + C_n \cdot x_{2n}(t_0) &= x_2^0, \\
				\hdotsfor{2}, \\
				C_1 \cdot x_{n1}(t_0) + C_2 \cdot x_{n2}(t_0) + \ldots + C_n \cdot x_{nn}(t_0) &= x_n^0.
			\end{array}
		\right.
	\end{equation*}
	має єдиний розв'язок $C_1^0, C_2^0, \ldots, C_n^0$. \\

	Тоді лінійна комбінація
	\begin{equation*}
		x(t) = C_1^0 \cdot x_1(t) + C_2^0 \cdot x_2(t) + \ldots + C_n^0 \cdot x_n(t)
	\end{equation*}
	є розв'язком поставленої задачі Коші. Теорема доведена.
\end{proof}

\begin{remark}
	Максимальне число незалежних розв'язків дорівнює кількості рівнянь $n$.
\end{remark}

\begin{proof}
	Це випливає з теореми про загальний розв'язок системи однорідних рівнянь, тому що будь-який інший розв'язок може бути представлений у вигляді лінійної комбінації $n$ лінійно незалежних розв'язків.
\end{proof}

\begin{definition}
	Матриця, складена з будь-яких $n$ лінійно незалежних роз\-в'яз\-ків, називається фундаментальною матрицею розв'язків системи.
\end{definition}

Якщо лінійно незалежними розв'язками будуть \[ x_1 = \begin{pmatrix} x_{11}(t) \\ x_{21}(t) \\ \vdots \\ x_{n1}(t) \end{pmatrix}, \quad x_2 = \begin{pmatrix} x_{12}(t) \\ x_{22}(t) \\ \vdots \\ x_{n2}(t) \end{pmatrix}, \quad \ldots, \quad x_n = \begin{pmatrix} x_{1n}(t) \\ x_{2n}(t) \\ \vdots \\ x_{nn}(t) \end{pmatrix} \] то матриця
\begin{equation*}
	X(t) = \begin{pmatrix} x_{11} (t) & x_{12} (t) & \ldots & x_{1n} (t) \\ x_{21} (t) & x_{22} (t) & \ldots & x_{2n} (t) \\ \vdots & \vdots & \ddots & \vdots \\ x_{n1} (t) & x_{n2} (t) & \ldots & x_{nn} (t) \end{pmatrix}
\end{equation*}
буде фундаментальною матрицею розв'язків. \\

Як випливає з попередньої теореми загальний розв'язок може бути представлений у вигляді
\begin{equation*}
	x_{\text{homo}} = \sum_{i = 1}^n C_i \cdot x_i(t),
\end{equation*}
де $C_i$ --- довільні сталі. Якщо ввести вектор $C = (C_1, C_2, \ldots, C_n)^T$, то загальний розв'язок можна записати у вигляді $x_{\text{homo}} = X(t) \cdot C$.
