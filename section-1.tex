Рівняння першого порядку, що розв'язане відносно похідної, має вигляд
\begin{equation*}
	\frac{\diff y}{\diff x} = f(x, y).	
\end{equation*}

Диференціальне рівняння встановлює зв'язок між координатами точки та кутовим коефіцієнтом дотичної $\diff y / \diff x$ до графіка розв'язку в цій же точці. Якщо знати $x$ та $y$, то можна обчислити $f(x, y)$ тобто $\diff y / \diff x$. \parvskip

Таким чином, диференціальне рівняння визначає поле напрямків, і задача інтегрування рівнянь зводиться до знаходження кривих, що звуться інтегральними кривими, напрям дотичних до яких в кожній точці збігається з напрямом поля.
