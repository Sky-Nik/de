Розглянемо застосування формули Остроградського-Ліувіля до рівняння 2-го порядку
\begin{equation*}
	%\label{eq:3.1.36}
	y'' + p_1(x) \cdot y' + p_2(x) \cdot y = 0.
\end{equation*}

Нехай $y_1(x)$ -- один з розв’язків. Тоді
\begin{equation*}
	%\label{eq:3.1.37}
	\begin{vmatrix}
		y_1(x) & y(x) \\
		y_1'(x) & y'(x)
	\end{vmatrix} = C_2 \cdot \exp \left\{ - \int p_1(x) \diff x \right\}.
\end{equation*}

Розкривши визначник, одержимо
\begin{equation*}
	%\label{eq:3.1.38}
	y_1(x) \cdot y'(x) - y(x) \cdot y_1'(x) = C_2 \cdot \exp \left\{ - \int p_1(x) \diff x \right\}.
\end{equation*}
 
Розділивши на $y_1^2(x)$, запишемо
\begin{equation*}
	%\label{eq:3.1.39}
	\frac{y_1(x) \cdot y'(x) - y(x) \cdot y_1'(x)}{y_1^2(x)} = \frac{C_2}{y_1^2(x)} \cdot \exp \left\{ - \int p_1(x) \diff x \right\},
\end{equation*}
або
\begin{equation*}
	%\label{eq:3.1.40}
	\frac{\diff}{\diff x} \left( \frac{y(x)}{y_1(x)} \right) = \frac{C_2}{y_1^2(x)} \cdot \exp \left\{ - \int p_1(x) \diff x \right\},
\end{equation*}

Проінтегрувавши, одержимо
\begin{equation*}
	%\label{eq:3.1.41}
	\frac{y(x)}{y_1(x)} = C_2 \cdot \int \left( \frac{1}{y_1^2(x)} \cdot \exp \left\{ - \int p_1(x) \diff x \right\} \right) \diff x + C_1,
\end{equation*}

Остаточно
\begin{equation*}
	%\label{eq:3.1.42}
	y(x) = C_1 \cdot y_1(x) + C_2 \cdot y_1(x) \cdot \int \left( \frac{1}{y_1^2(x)} \cdot \exp \left\{ - \int p_1(x) \diff x \right\} \right) \diff x,
\end{equation*}

Отримана формула називається формулою Абеля. Вона дозволяє по одному відомому роз\-в'яз\-ку знайти загальний роз\-в'яз\-ок однорідного лінійного рівняння другого порядку.
