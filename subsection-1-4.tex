\subsubsection{Загальна теорія}
Якщо ліва частина диференціального рівняння
\begin{equation}
	\label{eq:1.4.1}
	M(x, y) \cdot \diff x + N(x, y) \cdot \diff y = 0,
\end{equation}
є повним диференціалом деякої функції $u(x, y)$, тобто
\begin{equation}
	\label{eq:1.4.2}
	\diff u(x, y) = M(x, y) \cdot \diff x + N(x, y) \cdot \diff y,
\end{equation}
і, таким чином, \eqref{eq:1.4.1} приймає вигляд $\diff u (x, y) = 0$ то рівняння називається рівнянням в повних диференціалах. Звідси вираз
\begin{equation}
	\label{eq:1.4.3}
	u(x, y) = C
\end{equation}
є загальним інтегралом диференціального рівняння. \\

Критерієм того, що рівняння є рівнянням в повних диференціалах, тобто необхідною та достатньою умовою, є виконання рівності
\begin{equation}
	\label{eq:1.4.4}
	\frac{\partial M(x, y)}{\partial y} = \frac{\partial N(x, y)}{\partial x}.
\end{equation}
 
Нехай маємо рівняння в повних диференціалах. Тоді
\begin{equation}
	\label{eq:1.4.5}
	\frac{\partial u(x, y)}{\partial x} = M(x, y), \quad \frac{\partial u(x, y)}{\partial y} = N(x, y).
\end{equation}
Звідси $u(x, y) = \int M(x, y) \diff x + \phi(y)$ де $\phi(y)$ -- невідома функція. Для її визначення продиференціюємо співвідношення по $y$ і прирівняємо $N(x, y)$:
\begin{equation}
	\label{eq:1.4.6}
	\frac{\partial u(x, y)}{\partial y} = \frac{\partial}{\partial y} \left( \int M(x, y) \diff x \right) + \frac{\diff \phi(y)}{\diff y} = N(x, y).
\end{equation}
Звідси
\begin{equation}
	\label{eq:1.4.7}
	\phi(y) = \int \left( N(x, y) - \frac{\partial}{\partial y} \left( \int M(x, y) \diff x \right) \right) \diff y.
\end{equation}
Остаточно, загальний інтеграл має вигляд
\begin{equation}
	\label{eq:1.4.8}
	\int M(x, y) \diff x + \int \left( N(x, y) - \frac{\partial}{\partial y} \left( \int M(x, y) \diff x \right) \right) \diff y = C.
\end{equation}
Як відомо з математичного аналізу, якщо відомий повний диференціал \eqref{eq:1.4.2}, то $u(x, y)$ можна визначити, взявши криволінійний інтеграл по довільному контуру, що з’єднує фіксовану точку $(x_0, y_0)$ і точку із змінними координатами $(x, y)$. \\

Більш зручно брати криву, що складається із двох відрізків прямих. В цьому випадку криволінійний інтеграл розпадається на два простих інтеграла
\begin{multline}
	\label{eq:1.4.9}
	u(x, y) = \int_{(x_0, y_0)}^{(x,y)} M(x,y) \cdot \diff x + N(x, y) \cdot \diff y = \\
	= \int_{(x_0, y_0)}^{(x,y_0)} M(x,y) \cdot \diff x + \int_{(x,y_0)}^{(x,y)} N(x, y) \cdot \diff y = \\
	= \int_{x_0}^{x} M(\xi,y_0) \cdot \diff \xi + \int_{y_0}^{y} N(x, y) \cdot \diff y.
\end{multline}
В цьому випадку одразу одержуємо розв’язок задачі Коші.
\begin{equation}
	\label{eq:1.4.10}
	\int_{x_0}^{x} M(\xi,y_0) \cdot \diff \xi + \int_{y_0}^{y} N(x, y) \cdot \diff y = 0.
\end{equation}

\subsubsection{Множник, що Інтегрує}
В деяких випадках рівняння \eqref{eq:1.4.1} не є рівнянням в повних диференціалах, але існує функція $\mu = \mu(x,y)$ така, що рівняння
\begin{equation}
	\label{eq:1.4.11}
	\mu(x,y) \cdot M(x, y) \cdot \diff x + \mu(x,y) \cdot N(x, y) \cdot \diff y = 0,
\end{equation}
вже буде рівнянням в повних диференціалах. Необхідною та достатньою умовою цього є рівність
\begin{equation}
	\label{eq:1.4.12}
	\frac{\partial}{\partial y} (\mu(x,y) \cdot M(x, y)) = \frac{\partial}{\partial x} (\mu(x,y) \cdot N(x, y)),
\end{equation}
або
\begin{equation}
	\label{eq:1.4.13}
	\frac{\partial \mu}{\partial y} \cdot M + \mu \cdot \frac{\partial M}{\partial y} = \frac{\partial \mu}{\partial x} \cdot N + \mu \cdot \frac{\partial N}{\partial x}.
\end{equation}
Таким чином замість звичайного диференціального рівняння відносно функції $y(x)$ одержимо диференціальне рівняння в частинних похідних відносно функції $\mu(x, y)$. \\

Задача інтегрування його значно спрощується, якщо відомо в якому вигляді шукати функцію $\mu(x,y)$, наприклад $\mu = \mu(\omega(x,y))$ де $\omega(x,y)$ -- відома функція. В цьому випадку одержуємо
\begin{equation}
	\label{eq:1.4.14}
	\frac{\partial \mu}{\partial y} = \frac{\diff \mu}{\diff \omega} \cdot \frac{\partial \omega}{\partial y}, \quad \frac{\partial \mu}{\partial x} = \frac{\diff \mu}{\diff \omega} \cdot \frac{\partial \omega}{\partial x}
\end{equation}
Після підстановки в \eqref{eq:1.4.13} маємо
\begin{equation}
	\label{eq:1.4.15}
	\frac{\diff \mu}{\diff \omega} \cdot \frac{\partial \omega}{\partial y} \cdot M + \mu \cdot \frac{\partial M}{\partial y} = \frac{\diff \mu}{\diff \omega} \cdot \frac{\partial \omega}{\partial x} \cdot N + \mu \cdot \frac{\partial N}{\partial x}.
\end{equation}
або
\begin{equation}
	\label{eq:1.4.16}
	\frac{\diff \mu}{\diff \omega} \left( \frac{\partial \omega}{\partial x} \cdot N - \frac{\partial \omega}{\partial y} \cdot M \right) = \mu \left( \frac{\partial M}{\partial y} - \frac{\partial N}{\partial x} \right).
\end{equation}
Розділимо змінні
\begin{equation}
	\label{eq:1.4.17}
	\frac{\diff \mu}{\mu} = \frac{\frac{\partial M}{\partial y} - \frac{\partial N}{\partial x} }{\frac{\partial \omega}{\partial x} \cdot N - \frac{\partial \omega}{\partial y} \cdot M} \cdot \diff \omega.
\end{equation}
Проінтегрувавши і поклавши сталу інтегрування одиницею, одержимо:
\begin{equation}
	\label{eq:1.4.18}
	\mu(\omega(x,y)) = \exp\left\{\int \frac{\frac{\partial M}{\partial y} - \frac{\partial N}{\partial x} }{\frac{\partial \omega}{\partial x} \cdot N - \frac{\partial \omega}{\partial y} \cdot M} \cdot \diff \omega\right\}.
\end{equation}
Розглянемо частинні випадки.
\begin{enumerate}
	\item Нехай $\omega(x, y) = x$. Тоді $\frac{\partial \omega}{\partial x} = 1$, $\frac{\partial \omega}{\partial y} = 0$, $\diff \omega = \diff x$ і формула має вигляд
	\begin{equation}
		\label{eq:1.4.19}
		\mu(\omega(x,y)) = \exp\left\{\int \frac{\frac{\partial M}{\partial y} - \frac{\partial N}{\partial x} }{N} \cdot \diff x\right\}.
	\end{equation}	
	\item Нехай $\omega(x, y) = y$. Тоді $\frac{\partial \omega}{\partial x} = 0$, $\frac{\partial \omega}{\partial y} = 1$, $\diff \omega = \diff y$ і формула має вигляд
	\begin{equation}
		\label{eq:1.4.20}
		\mu(\omega(x,y)) = \exp\left\{\int \frac{\frac{\partial M}{\partial y} - \frac{\partial N}{\partial x} }{-M} \cdot \diff y\right\}.
	\end{equation}
	\item Нехай $\omega(x, y) = x^2 \pm y^2$. Тоді $\frac{\partial \omega}{\partial x} = 2 x$, $\frac{\partial \omega}{\partial y} = \pm 2y$, $\diff \omega = \diff (x^2 \pm y^2)$ і формула має вигляд
	\begin{equation}
		\label{eq:1.4.21}
		\mu(\omega(x,y)) = \exp\left\{\int \frac{\frac{\partial M}{\partial y} - \frac{\partial N}{\partial x} }{2 x N \mp 2 y M} \cdot \diff (x^2 \pm y^2)\right\}.
	\end{equation}
	\item Нехай $\omega(x, y) = x y$. Тоді $\frac{\partial \omega}{\partial x} = y$, $\frac{\partial \omega}{\partial y} = x$, $\diff \omega = \diff (xy)$ і формула має вигляд
	\begin{equation}
		\label{eq:1.4.22}
		\mu(\omega(x,y)) = \exp\left\{\int \frac{\frac{\partial M}{\partial y} - \frac{\partial N}{\partial x} }{yN-xM} \cdot \diff (xy)\right\}.
	\end{equation}
\end{enumerate}

\subsubsection{Вправи для самостійної роботи}
Як вже було сказано, рівняння \[M(x, y) \cdot \diff x + N(x, y) \cdot \diff y = 0\] буде рівнянням в повних диференціалах, якщо його ліва частина є повним диференціалом деякої функції. Це має місце при \[\frac{\partial M(x, y)}{\partial y} = \frac{\partial N(x, y)}{\partial x}.\]

\begin{example}
	Розв’язати рівняння \[(2x + 3x^2y) \cdot \diff x + (x^3 - 3y^2) \cdot \diff y = 0.\]
\end{example}
\begin{solution}
	Перевіримо, що це рівняння є рівнянням в повних диференціалах. Обчислимо
	\[ \frac{\partial}{\partial y} (2x + 3x^2y) = 3x^2, \quad \frac{\partial}{\partial x} (x^3 - 3y^2) = 3x^2. \]
	Таким чином існує функція $u(x,y)$, що \[\frac{\partial u(x,y)}{\partial x} = 2x + 3x^2y.\] Проінтегруємо по $x$. Отримаємо
	\[ u(x,y) = \int(2x+3x^2y)\cdot\diff x+\Phi(y)=x^2+x^3y+\Phi(y).\]
	Для знаходження функції $\Phi(y)$ візьмемо похідну від $u(x,y)$ по $y$ і прирівняємо до $x^3-3y^2$. Отримаємо
	\[ \frac{\partial u(x,y)}{\partial y} = x^3 + \Phi'(y) = x^3 - 3y^2.\]
	Звідси $\Phi'(y) = -3y^2$ і $\Phi(y) = -y^3$. Таким чином, \[u(x,y)=x^2+x^3y-y^3\] і загальний інтеграл диференціального рівняння має вигляд \[x^2+x^3y-y^3=C.\]
\end{solution}

Перевірити, що дані рівняння є рівняннями в повних диференціалах, і розв’язати їх:
\begin{problem}
	\[ 2 x y \cdot \diff x + (x^2 - y^2) \cdot \diff y = 0;\]
\end{problem}
\begin{problem}
	\[(2-9xy^2)\cdot x\cdot \diff x + (4y^2-6x^3)\cdot y\cdot \diff y=0;\]
\end{problem}
\begin{problem}
	\[e^{-y}\cdot\diff x-(2y+x\cdot e^{-y})\cdot\diff y=0;\]
\end{problem}
\begin{problem}
	\[ \frac yx\cdot\diff x+(y^3+\ln x)\cdot\diff y=0;\]
\end{problem}
\begin{problem}
	\[\frac{3x^2+y^2}{y^2}\cdot\diff x-\frac{2x^3+5y}{y^3}\cdot\diff y=0;\]
\end{problem}
\begin{problem}
	\[2x\cdot\left(1+\sqrt{x^2-y}\right)\cdot\diff x-\sqrt{x^2-y}\cdot\diff y=0;\]
\end{problem}
\begin{problem}
	\[(1+y^2\cdot\sin 2x)\cdot\diff x-2y\cdot\cos^2x\cdot\diff y=0;\]
\end{problem}
\begin{problem}
	\[3x^2\cdot(1+\ln y)\cdot\diff x=\left(2y-\frac{x^3}y\right)\cdot\diff y;\]
\end{problem}
\begin{problem}
	\[\left(\frac x{\sin y}+2\right)\cdot\diff x+\frac{(x^2+1)\cdot\cos y}{\cos2y-1}\cdot\diff y=0;\]
\end{problem}
\begin{problem}
	\[(2x+y\cdot e^{xy})\cdot\diff x+(x\cdot e^{xy}+3y^2)\cdot\diff y=0;\]
\end{problem}
\begin{problem}
	\[\left(2+\frac{1}{x^2+y^2}\right)\cdot x\cdot\diff x+\frac{y}{x^2+y^2}\cdot \diff y=0;\]
\end{problem}
\begin{problem}
	\[\left(3y^2-\frac{y}{x^2+y^2}\right)\cdot\diff x+\left(6xy+\frac{x}{x^2+y^2}\right)\cdot \diff y=0.\]
\end{problem}

Розв’язати, використовуючи інтегруючий множник:
\begin{problem} $\mu=\mu(x-y)$,
	\[(2x^3+3x^2y+y^2-y^3)\cdot\diff x+(2y^3+3xy^2+x^2-x^3)\cdot\diff x=0;\]
\end{problem}
\begin{problem}
	\[ \left(y-\frac{ay}{x}+x\right)\cdot\diff x+a\cdot\diff y=0, \quad \mu=\mu(x+y);\]
\end{problem}
\begin{problem}
	\[(x^2+y)\cdot\diff y+x\cdot(1-y)\cdot\diff x=0, \quad \mu=\mu(xy);\]
\end{problem}
\begin{problem}
	\[(x^2-y^2+y)\cdot\diff x+x\cdot(2y-1)\cdot\diff y=0;\]
\end{problem}
\begin{problem}
	\[(2x^2y^2+y)\cdot\diff x+(x^3y-x)\cdot\diff y=0.\]
\end{problem}