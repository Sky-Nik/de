% cd ..\..\Users\NikitaSkybytskyi\Desktop\differential-equations
% pdflatex main.tex && cls && pdflatex main.tex && cls && pdflatex main.tex && del main.toc, main.log, main.aux, main.out && start main.pdf

\documentclass[a4paper, 12pt]{article}
\usepackage[utf8]{inputenc}
\usepackage[T2A,T1]{fontenc}
\usepackage[english, ukrainian]{babel}
\usepackage{amsmath, amssymb}
%\usepackage[showframe]{geometry}
\usepackage{natbib}

\allowdisplaybreaks
\setlength\parindent{0pt}

\title{{\Huge ДИФЕРЕНЦІАЛЬНІ РІВНЯННЯ}}
\author{Скибицький Нікіта}
% \title{{\Huge ДИФЕРЕНЦІАЛЬНІ РІВНЯННЯ} \\ {\LARGE (КУРС ЛЕКЦІЙ)}}
% \author{{\Large Скибицький Нікіта} \\ {\tt n.skybytskyi@knu.ua} \\ {Київський національний університет імені Тараса Шевченка}}
\date{\today}

\usepackage{float, multirow, multicol, xcolor, hyperref}
\hypersetup{unicode=true, colorlinks=true, linktoc=all, linkcolor=red}

\usepackage{amsthm}
\newtheorem{theorem}{Теорема}[section]
\newtheorem{lemma}{Лема}[section]
\theoremstyle{definition}
\newtheorem*{definition}{Визначення}
\newtheorem{problem}{Задача}[subsection]
\newtheorem*{example*}{Приклад}
\newtheorem{example}[problem]{Приклад}
\newtheorem{property}{Властивість}
\newtheorem*{solution}{Розв'язок}
\newtheorem*{remark}{Зауваження}

\renewcommand{\phi}{\varphi}
\renewcommand{\epsilon}{\varepsilon}
\newcommand{\RR}{\mathbb{R}}
\newcommand{\NN}{\mathbb{N}}

\DeclareMathOperator{\trace}{tr}

\newcommand{\todo}{\texttt{[TO DO]}}

\newcommand*\diff{\mathop{}\!\mathrm{d}}
\newcommand*\rfrac[2]{{}^{#1}\!/_{\!#2}}

\numberwithin{equation}{section}% reset equation counter for sections
\numberwithin{equation}{subsection}% Omit `.0` in equation numbers for non-existent subsections.
\renewcommand*{\theequation}{%
	\ifnum\value{subsection}=0%
		\thesection%
	\else%
		\thesubsection%
	\fi%
	.\arabic{equation}%
}

\makeatletter
\def\old@comma{,}
\catcode`\,=13
\def,{%
	\ifmmode%
		\old@comma\discretionary{}{}{}%
	\else%
		\old@comma%
	\fi%
}
\makeatother

\begin{document}

\maketitle \thispagestyle{empty} \newpage 

У ваших руках конспект лекцій з нормативного курсу ``Диференціальні рівняння'' прочитаного проф., д.ф.-м.н. Хусаїновим Денисом Ях'євичем на другому курсі спеціальності ``прикладна математика'' факультету ком\-п'ю\-тер\-них наук та кібернетики Київського національного університету імені Тараса Шевченка восени 2017-го та навесні 2018-го року. \\

Конспект у компактній формі відображає матеріал курсу, допомагає сформувати загальне уявлення про предмет вивчення, правильно зорієнтуватися в даній галузі знань. Конспект лекцій з названої дисципліни сприятиме більш успішному вивченню дисципліни, причому більшою мірою для студентів заочної форми, екстернату, дистанційного та індивідуального навчання. \\

Комп'ютерний набір та верстка -- Скибицький Нікіта Максимович. \newpage

\tableofcontents \newpage

\section*{Вступ}
Наведемо декілька основних визначень теорії диференціальних рівнянь, що будуть використовуватися надалі.

\begin{definition}
	Рівняння, що містять похідні від шуканої функції та можуть містити шукану функцію та незалежну змінну, називаються диференціальними рівняннями.
\end{definition}

\begin{definition}
	Якщо в диференціальному рівнянні невідомі функції є функціями однієї змінної:
	\begin{equation*}
		F \left( x, y, y', y'', \ldots, y^{(n)} \right) = 0,
	\end{equation*}
	то диференціальне рівняння називається звичайним.
\end{definition}

\begin{definition}
	Якщо невідома функція, що входить в диференціальне рівняння, є функцією двох або більшої кількості незалежних змінних:
	\begin{equation*}
		F \left( x, y, z, \frac{\partial z}{\partial x}, \frac{\partial z}{\partial y}, \ldots, \frac{\partial^k z}{\partial x^\ell \partial y^{k - \ell}}, \ldots, \frac{\partial^n z}{\partial y^n} \right) = 0,
	\end{equation*}
	то диференціальне рівняння називається рівнянням у частинних похідних.
\end{definition}

\begin{definition}
	Порядком диференціального рівняння називається максимальний порядок похідної від невідомої функції, що входить в диференціальне рівняння.
\end{definition}

\begin{definition}
	Розв'язком диференціального рівняння називається функція, що має необхідний ступінь гладкості, і яка при підстановці в диференціальне рівняння обертає його в тотожність. 
\end{definition}

\begin{definition}
	Процес знаходження розв'язку диференціального рівняння називається інтегруванням диференціального рівняння.
\end{definition}


\section{Диференціальні рівняння першого порядку}
% version 1.0
Рівняння першого порядку, що розв'язане відносно похідної, має вигляд
\begin{equation*}
	\frac{\diff y}{\diff x} = f(x, y).	
\end{equation*}

Диференціальне рівняння встановлює зв'язок між координатами точки та кутовим коефіцієнтом дотичної $\frac{\diff y}{\diff x}$ до графіку розв'язку в цій же точці. Якщо знати $x$ та $y$, то можна обчислити $f(x, y)$ тобто $\frac{\diff y}{\diff x}$. \\

Таким чином, диференціальне рівняння визначає поле напрямків, і задача інтегрування рівнянь зводиться до знаходження кривих, що звуться інтегральними кривими, напрям дотичних до яких в кожній точці співпадає з напрямом поля.


	\subsection{Рівняння зі змінними, що розділяються}
	% version 1.0

		\subsubsection{Загальна теорія}
		Рівняння вигляду
\begin{equation*}
    \frac{\diff y}{\diff x} = f(x) g(y),
\end{equation*}
або більш загального вигляду
\begin{equation*}
    f_1(x) f_2(y) \diff x + g_1(x) g_2(y) \diff y = 0
\end{equation*}
називаються рівняннями зі змінними, що розділяються. Розділимо його на $f_2(y) g_1(x)$ і одержимо рівняння з розділеними змінними:
\begin{equation*}
    \frac{f_1(x)}{g_1(x)} \diff x + \frac{g_2(y)}{f_2(y)} \diff y = 0.
\end{equation*}

Узявши інтеграли, отримаємо
\begin{equation*}
    \int \frac{f_1(x)}{g_1(x)} \diff x + \int \frac{g_2(y)}{f_2(y)} \diff y = C,
\end{equation*}
або
\begin{equation*}
    \Phi(x, y) = C.
\end{equation*}

\begin{definition}
    Це кінцеве рівняння, що визначає розв'язок диференціального рівняння як неявну функцію від $x$, називається інтегралом розглянутого рівняння.
\end{definition}

\begin{definition}
    Це ж рівняння, що визначає всі без винятку розв'язки даного диференціального рівняння, називається загальним інтегралом.
\end{definition}

Бувають випадки (в основному), що невизначені інтеграли з рівняння з розділеними змінними не можна записати в елементарних функціях. Попри це, задача інтегрування вважається виконаною. Кажуть, що диференціальне рівняння розв'язне у квадратурах. \parvskip

Можливо, що інтеграл рівняння розв'язується відносно $y$:
\begin{equation*}
    y = y(x, C).
\end{equation*}

Тоді, завдяки вибору $C$, можна одержати всі розв'язки.

\begin{definition}
    Ця залежність, що тотожно задовольняє вихідному диференціальному рівнянню, де $C$ --- довільна стала, називається загальним розв'язком диференціального рівняння.
\end{definition}

Геометрично загальний розв'язок являє собою сім'ю кривих, що не перетинаються, які заповнюють деяку область. Іноді треба виділити одну криву сім'ї, що проходить через задану точку $M(x_0, y_0)$.

\begin{definition}
    Знаходження розв'язку $y = y(x)$, що проходить через задану точку $M(x_0, y_0)$, називається розв'язком задачі Коші.
\end{definition}

\begin{definition}
    Розв'язок, який записаний у вигляді $y = y(x, x_0, y_0)$ і задовольняє умові $y(x, x_0, y_0) = y_0$, називається розв'язком у формі Коші.
\end{definition}

		\subsubsection{Рівняння, що зводяться до рівнянь зі змінними, що розділяються}
		Розглянемо рівняння вигляду
\begin{equation}
	\label{eq:1.1.8}
	\frac{\diff y}{\diff x} = f(a x + b y + c)
\end{equation}
де $a$, $b$, $c$ -- сталі. \\

Зробимо заміну $a x + b y + c = z$. Тоді $a \cdot \diff x + b \cdot \diff y = \diff z$ і $\frac{\diff y}{\diff x} = \frac1b \cdot \left( \frac{\diff z}{\diff x} - a \right)$. \\

Підставивши в \eqref{eq:1.1.8}, одержимо
\begin{equation}
	\label{eq:1.1.9}
	\frac1b \cdot \left( \frac{\diff z}{\diff x} - a \right) = f(z),
\end{equation}
або
\begin{equation}
	\label{eq:1.1.10}
	\frac{\diff z}{\diff x} = a + b \cdot f (z),
\end{equation}

Розділивши змінні, запишемо
\begin{equation}
	\label{eq:1.1.11}
	\frac{\diff z}{a + b \cdot f (z)} - \diff x = 0
\end{equation}
і
\begin{equation}
	\label{eq:1.1.12}
	\int \frac{\diff z}{a + b \cdot f (z)} - x = C.
\end{equation}

Загальний інтеграл має вигляд $\Phi(a x + b y + c, x) = C$.

		\subsubsection{Вправи для самостійної роботи}
		Рівняння зі змінними, що розділяються могуть бути записані у вигляді $y' = f(x) \cdot g(y)$ або  $f_1(x) \cdot f_2(y) \cdot \diff x + g_1(x) \cdot g_2(y) \cdot \diff y$. Для розв’язків такого рівняння необхідно обидві частини помножити або розділити на такий вираз, щоб в одну частину входило тільки $x$, а в другу -- тільки $y$. Тоді обидві частини рівняння можна проінтегрувати.
Якщо ділити на вираз, що містить $x$ та $y$, може бути загублений розв’язок, що обертає цей вираз в нуль.

\begin{example}
	Розв’язати рівняння
	\begin{equation}
		\label{eq:1.1.13}
		x^2 y^2 y' + y = 1.
	\end{equation}
\end{example}

\begin{solution}
	Підставивши $y = \frac{\diff y}{\diff x}$ в \eqref{eq:1.1.13}, отримаємо 
	\begin{equation}
		\label{eq:1.1.14}
		x^2 y^2 \cdot \frac{\diff y}{\diff x} + y = 1.
	\end{equation}

	Помножимо обидві частини рівняння на $\diff x$ і розділимо на $x^2 \cdot (y - 1)$. Перевіримо, що $y = 1$ при цьому є розв’язком, а $x = 0$ цим розв’язком не є:
	\begin{equation}
		\label{eq:1.1.15}
		\frac{y^2}{y - 1} \cdot \diff y = - \frac{\diff x}{x^2}.
	\end{equation}

	Проінтегрируємо обидві частини рівняння:
	\begin{align}
		\label{eq:1.1.16}
		\int \frac{y^2}{y - 1} \cdot \diff y &= - \int \frac{\diff x}{x^2}. \\
		\frac{y^2}{2} + y + \ln |y - 1| &= \frac{1}{x} + C
	\end{align}
\end{solution}

\begin{example}
	Розв’язати рівняння
	\begin{equation}
		\label{eq:1.1.17}
		y' = \sqrt{4x + 2y - 1}.
	\end{equation}
\end{example}

\begin{solution}
	Введемо заміну змінних $z = 4 x + 2 y - 1$. Тоді $x' = 4 + 2 y'$. Рівняння \eqref{eq:1.1.17} перетвориться до вигляду $z' - 4 = 2 \sqrt{z}$; $z' = 4 + 2 \sqrt{z}$; $\frac{\diff z}{2 + \sqrt{z}} = 2 \diff x$. Проінтегруємо обидві частини рівняння:
	\begin{equation}
		\label{eq:1.1.18}
		\int \frac{\diff z}{2 + \sqrt{z}} = \int 2 \diff x
	\end{equation}
	Обчислимо інтеграл, що стоїть зліва. При обчисленні будемо використовувати таку заміну: 
	\begin{equation*}
		\sqrt{z} = t, \quad \diff z = 2 t \diff t, \quad 2 + \sqrt{z} = 2 + t,
	\end{equation*}
	\begin{multline}
		\label{eq:1.1.19}
		\int \frac{\diff z}{2 + \sqrt{z}} = \int \frac{2 t \diff t}{2 + t} = 2 \int \frac{t + 2 - 2}{t + 2} \cdot \diff t = \\
		= 2 t - 4 \ln |2 + t| = 2 \sqrt{z} - 4 \ln \left(2 + \sqrt{z}\right).
	\end{multline}
	Після інтегрування отримаємо $2 \sqrt{z} - 4 \ln \left(2 + \sqrt{z}\right) = 2 x + 2 C$. Зробимо обернену заміну: $z = 4x + 2y - 1$;
	\begin{equation}
		\label{eq:1.1.20}
		\sqrt{4x + 2y - 1} - 2 \ln \left(2 + \sqrt{4x + 2y - 1}\right) = x + C.
	\end{equation}
\end{solution}

Розв’язати рівняння:
\begin{multicols}{2}
\begin{problem}
	\[ x y \cdot \diff x + (x + 1) \cdot \diff y = 0; \]
\end{problem}
\begin{problem}
	\[ x \cdot (1 + y) \cdot \diff x = y \cdot (1 + x^2) \cdot \diff y; \]
\end{problem}
\begin{problem}
	\[ y' = 10^{x + y}; \]
\end{problem}
\begin{problem}
	\[ y' - xy^2 = 2xy; \]
\end{problem}
\begin{problem}
	\[ \sqrt{y^2 + 1} \diff x = x y \cdot \diff y; \]
\end{problem}
\begin{problem}
	\[ y' = x \tan (y); \]
\end{problem}
\begin{problem}
	\[ y y' + x = 1; \]
\end{problem}
\begin{problem}
	\[ 3 y^2 y' + 15 x = 2 x y^3; \]
\end{problem}
\begin{problem}
	\[ y' = \cos (y - x); \]
\end{problem}
\begin{problem}
	\[ y' - y = 2x - 3; \]
\end{problem}
\begin{problem}
	\[ x y' + y = y^2; \]
\end{problem}
\begin{problem}
	\[ e^{-y} \cdot (1 + y') = 1; \]
\end{problem}
\begin{problem}
	\[ 2 x^2 y y' + y^2 = 2; \]
\end{problem}
\begin{problem}
	\[ y' - x y^3 = 2 x y^2. \]
\end{problem}
\end{multicols}

Знайти частинні розв’язки, що задовольняють заданим початковим умовам:
\begin{problem}
	\[ (x^2 - 1) \cdot y' + 2 x y^2 = 0, \quad y(0) = 1; \]
\end{problem}
\begin{problem}
	\[ y' \cdot \cot (x) + y = 2, \quad y(0) = - 1; \]
\end{problem}
\begin{problem}
	\[ y' = 3 \sqrt[3]{y^2}, \quad y(2) = 0. \]
\end{problem}

	\subsection{Однорідні рівняння}
	\subsubsection{Загальна теорія}
Нехай рівняння має вигляд
\begin{equation}
	\label{eq:1.2.1}
	M(x, y) \cdot \diff	x + N(x, y) \cdot \diff y = 0.
\end{equation}

Якщо функції $M(x, y)$ та $N(x, y)$ однорідні одного ступеня, то рівняння називається однорідним. Нехай функції $M(x, y)$ та $N(x, y)$ однорідні ступеня $k$, тобто
\begin{equation}
	\label{eq:1.2.2}
	M(t \cdot x, t \cdot y) = t^k \cdot M(x, y), \qquad N(t \cdot x, t \cdot y) = t^k \cdot N(x, y).
\end{equation}

Робимо заміну 
\begin{equation}
	\label{eq:1.2.2_5}
	y = u x, \quad \diff y = u \diff x + x \diff u.
\end{equation}
Після підстановки одержуємо
\begin{equation}
	\label{eq:1.2.3}
	M(x, u x) \cdot \diff x + N(x, u x) \cdot (u \diff x + x \diff u) = 0,
\end{equation}
або 
\begin{equation}
	\label{eq:1.2.4}
	x^k M(1, u) \cdot \diff x + x^k N(1, u) \cdot (u \diff x + x \diff u) = 0.
\end{equation}

Скоротивши на $x^k$ і розкривши дужки, запишемо 
\begin{equation}
	\label{eq:1.2.5}
	M(1, u) \cdot \diff x + N(1, u) \cdot u \diff x + N(1, u) \cdot x \diff u = 0.
\end{equation}
Згрупувавши, одержимо рівняння зі змінними, що розділяються
\begin{equation}
	\label{eq:1.2.6}
	(M(1, u) + N(1, u) \cdot u) \diff x + N(1, u) \cdot x \diff u = 0,
\end{equation}
або 
\begin{equation}
	\label{eq:1.2.7}
	\int \frac{\diff x}{x} + \int \frac{N(1, u) \cdot \diff u}{M(1, u) + N(1, u) \cdot u} = C.
\end{equation}

Взявши інтеграли та замінивши $u = y / x$, отримаємо загальний інтеграл $\Phi(x, y / x) = C$.

\subsubsection{Рівняння, що зводяться до однорідних}

Нехай маємо рівняння вигляду
\begin{equation}
	\label{eq:1.2.8}
	\frac{\diff y}{\diff x} = f \left( \frac{a_1 x + b_1 y + c_1}{a_2 x + b_2 y + c_2} \right).
\end{equation}

Розглянемо два випадки
\begin{enumerate}
	\item 
	\begin{equation}
		\label{eq:1.2.9}
		\Delta = \begin{vmatrix} a_1 & b_1 \\ a_2 & b_2 \end{vmatrix} \ne 0.
	\end{equation}

	Тоді система алгебраїчних рівнянь
	\begin{equation}
		\label{eq:1.2.10}
		\left\{
			\begin{aligned}
				a_1 x + b_1 y + c_1 &= 0, \\
				a_2 x + b_2 y + c_2 &= 0,
			\end{aligned}
		\right.
	\end{equation}
	має єдиний розв’язок $(x_0, y_0)$. Проведемо заміну 
	\begin{equation}
		\label{eq:1.2.10_5}
		\left\{\begin{aligned}
			x &= x_1 + x_0, \\
			y &= y_1 + y_0
		\end{aligned}\right.
	\end{equation}
	та отримаємо
	\begin{multline}
		\label{eq:1.2.11}
		\frac{\diff y_1}{\diff x_1} = f \left( \frac{a_1 \cdot (x_1 + x_0) + b_1 \cdot (y_1 + y_0) + c_1}{a_2 \cdot (x_1 + x_0) + b_2 \cdot (y_1 + y_0) + c_2} \right) = \\
		= f \left( \frac{a_1 x_1 + b_1 y_1 + (a_1 x_0 + b_1 y_0 + c_1)}{a_2 x_1 + b_2 y_1 + (a_2 x_0 + b_2 y_0 + c_2)} \right)
	\end{multline}

	Оскільки $(x_0, y_0)$ -- розв’язок \eqref{eq:1.2.10}, то \eqref{eq:1.2.8} набуде вигляду
	\begin{equation}
		\label{eq:1.2.12}
		\frac{\diff y_1}{\diff x_1} = f \left( \frac{a_1 x_1 + b_1 y_1}{a_2 x_1 + b_2 y_1} \right)
	\end{equation}
	і є однорідним нульового ступеня. Робимо заміну 
	\begin{equation}
		\label{eq:1.2.12_5}
		y_1 = u x_1, \quad \diff y_1 = u \cdot \diff x_1 + x_1 \cdot \diff u.
	\end{equation}

	Підставимо в \eqref{eq:1.2.12}
	\begin{equation}
		\label{eq:1.2.13}
		u + x_1 \cdot \frac{\diff u}{\diff x_1} = f \left( \frac{a_1 x_1 + b_1 u x_1}{a_2 x_1 + b_2 u x_1} \right).
	\end{equation}
	
	Одержимо
	\begin{equation}
		\label{eq:1.2.14}
		x_1 \cdot \diff u + \left( u - f \left( \frac{a_1 x_1 + b_1 u x_1}{a_2 x_1 + b_2 u x_1} \right) \right) \diff x_1 = 0.
	\end{equation}

	Розділивши змінні, маємо
	\begin{equation}
		\label{eq:1.2.15}
		\int \frac{\diff u}{u - f \left( \frac{a_1 x_1 + b_1 u x_1}{a_2 x_1 + b_2 u x_1} \right)} + \ln (x_1) = C.
	\end{equation}

	І загальний інтеграл диференціального рівняння має вигляд $\Phi(u, x_1) = C$. Повернувшись до вихідних змінних, запишемо
	\begin{equation}
		\label{eq:1.2.16}
		\Phi \left( \frac{y - y_0}{x - x_0}, x - x_0 \right) = C.
	\end{equation}

	\item Нехай 
	\begin{equation}
		\label{eq:1.2.17}
		\Delta = \begin{vmatrix} a_1 & b_1 \\ a_2 & b_2 \end{vmatrix} = 0,
	\end{equation}
	тобто коефіцієнти рядків лінійно залежні і
	\begin{equation}
		\label{eq:1.2.18}
		a_1 x + b_1 y = \alpha \cdot (a_2 x + b_2 y).
	\end{equation}

	Робимо заміну $a_2 x + b_2 y = z$. Звідси $\frac{\diff y}{\diff x} = \frac{1}{b_2} \cdot \left( \frac{\diff z}{\diff x} - a_2 \right)$. \\

	Підставивши в диференціальне рівняння, одержимо
	\begin{equation}
		\label{eq:1.2.19}
		\frac{1}{b_2} \cdot \left( \frac{\diff z}{\diff x} - a_2 \right) = f \left ( \frac{\alpha z + c_1}{z + c_2} \right),
	\end{equation}
	або
	\begin{equation}
		\label{eq:1.2.20}
		\frac{\diff z}{\diff x} = a_2 + b_2 \cdot f \left ( \frac{\alpha z + c_1}{z + c_2} \right),
	\end{equation}
	Розділивши змінні, отримаємо
	\begin{equation}
		\label{eq:1.2.21}
		\int \frac{\diff z}{a_2 + b_2 \cdot f \left ( \frac{\alpha z + c_1}{z + c_2} \right)} - x = C,
	\end{equation}

	Загальний інтеграл має вигляд $\Phi(a_2 x + b_2 y, x) = C$
\end{enumerate}

\subsubsection{Вправи для самостійної роботи}

Однорідні рівняння можуть бути записані у вигляді \[y' = f \left( \frac{y}{x} \right)\] або  \[ M(x, y)\cdot\diff y + N(x, y) \cdot \diff y = 0, \] де $M(x, y)$ і $N(x, y)$ -- однорідні функції одного й того ж ступеня. Для того, щоб розв’язати однорідне рівняння, необхідно провести заміну \[y = u x, \quad \diff y = u \cdot \diff x + x \cdot \diff u,\] в результаті якої отримаємо рівняння зі змінними, що розділяються. 

\begin{example}
	Розв’язати рівняння $x \cdot \diff y = (x + y) \cdot \diff y$. 
\end{example}

\begin{solution}
	Дане рівняння однорідне, оскільки $x$ та $x + y$ є однорідними функціями першого ступеня. \\

	Проведемо заміну: $y = u x$. Тоді $\diff y = u \diff x + x \diff y$. Підставивши $y$ та $\diff y$ в задане рівняння, отримаємо  
	\begin{align*}
		x \cdot (x \diff u + u \diff x) &= (x + x u) \diff x, \\
		x^2 \diff u &= x \diff x
	\end{align*}

	Розв’яжемо це рівняння зі змінними, що розділяються:
	\begin{align*}
		\diff u &= \frac{\diff x}{x}, \\
		u &= \ln |x| + C.
	\end{align*}
	Повернувшись до вихідних змінних $u = y / x$, отримаємо \[y = x \cdot (\ln |x| + C).\] Крім того розв’язком є $x = 0$, що було загублене при поділенні рівняння на $x$.
\end{solution}

Розв’язати рівняння:
\begin{multicols}{2}
\begin{problem}
	\[ (x + 2y) \cdot \diff x - x \diff y = 0; \]
\end{problem}
\begin{problem}
	\[ (x - y) \cdot \diff x + (x + y) \cdot \diff y = 0; \]
\end{problem}
\begin{problem}
	\[y^2 + x^2 y' = x y y'; \]
\end{problem}
\begin{problem}
	\[ (x^2 + y^2) \cdot y' = 2 xy; \]
\end{problem}
\begin{problem}
	\[ xy' - y = x \cdot \tan \left( \frac{y}{x} \right); \]
\end{problem}
\begin{problem}
	\[ x y' = y - x e^{y / x}; \]
\end{problem}
\begin{problem}
	\[x y' - y = (x + y) \cdot \ln \left( \frac{x + y}{x} \right); \]
\end{problem}
\begin{problem}
	\[ (3x + y) \cdot \diff x - (2x + 3y) \cdot \diff y = 0; \]
\end{problem}
\begin{problem}
	\[ x y' = y \cos \left(\ln \left(\frac{y}{x} \right)\right); \]
\end{problem}
\begin{problem}
	\[ \left(y+\sqrt{xy}\right)\cdot \diff x = x \diff y; \]
\end{problem}
\begin{problem}
	\[ xy' = \sqrt{x^2 - y^2} + y; \]
\end{problem}
\begin{problem}
	\[ x^2 y' = y \cdot (x + y); \]
\end{problem}
\begin{problem}
	\[ y \cdot (-y + xy') = \sqrt{x^4 + y^4}; \]
\end{problem}
\begin{problem}
	\[ x \diff y - y \diff x = \sqrt{x^2 + y^2} \diff x; \]
\end{problem}
\begin{problem}
	\[ (y^2 - 2xy) \cdot \diff x + x^2 \cdot \diff y = 0; \]
\end{problem}
\begin{problem}
	\[ 2x^3 y' = y \cdot (2x^2 - y^2); \]
\end{problem}
\begin{problem}
	\[ \left( x - y \cos \left(\rfrac{y}{x}\right)\right) \diff x + x \cos \left(\rfrac{y}{x}\right) \diff y = 0; \]
\end{problem}
\begin{problem}
	\[ y' (xy - x^2) = y^2; \]
\end{problem}
\begin{problem}
	\[ 2xyy' = x^2 + y^2; \]
\end{problem}
\begin{problem}
	\[ (6x + 3y) \cdot \diff x = (7x - 2y) \cdot \diff y; \]
\end{problem}
\begin{problem}
	\[ y^2 x \diff x = y \cdot (xy - 2y^2) \cdot \diff y; \]
\end{problem}
\begin{problem}
	\[ x^2 y \diff x = y \cdot (xy - 2y^2) \cdot \diff y; \]
\end{problem}
\begin{problem}
	\[ 2y^3 = xy' \cdot (2y^2 - x^2); \]
\end{problem}
\begin{problem}
	\[ \left( x + \sqrt{xy}\right) \cdot \diff y = y \diff x; \]
\end{problem}
\begin{problem}
	\[ y = \left( \sqrt{y^2 - x^2} + x\right) y'; \]
\end{problem}
\begin{problem}
	\[ (3x - 2y) \cdot \diff x - (2x + y) \cdot \diff y = 0; \]
\end{problem}
\begin{problem}
	\[ (7x + 6y) \cdot \diff x - (x + 3y) \cdot \diff y = 0; \]
\end{problem}
\begin{problem}
	\[ xy' = y + x \cdot \cot \left(\frac{y}{x}\right). \]
\end{problem}
\end{multicols}

Знайти частинні розв’язки, що задовольняють задані початкові умови:
\begin{problem}
	\[ x y' = 4 \sqrt{2x^2 + y^2} + y, \quad y(1) = 2; \]
\end{problem}
\begin{problem}
	\[ (2y^2 + 3x^2) \cdot xy' = 3y^3 + 6yx^2, \quad y(2) = 1; \]
\end{problem}
\begin{problem}
	\[ y' (x^2 - 2xy) = x^2 + xy - y^2, \quad y(3) = 0; \]
\end{problem}
\begin{problem}
	\[ 2 y' = \frac{y^2}{x^2} + 8 \cdot \frac{y}{x} + 8, \quad y(1) = 1; \]
\end{problem}
\begin{problem}
	\[ y' (x^2 - 4xy) = x^2 + xy - 3 y^2, \quad y(1) = 1; \]
\end{problem}
\begin{problem}
	\[ xy' = 3 \sqrt{2x^2 + y^2} + y, \quad y(1) = 1; \]
\end{problem}
\begin{problem}
	\[ (2y^2 + 7x^2)\cdot x y' = 3y^3 + 14yx^2, \quad y(1) = 1; \]
\end{problem}
\begin{problem}
	\[ 2y' = \frac{y^2}{x^2} + 6 \cdot \frac{y}{x} + 3, \quad y(3) = 1; \]
\end{problem}
\begin{problem}
	\[ x^2 y' = y^2 + 4xy + 2x^2, \quad y(1) = 1; \]
\end{problem}
\begin{problem}
	\[ xy' = \sqrt{2x^2 + y^2} + y, \quad y(1) = 1; \]
\end{problem}
\begin{problem}
	\[ xy' = 3 \sqrt{x^2 + y^2} + y, \quad y(3) = 4; \]
\end{problem}
\begin{problem}
	\[ xy' = 2 \sqrt{x^2 + y^2} + y, \quad y(4) = 3; \]
\end{problem}
\begin{problem}
	\[ 2y' = \frac{y^2}{x^2} + 8 \cdot \frac{y}{x} + 8, \quad y(1) = 1; \]
\end{problem}
\begin{problem}
	\[ y' = \frac{x + 2y}{2x - y}, \quad y(3) = 8. \]
\end{problem}

		\subsubsection{Загальна теорія}
		% version 1.0
Нехай рівняння має вигляд
\begin{equation*}
	M(x, y) \cdot \diff	x + N(x, y) \cdot \diff y = 0.
\end{equation*}

Якщо функції $M(x, y)$ та $N(x, y)$ однорідні одного ступеня, то рівняння називається однорідним. Нехай функції $M(x, y)$ та $N(x, y)$ однорідні ступеня $k$, тобто
\begin{equation*}
	M(t \cdot x, t \cdot y) = t^k \cdot M(x, y), \qquad N(t \cdot x, t \cdot y) = t^k \cdot N(x, y).
\end{equation*}

Робимо заміну 
\begin{equation*}
	y = u x, \quad \diff y = u \diff x + x \diff u.
\end{equation*}

Після підстановки одержуємо
\begin{equation*}
	M(x, u x) \cdot \diff x + N(x, u x) \cdot (u \diff x + x \diff u) = 0,
\end{equation*}
або 
\begin{equation*}
	x^k M(1, u) \cdot \diff x + x^k N(1, u) \cdot (u \diff x + x \diff u) = 0.
\end{equation*}

Скоротивши на $x^k$ і розкривши дужки, запишемо 
\begin{equation*}
	M(1, u) \cdot \diff x + N(1, u) \cdot u \diff x + N(1, u) \cdot x \diff u = 0.
\end{equation*}

Згрупувавши, одержимо рівняння зі змінними, що розділяються
\begin{equation*}
	(M(1, u) + N(1, u) \cdot u) \diff x + N(1, u) \cdot x \diff u = 0,
\end{equation*}
або 
\begin{equation*}
	\int \frac{\diff x}{x} + \int \frac{N(1, u) \cdot \diff u}{M(1, u) + N(1, u) \cdot u} = C.
\end{equation*}

Взявши інтеграли та замінивши $u = y / x$, отримаємо загальний інтеграл $\Phi(x, y / x) = C$.

		\subsubsection{Рівняння, що зводяться до однорідних}
		Нехай маємо рівняння вигляду
\begin{equation*}
	%\label{eq:1.2.8}
	\label{eq:linear-fractional-equation}
	\frac{\diff y}{\diff x} = f \left( \frac{a_1 x + b_1 y + c_1}{a_2 x + b_2 y + c_2} \right).
\end{equation*}

Розглянемо два випадки
\begin{enumerate}
	\item 
	\begin{equation*}
		%\label{eq:1.2.9}
		\Delta = \begin{vmatrix} a_1 & b_1 \\ a_2 & b_2 \end{vmatrix} \ne 0.
	\end{equation*}

	Тоді система алгебраїчних рівнянь
	\begin{equation*}
		%\label{eq:1.2.10}
		\left\{
			\begin{aligned}
				a_1 x + b_1 y + c_1 &= 0, \\
				a_2 x + b_2 y + c_2 &= 0,
			\end{aligned}
		\right.
	\end{equation*}
	має єдиний розв’язок $(x_0, y_0)$. Проведемо заміну 
	\begin{equation*}
		%\label{eq:1.2.10_5}
		\left\{\begin{aligned}
			x &= x_1 + x_0, \\
			y &= y_1 + y_0
		\end{aligned}\right.
	\end{equation*}
	та отримаємо
	\begin{multline*}
		%\label{eq:1.2.11}
		\frac{\diff y_1}{\diff x_1} = f \left( \frac{a_1 \cdot (x_1 + x_0) + b_1 \cdot (y_1 + y_0) + c_1}{a_2 \cdot (x_1 + x_0) + b_2 \cdot (y_1 + y_0) + c_2} \right) = \\
		= f \left( \frac{a_1 x_1 + b_1 y_1 + (a_1 x_0 + b_1 y_0 + c_1)}{a_2 x_1 + b_2 y_1 + (a_2 x_0 + b_2 y_0 + c_2)} \right)
	\end{multline*}

	Оскільки $(x_0, y_0)$ -- розв’язок алгебраїчної системи, то диференціальне рівняння набуде вигляду
	\begin{equation*}
		%\label{eq:1.2.12}
		\frac{\diff y_1}{\diff x_1} = f \left( \frac{a_1 x_1 + b_1 y_1}{a_2 x_1 + b_2 y_1} \right)
	\end{equation*}
	і є однорідним нульового ступеня. Робимо заміну 
	\begin{equation*}
		%\label{eq:1.2.12_5}
		y_1 = u x_1, \quad \diff y_1 = u \cdot \diff x_1 + x_1 \cdot \diff u.
	\end{equation*}

	Підставимо в рівняння
	\begin{equation*}
		%\label{eq:1.2.13}
		u + x_1 \cdot \frac{\diff u}{\diff x_1} = f \left( \frac{a_1 x_1 + b_1 u x_1}{a_2 x_1 + b_2 u x_1} \right).
	\end{equation*}
	
	Одержимо
	\begin{equation*}
		%\label{eq:1.2.14}
		x_1 \cdot \diff u + \left( u - f \left( \frac{a_1 x_1 + b_1 u x_1}{a_2 x_1 + b_2 u x_1} \right) \right) \diff x_1 = 0.
	\end{equation*}

	Розділивши змінні, маємо
	\begin{equation*}
		%\label{eq:1.2.15}
		\int \frac{\diff u}{u - f \left( \frac{a_1 x_1 + b_1 u x_1}{a_2 x_1 + b_2 u x_1} \right)} + \ln (x_1) = C.
	\end{equation*}

	І загальний інтеграл рівняння має вигляд $\Phi(u, x_1) = C$. Повернувшись до вихідних змінних, запишемо
	\begin{equation*}
		%\label{eq:1.2.16}
		\Phi \left( \frac{y - y_0}{x - x_0}, x - x_0 \right) = C.
	\end{equation*}

	\item Нехай 
	\begin{equation*}
		%\label{eq:1.2.17}
		\Delta = \begin{vmatrix} a_1 & b_1 \\ a_2 & b_2 \end{vmatrix} = 0,
	\end{equation*}
	тобто коефіцієнти рядків лінійно залежні і
	\begin{equation*}
		%\label{eq:1.2.18}
		a_1 x + b_1 y = \alpha \cdot (a_2 x + b_2 y).
	\end{equation*}

	Робимо заміну $a_2 x + b_2 y = z$. Звідси $\frac{\diff y}{\diff x} = \frac{1}{b_2} \cdot \left( \frac{\diff z}{\diff x} - a_2 \right)$. \\

	Підставивши в диференціальне рівняння, одержимо
	\begin{equation*}
		%\label{eq:1.2.19}
		\frac{1}{b_2} \cdot \left( \frac{\diff z}{\diff x} - a_2 \right) = f \left ( \frac{\alpha z + c_1}{z + c_2} \right),
	\end{equation*}
	або
	\begin{equation*}
		%\label{eq:1.2.20}
		\frac{\diff z}{\diff x} = a_2 + b_2 \cdot f \left ( \frac{\alpha z + c_1}{z + c_2} \right),
	\end{equation*}
	Розділивши змінні, отримаємо
	\begin{equation*}
		%\label{eq:1.2.21}
		\int \frac{\diff z}{a_2 + b_2 \cdot f \left ( \frac{\alpha z + c_1}{z + c_2} \right)} - x = C,
	\end{equation*}

	Загальний інтеграл має вигляд $\Phi(a_2 x + b_2 y, x) = C$
\end{enumerate}


		\subsubsection{Вправи для самостійної роботи}
		% version 1.0
Однорідні рівняння можуть бути записані у вигляді \[y' = f \left( \frac{y}{x} \right)\] або  \[ M(x, y)\cdot\diff y + N(x, y) \cdot \diff y = 0, \] 

де $M(x, y)$ і $N(x, y)$ -- однорідні функції одного й того ж ступеня. Для того, щоб розв'язати однорідне рівняння, необхідно провести заміну \[y = u x, \quad \diff y = u \cdot \diff x + x \cdot \diff u,\] 

в результаті якої отримаємо рівняння зі змінними, що розділяються. 

\begin{example}
	Розв'язати рівняння $x \cdot \diff y = (x + y) \cdot \diff y$. 
\end{example}

\begin{solution}
	Дане рівняння однорідне, оскільки $x$ та $x + y$ є однорідними функціями першого ступеня. \\

	Проведемо заміну: $y = u x$. Тоді $\diff y = u \diff x + x \diff y$. Підставивши $y$ та $\diff y$ в задане рівняння, отримаємо  
	\begin{align*}
		x \cdot (x \diff u + u \diff x) &= (x + x u) \diff x, \\
		x^2 \diff u &= x \diff x
	\end{align*}

	Розв'яжемо це рівняння зі змінними, що розділяються:
	\begin{align*}
		\diff u &= \frac{\diff x}{x}, \\
		u &= \ln |x| + C.
	\end{align*}

	Повернувшись до вихідних змінних $u = y / x$, отримаємо \[y = x \cdot (\ln |x| + C).\] 

	Крім того розв'язком є $x = 0$, що було загублене при поділенні рівняння на $x$.
\end{solution}

Розв'язати рівняння:
\begin{multicols}{2}
	\begin{problem}
		\[ (x + 2y) \cdot \diff x - x \diff y = 0; \]
	\end{problem}

	\begin{problem}
		\[ (x - y) \cdot \diff x + (x + y) \cdot \diff y = 0; \]
	\end{problem}
	
	\begin{problem}
		\[y^2 + x^2 y' = x y y'; \]
	\end{problem}
	
	\begin{problem}
		\[ (x^2 + y^2) \cdot y' = 2 xy; \]
	\end{problem}
	
	\begin{problem}
		\[ xy' - y = x \cdot \tan \left( \frac{y}{x} \right); \]
	\end{problem}
	
	\begin{problem}
		\[ x y' = y - x e^{y / x}; \]
	\end{problem}
	
	\begin{problem}
		\[x y' - y = (x + y) \cdot \ln \left( \frac{x + y}{x} \right); \]
	\end{problem}
	
	\begin{problem}
		\[ (3x + y) \cdot \diff x - (2x + 3y) \cdot \diff y = 0; \]
	\end{problem}
	
	\begin{problem}
		\[ x y' = y \cos \left(\ln \left(\frac{y}{x} \right)\right); \]
	\end{problem}
	
	\begin{problem}
		\[ \left(y+\sqrt{xy}\right)\cdot \diff x = x \diff y; \]
	\end{problem}
	
	\begin{problem}
		\[ xy' = \sqrt{x^2 - y^2} + y; \]
	\end{problem}
	
	\begin{problem}
		\[ x^2 y' = y \cdot (x + y); \]
	\end{problem}
	
	\begin{problem}
		\[ y \cdot (-y + xy') = \sqrt{x^4 + y^4}; \]
	\end{problem}
	
	\begin{problem}
		\[ x \diff y - y \diff x = \sqrt{x^2 + y^2} \diff x; \]
	\end{problem}
	
	\begin{problem}
		\[ (y^2 - 2xy) \cdot \diff x + x^2 \cdot \diff y = 0; \]
	\end{problem}
	
	\begin{problem}
		\[ 2x^3 y' = y \cdot (2x^2 - y^2); \]
	\end{problem}
	
	\begin{problem}
		\[ \left( x - y \cos \left(\rfrac{y}{x}\right)\right) \diff x = - x \cos \left(\rfrac{y}{x}\right) \diff y; \]
	\end{problem}
	
	\begin{problem}
		\[ y' (xy - x^2) = y^2; \]
	\end{problem}
	
	\begin{problem}
		\[ 2xyy' = x^2 + y^2; \]
	\end{problem}
	
	\begin{problem}
		\[ (6x + 3y) \cdot \diff x = (7x - 2y) \cdot \diff y; \]
	\end{problem}
	
	\begin{problem}
		\[ y^2 x \diff x = y \cdot (xy - 2y^2) \cdot \diff y; \]
	\end{problem}
	
	\begin{problem}
		\[ x^2 y \diff x = y \cdot (xy - 2y^2) \cdot \diff y; \]
	\end{problem}
	
	\begin{problem}
		\[ 2y^3 = xy' \cdot (2y^2 - x^2); \]
	\end{problem}
	
	\begin{problem}
		\[ \left( x + \sqrt{xy}\right) \cdot \diff y = y \diff x; \]
	\end{problem}
	
	\begin{problem}
		\[ y = \left( \sqrt{y^2 - x^2} + x\right) y'; \]
	\end{problem}
	
	\begin{problem}
		\[ (3x - 2y) \cdot \diff x - (2x + y) \cdot \diff y = 0; \]
	\end{problem}
	
	\begin{problem}
		\[ (7x + 6y) \cdot \diff x - (x + 3y) \cdot \diff y = 0; \]
	\end{problem}
	
	\begin{problem}
		\[ xy' = y + x \cdot \cot \left(\frac{y}{x}\right). \]
	\end{problem}
\end{multicols}

Знайти частинні розв'язки, що задовольняють задані початкові умови:
\begin{problem}
	\[ x y' = 4 \sqrt{2x^2 + y^2} + y, \quad y(1) = 2; \]
\end{problem}

\begin{problem}
	\[ (2y^2 + 3x^2) \cdot xy' = 3y^3 + 6yx^2, \quad y(2) = 1; \]
\end{problem}

\begin{problem}
	\[ y' (x^2 - 2xy) = x^2 + xy - y^2, \quad y(3) = 0; \]
\end{problem}

\begin{problem}
	\[ 2 y' = \frac{y^2}{x^2} + 8 \cdot \frac{y}{x} + 8, \quad y(1) = 1; \]
\end{problem}

\begin{problem}
	\[ y' (x^2 - 4xy) = x^2 + xy - 3 y^2, \quad y(1) = 1; \]
\end{problem}

\begin{problem}
	\[ xy' = 3 \sqrt{2x^2 + y^2} + y, \quad y(1) = 1; \]
\end{problem}

\begin{problem}
	\[ (2y^2 + 7x^2)\cdot x y' = 3y^3 + 14yx^2, \quad y(1) = 1; \]
\end{problem}

\begin{problem}
	\[ 2y' = \frac{y^2}{x^2} + 6 \cdot \frac{y}{x} + 3, \quad y(3) = 1; \]
\end{problem}

\begin{problem}
	\[ x^2 y' = y^2 + 4xy + 2x^2, \quad y(1) = 1; \]
\end{problem}

\begin{problem}
	\[ xy' = \sqrt{2x^2 + y^2} + y, \quad y(1) = 1; \]
\end{problem}

\begin{problem}
	\[ xy' = 3 \sqrt{x^2 + y^2} + y, \quad y(3) = 4; \]
\end{problem}

\begin{problem}
	\[ xy' = 2 \sqrt{x^2 + y^2} + y, \quad y(4) = 3; \]
\end{problem}

\begin{problem}
	\[ 2y' = \frac{y^2}{x^2} + 8 \cdot \frac{y}{x} + 8, \quad y(1) = 1; \]
\end{problem}

\begin{problem}
	\[ y' = \frac{x + 2y}{2x - y}, \quad y(3) = 8. \]
\end{problem}

	\subsection{Лінійні рівняння першого порядку}
	\subsubsection{Загальна теорія}

Рівняння, що є лінійним відносно невідомої функції та її похідної, називається лінійним диференціальним рівнянням. Його загальний вигляд такий:
\begin{equation*}
	%\label{eq:1.3.1}
	\frac{\diff y}{\diff x} + p(x) \cdot y = q(x).
\end{equation*}

Якщо $q(x) \equiv 0$, тобто рівняння має вигляд
\begin{equation*}
	%\label{eq:1.3.2}
	\frac{\diff y}{\diff x} + p(x) \cdot y = 0,
\end{equation*}
то воно зветься однорідним. Однорідне рівняння є рівнянням зі змінними, що розділяються і розв’язується таким чином:
\begin{align}
	%\label{eq:1.3.2_25}
	\frac{\diff y}{y} &= -p(x) \cdot \diff x, \\
	\int \frac{\diff y}{y} &= - \int p(x) \cdot \diff x, \\
	\ln y &= - \int p(x) \cdot \diff x + \ln C.
\end{align}
Нарешті 
\begin{equation*}
	%\label{eq:1.3.2_5}
	y = C \cdot \exp \left\{ - \int p(x) \cdot \diff x \right\}
\end{equation*}

Розв’язок неоднорідного рівняння будемо шукати методом варіації довільних сталих (методом невизначених множників Лагранжа). Він складається в тому, що розв’язок неоднорідного рівняння шукається в такому ж вигляді, як і розв’язок однорідного, але $C$ вважається невідомою функцією від $x$, тобто $C = C(x)$ і 
\begin{equation*}
	%\label{eq:1.3.2_75}
	y = C(x) \cdot \exp \left\{ - \int p(x) \cdot \diff x \right\}	
\end{equation*}

Для знаходження $C(x)$ підставимо $y$ у рівняння
\begin{multline} 
	%\label{eq:1.3.3}
	\frac{\diff C(x)}{\diff x} \cdot \exp \left\{ - \int p(x) \cdot \diff x \right\} = - C(x) \cdot p(x) \cdot \exp \left\{ - \int p(x) \cdot \diff x \right\} + \\
	+ p(x) \cdot C(x) \cdot \exp \left\{ - \int p(x) \cdot \diff x \right\} = q(x).
\end{multline}

Звідси
\begin{equation*} 
	%\label{eq:1.3.4}
	\diff C(x) = q(x) \cdot \exp \left\{\int p(x) \cdot \diff x \right\} \cdot \diff x.
\end{equation*}

Проінтегрувавши, одержимо
\begin{equation*} 
	%\label{eq:1.3.5}
	C(x) = \int q(x) \cdot \exp \left\{\int p(x) \cdot \diff x \right\} \cdot \diff x + C.
\end{equation*}

І загальний розв’язок неоднорідного рівняння має вигляд
\begin{multline} 
	%\label{eq:1.3.6}
	y = \exp \left\{ - \int p(x) \cdot \diff x \right\} \cdot \\
	\cdot \left( \int q(x) \cdot \exp \left\{\int p(x) \cdot \diff x \right\} \cdot \diff x + C\right).
\end{multline}

Якщо використовувати початкові умови $y(x_0) = y_0$, то розв’язок можна записати у формі Коші:
\begin{multline} 
	%\label{eq:1.3.7}
	y(x, x_0, y_0) = \exp \left\{ - \int_{x_0}^x p(t) \cdot \diff t \right\} \cdot \\
	\cdot \left( \int_{x_0}^x q(t) \cdot \exp \left\{\int_t^x p(\xi) \cdot \diff \xi \right\} \cdot \diff t + y_0\right).
\end{multline}

\subsubsection{Рівняння Бернуллі}

Рівняння вигляду
\begin{equation*}
	%\label{eq:1.3.8}
	\frac{\diff y}{\diff x} + p(x) \cdot y = q(x) \cdot y^m, \quad m \ne 1
\end{equation*}
називається рівнянням Бернуллі. Розділимо на $y^m$ і одержимо 
\begin{equation*}
	%\label{eq:1.3.9}
	y^{-m} \cdot \frac{\diff y}{\diff x} + p(x) \cdot y^{1-m} = q(x).
\end{equation*}

Зробимо заміну: 
\begin{equation*}
	%\label{eq:1.3.9_5}
	y^{1-m} = z, \quad (1 - m) \cdot y^{-m} \cdot \frac{\diff y}{\diff x} = \diff z.
\end{equation*}

Підставивши в рівняння, отримаємо
\begin{equation*}
	%\label{eq:1.3.10}
	\frac{1}{1-m} \cdot \frac{\diff z}{\diff x} + p(x) \cdot z = q(x).
\end{equation*}

Одержали лінійне диференціальне рівняння. Його розв’язок має вигляд
\begin{multline}
	%\label{eq:1.3.11}
	z = \exp\left\{ -(1 - m) \cdot \int p(x) \diff x \right\} \cdot \\
	\cdot \left( (1-m) \cdot \int q(x) \cdot \exp\left\{ (1 - m) \cdot \int p(x) \diff x \right\} + C\right).
\end{multline}
 
\subsubsection{Рівняння Рікатті}

Рівняння вигляду 
\begin{equation*}
	%\label{eq:1.3.12}
	\frac{\diff y}{\diff x} + p(x) \cdot y + r(x) \cdot y^2 = q(x)
\end{equation*} 
називається рівнянням Рікатті. В загальному випадку рівняння Рікатті не інтегрується. Відомі лише деякі частинні випадки рівнянь Рікатті, що інтегруються в квадратурах. Розглянемо один з них. Нехай відомий один частинний розв’язок $y = y_1(x)$. Робимо заміну $y = y_1(x) + z$ і одержуємо
\begin{equation*}
	%\label{eq:1.3.13}
	\frac{\diff y_1(x)}{\diff x} + \frac{\diff z}{\diff x} + p(x) \cdot (y_1(x) + z) + r(x) \cdot (y_1(x) + z)^2 = q(x).
\end{equation*}

Оскільки $y_1(x)$ -- частинний розв’язок, то
\begin{equation*}
	%\label{eq:1.3.14}
	\frac{\diff y_1(x)}{\diff x} + p(x) \cdot y_1 + r(x) \cdot y_1^2 = q(x).
\end{equation*}

Розкривши в попередній рівності дужки і використовуючи останнє зауваження, одержуємо
\begin{equation*}
	%\label{eq:1.3.15}
	\frac{\diff z}{\diff x} + p(x) \cdot z + 2 r(x) \cdot y_1(x) \cdot z + r(x) \cdot z^2 = 0.
\end{equation*}

Перепишемо одержане рівняння у вигляді
\begin{equation*}
	%\label{eq:1.3.16}
	\frac{\diff z}{\diff x} + \left(p(x) + 2 r(x) \cdot y_1(x)\right) \cdot z = - r(x) \cdot z^2 ,
\end{equation*}
це рівняння Бернуллі з $m = 2$.

\subsubsection{Вправи для самостійної роботи}

\begin{example}
	Розв’язати рівняння \[ y' - y \cdot \tan x = \cos x.\]
\end{example}
\begin{solution}
	Використовуючи вигляд загального розв’язку, отримаємо
	\[ y = \exp\left\{\int \tan x \diff x\right\} \cdot \left(\int \exp\left\{-\int\tan x\diff x\right\}\cdot \cos x \diff x+C\right). \]

	Оскільки \[\int \tan x \diff x = - \ln |\cos x|,\] то отримаємо
	\begin{align*}
		y &= e^{-\ln|\cos x|} \cdot \left(\int e^{\ln|\cos x|} \cdot \cos x \diff x+C\right) = \\
		&= \frac{1}{\cos x} \cdot \left( \int \cos^2 x \diff x + C \right) = \\
		&= \frac{1}{\cos x} \cdot \left( \frac x2 + \frac{\sin 2x}{4} + C \right).
	\end{align*}
	Або
	\[ y = \frac{C}{\cos x} + \frac{x}{2 \cos x} + \frac{\sin x}{2}. \]
\end{solution}

\begin{example}
	Знайти частинний розв’язок рівняння \[ y' - \frac yx = x^2,\] що задовольняє початковій умові $y(2) = 2$.
\end{example}
\begin{solution}
	Використовуючи вигляд загального розв’язку, отримаємо
	\begin{align*}
		y &= \exp\left\{\int \frac1x \diff x\right\} \cdot \left(\int \exp\left\{-\int\frac1x\diff x\right\}\cdot x^2 \diff x+C\right) = \\
		&= e^{\ln|x|} \cdot \left(\int e^{-\ln|x|} \cdot x^2 \diff x+C\right) = \\
		&= x \cdot \left(\int x \diff x+C\right) = \\
		&= x \cdot \left(\frac{x^2}{2} + C\right).
	\end{align*}
	Таким чином 
	\[ y = C x + \frac{x^3}{2}.\]
	Підставивши початкові умови $y(2) = 2$, одержимо $2 = 2C + 4$. Звідси $C = -1$ і частинний розв’язок має вигляд \[ y_{\text{част.}} = \frac{x^3}{2} - x.\]
\end{solution}

Розв’язати рівняння:
\begin{multicols}{2}
\begin{problem}
	\[x y' + (x + 1) \cdot y = 3 x^2 e^{-x};\]
\end{problem}
\begin{problem}
	\[(2x + 1) \cdot y' =4x+2y;\]
\end{problem}
\begin{problem}
	\[y'=2x\cdot(x^2+y);\]
\end{problem}
\begin{problem}
	\[x^2y'+xy+1=0;\]
\end{problem}
\begin{problem}
	\[y'+y\cdot\tan x=\sec x;\]
\end{problem}
\begin{problem}
	\[x\cdot(y'-y)=e^x;\]
\end{problem}
\begin{problem}
	\[(xy'-1)\cdot\ln x=2y;\]
\end{problem}
\begin{problem}
	\[(y+x^2)\cdot \diff x=x \cdot\diff y;\]
\end{problem}
\begin{problem}
	\[(2e^x-y)\cdot\diff x=\diff y;\]
\end{problem}
\begin{problem}
	\[\sin^2 y + x \cdot \cot y = \frac1{y^2};\]
\end{problem}
\begin{problem}
	\[(x+y^2)\cdot y'=y;\]
\end{problem}
\begin{problem}
	\[(3e^y-x)\cdot y' = 1;\]
\end{problem}
\begin{problem}
	\[y = x\cdot(y'- x \cdot \cos x).\]
\end{problem}
\end{multicols}

Знайти частинні розв’язки рівняння з заданими початковими умовами:
\begin{problem}
	\[y'-\frac yx=-\frac{\ln x}x, \quad y(1)=1;\]
\end{problem}
\begin{problem}
	\[y'-\frac{2xy}{1+x^2}=1+x^2, \quad y(1)=3;\]
\end{problem}
\begin{problem}
	\[y'-\frac{2y}{x+1}=e^{x}\cdot(x+1)^2, \quad y(0)=1;\]
\end{problem}
\begin{problem}
	\[xy'+2y=x64,\quad y(1)=-\frac58;\]	
\end{problem}
\begin{problem}
	\[ y' - \frac yx  = x \cdot \sin x, \quad y\left(\frac\pi2\right)=1;\]
\end{problem}
\begin{problem}
	\[y'+\frac yx=\sin x, \quad y(\pi)=\frac1\pi;\]
\end{problem}
\begin{problem}
	\[(13y^3-x)\cdot y'=4y, \quad y(5)=1;\]
\end{problem}
\begin{problem}
	\[2\cdot(x+\ln^2y-\ln y)\cdot y'= y, \quad y(2)=1.\]
\end{problem}

Розв’язати рівняння Бернуллі:
\begin{multicols}{2}
\begin{problem}
	\[y'+xy=(1+x)\cdot e^{-x}\cdot y^2;\]
\end{problem}
\begin{problem}
	\[xy'+y=2y^2\cdot \ln x;\]
\end{problem}
\begin{problem}
	\[2\cdot(2xy'+y)=xy^2;\]
\end{problem}
\begin{problem}
	\[3\cdot(xy'+y)=y^2\cdot \ln x;\]
\end{problem}
\begin{problem}
	\[2\cdot(y'+y)=xy^2.\]
\end{problem}
\end{multicols}

Розв’язати рівняння Рікатті:
\begin{multicols}{2}
\begin{problem}
	\[x^2\cdot y' + xy +x^2y^2=4;\]
\end{problem}
\begin{problem}
	\[3y'+y^2+\frac2x=0;\]
\end{problem}
\begin{problem}
	\[xy'-(2x+1)\cdot y+y^2=5-x^2;\]
\end{problem}
\begin{problem}
	\[y'-2xy+y^2=5-x^2;\]
\end{problem}
\begin{problem}
	\[y'+2y \cdot e^x - y^2 = e^{2x} + e^x.\]
\end{problem}
\end{multicols}

		\subsubsection{Загальна теорія}
		Рівняння, що є лінійним відносно невідомої функції та її похідної, називається лінійним диференціальним рівнянням. Його загальний вигляд такий:
\begin{equation*}
	\frac{\diff y}{\diff x} + p(x) y = q(x).
\end{equation*}

Якщо $q(x) \equiv 0$, тобто рівняння має вигляд
\begin{equation*}
	\frac{\diff y}{\diff x} + p(x) y = 0,
\end{equation*}
то воно зветься однорідним. Однорідне рівняння є рівнянням зі змінними, що розділяються і розв'язується таким чином:
\begin{align*}
	\frac{\diff y}{y} &= -p(x) \diff x, \\
	\int \frac{\diff y}{y} &= - \int p(x) \diff x, \\
	\ln y &= - \int p(x) \diff x + \ln C.
\end{align*}

Нарешті 
\begin{equation*}
	y = C \exp \left\{ - \int p(x) \diff x \right\}
\end{equation*}

Розв'язок неоднорідного рівняння будемо шукати методом варіації довільних сталих (методом невизначених множників Лагранжа). Він складається в тому, що розв'язок неоднорідного рівняння шукається в такому ж вигляді, як і розв'язок однорідного, але $C$ вважається невідомою функцією від $x$, тобто $C = C(x)$ і 
\begin{equation*}
	y = C(x) \exp \left\{ - \int p(x) \diff x \right\}	
\end{equation*}

Для знаходження $C(x)$ підставимо $y$ у рівняння
\begin{multline*} 
	\frac{\diff C(x)}{\diff x} \exp \left\{ - \int p(x) \diff x \right\} = - C(x) p(x) \exp \left\{ - \int p(x) \diff x \right\} + \\
	+ p(x) C(x) \exp \left\{ - \int p(x) \diff x \right\} = q(x).
\end{multline*}

Звідси
\begin{equation*} 
	\diff C(x) = q(x) \exp \left\{\int p(x) \diff x \right\} \diff x.
\end{equation*}

Проінтегрувавши, одержимо
\begin{equation*} 
	C(x) = \int q(x) \exp \left\{\int p(x) \diff x \right\} \diff x + C.
\end{equation*}

І загальний розв'язок неоднорідного рівняння має вигляд
\begin{equation*} 
	y = \exp \left\{ - \int p(x) \diff x \right\} \left( \int q(x) \exp \left\{\int p(x) \diff x \right\} \diff x + C\right).
\end{equation*}

Якщо використовувати початкові умови $y(x_0) = y_0$, то розв'язок можна записати у формі Коші:
\begin{equation*} 
	y(x, x_0, y_0) = \exp \left\{ - \int_{x_0}^x p(t) \diff t \right\} \left( \int_{x_0}^x q(t) \exp \left\{\int_t^x p(\xi) \diff \xi \right\} \diff t + y_0\right).
\end{equation*}


		\subsubsection{Рівняння Бернуллі}
		Рівняння вигляду
\begin{equation*}
	\frac{\diff y}{\diff x} + p(x) y = q(x) y^m, \quad m \ne 1
\end{equation*}
називається рівнянням Бернуллі. Розділимо на $y^m$ і одержимо 
\begin{equation*}
	y^{-m} \frac{\diff y}{\diff x} + p(x) y^{1-m} = q(x).
\end{equation*}

Зробимо заміну: 
\begin{equation*}
	y^{1-m} = z, \quad (1 - m) y^{-m} \frac{\diff y}{\diff x} = \frac{\diff z}{\diff x}. % thanks to Denys Chergykalo for a valuable suggestion here
\end{equation*}

Підставивши в рівняння, отримаємо
\begin{equation*}
	\frac{1}{1-m} \cdot \frac{\diff z}{\diff x} + p(x) z = q(x).
\end{equation*}

Одержали лінійне диференціальне рівняння. Його розв’язок має вигляд
\begin{multline*}
	z = \exp\left\{ -(1 - m) \int p(x) \diff x \right\} \cdot \\ 
	\cdot \left( (1-m) \int q(x) \exp\left\{ (1 - m) \int p(x) \diff x \right\} \diff x + C\right).
\end{multline*}


		\subsubsection{Рівняння Рікатті}
		Рівняння вигляду 
\begin{equation*}
	\frac{\diff y}{\diff x} + p(x) y + r(x) y^2 = q(x)
\end{equation*} 
називається рівнянням Рікатті. В загальному випадку рівняння Рікатті не інтегрується. Відомі лише деякі частинні випадки рівнянь Рікатті, що інтегруються в квадратурах. Розглянемо один з них. Нехай відомий один частинний розв’язок $y = y_1(x)$. Робимо заміну $y = y_1(x) + z$ і одержуємо
\begin{equation*}
	\frac{\diff y_1(x)}{\diff x} + \frac{\diff z}{\diff x} + p(x) (y_1(x) + z) + r(x) (y_1(x) + z)^2 = q(x).
\end{equation*}

Оскільки $y_1(x)$ --- частинний розв’язок, то
\begin{equation*}
	\frac{\diff y_1(x)}{\diff x} + p(x) y_1 + r(x) y_1^2 = q(x).
\end{equation*}

Розкривши в попередній рівності дужки і використовуючи останнє зауваження, одержуємо
\begin{equation*}
	\frac{\diff z}{\diff x} + p(x) z + 2 r(x) y_1(x) z + r(x) z^2 = 0.
\end{equation*}

Перепишемо одержане рівняння у вигляді
\begin{equation*}
	\frac{\diff z}{\diff x} + \left(p(x) + 2 r(x) y_1(x)\right) z = - r(x) z^2 ,
\end{equation*}
це рівняння Бернуллі з $m = 2$.


		\subsubsection{Вправи для самостійної роботи}
		\begin{example}
	Розв'язати рівняння \[ y' - y \tan x = \cos x.\]
\end{example}

\begin{solution}
	Використовуючи вигляд загального розв'язку, отримаємо \[ y = \exp\left\{\int \tan x \diff x\right\} \left(\int \exp\left\{-\int\tan x\diff x\right\} \cos x \diff x + C \right). \]

	Оскільки \[\int \tan x \diff x = - \ln |\cos x|,\] 

	то отримаємо
	\begin{align*}
		y &= e^{-\ln|\cos x|} \left(\int e^{\ln|\cos x|} \cos x \diff x+C\right) = \\
		&= \frac{1}{\cos x} \left( \int \cos^2 x \diff x + C \right) = \\
		&= \frac{1}{\cos x} \left( \frac x2 + \frac{\sin 2x}{4} + C \right).
	\end{align*}
	
	Або
	\[ y = \frac{C}{\cos x} + \frac{x}{2 \cos x} + \frac{\sin x}{2}. \]
\end{solution}

\begin{example}
	Знайти частинний розв'язок рівняння \[ y' - \frac yx = x^2,\] 

	що задовольняє початковій умові $y(2) = 2$.
\end{example}

\begin{solution}
	Використовуючи вигляд загального розв'язку, отримаємо
	\begin{align*}
		y &= \exp\left\{\int \frac1x \diff x\right\} \left(\int \exp\left\{-\int\frac1x\diff x\right\} x^2 \diff x+C\right) = \\
		&= e^{\ln|x|} \left(\int e^{-\ln|x|} x^2 \diff x+C\right) = \\
		&= x \left(\int x \diff x+C\right) = \\
		&= x \left(\frac{x^2}{2} + C\right).
	\end{align*}

	Таким чином \[ y = C x + \frac{x^3}{2}.\]
	
	Підставивши початкові умови $y(2) = 2$, одержимо $2 = 2C + 4$. Звідси $C = -1$ і частинний розв'язок має вигляд \[ y_{\text{част.}} = \frac{x^3}{2} - x.\]
\end{solution}

Розв'язати рівняння:
\begin{multicols}{2}
	\begin{problem}
		\[x y' + (x + 1) y = 3 x^2 e^{-x};\]
	\end{problem}
	
	\begin{problem}
		\[(2x + 1) y' =4x+2y;\]
	\end{problem}
	
	\begin{problem}
		\[y'=2x(x^2+y);\]
	\end{problem}
	
	\begin{problem}
		\[x^2y'+xy+1=0;\]
	\end{problem}
	
	\begin{problem}
		\[y'+y\tan x=\sec x;\]
	\end{problem}
	
	\begin{problem}
		\[x(y'-y)=e^x;\]
	\end{problem}
	
	\begin{problem}
		\[(xy'-1)\ln x=2y;\]
	\end{problem}
	
	\begin{problem}
		\[(y+x^2) \diff x=x \diff y;\]
	\end{problem}
	
	\begin{problem}
		\[(2e^x-y)\diff x=\diff y;\]
	\end{problem}
	
	\begin{problem}
		\[\sin^2 y + x \cot y = \frac1{y^2};\]
	\end{problem}
	
	\begin{problem}
		\[(x+y^2) y'=y;\]
	\end{problem}
	
	\begin{problem}
		\[(3e^y-x) y' = 1;\]
	\end{problem}
	
	\begin{problem}
		\[y = x(y'- x \cos x).\]
	\end{problem}
\end{multicols}

Знайти частинні розв'язки рівняння з заданими початковими умовами:
\begin{problem}
	\[y'-\frac yx=-\frac{\ln x}x, \quad y(1)=1;\]
\end{problem}

\begin{problem}
	\[y'-\frac{2xy}{1+x^2}=1+x^2, \quad y(1)=3;\]
\end{problem}

\begin{problem}
	\[y'-\frac{2y}{x+1}=e^{x}(x+1)^2, \quad y(0)=1;\]
\end{problem}

\begin{problem}
	\[xy'+2y=x64,\quad y(1)=-\frac58;\]	
\end{problem}

\begin{problem}
	\[ y' - \frac yx = x \sin x, \quad y\left(\frac\pi2\right)=1;\]
\end{problem}

\begin{problem}
	\[y'+\frac yx=\sin x, \quad y(\pi)=\frac1\pi;\]
\end{problem}

\begin{problem}
	\[(13y^3-x) y'=4y, \quad y(5)=1;\]
\end{problem}

\begin{problem}
	\[2(x+\ln^2y-\ln y) y'= y, \quad y(2)=1.\]
\end{problem}

Розв'язати рівняння Бернуллі:
\begin{multicols}{2}
	\begin{problem}
		\[y'+xy=(1+x) e^{-x} y^2;\]
	\end{problem}
	
	\begin{problem}
		\[xy'+y=2y^2 \ln x;\]
	\end{problem}
	
	\begin{problem}
		\[2(2xy'+y)=xy^2;\]
	\end{problem}
	
	\begin{problem}
		\[3(xy'+y)=y^2 \ln x;\]
	\end{problem}

	\begin{problem}
		\[2(y'+y)=xy^2.\]
	\end{problem}
\end{multicols}

Розв'язати рівняння Рікатті:
\begin{multicols}{2}
	\begin{problem}
		\[x^2 y' + xy +x^2y^2=4;\]
	\end{problem}

	\begin{problem}
		\[3y'+y^2+\frac2x=0;\]
	\end{problem}

	\begin{problem}
		\[xy'-(2x+1) y+y^2=5-x^2;\]
	\end{problem}

	\begin{problem}
		\[y'-2xy+y^2=5-x^2;\]
	\end{problem}
	
	\begin{problem}
		\[y'+2y e^x - y^2 = e^{2x} + e^x.\]
	\end{problem}
\end{multicols}

	\subsection{Рівняння в повних диференціалах}
	\subsubsection{Загальна теорія}
Якщо ліва частина диференціального рівняння
\begin{equation}
	\label{eq:1.4.1}
	M(x, y) \cdot \diff x + N(x, y) \cdot \diff y = 0,
\end{equation}
є повним диференціалом деякої функції $u(x, y)$, тобто
\begin{equation}
	\label{eq:1.4.2}
	\diff u(x, y) = M(x, y) \cdot \diff x + N(x, y) \cdot \diff y,
\end{equation}
і, таким чином, \eqref{eq:1.4.1} приймає вигляд $\diff u (x, y) = 0$ то рівняння називається рівнянням в повних диференціалах. Звідси вираз
\begin{equation}
	\label{eq:1.4.3}
	u(x, y) = C
\end{equation}
є загальним інтегралом диференціального рівняння. \\

Критерієм того, що рівняння є рівнянням в повних диференціалах, тобто необхідною та достатньою умовою, є виконання рівності
\begin{equation}
	\label{eq:1.4.4}
	\frac{\partial M(x, y)}{\partial y} = \frac{\partial N(x, y)}{\partial x}.
\end{equation}
 
Нехай маємо рівняння в повних диференціалах. Тоді
\begin{equation}
	\label{eq:1.4.5}
	\frac{\partial u(x, y)}{\partial x} = M(x, y), \quad \frac{\partial u(x, y)}{\partial y} = N(x, y).
\end{equation}
Звідси $u(x, y) = \int M(x, y) \diff x + \phi(y)$ де $\phi(y)$ -- невідома функція. Для її визначення продиференціюємо співвідношення по $y$ і прирівняємо $N(x, y)$:
\begin{equation}
	\label{eq:1.4.6}
	\frac{\partial u(x, y)}{\partial y} = \frac{\partial}{\partial y} \left( \int M(x, y) \diff x \right) + \frac{\diff \phi(y)}{\diff y} = N(x, y).
\end{equation}
Звідси
\begin{equation}
	\label{eq:1.4.7}
	\phi(y) = \int \left( N(x, y) - \frac{\partial}{\partial y} \left( \int M(x, y) \diff x \right) \right) \diff y.
\end{equation}
Остаточно, загальний інтеграл має вигляд
\begin{equation}
	\label{eq:1.4.8}
	\int M(x, y) \diff x + \int \left( N(x, y) - \frac{\partial}{\partial y} \left( \int M(x, y) \diff x \right) \right) \diff y = C.
\end{equation}
Як відомо з математичного аналізу, якщо відомий повний диференціал \eqref{eq:1.4.2}, то $u(x, y)$ можна визначити, взявши криволінійний інтеграл по довільному контуру, що з’єднує фіксовану точку $(x_0, y_0)$ і точку із змінними координатами $(x, y)$. \\

Більш зручно брати криву, що складається із двох відрізків прямих. В цьому випадку криволінійний інтеграл розпадається на два простих інтеграла
\begin{multline}
	\label{eq:1.4.9}
	u(x, y) = \int_{(x_0, y_0)}^{(x,y)} M(x,y) \cdot \diff x + N(x, y) \cdot \diff y = \\
	= \int_{(x_0, y_0)}^{(x,y_0)} M(x,y) \cdot \diff x + \int_{(x,y_0)}^{(x,y)} N(x, y) \cdot \diff y = \\
	= \int_{x_0}^{x} M(x,y_0) \cdot \diff x + \int_{y_0}^{y} N(x, y) \cdot \diff y.
\end{multline}
В цьому випадку одразу одержуємо розв’язок задачі Коші.
\begin{equation}
	\label{eq:1.4.10}
	\int_{x_0}^{x} M(x,y_0) \cdot \diff x + \int_{y_0}^{y} N(x, y) \cdot \diff y = 0.
\end{equation}

\subsubsection{Множник, що Інтегрує}
В деяких випадках рівняння \eqref{eq:1.4.1} не є рівнянням в повних диференціалах, але існує функція $\mu = \mu(x,y)$ така, що рівняння
\begin{equation}
	\label{eq:1.4.11}
	\mu(x,y) \cdot M(x, y) \cdot \diff x + \mu(x,y) \cdot N(x, y) \cdot \diff y = 0,
\end{equation}
вже буде рівнянням в повних диференціалах. Необхідною та достатньою умовою цього є рівність
\begin{equation}
	\label{eq:1.4.12}
	\frac{\partial}{\partial y} (\mu(x,y) \cdot M(x, y)) = \frac{\partial}{\partial x} (\mu(x,y) \cdot N(x, y)),
\end{equation}
або
\begin{equation}
	\label{eq:1.4.13}
	\frac{\partial \mu}{\partial y} \cdot M + \mu \cdot \frac{\partial M}{\partial y} = \frac{\partial \mu}{\partial x} \cdot N + \mu \cdot \frac{\partial N}{\partial x}.
\end{equation}
Таким чином замість звичайного диференціального рівняння відносно функції $y(x)$ одержимо диференціальне рівняння в частинних похідних відносно функції $\mu(x, y)$. \\

Задача інтегрування його значно спрощується, якщо відомо в якому вигляді шукати функцію $\mu(x,y)$, наприклад $\mu = \mu(\omega(x,y))$ де $\omega(x,y)$ -- відома функція. В цьому випадку одержуємо
\begin{equation}
	\label{eq:1.4.14}
	\frac{\partial \mu}{\partial y} = \frac{\diff \mu}{\diff \omega} \cdot \frac{\partial \omega}{\partial y}, \quad \frac{\partial \mu}{\partial x} = \frac{\diff \mu}{\diff \omega} \cdot \frac{\partial \omega}{\partial x}
\end{equation}
Після підстановки в \eqref{eq:1.4.13} маємо
\begin{equation}
	\label{eq:1.4.15}
	\frac{\diff \mu}{\diff \omega} \cdot \frac{\partial \omega}{\partial y} \cdot M + \mu \cdot \frac{\partial M}{\partial y} = \frac{\diff \mu}{\diff \omega} \cdot \frac{\partial \omega}{\partial x} \cdot N + \mu \cdot \frac{\partial N}{\partial x}.
\end{equation}
або
\begin{equation}
	\label{eq:1.4.16}
	\frac{\diff \mu}{\diff \omega} \left( \frac{\partial \omega}{\partial x} \cdot N - \frac{\partial \omega}{\partial y} \cdot M \right) = \mu \left( \frac{\partial M}{\partial y} - \frac{\partial N}{\partial x} \right).
\end{equation}
Розділимо змінні
\begin{equation}
	\label{eq:1.4.17}
	\frac{\diff \mu}{\mu} = \frac{\frac{\partial M}{\partial y} - \frac{\partial N}{\partial x} }{\frac{\partial \omega}{\partial x} \cdot N - \frac{\partial \omega}{\partial y} \cdot M} \cdot \diff \omega.
\end{equation}
Проінтегрувавши і поклавши сталу інтегрування одиницею, одержимо:
\begin{equation}
	\label{eq:1.4.17}
	\mu(\omega(x,y)) = \exp\left\{\int \frac{\frac{\partial M}{\partial y} - \frac{\partial N}{\partial x} }{\frac{\partial \omega}{\partial x} \cdot N - \frac{\partial \omega}{\partial y} \cdot M} \cdot \diff \omega\right\}.
\end{equation}
Розглянемо частинні випадки.
\begin{enumerate}
	\item Нехай $\omega(x, y) = x$. Тоді $\frac{\partial \omega}{\partial x} = 1$, $\frac{\partial \omega}{\partial y} = 0$, $\diff \omega = \diff x$ і формула має вигляд
	\begin{equation}
		\label{eq:1.4.18}
		\mu(\omega(x,y)) = \exp\left\{\int \frac{\frac{\partial M}{\partial y} - \frac{\partial N}{\partial x} }{N} \cdot \diff x\right\}.
	\end{equation}	
	\item Нехай $\omega(x, y) = y$. Тоді $\frac{\partial \omega}{\partial x} = 0$, $\frac{\partial \omega}{\partial y} = 1$, $\diff \omega = \diff y$ і формула має вигляд
	\begin{equation}
		\label{eq:1.4.19}
		\mu(\omega(x,y)) = \exp\left\{\int \frac{\frac{\partial M}{\partial y} - \frac{\partial N}{\partial x} }{-M} \cdot \diff y\right\}.
	\end{equation}
	\item Нехай $\omega(x, y) = x^2 \pm y^2$. Тоді $\frac{\partial \omega}{\partial x} = 2 x$, $\frac{\partial \omega}{\partial y} = \pm 2y$, $\diff \omega = \diff (x^2 \pm y^2)$ і формула має вигляд
	\begin{equation}
		\label{eq:1.4.20}
		\mu(\omega(x,y)) = \exp\left\{\int \frac{\frac{\partial M}{\partial y} - \frac{\partial N}{\partial x} }{2 x N \mp 2 y M} \cdot \diff (x^2 \pm y^2)\right\}.
	\end{equation}
	\item Нехай $\omega(x, y) = x y$. Тоді $\frac{\partial \omega}{\partial x} = y$, $\frac{\partial \omega}{\partial y} = x$, $\diff \omega = \diff (xy)$ і формула має вигляд
	\begin{equation}
		\label{eq:1.4.21}
		\mu(\omega(x,y)) = \exp\left\{\int \frac{\frac{\partial M}{\partial y} - \frac{\partial N}{\partial x} }{yN-xM} \cdot \diff (xy)\right\}.
	\end{equation}
\end{enumerate}

\subsubsection{Вправи для самостійної роботи}
Як вже було сказано, рівняння \[M(x, y) \cdot \diff x + N(x, y) \cdot \diff y = 0\] буде рівнянням в повних диференціалах, якщо його ліва частина є повним диференціалом деякої функції. Це має місце при \[\frac{\partial M(x, y)}{\partial y} = \frac{\partial N(x, y)}{\partial x}.\]

\begin{example}
	Розв’язати рівняння \[(2x + 3x^2y) \cdot \diff x + (x^3 - 3y^2) \cdot \diff y = 0.\]
\end{example}
\begin{solution}
	Перевіримо, що це рівняння є рівнянням в повних диференціалах. Обчислимо
	\[ \frac{\partial}{\partial y} (2x + 3x^2y) = 3x^2, \quad \frac{\partial}{\partial x} (x^3 - 3y^2) = 3x^2. \]
	Таким чином існує функція $u(x,y)$, що \[\frac{\partial u(x,y)}{\partial x} = 2x + 3x^2y.\] Проінтегруємо по $x$. Отримаємо
	\[ u(x,y) = \int(2x+3x^2y)\cdot\diff x+\Phi(y)=x^2+x^3y+\Phi(y).\]
	Для знаходження функції $\Phi(y)$ візьмемо похідну від $u(x,y)$ по $y$ і прирівняємо до $x^3-3y^2$. Отримаємо
	\[ \frac{\partial u(x,y)}{\partial y} = x^3 + \Phi'(y) = x^3 - 3y^2.\]
	Звідси $\Phi'(y) = -3y^2$ і $\Phi(y) = -y^3$. Таким чином, \[u(x,y)=x^2+x^3y-y^3\] і загальний інтеграл диференціального рівняння має вигляд \[x^2+x^3y-y^3=C.\]
\end{solution}

Перевірити, що дані рівняння є рівняннями в повних диференціалах, і розв’язати їх:
\begin{problem}
	\[ 2 x y \cdot \diff x + (x^2 - y^2) \cdot \diff y = 0;\]
\end{problem}
\begin{problem}
	\[(2-9xy^2)\cdot x\cdot \diff x + (4y^2-6x^3)\cdot y\cdot \diff y=0;\]
\end{problem}
\begin{problem}
	\[e^{-y}\cdot\diff x-(2y+x\cdot e^{-y})\cdot\diff y=0;\]
\end{problem}
\begin{problem}
	\[ \frac yx\cdot\diff x+(y^3+\ln x)\cdot\diff y=0;\]
\end{problem}
\begin{problem}
	\[\frac{3x^2+y^2}{y^2}\cdot\diff x-\frac{2x^3+5y}{y^3}\cdot\diff y=0;\]
\end{problem}
\begin{problem}
	\[2x\cdot\left(1+\sqrt{x^2-y}\right)\cdot\diff x-\sqrt{x^2-y}\cdot\diff y=0;\]
\end{problem}
\begin{problem}
	\[(1+y^2\cdot\sin 2x)\cdot\diff x-2y\cdot\cos^2x\cdot\diff y=0;\]
\end{problem}
\begin{problem}
	\[3x^2\cdot(1+\ln y)\cdot\diff x=\left(2y-\frac{x^3}y\right)\cdot\diff y;\]
\end{problem}
\begin{problem}
	\[\left(\frac x{\sin y}+2\right)\cdot\diff x+\frac{(x^2+1)\cdot\cos y}{\cos2y-1}\cdot\diff y=0;\]
\end{problem}
\begin{problem}
	\[(2x+y\cdot e^{xy})\cdot\diff x+(x\cdot e^{xy}+3y^2)\cdot\diff y=0;\]
\end{problem}
\begin{problem}
	\[\left(2+\frac{1}{x^2+y^2}\right)\cdot x\cdot\diff x+\frac{y}{x^2+y^2}\cdot \diff y=0;\]
\end{problem}
\begin{problem}
	\[\left(3y^2-\frac{y}{x^2+y^2}\right)\cdot\diff x+\left(6xy+\frac{x}{x^2+y^2}\right)\cdot \diff y=0.\]
\end{problem}

Розв’язати, використовуючи інтегруючий множник:
\begin{problem} $\mu=\mu(x-y)$,
	\[(2x^3+3x^2y+y^2-y^3)\cdot\diff x+(2y^3+3xy^2+x^2-x^3)\cdot\diff x=0;\]
\end{problem}
\begin{problem}
	\[ \left(y-\frac{ay}{x}+x\right)\cdot\diff x+a\cdot\diff y=0, \quad \mu=\mu(x+y);\]
\end{problem}
\begin{problem}
	\[(x^2+y)\cdot\diff y+x\cdot(1-y)\cdot\diff x=0, \quad \mu=\mu(xy);\]
\end{problem}
\begin{problem}
	\[(x^2-y^2+y)\cdot\diff x+x\cdot(2y-1)\cdot\diff y=0;\]
\end{problem}
\begin{problem}
	\[(2x^2y^2+y)\cdot\diff x+(x^3y-x)\cdot\diff y=0.\]
\end{problem}

		\subsubsection{Загальна теорія}
		Якщо ліва частина диференціального рівняння
\begin{equation*}
	M(x, y) \diff x + N(x, y) \diff y = 0,
\end{equation*}
є повним диференціалом деякої функції $u(x, y)$, тобто
\begin{equation*}
	\diff u(x, y) = M(x, y) \diff x + N(x, y) \diff y,
\end{equation*}
і, таким чином, рівняння набуває вигляду $\diff u (x, y) = 0$ то рівняння називається рівнянням в повних диференціалах. Звідси вираз
\begin{equation*}
	u(x, y) = C
\end{equation*}
є загальним інтегралом диференціального рівняння. \parvskip

Критерієм того, що рівняння є рівнянням в повних диференціалах, тобто необхідною та достатньою умовою, є виконання рівності
\begin{equation*}
	\frac{\partial M(x, y)}{\partial y} = \frac{\partial N(x, y)}{\partial x}.
\end{equation*}
 
Нехай маємо рівняння в повних диференціалах. Тоді
\begin{equation*}
	\frac{\partial u(x, y)}{\partial x} = M(x, y), \quad \frac{\partial u(x, y)}{\partial y} = N(x, y).
\end{equation*}

Звідси 
\begin{equation*}
    u(x, y) = \int M(x, y) \diff x + \phi(y),
\end{equation*}
де $\phi(y)$ --- невідома функція. Для її визначення продиференціюємо співвідношення по $y$ і прирівняємо $N(x, y)$:
\begin{equation*}
	\frac{\partial u(x, y)}{\partial y} = \frac{\partial}{\partial y} \left( \int M(x, y) \diff x \right) + \frac{\diff \phi(y)}{\diff y} = N(x, y).
\end{equation*}

Звідси
\begin{equation*}
	\phi(y) = \int \left( N(x, y) - \frac{\partial}{\partial y} \left( \int M(x, y) \diff x \right) \right) \diff y.
\end{equation*}

Остаточно, загальний інтеграл має вигляд
\begin{equation*}
	\int M(x, y) \diff x + \int \left( N(x, y) - \frac{\partial}{\partial y} \left( \int M(x, y) \diff x \right) \right) \diff y = C.
\end{equation*}

Як відомо з математичного аналізу, якщо відомий повний диференціал, то функцію $u(x, y)$ можна визначити, взявши криволінійний інтеграл по довільному контуру, що з'єднує фіксовану точку $(x_0, y_0)$ і точку із змінними координатами $(x, y)$. \parvskip

Більш зручно брати криву, що складається із двох відрізків прямих. В цьому випадку криволінійний інтеграл розпадається на два простих інтеграла
\begin{multline*}
	u(x, y) = \int_{(x_0, y_0)}^{(x,y)} M(x,y) \diff x + N(x, y) \diff y = \\
	= \int_{(x_0, y_0)}^{(x,y_0)} M(x,y) \diff x + \int_{(x,y_0)}^{(x,y)} N(x, y) \diff y = \\
	= \int_{x_0}^{x} M(\xi,y_0) \diff \xi + \int_{y_0}^{y} N(x, \eta) \diff \eta.
\end{multline*}

У цьому випадку одразу одержуємо розв'язок задачі Коші.
\begin{equation*}
	\int_{x_0}^{x} M(\xi,y_0) \diff \xi + \int_{y_0}^{y} N(x, \eta) \diff \eta = 0.
\end{equation*}


		\subsubsection{Множник, що інтегрує}
		В деяких випадках рівняння
\begin{equation*}
	%\label{eq:1.4.1}
	M(x, y) \cdot \diff x + N(x, y) \cdot \diff y = 0,
\end{equation*}
не є рівнянням в повних диференціалах, але існує функція $\mu = \mu(x,y)$ така, що рівняння
\begin{equation*}
	%\label{eq:1.4.11}
	\mu(x,y) \cdot M(x, y) \cdot \diff x + \mu(x,y) \cdot N(x, y) \cdot \diff y = 0,
\end{equation*}
вже буде рівнянням в повних диференціалах. Необхідною та достатньою умовою цього є рівність
\begin{equation*}
	%\label{eq:1.4.12}
	\frac{\partial}{\partial y} (\mu(x,y) \cdot M(x, y)) = \frac{\partial}{\partial x} (\mu(x,y) \cdot N(x, y)),
\end{equation*}
або
\begin{equation*}
	%\label{eq:1.4.13}
	\frac{\partial \mu}{\partial y} \cdot M + \mu \cdot \frac{\partial M}{\partial y} = \frac{\partial \mu}{\partial x} \cdot N + \mu \cdot \frac{\partial N}{\partial x}.
\end{equation*}
Таким чином замість звичайного диференціального рівняння відносно функції $y(x)$ одержимо диференціальне рівняння в частинних похідних відносно функції $\mu(x, y)$. \\

Задача інтегрування його значно спрощується, якщо відомо в якому вигляді шукати функцію $\mu(x,y)$, наприклад $\mu = \mu(\omega(x,y))$ де $\omega(x,y)$ -- відома функція. В цьому випадку одержуємо
\begin{equation*}
	%\label{eq:1.4.14}
	\frac{\partial \mu}{\partial y} = \frac{\diff \mu}{\diff \omega} \cdot \frac{\partial \omega}{\partial y}, \quad \frac{\partial \mu}{\partial x} = \frac{\diff \mu}{\diff \omega} \cdot \frac{\partial \omega}{\partial x}
\end{equation*}
Після підстановки в попереднє рівняння маємо
\begin{equation*}
	%\label{eq:1.4.15}
	\frac{\diff \mu}{\diff \omega} \cdot \frac{\partial \omega}{\partial y} \cdot M + \mu \cdot \frac{\partial M}{\partial y} = \frac{\diff \mu}{\diff \omega} \cdot \frac{\partial \omega}{\partial x} \cdot N + \mu \cdot \frac{\partial N}{\partial x}.
\end{equation*}
або
\begin{equation*}
	%\label{eq:1.4.16}
	\frac{\diff \mu}{\diff \omega} \left( \frac{\partial \omega}{\partial x} \cdot N - \frac{\partial \omega}{\partial y} \cdot M \right) = \mu \left( \frac{\partial M}{\partial y} - \frac{\partial N}{\partial x} \right).
\end{equation*}
Розділимо змінні
\begin{equation*}
	%\label{eq:1.4.17}
	\frac{\diff \mu}{\mu} = \frac{\frac{\partial M}{\partial y} - \frac{\partial N}{\partial x} }{\frac{\partial \omega}{\partial x} \cdot N - \frac{\partial \omega}{\partial y} \cdot M} \cdot \diff \omega.
\end{equation*}
Проінтегрувавши і поклавши сталу інтегрування одиницею, одержимо:
\begin{equation*}
	%\label{eq:1.4.18}
	\mu(\omega(x,y)) = \exp\left\{\int \frac{\frac{\partial M}{\partial y} - \frac{\partial N}{\partial x} }{\frac{\partial \omega}{\partial x} \cdot N - \frac{\partial \omega}{\partial y} \cdot M} \cdot \diff \omega\right\}.
\end{equation*}
Розглянемо частинні випадки.
\begin{enumerate}
	\item Нехай $\omega(x, y) = x$. Тоді $\frac{\partial \omega}{\partial x} = 1$, $\frac{\partial \omega}{\partial y} = 0$, $\diff \omega = \diff x$ і формула має вигляд
	\begin{equation*}
		%\label{eq:1.4.19}
		\mu(\omega(x,y)) = \exp\left\{\int \frac{\frac{\partial M}{\partial y} - \frac{\partial N}{\partial x} }{N} \cdot \diff x\right\}.
	\end{equation*}	
	\item Нехай $\omega(x, y) = y$. Тоді $\frac{\partial \omega}{\partial x} = 0$, $\frac{\partial \omega}{\partial y} = 1$, $\diff \omega = \diff y$ і формула має вигляд
	\begin{equation*}
		%\label{eq:1.4.20}
		\mu(\omega(x,y)) = \exp\left\{\int \frac{\frac{\partial M}{\partial y} - \frac{\partial N}{\partial x} }{-M} \cdot \diff y\right\}.
	\end{equation*}
	\item Нехай $\omega(x, y) = x^2 \pm y^2$. Тоді $\frac{\partial \omega}{\partial x} = 2 x$, $\frac{\partial \omega}{\partial y} = \pm 2y$, $\diff \omega = \diff (x^2 \pm y^2)$ і формула має вигляд
	\begin{equation*}
		%\label{eq:1.4.21}
		\mu(\omega(x,y)) = \exp\left\{\int \frac{\frac{\partial M}{\partial y} - \frac{\partial N}{\partial x} }{2 x N \mp 2 y M} \cdot \diff (x^2 \pm y^2)\right\}.
	\end{equation*}
	\item Нехай $\omega(x, y) = x y$. Тоді $\frac{\partial \omega}{\partial x} = y$, $\frac{\partial \omega}{\partial y} = x$, $\diff \omega = \diff (xy)$ і формула має вигляд
	\begin{equation*}
		%\label{eq:1.4.22}
		\mu(\omega(x,y)) = \exp\left\{\int \frac{\frac{\partial M}{\partial y} - \frac{\partial N}{\partial x} }{yN-xM} \cdot \diff (xy)\right\}.
	\end{equation*}
\end{enumerate}


		\subsubsection{Вправи для самостійної роботи}
		Як вже було сказано, рівняння \[M(x, y) \diff x + N(x, y) \diff y = 0\]

 буде рівнянням в повних диференціалах, якщо його ліва частина є повним диференціалом деякої функції. Це має місце при \[\frac{\partial M(x, y)}{\partial y} = \frac{\partial N(x, y)}{\partial x}.\]

\begin{example}
	Розв'язати рівняння \[(2x + 3x^2y) \diff x + (x^3 - 3y^2) \diff y = 0.\]
\end{example}

\begin{solution}
	Перевіримо, що це рівняння є рівнянням в повних диференціалах. Обчислимо \[ \frac{\partial}{\partial y} (2x + 3x^2y) = 3x^2, \quad \frac{\partial}{\partial x} (x^3 - 3y^2) = 3x^2. \]
	
	Таким чином існує функція $u(x,y)$, що \[\frac{\partial u(x,y)}{\partial x} = 2x + 3x^2y.\] 
	
	Проінтегруємо по $x$. Отримаємо \[ u(x,y) = \int(2x+3x^2y)\diff x+\Phi(y)=x^2+x^3y+\Phi(y).\]
	
	Для знаходження функції $\Phi(y)$ візьмемо похідну від $u(x,y)$ по $y$ і прирівняємо до $x^3-3y^2$. Отримаємо \[ \frac{\partial u(x,y)}{\partial y} = x^3 + \Phi'(y) = x^3 - 3y^2.\]

	Звідси $\Phi'(y) = -3y^2$ і $\Phi(y) = -y^3$. Таким чином, \[u(x,y)=x^2+x^3y-y^3\] і загальний інтеграл диференціального рівняння має вигляд \[x^2+x^3y-y^3=C.\]
\end{solution}

Перевірити, що дані рівняння є рівняннями в повних диференціалах, і роз\-в'яз\-а\-ти їх:
\begin{problem}
	\[ 2 x y \diff x + (x^2 - y^2) \diff y = 0;\]
\end{problem}

\begin{problem}
	\[(2-9xy^2) x \diff x + (4y^2-6x^3) y \diff y=0;\]
\end{problem}

\begin{problem}
	\[e^{-y}\diff x-(2y+x e^{-y})\diff y=0;\]
\end{problem}

\begin{problem}
	\[ \frac yx\diff x+(y^3+\ln x)\diff y=0;\]
\end{problem}

\begin{problem}
	\[\frac{3x^2+y^2}{y^2}\diff x-\frac{2x^3+5y}{y^3}\diff y=0;\]
\end{problem}

\begin{problem}
	\[2x\left(1+\sqrt{x^2-y}\right)\diff x-\sqrt{x^2-y}\diff y=0;\]
\end{problem}

\begin{problem}
	\[(1+y^2\sin 2x)\diff x-2y\cos^2x\diff y=0;\]
\end{problem}

\begin{problem}
	\[3x^2(1+\ln y)\diff x=\left(2y-\frac{x^3}y\right)\diff y;\]
\end{problem}

\begin{problem}
	\[\left(\frac x{\sin y}+2\right)\diff x+\frac{(x^2+1)\cos y}{\cos2y-1}\diff y=0;\]
\end{problem}

\begin{problem}
	\[(2x+y e^{xy})\diff x+(x e^{xy}+3y^2)\diff y=0;\]
\end{problem}

\begin{problem}
	\[\left(2+\frac{1}{x^2+y^2}\right) x\diff x+\frac{y}{x^2+y^2} \diff y=0;\]
\end{problem}

\begin{problem}
	\[\left(3y^2-\frac{y}{x^2+y^2}\right)\diff x+\left(6xy+\frac{x}{x^2+y^2}\right) \diff y=0.\]
\end{problem}

Розв'язати, використовуючи множник, що інтегрує:
\begin{problem} $\mu=\mu(x-y)$,
	\[(2x^3+3x^2y+y^2-y^3)\diff x+(2y^3+3xy^2+x^2-x^3)\diff x=0;\]
\end{problem}

\begin{problem}
	\[ \left(y-\frac{ay}{x}+x\right)\diff x+a\diff y=0, \quad \mu=\mu(x+y);\]
\end{problem}

\begin{problem}
	\[(x^2+y)\diff y+x(1-y)\diff x=0, \quad \mu=\mu(xy);\]
\end{problem}

\begin{problem}
	\[(x^2-y^2+y)\diff x+x(2y-1)\diff y=0;\]
\end{problem}

\begin{problem}
	\[(2x^2y^2+y)\diff x+(x^3y-x)\diff y=0.\]
\end{problem}

	\subsection{Диференціальні рівняння першого порядку, не розв’язані відносно похідної}
	Диференціальне рівняння першого порядку, не розв’язане відносно похідної, має такий вигляд
\begin{equation}
	\label{eq:1.5.1}
	F(x, y, y') = 0. 	
\end{equation}

\subsubsection{Частинні випадки рівнянь, що інтегруються в квадратурах}

Розглянемо ряд диференціальних рівнянь, що інтегруються в квадратурах.
\begin{enumerate}
	\item Рівняння вигляду 
	\begin{equation}
		\label{eq:1.5.2}
		F(y') = 0.
	\end{equation}
	Нехай алгебраїчне рівняння $F(k) = 0$ має принаймні один дійсний корінь $k = k_0$. Тоді, інтегруючи $y' = k_0$, одержимо $y = k_0 \cdot x + C$. Звідси $k_0 = (y - C) / x$ і вираз
	\begin{equation}
		\label{eq:1.5.3}
		F \left( \frac{y - c}{x} \right) = 0	
	\end{equation}
	містить всі розв’язки вихідного диференціального рівняння.
	\item Рівняння вигляду 
	\begin{equation}
		\label{eq:1.5.4}
		F(x, y') = 0.
	\end{equation}
	Нехай це рівняння можна записати у параметричному вигляді
	\begin{equation}
		\label{eq:1.5.5}
		\left\{\begin{aligned}
			x &= \phi(t), \\
			y' &= \psi(t).
		\end{aligned}\right.
	\end{equation}
	Використовуючи співвідношення $\diff y = y ' \cdot \diff x$, одержимо 
	\begin{equation}
		\label{eq:1.5.6}
		\diff y = \psi(t) \cdot \phi'(t) \cdot \diff t.
	\end{equation}
	Проінтегрувавши, запишемо
	\begin{equation}
		\label{eq:1.5.7}
		y = \int \psi(t) \cdot \phi'(t) \cdot \diff t + C.
	\end{equation}
	І загальний розв’язок в параметричній формі має вигляд
	\begin{equation}
		\label{eq:1.5.8}
		\left\{\begin{aligned}
		x &= \phi(t), \\
		y &= \int \psi(t) \cdot \phi'(t) \cdot \diff t + C.
		\end{aligned}\right.
	\end{equation}
	\item Рівняння вигляду 
	\begin{equation}
		\label{eq:1.5.9}
		F(y, y') = 0.
	\end{equation}
	Нехай це рівняння можна записати у параметричному вигляді
	\begin{equation}
		\label{eq:1.5.10}
		\left\{\begin{aligned}
			y &= \phi(t), \\
			y' &= \psi(t).
		\end{aligned}\right.
	\end{equation}
	Використовуючи співвідношення $\diff y = y ' \cdot \diff x$, одержимо 
	\begin{equation}
		\label{eq:1.5.11}
		\phi'(t) \cdot \diff t = \psi(t) \cdot \diff x
	\end{equation}
	і
	\begin{equation}
		\label{eq:1.5.12}
		\diff x = \frac{\phi'(t)}{\psi(t)} \cdot \diff t
	\end{equation}
	Проінтегрувавши, запишемо
	\begin{equation}
		\label{eq:1.5.13}
		x = \int \frac{\phi'(t)}{\psi(t)}\cdot \diff t + C.
	\end{equation}
	І загальний розв’язок в параметричній формі має вигляд
	\begin{equation}
		\label{eq:1.5.14}
		\left\{\begin{aligned}
		x &= \int \frac{\phi'(t)}{\psi(t)}\cdot \diff t + C, \\
		y &= \phi(t).
		\end{aligned}\right.
	\end{equation}
	\item Рівняння Лагранжа
	\begin{equation}
		\label{eq:1.5.15}
		y = \phi(y') \cdot x + \psi(y').
	\end{equation}
	Введемо параметр $y' = \frac{\diff y}{\diff x} = p$ і отримаємо
	\begin{equation}
		\label{eq:1.5.16}
		y = \phi(p) \cdot x + \psi(p).
	\end{equation}
	Продиференціювавши, запишемо
	\begin{equation}
		\label{eq:1.5.17}
		\diff y = \phi'(p) \cdot x \cdot \diff p + \phi(p) \cdot \diff x + \psi'(p) \cdot \diff p.
	\end{equation}
	Замінивши $\diff y = p \cdot \diff x$ одержимо
	\begin{equation}
		\label{eq:1.5.18}
		p \cdot \diff x = \phi'(p) \cdot x \cdot \diff p + \phi(p) \cdot \diff x + \psi'(p) \cdot \diff p.
	\end{equation}
	Звідси
	\begin{equation}
		\label{eq:1.5.19}
		(p - \phi(p)) \cdot \diff x - \phi'(p) \cdot x \cdot \diff p = \psi'(p) \cdot \diff p.
	\end{equation}
	І отримали лінійне неоднорідне диференціальне рівняння
	\begin{equation}
		\label{eq:1.5.20}
		\frac{\diff x}{\diff p} + \frac{\phi'(p)}{\phi(p)-p} \cdot x = \frac{\phi'(p)}{p-\phi(p)}.
	\end{equation}
	Його розв’язок
	\begin{multline}
		\label{eq:1.5.21}
		x = \exp\left\{\int \frac{\phi'(p)}{p-\phi(p)} \cdot \diff p\right\} \cdot \\
		\cdot \left(\int \frac{\phi'(p)}{p-\phi(p)} \cdot \exp\left\{\int \frac{\phi'(p)}{\phi(p)-p} \cdot \diff p\right\} \diff p + C \right) = \\
		= \Psi(p, C).
	\end{multline}
	І остаточний розв’язок рівняння Лагранжа в параметричній формі запишеться у вигляді
	\begin{equation}
		\label{eq:1.5.22}
		\left\{\begin{aligned}
			x &= \Psi(p,C), \\
			y &= \phi(p) \cdot \Phi(p, C) + \psi(p).
		\end{aligned}\right.
	\end{equation}
\end{enumerate}


		\subsubsection{Частинні випадки рівнянь, що інтегруються в квадратурах}
		% version 1.0
Розглянемо ряд диференціальних рівнянь, що інтегруються в квадратурах.
\begin{enumerate}
	\item Рівняння вигляду 
	\begin{equation*}
		F(y') = 0.
	\end{equation*}

	Нехай алгебраїчне рівняння $F(k) = 0$ має принаймні один дійсний корінь $k = k_0$. Тоді, інтегруючи $y' = k_0$, одержимо $y = k_0 \cdot x + C$. Звідси $k_0 = (y - C) / x$ і вираз
	\begin{equation*}
		F \left( \frac{y - c}{x} \right) = 0	
	\end{equation*}
	містить всі розв'язки вихідного диференціального рівняння.

	\item Рівняння вигляду 
	\begin{equation*}
		F(x, y') = 0.
	\end{equation*}

	Нехай це рівняння можна записати у параметричному вигляді
	\begin{equation*}
		\left\{\begin{aligned}
			x &= \phi(t), \\
			y' &= \psi(t).
		\end{aligned}\right.
	\end{equation*}

	Використовуючи співвідношення $\diff y = y ' \cdot \diff x$, одержимо 
	\begin{equation*}
		\diff y = \psi(t) \cdot \phi'(t) \cdot \diff t.
	\end{equation*}

	Проінтегрувавши, запишемо
	\begin{equation*}
		y = \int \psi(t) \cdot \phi'(t) \cdot \diff t + C.
	\end{equation*}

	І загальний розв'язок в параметричній формі має вигляд
	\begin{equation*}
		\left\{
			\begin{aligned}
				x &= \phi(t), \\
				y &= \int \psi(t) \cdot \phi'(t) \cdot \diff t + C.
			\end{aligned}
		\right.
	\end{equation*}
	
	\item Рівняння вигляду 
	\begin{equation*}
		F(y, y') = 0.
	\end{equation*}

	Нехай це рівняння можна записати у параметричному вигляді
	\begin{equation*}
		\left\{
			\begin{aligned}
				y &= \phi(t), \\
				y' &= \psi(t).
			\end{aligned}
		\right.
	\end{equation*}
	
	Використовуючи співвідношення $\diff y = y ' \cdot \diff x$, одержимо 
	\begin{equation*}
		\phi'(t) \cdot \diff t = \psi(t) \cdot \diff x
	\end{equation*}
	і
	\begin{equation*}
		\diff x = \frac{\phi'(t)}{\psi(t)} \cdot \diff t
	\end{equation*}
	
	Проінтегрувавши, запишемо
	\begin{equation*}
		x = \int \frac{\phi'(t)}{\psi(t)}\cdot \diff t + C.
	\end{equation*}
	
	І загальний розв'язок в параметричній формі має вигляд
	\begin{equation*}
		\left\{
			\begin{aligned}
				x &= \int \frac{\phi'(t)}{\psi(t)}\cdot \diff t + C, \\
				y &= \phi(t).
			\end{aligned}
		\right.
	\end{equation*}
	
	\item Рівняння Лагранжа
	\begin{equation*}
		y = \phi(y') \cdot x + \psi(y').
	\end{equation*}
	
	Введемо параметр $y' = \frac{\diff y}{\diff x} = p$ і отримаємо
	\begin{equation*}
		y = \phi(p) \cdot x + \psi(p).
	\end{equation*}
	
	Продиференціювавши, запишемо
	\begin{equation*}
		\diff y = \phi'(p) \cdot x \cdot \diff p + \phi(p) \cdot \diff x + \psi'(p) \cdot \diff p.
	\end{equation*}
	
	Замінивши $\diff y = p \cdot \diff x$ одержимо
	\begin{equation*}
		p \cdot \diff x = \phi'(p) \cdot x \cdot \diff p + \phi(p) \cdot \diff x + \psi'(p) \cdot \diff p.
	\end{equation*}
	
	Звідси
	\begin{equation*}
		(p - \phi(p)) \cdot \diff x - \phi'(p) \cdot x \cdot \diff p = \psi'(p) \cdot \diff p.
	\end{equation*}
	
	І отримали лінійне неоднорідне диференціальне рівняння
	\begin{equation*}
		\frac{\diff x}{\diff p} + \frac{\phi'(p)}{\phi(p)-p} \cdot x = \frac{\phi'(p)}{p-\phi(p)}.
	\end{equation*}
	
	Його роз\-в'яз\-ок
	\begin{multline*}
		x = \exp\left\{\int \frac{\phi'(p)}{p-\phi(p)} \cdot \diff p\right\} \cdot \\
		\cdot \left(\int \frac{\phi'(p)}{p-\phi(p)} \cdot \exp\left\{\int \frac{\phi'(p)}{\phi(p)-p} \cdot \diff p\right\} \diff p + C \right) = \Psi(p, C).
	\end{multline*}
	
	І остаточний розв'язок рівняння Лагранжа в параметричній формі запишеться у вигляді
	\begin{equation*}
		\left\{
			\begin{aligned}
				x &= \Psi(p,C), \\
				y &= \phi(p) \cdot \Phi(p, C) + \psi(p).
			\end{aligned}
		\right.
	\end{equation*}
	
	\item Рівняння Клеро. \\

	Частинним випадком рівняння Лагранжа, що відповідає $\phi(y') = y'$ є рівняння Клеро
 	\begin{equation*}
 		y = y' x + \psi(y').
 	\end{equation*}
	
	Поклавши $y' = \frac{\diff y}{\diff x} = p$, отримаємо $y = p x + \psi(p)$. Продиференціюємо 
	\begin{equation*}
		\diff y = p \cdot \diff x + x \cdot \diff p + \psi'(p) \cdot \diff p.
	\end{equation*}
	
	Оскільки $\diff y = p \cdot \diff x$, то
	\begin{equation*}
		p \cdot \diff x = p \cdot \diff x + x \cdot \diff p + \psi'(p) \cdot \diff p.
	\end{equation*}
	
	Скоротивши, одержимо
	\begin{equation*}
		(x + \psi'(p)) \cdot \diff p = 0.
	\end{equation*}
	
	Можливі два випадки.
	\begin{enumerate}
		\item $x + \psi'(p) - 0$ і розв'язок має вигляд
		\begin{equation*}
			\left\{
				\begin{aligned}
					x &= - \psi'(p), \\
					y &= -p \cdot \psi'(p) + \psi(p).
				\end{aligned}
			\right.
		\end{equation*}
		
		\item $\diff p = 0$, $p = C$ і розв'язок має вигляд
		\begin{equation*}
			y = C x + \psi(C).
		\end{equation*}
	\end{enumerate}
	
	Загальним розв'язком рівняння Клеро буде сім'я ``прямих''. Її огинає особлива крива.
	
	\item Параметризація загального вигляду. Нехай диференціальне рівняння $F(x, y, y') = 0$ вдалося записати у вигляді системи рівнянь з двома параметрами
	\begin{equation*}
		x = \phi(u, v), \quad y = \psi(u, v), \quad y' = \theta(u, v).	
	\end{equation*}
	
	Використовуючи співвідношення $\diff y = y' \cdot \diff x$, одержимо
	\begin{multline*}
		\frac{\partial \psi(u,v)}{\partial u} \cdot \diff u + \frac{\partial \psi(u, v)}{\partial v} \cdot \diff v = \\
		= \theta(u,v) \cdot \left( \frac{\partial \phi(u,v)}{\partial u} \cdot \diff u + \frac{\partial \phi(u, v)}{\partial v} \cdot \diff v\right)
	\end{multline*}
	
	Перегрупувавши члени, одержимо
	\begin{multline*}
		\left( \frac{\partial \psi(u,v)}{\partial u} - \theta(u, v) \cdot \frac{\partial \phi(u,v)}{\partial u} \right) \diff u = \\
		= \left( \theta(u,v) \cdot \frac{\partial \phi(u, v)}{\partial v} - \frac{\partial \psi(u, v)}{\partial v} \right) \diff v.
	\end{multline*}
	
	Звідси
	\begin{equation*}
		\frac{\diff u}{\diff v} = \frac{\theta(u,v) \cdot \frac{\partial \phi(u, v)}{\partial v} - \frac{\partial \psi(u, v)}{\partial v}}{\frac{\partial \psi(u,v)}{\partial u} - \theta(u, v) \cdot \frac{\partial \phi(u,v)}{\partial u}}.
	\end{equation*}

	Або отримали рівняння вигляду
	\begin{equation*}
		\frac{\diff u}{\diff v} = f(u, v).
	\end{equation*}

	Параметризація загального вигляду не дає інтеграл диференціального рівняння. Вона дозволяє звести диференціальне рівняння, не роз\-в'яз\-а\-не відносно похідної, до диференціального рівняння, роз\-в'яз\-а\-но\-го відносно похідної.

	\item Нехай рівняння $F(x, y, y') = 0$ можна розв'язати відносно $y'$ і воно має $n$ коренів, тобто його  можна записати у вигляді  
	\begin{equation*}
		\prod_{i=1}^n (y' - f_i(x, y)) = 0.
	\end{equation*}
	
	Розв'язавши кожне з рівнянь $y' = f_i(x, y)$, $i=\overline{1,n}$, отримаємо $n$ загальних розв'язків (або інтервалів) $y = \phi_i(x, C)$, $i=\overline{1,n}$ (або $\phi_u(x,y)=C$, $i=\overline{1,n}$). І загальний розв'язок вихідного рівняння, не розв'язаного відносно похідної має вигляд
	\begin{equation*}
		\prod_{i=1}^n (y - \phi_i(x, C)) = 0,
	\end{equation*}
	або
	\begin{equation*}
		\prod_{i=1}^n (\phi_i(x, y) - C) = 0.
	\end{equation*}
\end{enumerate}


		\subsubsection{Вправи для самостійної роботи}
		\begin{enumerate}
	\item Розв'язати рівняння вигляду $F(y') = 0$:
	\begin{example}
		$(y')^3 - 1 = 0$;
	\end{example}
	
	\begin{solution}
		Рівняння має дійсний розв'язок, тобто воно поставлене коректно. Тому його розв'язком буде \[\left(\frac{y - C}{x}\right)^3 - 1 = 0.\]
	\end{solution}
	
	\begin{multicols}{2}
		\begin{problem}
			\[(y')^2 - 2 y' + 1 = 0;\]
		\end{problem}
		
		\begin{problem}
			\[ (y')^4 - 16 = 0. \]
		\end{problem}
	\end{multicols}

	\item Розв'язати рівняння вигляду $F(x, y^\prime) = 0$:
	\begin{example}
		$x = \left( y^\prime \right)^3 + y^\prime$;
	\end{example}
	
	\begin{solution}
		Робимо параметризацію $y' = t$, $x = t^3 + t$. Використовуючи основну форму запису $\diff y = y' \diff x$ одержимо \[ \diff y = t (3t^2 + 1) \diff t.\]
	
		Звідси \[ y = \int t (3t^2 + 1) \diff t = \frac{3t^4}{4} + \frac{t^2}{2} + C.\]
	
		Остаточний розв'язок у параметричній формі має вигляд\[ x = t^3 + t, \quad y = \frac{3t^4}{4} + \frac{t^2}{2} + C.\]
	\end{solution}

	\begin{multicols}{2}
		\begin{problem}
			\[ x ((y')^2 - 1) = 2y'; \]
		\end{problem}
		
		\begin{problem}
			\[ x = y' \sqrt{(y')^2 - 1}; \]
		\end{problem}
		
		\begin{problem}
		 	\[ y' (x - \ln y') - 1. \]
		\end{problem}
	\end{multicols}

	\item Розв'язати рівняння вигляду $F(y, y') = 0$:
	\begin{example}
		$y = (y')^2 + 2 (y')^3$;
	\end{example}

	\begin{solution}
		Робимо параметризацію $y' = t$, $y = t^2 + 2t^3$. Використовуючи основну форму запису $\diff y = y' \diff x$, одержуємо
		\[ (2t + 6t^2) \diff t = t \diff x.\]

		Звідси \[ \diff x = (2 + 6t) \diff t, \quad x = \int(2+6t) \diff t = 2t + 3t^2 + C.\]

		Остаточний розв'язок у параметричній формі має вигляд 
		\[ x = 2t + 3t^2, \quad y = t^2 + 2t^.\]

		Крім того за рахунок скорочення втрачено $y \equiv 0$.
	\end{solution}
	
	\begin{multicols}{2}
		\begin{problem}
			\[ y = \ln ( 1 + (y')^2); \]
		\end{problem}
		
		\begin{problem}
			\[ y = (y' - 1) e^{y'}; \]
		\end{problem}
		
		\begin{problem}
		 	\[ (y')^4 - (y')^2 = y^2. \]
		\end{problem}
	\end{multicols}

	\item Розв'язати рівняння Лагранжа
	\begin{example}
		$y = - x y' + 4 \sqrt{y'}$;
	\end{example}

	\begin{solution}
		Робимо параметризацію \[ y' = t, \quad y = - xt + 4 \sqrt{t}.\] Диференціюємо друге рівняння.
		\[ \diff y = - x \diff t - t \diff x + \frac{2}{\sqrt{t}} \diff t.\]

		Оскільки зроблено заміну $\diff y = t \diff x$, то одержимо \[ t \diff x = - x \diff t - t \diff x + \frac{2\diff t}{\sqrt{t}},\] або \[ 2 t \diff x = - x \diff t + \frac{2 \diff t}{\sqrt{t}}.\]

		Звідси \[ \frac{\diff x}{\diff t} + \frac{x}{2t} = \frac{1}{t \sqrt{t}}.\]

		Розв'язок лінійного неоднорідного рівняння може бути представлений у вигляді
		\begin{multline*}
			x = \exp\left\{ - \int \frac{\diff t}{2t} \right\} \left( \int \exp\left\{\int \frac{\diff t}{2t} \right\} \frac{1}{t\sqrt{t}} + C\right) = \\
			= \frac{1}{\sqrt{t}} \left(\int \frac {\diff t}{t} + C\right) = \frac{\ln |t| + C}{\sqrt{t}}.
		\end{multline*}

		Остаточно маємо \[ x = \frac{\ln |t| + C}{\sqrt{t}}, \quad y = - \sqrt{t} (\ln |t| + C) + 4 \sqrt{t}. \] 

		Крім того при діленні на $t$ втратили $y \equiv 0$.
	\end{solution}

	\begin{multicols}{2}
		\begin{problem}
			\[ y = 2 x y' - 4 (y')^3;\]
		\end{problem}
	
		\begin{problem}
			\[ y = x (y')^2 - 2 (y')^3;\]
		\end{problem}
	
		\begin{problem}
		 	\[ x y' (y' + 2) = y;\]
		\end{problem}
	
		\begin{problem}
		 	\[ 2 x y' - y = \ln y'.\]
		\end{problem}
	\end{multicols}

	\item Розв'язати рівняння Клеро
	\begin{example}
		$y = x y' - (y')^2$;
	\end{example}
	
	\begin{solution}
		Робимо параметризацію $y' = t$, $y = x t - t^2$. Диференціюємо друге рівняння:
		\[ \diff y = x \diff t + t \diff x - 2 t \diff t.\]

		Оскільки зроблено заміну $\diff y = t \diff x$, то одержимо
		\[ t \diff x = x \diff t + t \diff x - 2 t \diff t.\]

		Звідси $(x - 2t) \diff t = 0$. І маємо дві гілки
		\begin{enumerate}
			\item Особливий розв'язок $x = 2t$, $y = t^2$, або $y = x^2 / 4$.
			\item Загальний розвозок $y = C x - C^2$.
		\end{enumerate}
	\end{solution}

	\begin{multicols}{2}
		\begin{problem}
			\[ y=xy'+4\sqrt{y'}; \]
		\end{problem}
		
		\begin{problem}
			\[ y=xy'+2-y'; \]
		\end{problem}
		
		\begin{problem}
			\[ y=xy'-\ln y'; \]
		\end{problem}
		
		\begin{problem}
			\[ y=xy'+\sin y'; \]
		\end{problem}
		
		\begin{problem}
			\[ y=xy'+\sqrt{1+(y')^2}; \]
		\end{problem}
		
		\begin{problem}
			\[ y=xy'+(y')^3; \]
		\end{problem}
		
		\begin{problem}
			\[ y=xy'+\cos(2+y'); \]
		\end{problem}
		
		\begin{problem}
			\[ y=xy'-\ln\sqrt{1+(y')^2}; \]
		\end{problem}
		
		\begin{problem}
			\[ y=xy'-y'-(y')^3; \]
		\end{problem}
		
		\begin{problem}
			\[ y=xy'-\sqrt{2-(y')^2}; \]
		\end{problem}
		
		\begin{problem}
			\[ y=xy'+\sqrt{2y'+2}; \]
		\end{problem}
		
		\begin{problem}
			\[ y=xy'-e^{y'}; \]
		\end{problem}
		
		\begin{problem}
			\[ y=xy'-\tan y'; \]
		\end{problem}
		
		\begin{problem}
			\[ (y')^3=3(xy'-y). \]
		\end{problem}
	\end{multicols}

	\item Розв'язати рівняння параметризацією загального виду
	\begin{example}
		$(y')^2 - 2 x y' = x^2 - 4y$;
	\end{example}

	\begin{solution}
		Введемо параметризацію рівняння \[x = u, \quad y' = v, \quad y = \frac {u^2 + 2 u v - v^2}{4}.\] 

		Використовуючи співвідношення $\diff y = y' \diff x$, одержимо рівняння
		\[ \frac1u (2u \diff u + 2 u \diff v + 2 v \diff u - 2 v \diff v) = v \diff u.\]

		Перепишемо його у вигляді \[ (u+v)\diff u+(u-v)\diff v=2v\diff u,\] або \[ (u-v)\diff u+(u-v)\diff v=0,\]
	
		Воно розділяється на два 
		\begin{enumerate}
			\item $\diff u + \diff v = 0 \implies v = - u + C$. \parvskip

			Підставивши в параметризовану систему, одержуємо \[x = u, \quad y = \frac{u^2+2u(-u+C)-(-u+C)^2}{4},\]

			 або \[y = \frac{x^2+2x(-x+C)-(-x+C)^2}{4} = \frac{-2x^2+4Cx-C^2}{4}.\]

			\item $u - v = 0 \implies v = u$. І розв'язок має вигляд $y = x^2/ 2$.
		\end{enumerate}
	\end{solution}

	\begin{multicols}{2}
		\begin{problem}
			\[5y+(y')^2=x(x+y');\]
		\end{problem}

		\begin{problem}
			\[x^2(y')^2=xyy'+1;\]
		\end{problem}

		\begin{problem}
			\[(y')^3+y^2=xyy';\]
		\end{problem}

		\begin{problem}
			\[y=x(y')^2-2(y')^3;\]
		\end{problem}

		\begin{problem}
			\[2xy'-y=y'\ln(yy');\]
		\end{problem}

		\begin{problem}
			\[y'=e^{xy'/y}.\]
		\end{problem}
	\end{multicols}

	\item Розв'язати рівняння
	\begin{example}
		$(y')^2 - y^2 = 0$;
	\end{example}

	\begin{solution}
		Це рівняння розв'язується відносно $y'$. Маємо \[y' = y, \quad y' = - y.\]

		Розв'язок першого має вигляд $y = ce^x$, другого $Ce^{-x}$. Загальний роз\-в'яз\-ок має вигляд \[ (y - ce^x)(y-Ce^{-x})=0.\]
	\end{solution}
	
	\begin{multicols}{2}
		\begin{problem}
			\[y^2((y')^2+1)=1;\]
		\end{problem}
		
		\begin{problem}
			\[(y')^2-4y^3=0;\]
		\end{problem}
		
		\begin{problem}
			\[x(y')^2=y;\]
		\end{problem}
		
		\begin{problem}
			\[(y')^2+xy=y^2+xy';\]
		\end{problem}
		
		\begin{problem}
			\[xy'(xy'+y)=2y^2.\]
		\end{problem}
	\end{multicols}
\end{enumerate}

	\subsection{Існування та єдиність розв’язків диференціальних рівнянь першого порядку. Неперервна залежність та диференційованість}
	Клас диференціальних рівнянь, що інтегруються в квадратурах, досить невеликий, тому мають велике значення наближені методи розв’язку диференціальних рівнянь. Але, щоб використовувати ці методи, треба бути впевненим в існуванні розв’язку шуканого рівняння та в його єдиності. \\

Зараз значна частина теорем існування  та єдиності розв’язків не тільки диференціальних, але й рівнянь інших видів доводиться методом стискаючих відображень. \\

\begin{definition} 
	Простір $M$ називається метричним, якщо для довільних двох точок $x,y\in M$ визначена функція $\rho(x,y)$, що задовольняє аксіомам:
	\begin{enumerate}
		\item $\rho(x, y)\ge0$, причому $\rho(x,y)=0$ тоді і тільки тоді, коли $x=y$;
		\item $\rho(x,y)=\rho(y,x)$ (комутативність);
		\item $\rho(x,y)+\rho(y,z)\ge\rho(x,z)$ (нерівність трикутника).
	\end{enumerate}
	Функція $\rho(x,y)$ називається відстанню (метрикою) в просторі $M$.
\end{definition}
\begin{example*} 
	Векторний $n$-вимірний простір $\RR^n$. \\

	Нехай $x=(x_1,x_2,\ldots,x_n)$, $y=(y_1,y_2,\ldots,y_n)$. За метрику можна взяти: 
	\begin{equation*}
		%\label{eq:1.6.1}
		\rho(x,y)=\left(\sum_{i=1}^n (x_i-y_i)^2\right)^{1/2},
	\end{equation*}
	або 
	\begin{equation*}
		%\label{eq:1.6.2}
		\rho(x,y)=\max_{i=\overline{1,n}}|x_i-y_i|.
	\end{equation*}
\end{example*}
\begin{example*} 
	Простір неперервних функцій на відрізку $[a,b]$ позначається $C([a,b])$. За метрику можна взяти
		\begin{equation*}
		%\label{eq:1.6.3}
		\rho(x(t), y(t)) = \left(\int_a^b (x(t)-y(t))^2 \diff t\right)^{1/2},
	\end{equation*}
	або
	\begin{equation*}
		%\label{eq:1.6.4}
		\rho(x(t), y(t)) = \max_{t\in[a,b]} |x(t)-y(t)|.
	\end{equation*}
\end{example*}
\begin{definition} 
	Послідовність $\{x_n\}_{n=1}^\infty$ називається фундаментальною, як\-що для довільного $\epsilon > 0$ існує $n \ge N(\epsilon)$ таке, що при $n \ge N(\epsilon)$ і довільному $m\in\NN$ буде $\rho(x_n,x_{n+m}) < \epsilon$.
\end{definition}
\begin{definition} 
	Метричний простір $M$ називається повним, якщо довільна фундаментальна послідовність точок $\{x_n\}$ простору $M$ збігається до деякої точки $x_0$ простору $M$.
\end{definition}
\begin{theorem}[принцип стискаючих відображень] 
	Нехай в повному метричному просторі $M$ задано оператор $A$, що задовольняє умовам.
	\begin{enumerate}
		\item Оператор $A$ переводить точки простору $M$ в точки цього ж простору, тобто якщо $x\in M$, то і $Ax \in M$.
		\item Оператор $A$ є оператором стиску, тобто $\rho(Ax,Ay)\le\alpha\rho(x,y)$, де $0<\alpha<1$, $x,y$ -- довільні точки $M$. 
	\end{enumerate}
	Тоді існує єдина нерухома точка $\bar x \in M$, яка є розв’язком операторного рівняння $A\bar x=\bar x$ і вона може бути знайдена методом послідовних відображень, тобто $x = \lim_{n\to\infty} x_n$, де $x_{n+1} = A x_n$, причому $x_0$ вибирається довільно.
\end{theorem}
\begin{proof}
	Візьмемо довільну точку $x_0\in M$ і побудуємо послідовність $A^nx_0$. Покажемо, що побудована послідовність є фундаментальною. Дійсно
	\begin{align*}
		%\label{eq:1.6.5}
		\rho(x_2, x_1) &= \rho(A x_1, A x_0) \le \alpha \rho (x_1, x_0), \\
		%\label{eq:1.6.6}
		\rho(x_3, x_2) &= \rho(A x_2, A x_1) \le \alpha \rho (x_2, x_1) \le \alpha^2 \rho(x_1, x_0), \\
		%\label{eq:1.6.7}
		\rho(x_{n+1}, x_n) &= \rho(A x_n, A x_{n-1}) \le \alpha \rho (x_n, x_{n-1}) \le \ldots \le \alpha^n \rho(x_1, x_0).
	\end{align*}
	Оцінимо $\rho(x_n, x_{n+m})$. Застосувавши $m-1$ раз нерівність трикутника, отримуємо 
	\begin{multline*}
		%\label{eq:1.6.8}
		\rho(x_n, x_{n+m}) \le \rho(x_n, x_{n+1}) + \rho(x_{n+1}, x_{n+2}) + \ldots + \rho(x_{n+m-1},x_{n+m}) \le \\
		\le \alpha^n \rho(x_1, x_0) + \alpha^{n+1} \rho(x_1, x_0) +\alpha^{n+m-1} \rho(x_1, x_0) = \\
		= (\alpha^n + \alpha^{n+1} + \ldots + \alpha^{n + m}) \cdot \rho(x_1, x_0) < \frac{\alpha^n}{1 - \alpha} \cdot \rho(x_1, x_0) \xrightarrow[n\to\infty]{} 0.
	\end{multline*}
	Тобто послідовність $\{x_n\}$ є фундаментальною і, в силу повноти простору $M$, збігається до деякого елемента цього ж простора $x$. \\

	Покажемо, що $x$ є нерухомою точкою $A$, тобто $Ax=x$.\\

	Нехай від супротивного $Ax=\bar x$ і $x\ne\bar x$. Застосувавши нерівність трикутника, одержимо $\rho(x,\bar x) < \rho(x, x_{n+1}) + \rho(x_{n+1}, \bar x)$. Оцінимо кожний з доданків.
	\begin{enumerate}
		\item $\rho(x, x_{n+1}) \xrightarrow[n\to\infty]{} 0$.
		\item $\rho(x_{n+1}, \bar x) = \rho(Ax_n, Ax) \le \alpha \rho(x_n, x) \xrightarrow[n\to\infty]{} 0$.
	\end{enumerate}
	Таким чином $\rho(x, \bar x) \le 0$, а в силу невід'ємності метрики це значить, що $x = \bar x$. \\

	Покажемо, що нерухома точка єдина. Нехай, від супротивного, існують дві точки $x$ і $y$: $A x = x$ і $A y = y$. Але тоді
	\begin{equation*}
		%\label{eq:1.6.9}
		\rho(x, y) = \rho(A x, A y) \le \alpha \rho(x, y) < \rho(x, y),
	\end{equation*}
	 що суперечить припущенню про стислість оператора. Таким чином, припущення про неєдиність нерухомої точки помилкове.
\end{proof}

З використанням теореми про нерухому точку доведемо теорему про існування та єдиність розв’язку задачі Коші диференціального рівняння, розв’язаного відносно похідної.

\begin{theorem}[про існування та єдиність розв’язку задачі Коші]
	Нехай у диференціальному рівнянні $\frac{\diff y}{\diff x} = f(x, y)$ функція $f(x,y)$ визначена в прямокутнику
	\begin{equation*}
		%\label{eq:1.6.10}
		D = \{(x,y) : x_0 - a \le x \le x_0 + a, y_0 - b \le y \le y_0 + b\},
	\end{equation*}
	і задовольняє умовам:
	\begin{enumerate}
		\item $f(x,y)$ неперервна по $x$ та $y$ у $D$;
		\item $f(x,y)$ задовольняє умові Ліпшиця по змінній $y$, тобто 
		\begin{equation*}
			%\label{eq:1.6.11}
			|f(x, y_1) - f(x, y_2)| \le N \cdot |y_1 - y_2|, \quad N = const.
		\end{equation*}
	\end{enumerate}
	Тоді існує єдиний розв’язок $y = y(x)$ диференціального рівняння, який визначений при $x_0 - h \le x \le x_0 + h$, і задовольняє умові $y(x_0) = y_0$, де $h < \min \{a, b / M, 1 / N\}$, $M = \max_{(x, y) \in D} |f(x,y)|$.
\end{theorem}

\begin{proof}
	Розглянемо простір, елементами якого є функції $y(x)$, неперервні на відрізку $[x_0 - h, x_0 + h]$ й обмежені $|y(x) - y_0| \le b$. Введемо метрику $\rho(y(x), z(x))$. Одержимо повний метричний простір $C([x_0 - h, x_0 + h])$. Замінимо диференціальне рівняння
	\begin{equation*}
		%\label{eq:1.6.12}
		\frac{\diff y}{\diff x} = f(x, y), \quad y(x_0) = y_0
	\end{equation*}
	еквівалентним інтегральним рівнянням
	\begin{equation*}
		%\label{eq:1.6.13}
		y(x) = \int_{x_0}^x f(t, y(t)) \diff t + y_0 = A y.
	\end{equation*}
	Розглянемо оператор $A$. Через те, що  
	\begin{equation*}
		%\label{eq:1.6.14}
		\left|\int_{x_0}^x f(t, y(t)) \diff t \right| \le \int_{x_0}^x |f(t, y(t))| \diff t \le M \cdot |x-x_0| \le Mh \le b,
	\end{equation*}
	то оператор $A$ ставить у відповідність кожній неперервній функції $y(x)$, визначеній при $x\in[x_0 - h, x_0 + h]$ й обмеженій $|y(x)-y_0|\le b$ також неперервну функцію $Ay$,  визначену при $x\in[x_0 - h, x_0 + h]$ й обмежену $|y(x)-y_0|\le b$. \\

	Перевіримо, чи є оператор $A$ оператором стиску:
	\begin{align*}
		%\label{eq:1.6.15}
		\rho(Ay, Az) &= \max_{x \in[x_0-h,x_0+h]} \left|y_0 + \int_{x_0}^x f(t, y(t)) \diff y - y_0 - \int_{x_0}^x f(t, z(t)) \diff t\right| \le \\
		&\le \max_{x \in[x_0-h,x_0+h]} \int_{x_0}^x |f(t, y(t)) - f(t, z(t))| \diff t \le \\
		&\le N \cdot \max_{x \in[x_0-h,x_0+h]} \int_{x_0}^x |y(t) - z(t)| \diff t \le \\
		&\le N \cdot \max_{x \in[x_0-h,x_0+h]} |y(t) - z(t)| \cdot \int_{x_0}^x \diff t \le N \cdot \rho(y, z) \cdot h.
	\end{align*}
	І оскільки $Nh < 1$, то оператор $A$ є оператором стиску. Відповідно до принципу стискаючих відображень операторне рівняння $Ay=y$ має єдиний розв’язок, тобто інтегральне рівняння чи початкова задача Коші також має єдиний розв’язок.
\end{proof}

\begin{remark}
	Умову Ліпшиця можна замінити іншою, більш грубою, але легше перевіряємою умовою існування обмеженої по модулю частинної похідної $f_y^\prime (x,y)$ в області $D$. Дійсно,
	\begin{equation*}
		%\label{eq:1.6.16}
		|f(x,y_1)-f(x,y_2)|=|f_y^\prime(x,\xi)|\cdot|y_1-y_2|\le N\cdot|y_1-y_2|,
	\end{equation*}
	де $N = \max_{(x,y)\in D} |f_y^\prime(x,y)|$.
\end{remark}

Використовуючи доведену теорему про існування та єдиність роз\-в'яз\-ку задачі Коші розглянемо ряд теорем, що описують якісну поведінку роз\-в'яз\-ків.

\begin{theorem}[про неперервну залежність роз\-в'яз\-ків від параметру]
	Якщо права частина диференціального рівняння
	\begin{equation*}
		%\label{eq:1.6.17}
		\frac{\diff y}{\diff x} = f(x, y, \mu)
	\end{equation*}
	неперервна по $\mu$ при $\mu \in [\mu_1, \mu_2]$ і при кожному фіксованому $\mu$ задовольняє умовам теореми існування й єдиності, причому стала Ліпшиця $N$ не залежить від $\mu$, то розв’язок $y = y(x, \mu)$, що задовольняє початковій умові $y(x_0)=y_0$, неперервно залежить від $\mu$.
\end{theorem}
\begin{proof} 
	Оскільки члени послідовності
	\begin{equation*}
		%\label{eq:1.6.18}
		y_n(x, \mu) = y_0 + \int_{x_0}^x f(t, y_n(t, \mu)) \diff t
	\end{equation*}
	є неперервними функціями змінних $x$ і $\mu$, а стала $N$ не залежить від $\mu$, то послідовність $\{y_n\}$ збігається до $y$ рівномірно по $\mu$. І, як випливає з математичного аналізу, якщо послідовність неперервних функцій збігається рівномірно, то вона збігається до неперервної функції, тобто $y=y(x,\mu)$ -- функція, неперервна по $\mu$.
\end{proof}

\begin{theorem}[про неперервну залежність від початкових умов]
	Нехай виконані умови теореми про існування та єдиність роз\-в'я\-зків рівняння
	\begin{equation*}
		%\label{eq:1.6.19}
		\frac{\diff y}{\diff x} = f(x,y)
	\end{equation*}
	з початковими умовами $y(x_0) = y_0$. Тоді, розв’язки $y=y(x_0,y_0,x)$, що записані у формі Коші, неперервно залежать від початкових умов. 
\end{theorem}
\begin{proof}
	Роблячи заміну $x = y(x_0, y_0, x) - y_0$, $t = x - x_0$ одержимо диференціальне рівняння  
	\begin{equation*}
		%\label{eq:1.6.20}
		\frac{\diff z}{\diff t} = f(t + x_0, z + y_0)
	\end{equation*}
	з нульовими початковими умовами. На підставі попередньої теореми маємо неперервну залежність розв’язків від $x_0$, $y_0$ як від параметрів.
\end{proof}

\begin{theorem}[про диференційованість розв’язків]
	Якщо в околі точки $(x_0,y_0)$ функція $f(x,y)$ має неперервні змішані похідні до $k$-го порядку, то розв’язок $y(x)$ рівняння
	\begin{equation*}
		%\label{eq:1.6.21}
		\frac{\diff y}{\diff x} = f(x, y)
	\end{equation*}
	з початковими умовами $y(x_0)=y_0$ в деякому околі точки $(x_0,y_0)$ буде $k$ разів неперервно диференційований.
\end{theorem}
\begin{proof} 
	Підставивши $y(x)$ в рівняння, одержимо тотожність
	\begin{equation*}
		%\label{eq:1.6.22}
		\frac{\diff y(x)}{\diff x} \equiv f(x, y(x)),
	\end{equation*}
	яку можна диференціювати
	\begin{equation*}
		%\label{eq:1.6.23}
		\frac{\diff^2 y}{{\diff x}^2} = \frac{\partial f}{\partial x} + \frac{\partial f}{\partial y} \cdot \frac{\diff y}{\diff x} = \frac{\partial f}{\partial x} + \frac{\partial f}{\partial y} \cdot f.
	\end{equation*}
	Якщо $k > 1$, то праворуч функція неперервно диференційована. Продиференціюємо її ще раз
	\begin{multline*}
		%\label{eq:1.6.24}
		\frac{\diff^3 y}{{\diff x}^3} = \frac{\partial^2 f}{{\partial x}^2} + \frac{\partial^2 f}{\partial x \partial y} \cdot \frac{\diff y}{\diff x} + \left( \frac{\partial^2 f}{\partial y \partial x} + \frac{\partial^2 f}{{\partial y}^2} \cdot \frac{\diff y}{\diff x} \right) \cdot f + \\
		+ \frac{\partial f}{\partial y} \cdot \left( \frac{\partial f}{\partial x} + \frac{\partial f}{\partial y} \cdot \frac{\diff y}{\diff x} \right),
	\end{multline*}
	або
	\begin{equation*}
		%\label{eq:1.6.25}
		\frac{\diff^3 y}{{\diff x}^3} = \frac{\partial^2 f}{{\partial x}^2} + 2 \cdot \frac{\partial^2 f}{\partial x \partial y} \cdot f + \frac{\partial^2 f}{{\partial y}^2} \cdot f^2 + \frac{\partial f}{\partial y} \cdot \left( \frac{\partial f}{\partial x} + \frac{\partial f}{\partial y} \cdot F \right),
	\end{equation*}
	Проробивши це $k$ разів, отримаємо твердження теореми.
\end{proof}

Розглянемо диференціальне рівняння, не розв’язане відносно похідної
\begin{equation*}
	%\label{eq:1.6.26}
	F(x, y, y') = 0.
\end{equation*}
Нехай $(x_0, y_0)$ -- точка на площині. Підставивши її в рівняння, одержимо відносно $y'$ алгебраїчне рівняння
\begin{equation*}
	%\label{eq:1.6.27}
	F(x_0, y_0, y') = 0.
\end{equation*}
Це рівняння має корені $y_0^\prime, y_1^\prime, \ldots, y_n^\prime$. Задача Коші для диференціального рівняння, не розв’язаного відносно похідної, ставиться в такий спосіб. \\

Потрібно знайти розв’язок $y=y(x)$ диференціального, що задовольняє умовам $y(x_0)=y_0$, $y'(x_0)=y_i^\prime$, де $x_0,y_0$ -- довільні значення, а $y_i^\prime$ -- один з вибраних наперед коренів алгебраїчного рівняння.

\begin{theorem}[існування й єдиність розв’язку задачі Коші рівняння, не розв’язаного  відносно похідної]
	Нехай у замкненому околі точки $(x_0, y_0, y_i^\prime)$ функція $F(x,y,y')$ задовольняє умовам:
	\begin{enumerate}
		\item $F(x,y,y')$ -- неперервна по всіх аргументах;
		\item $\frac{\partial F}{\partial y'}$ існує і відмінна від нуля;
		\item $\left| \frac{\partial F}{\partial y}\right| \le N_0$.
	\end{enumerate}
	Тоді при $x \in [x_0 - h, x_0 + h]$, де $h$ -- досить мал е, існує єдиний розв’язок $y=y(x)$ рівняння $F(x, y, y') =0$, що задовольняє початковій умові $y(x_0)=y_0$, $y'(x_0)=y_i^\prime$.
\end{theorem}
\begin{proof}
	Як випливає з математичного аналізу відповідно до теореми про неявну функцію можна стверджувати, що умови 1) і 2) гарантують існування єдиної неперервної в околі точки $(x_0,y_0,y_i^\prime)$ функції $y'=f(x,y)$, обумовленої рівнянням $F(x,y,y')=0$, для якої $y'(x_0)=y_i^\prime$. Перевіримо, чи задовольняє $f(x,y)$ умові Ліпшиця чи більш грубій $\left|\frac{\partial f}{\partial y}\right| \le N$. Диференціюємо $F(x,y,y')=0$ по $y$. Оскільки $y'=f(x,y)$, то одержуємо
	\begin{equation*}
		%\label{eq:1.6.28}
		\frac{\partial F}{\partial y} + \frac{\partial F}{\partial y'} \cdot \frac{\partial f}{\partial y} = 0.
	\end{equation*}
	Звідси
	\begin{equation*}
		%\label{eq:1.6.29}
		\frac{\partial f}{\partial y} = - \frac{\frac{\partial F}{\partial y}}{\frac{\partial F}{\partial y'}} 
	\end{equation*} 
	З огляду на умови 2), 3), одержимо, що в деякому околі точки $(x_0,y_0)$ буде $\left|\frac{\partial f}{\partial y}\right| \le N$ і для рівняння $y'=f(x,y)$ виконані умови теореми існування й єдиності розв’язку задачі Коші.
\end{proof}


		\subsubsection{Особливі розв’язки}
		\begin{definition}
	Розв’язок $y = \phi(x)$ диференціального рівняння, в кожній точці якого $M(x,y)$ порушена єдиність розв’язку задачі Коші, називається особливим розв’язком. 
\end{definition}

Очевидно, особливі розв’язки треба шукати в тих точках області $D$, де порушені умови теореми про існування й єдиність розв’язку задачі Коші. Але, оскільки умови теореми носять достатній характер, то їхнє не виконання для існування особливих розв’язків, носить необхідний характер. І точки $N(x,y)$ області $D$, у яких порушені умови теореми про існування та єдиність розв’язку диференціального рівняння, є лише "підозрілими" на особливі розв’язки. \\

Розглянемо рівняння 
\begin{equation*}
	%\label{eq:1.6.30}
	y' = f(x,y).
\end{equation*}
Неперервність $f(x,y)$ в області $D$ зазвичай виконується, і особливі роз\-в'яз\-ки варто шукати там, де $\frac{\partial f}{\partial y} = +\pm \infty$. \\

Для диференціального рівняння, не роз\-в'яз\-а\-но\-го відносно похідної 
\begin{equation*}
	%\label{eq:1.6.31}
	F(x, y, y') = 0,
\end{equation*}
умови неперервності $F(x,y,y')$ й обмеженості $\frac{\partial F}{\partial y}$ зазвичай виконуються. І особливі розв’язки варто шукати там, де задовольняється остання рівність і 
\begin{equation*}
	%\label{eq:1.6.32}
	\frac{\partial F(x,y,y')}{\partial y'} = 0.
\end{equation*}

Вилучаючи із системи $y'$, одержимо $\Phi(x,y)=0$. Однак не в кожній точці $M(x,y)$, у якій $\Phi(x,y)$, порушується єдиність роз\-в'яз\-ку, тому що умови теореми мають лише достатній характер і не є необхідними. Якщо ж яка-небудь гілка $y=\phi(x)$ кривої $\Phi(x,y)$ є інтегральною кривою, то $y=\phi(x)$ називається особливим роз\-в'яз\-ком. \\

Таким чином, для знаходження особливого роз\-в'яз\-ку $F(x, y, y') = 0$ треба
\begin{enumerate}
	\item знайти $p$-дискримінантну криву з $F(x, y, y') = 0$ та $\frac{\partial F(x,y,y')}{\partial y'} = 0$.
	\item з'я\-су\-ва\-ти шляхом підстановки чи є серед гілок $p$-дискримінантної кривої інтегральні криві;
	\item з'я\-су\-ва\-ти чи порушена умова одиничності в точках цих кривих.
\end{enumerate}


		\subsubsection{Вправи для самостійної роботи}
		\setcounter{problem}{0}

\begin{example}
	Побудувати послідовні наближення $y_0(x)$, $y_1(x)$, $y_2(x)$ для рівняння $y' = x - y^2$, $y(0) = 0$.
\end{example}
\begin{solution}
	Візьмемо початкову функцію $y_0(0) \equiv 0$. Підставивши в ітераційну залежність \[ y_{n+1} (x) = y(x_0) + \int_{x_0}^x f(s, y_n(s)) \diff s \] отримаємо 
	\begin{align*} 
		y_1(x) &= \int_0^x s \diff s = \frac{x^2}{2}, \\
		y_2(x) &= \int_0^x (s - y_1^2(s)) \diff s = \int_0^x \left(s - \frac{s^4}{4}\right) \diff s = \frac{x^2}{2} - \frac{x^5}{20}.
	\end{align*}
\end{solution}

Побудувати послідовні наближення $y_0(x)$, $y_1(x)$, $y_2(x)$ для рівнянь
\begin{problem}
	\[y' = y^2 + 3x^2 - 1, \quad y(0) = 1;\]
\end{problem}
\begin{problem}
	\[y'=y+e^{y-1},\quad y(0)=1;\]
\end{problem}
\begin{problem}
	\[y'=1+x\cdot\sin y, \quad y(\pi)=2\pi.\]
\end{problem}

\begin{example}
	Вказати на проміжок з $a=1$, $b=1$, на якому гарантується існування та єдиність розв’язку диференціального рівняння $y'=y^3+x$, $y(0)=1$.
\end{example}
\begin{solution}
	Як випливає з теореми про існування та єдиність розв’язку, проміжок, на якому гарантується існування та єдиність розв’язку задачі Коші дорівнює $h = \min \left\{ a, \frac{b}{M}, \frac{1}{L}\right\}$, де \[ M = \max_{(x,y)\in D} |f(x,y)|, \quad L = \max_{(x,y)\in D} \left|\frac{\partial f(x,y)}{\partial y}\right|.\] Для цієї задачі отримаємо $D = \{ (x,y): |x| \le 1, |y| \le 1\}$, $M=2$, $L=3$. Тому $h = 1/3$.
\end{solution}	

Вказати проміжки, де гарантується існування та єдиність розв’язку задачі Коші рівняння
\begin{problem}
	\[y'=y+e^y,\quad x_0=0, \quad y_0=0,\quad a=1,\quad b=2;\]
\end{problem}
\begin{problem}
	\[y'=2xy+y^3,\quad x_0=1, \quad y_0=1,\quad a=2,\quad b=1;\]
\end{problem}
\begin{problem}
	\[y'=2+\sqrt[3]{y-2x},\quad x_0=0, \quad y_0=1,\quad a=1,\quad b=1.\]
\end{problem}

\begin{example}
	Знайти особливий розв’язок рівняння $y' - \sqrt{y}$.
\end{example}
\begin{solution}
	Особливий розв’язок слід шукати там, де $\frac{\partial f(x,y)}{\partial y}=\pm\infty$. Оскільки $\frac{\partial f(x,y)}{\partial y}=\frac{1}{2\sqrt{y}}$, то отримаємо $\bar y(x)=0$ -- крива, що підозріла на особливу. Перевірка показує, що це дійсно інтегральна крива. Щоб до кінця переконатися, що ця крива особлива, розв’язуємо рівняння \[y'=\sqrt{y}\implies \frac{\diff y}{\sqrt{y}}=\diff x\implies 2\sqrt{y}=x+C\implies y(x)=\frac{(x+c)^2}{4}.\]
	Легко переконатися, що $\bar y(x)=0$ є кривою, що огинає сім’ю інтегральних кривих $y(x)=\frac{(x+c)^2}{4}$. 
\end{solution}

\begin{example}
	Знайти особливий розв’язок рівняння $y = x + y' - \ln y'$.
\end{example}
\begin{solution}
	Складаємо рівняння $p$-дискриминантної кривої \[ y = x + p - \ln p, \quad 0 = 1 - \frac{1}{p}.\] Із другого рівняння $p=1$. Підставивши в перше, отримаємо, що крива, що є підозрілою як особлива, має вигляд $\bar y(x)=x+1$. \\

	Підставивши у рівняння, отримаємо $x + 1=x+1-\ln 1$, тобто впевнились, що $\bar y(x)=x+1$ є інтегральною кривою. \\

	Розв’яжемо рівняння методом введення параметру. Його загальний розв’язок має вигляд \[y=Ce^x-\ln C.\] Можна переконатися, що $\bar y(x)=x+1$ є кривою, що огинає сім’ю інтегральних кривих. \\

	Щоб перевірити це аналітично, запишемо умову дотику кривої $y=x+1$ та $y=Ce^x-\ln C$ в точці $(x_0,y_0)$. Вона має вигляд: \[ \bar y(x_0) = y(x_0, C), \quad \bar{y}'(x_0) = y'(x_0,C).\] Тобто \[x_0+1=Ce^{x_0}-\ln C, \quad 1 = Ce^{x_0}.\] З другого рівняння отримаємо $C = e^{-x_0}$. Підставивши у перше рівняння, маємо $x_0 + 1 = 1 - \ln e^{-x_0}$, тобто $x_0+1=x_0+1$ -- тотожність. Таким чином при кожному $x_0$ відбувається дотик інтегральних кривих та $\bar y(x)=x+1$, що огинає сім’ю інтегральних кривих.
\end{solution}

Знайти особливі розв’язки та зробити рисунок.
\begin{multicols}{2}
\begin{problem}
	\[8\cdot(y')^3-27y=0;\]
\end{problem}
\begin{problem}
	\[(y'+1)^3-27\cdot(x+y)^2=0;\]
\end{problem}
\begin{problem}
	\[y^2\cdot((y')^2+1)=1;\]
\end{problem}
\begin{problem}
	\[(y')^2-4y^3=0.\]
\end{problem}
\end{multicols}

\section{Нелінійні диференціальні рівняння вищих порядків}
\subsection{Загальні визначення. Існування та єдиність роз\-в’я\-з\-ків рівнянь}

Диференціальне рівняння $n$-го порядку має вигляд
\begin{equation*}
	%\label{eq:2.1.1}
	F \left( x, y, y', \ldots, y^{(n)} \right) = 0.
\end{equation*}
Якщо диференціальне рівняння розв’язане відносно старшої похідної, то воно має вигляд
\begin{equation*}
	%\label{eq:2.1.2}
	y^{(n)} = f \left( x, y, y', \ldots, y^{(n-1)} \right) = 0.
\end{equation*}
Іноді його називають диференціальним рівнянням у нормальній формі. Для диференціального рівняння, розв’язаного відносно похідної, задача Коші ставиться таким чином. Потрібно знайти функцію $y = y(x)$, $n$ разів неперервно диференційовану і таку, що при підстановці в останнє рівняння обертає його в тотожність і задовольняє початковим умовам 
\begin{equation*}
	%\label{eq:2.1.3}
	y(x_0) = y_0, y'(x_0) = y_0', \ldots, y^{(n - 1)} (x_0) = y_0^{(n-1)}.
\end{equation*}
Для диференціального рівняння, не розв’язаного відносно похідної, задача Коші полягає в знаходженні розв’язку $y = y(x)$, що задовольняє початковим даним 
\begin{equation*}
	%\label{eq:2.1.4}
	y(x_0) = y_0, y'(x_0) = y_0', \ldots, y^{(n - 1)} (x_0) = y_0^{(n-1)}, y^{(n)} (x_0) = y_0^{(n)},
\end{equation*}
де значення $x_0, y_0, y_0', \ldots, y_0^{(n-1)}$ довільні, а $y_0^{(n)}$ один з коренів алгебраїчного рівняння 
\begin{equation*}
	%\label{eq:2.1.5}
	F \left( x_0, y_0, y_0', \ldots, y_0^{(n)} \right) = 0.
\end{equation*}

\begin{theorem}[існування та єдиності розв’язку задачі Коші рівняння, розв’язаного відносно похідної]
	Нехай у деякому замкненому околі точки $\left(x_0, y_0, y_0', \ldots, y_0^{(n-1)}\right)$ функція $f\left(x,y,y',\ldots,y^{(n-1)}\right)$ задовольняє умовам:
	\begin{enumerate}
		\item вона визначена і неперервна по всім змінним;
		\item задовольняє умові Ліпшиця по всім змінним, починаючи з другої.
	\end{enumerate}
	Тоді при $x_0 - h \le x \le x_0 + h$, де $h$ -- досить мала величина, існує і єдиний розв’язок $y=y(x)$ рівняння
		\begin{equation*}
    	%\label{eq:2.1.2}
    	y^{(n)} = f \left( x, y, y', \ldots, y^{(n-1)} \right) = 0,
    \end{equation*}
    що задовольняє початковим умовам 
	\begin{equation*}
    	%\label{eq:2.1.3}
    	y(x_0) = y_0, y'(x_0) = y_0', \ldots, y^{(n - 1)} (x_0) = y_0^{(n-1)}.
    \end{equation*}
\end{theorem}
  
\begin{theorem}[існування та єдиності розв’язку задачі Коші рівняння, не розв’язаного відносно похідної]
	Нехай у деяком замкненому околі точки $\left(x_0, y_0, y_0', \ldots, y_0^{(n-1)}, y_0^{(n)}\right)$ функція $F\left(x,y,y',\ldots,y^{(n-1)},y^{(n)}\right)$ задовольняє умовам:
	\begin{enumerate}
		\item вона визначена і неперервна по всім змінним;
		\item її частинні похідні по всім змінним з другої пердеостанньої обмежені:
		\begin{equation*}
			%\label{eq:2.1.6}
			\left|\frac{\partial F}{\partial y}\right| < M_0, \quad \left|\frac{\partial F}{\partial y'}\right| < M_1, \quad \ldots, \quad \left|\frac{\partial F}{\partial y^{(n-1)}}\right| < M_{n-1}.
		\end{equation*}
		\item її частинна похідна по останній змінній не обертаєтсья на нуль: \[\left|\frac{\partial F}{\partial y^{(n)}}\right|\ne0.\]
	\end{enumerate}
	Тоді при $x_0 - h \le x \le x_0 + h$, де $h$ -- досить мала величина, існує і єдиний розв’язок $y=y(x)$ рівняння
    \begin{equation*}
	    %\label{eq:2.1.1}
    	F \left( x, y, y', \ldots, y^{(n)} \right) = 0.
    \end{equation*}
    що задовольняє початковим умовам
    \begin{equation*}
    	%\label{eq:2.1.4}
    	y(x_0) = y_0, y'(x_0) = y_0', \ldots, y^{(n - 1)} (x_0) = y_0^{(n-1)}, y^{(n)} (x_0) = y_0^{(n)}.
    \end{equation*}
\end{theorem}
\begin{definition}
	Загальним розв’язком диференціального рівняння $n$-го порядку називається $n$ разів неперервно диференційована функція $y=y(x,C_1,C_2,\ldots,C_n)$, що обертає при підстановці рівняння в тотожність, у якій вибором сталих $C_1, C_2, \ldots, C_n$ можна одержати розв’язок довільної задачі Коші в області існування та єдиності розв’язків.
\end{definition}



	\subsection{Загальні визначення. Існування та єдиність роз\-в’я\-з\-ків рівнянь}
	Диференціальне рівняння $n$-го порядку має вигляд
\begin{equation*}
	F \left( x, y, y', \ldots, y^{(n)} \right) = 0.
\end{equation*}

Якщо диференціальне рівняння розв'язане відносно старшої похідної, то воно має вигляд
\begin{equation*}
	y^{(n)} = f \left( x, y, y', \ldots, y^{(n-1)} \right) = 0.
\end{equation*}

Іноді його називають диференціальним рівнянням у нормальній формі. Для диференціального рівняння, розв'язаного відносно похідної, задача Коші ставиться таким чином. Потрібно знайти функцію $y = y(x)$, $n$ разів неперервно диференційовану і таку, що при підстановці в останнє рівняння обертає його в тотожність і задовольняє початковим умовам 
\begin{equation*}
	y(x_0) = y_0, \quad y'(x_0) = y_0', \quad \ldots, \quad y^{(n - 1)} (x_0) = y_0^{(n-1)}.
\end{equation*}

Для диференціального рівняння, не розв'язаного відносно похідної, задача Коші полягає в знаходженні розв'язку $y = y(x)$, що задовольняє початковим даним 
\begin{equation*}
	y(x_0) = y_0, \quad y'(x_0) = y_0', \quad \ldots, \quad y^{(n - 1)} (x_0) = y_0^{(n-1)}, \quad y^{(n)} (x_0) = y_0^{(n)},
\end{equation*}
де значення $x_0, y_0, y_0', \ldots, y_0^{(n-1)}$ довільні, а $y_0^{(n)}$ один з коренів алгебраїчного рівняння 
\begin{equation*}
	F \left( x_0, y_0, y_0', \ldots, y_0^{(n)} \right) = 0.
\end{equation*}

\begin{theorem}[існування та єдиності розв'язку задачі Коші рівняння, розв'язаного відносно похідної]
	Нехай у деякому замкненому околі точки $\left(x_0, y_0, y_0', \ldots, y_0^{(n-1)}\right)$ функція $f\left(x,y,y',\ldots,y^{(n-1)}\right)$ задовольняє умовам:
	\begin{enumerate}
		\item вона визначена і неперервна по всім змінним;
		\item задовольняє умові Ліпшиця по всім змінним, починаючи з другої.
	\end{enumerate}
	
	Тоді при $x_0 - h \le x \le x_0 + h$, де $h$ --- досить мала величина, існує і єдиний розв'язок $y=y(x)$ рівняння
		\begin{equation*}
		y^{(n)} = f \left( x, y, y', \ldots, y^{(n-1)} \right) = 0,
	\end{equation*}
	що задовольняє початковим умовам 
	\begin{equation*}
		y(x_0) = y_0, \quad y'(x_0) = y_0', \quad \ldots, \quad y^{(n - 1)} (x_0) = y_0^{(n-1)}.
	\end{equation*}
\end{theorem}

\begin{theorem}[існування та єдиності розв'язку задачі Коші рівняння, не розв'язаного відносно похідної]
	Нехай у деяком замкненому околі точки $\left(x_0, y_0, y_0', \ldots, y_0^{(n-1)}, y_0^{(n)}\right)$ функція $F\left(x,y,y',\ldots,y^{(n-1)},y^{(n)}\right)$ задовольняє умовам:
	\begin{enumerate}
		\item вона визначена і неперервна по всім змінним;
		\item її частинні похідні по всім змінним з другої до пердеостанньої обмежені:
		\begin{equation*}
			\left|\frac{\partial F}{\partial y}\right| < M_0, \quad \left|\frac{\partial F}{\partial y'}\right| < M_1, \quad \ldots, \quad \left|\frac{\partial F}{\partial y^{(n-1)}}\right| < M_{n-1}.
		\end{equation*}
		\item її частинна похідна по останній змінній не обертаєтсья на нуль: \[\left|\frac{\partial F}{\partial y^{(n)}}\right|\ne0.\]
	\end{enumerate}
	
	Тоді при $x_0 - h \le x \le x_0 + h$, де $h$ --- досить мала величина, існує і єдиний розв'язок $y=y(x)$ рівняння
	\begin{equation*}
		F \left( x, y, y', \ldots, y^{(n)} \right) = 0.
	\end{equation*}
	що задовольняє початковим умовам
	\begin{equation*}
		y(x_0) = y_0, \quad y'(x_0) = y_0', \quad \ldots, \quad y^{(n - 1)} (x_0) = y_0^{(n-1)}, \quad y^{(n)} (x_0) = y_0^{(n)}.
	\end{equation*}
\end{theorem}

\begin{definition}
	Загальним розв'язком диференціального рівняння $n$-го порядку називається $n$ разів неперервно диференційована функція вигляду $y = y(x,C_1, C_2, \ldots, C_n)$, що обертає при підстановці рівняння в тотожність, у якій вибором сталих $C_1, C_2, \ldots, C_n$ можна одержати розв'язок довільної задачі Коші в області існування та єдиності розв'язків.
\end{definition}



	\subsection{Диференціальні рівняння вищих порядків, що інтегруються в квадратурах}
	Розглянемо деякі типи диференціальних рівнянь, що інтегруються в квадратурах.

\begin{enumerate}
\item Рівняння вигляду
\begin{equation*}
	%\label{eq:2.2.1}
	y^{(n)} = f(x).
\end{equation*}
Проінтегрувавши його $n$ разів одержимо загальний розв’язок у вигляді
\begin{equation*}
	%\label{eq:2.2.2}
	y = \underset{n}{\underbrace{\int \cdots \int}} f(x) \,\underset{n}{\underbrace{\diff x \cdots \diff x}} + C_1 x^{n-1} + C_2 x^{n-2} + \ldots + C_{n-1} x + C_n.
\end{equation*}
Якщо задані умови Коші
\begin{equation*}
	%\label{eq:2.1.3}
	y(x_0) = y_0, y'(x_0) = y_0', \ldots, y^{(n - 1)} (x_0) = y_0^{(n-1)},
\end{equation*}
то розв’язок має вигляд
\begin{multline*}
	%\label{eq:2.2.3}
	y = \underset{n}{\underbrace{\int_{x_0}^x \cdots \int_{x_0}^x}} f(t) \, \underset{n}{\underbrace{\diff t \cdots \diff t}} + \frac{y_0}{(n-1)!} \cdot (x-x_0)^{n-1} + \\ 
	+ \frac{y_0'}{(n-2)!} \cdot (x-x_0)^{n-2} + \ldots + y_0^{(n-2)} \cdot (x-x_0) + y_0^{(n-1)}.
\end{multline*}
\item Рівняння вигляду
\begin{equation*}
	%\label{eq:2.2.4}
	F\left(x, y^{(n)}\right) = 0.
\end{equation*}
Нехай це рівняння вдалося записати в параметричному вигляді
\begin{equation*}
	%\label{eq:2.2.5}
	\left\{
		\begin{aligned}
			x &= \phi(t), \\
			y^{(n)} &= \psi (t).
		\end{aligned}
	\right.
\end{equation*}
Використовуючи основне співвідношення $\diff y^{(n-1)} = y^{(n)} \cdot \diff x$, одержимо
\begin{equation*}
	%\label{eq:2.2.6}
	\diff y^{(n-1)} = \psi(t) \cdot \phi(t) \cdot \diff t
\end{equation*}
Проінтегрувавши його, маємо 
  \begin{equation*}
	%\label{eq:2.2.7}
	y^{(n-1)} = \int \psi(t) \cdot \phi(t) \cdot \diff t + C_1 = \psi_1(t, C_1).
\end{equation*}
І одержимо параметричний запис рівняння $(n-1)$-го порядку:
\begin{equation*}
	%\label{eq:2.2.8}
	\left\{
		\begin{aligned}
			x &= \phi(t), \\
			y^{(n-1)} &= \psi_1(t, C_1).
		\end{aligned}
	\right.
\end{equation*}
Проробивши зазначений процес ще $(n-1)$ раз, одержимо загальний розв’язок рівняння в параметричному вигляді
\begin{equation*}
	%\label{eq:2.2.9}
	\left\{
		\begin{aligned}
			x &= \phi(t), \\
			y &= \psi_n(t, C_1, \ldots, C_n).
		\end{aligned}
	\right.
\end{equation*}
 
\item Рівняння вигляду
\begin{equation*}
	%\label{eq:2.2.10}
	F \left( y^{(n-1)}, y^{(n)} \right) = 0.
\end{equation*}
Нехай це рівняння вдалося записати в параметричному вигляді 
\begin{equation*}
	%\label{eq:2.2.11}
	\left\{
		\begin{aligned}
			y^{(n-1)} &= \phi(t), \\
			y^{(n)} &= \psi(t).
		\end{aligned}
	\right.
\end{equation*}
Використовуючи основне співвідношення $\diff y^{(n-1)} = y^{(n)} \cdot \diff x$, одержуємо
\begin{equation*}
	%\label{eq:2.2.12}
	\diff x = \frac{\diff y^{(n-1)}}{y^{(n)}} = \frac{\phi'(t)}{\psi(t)} \cdot \diff t.
\end{equation*}
Проінтегрувавши, маємо
\begin{equation*}
	%\label{eq:2.2.13}
	x = \int \frac{\phi'(t)}{\psi(t)} \cdot \diff t + C_1 = \psi_1(t, C_1).
\end{equation*}
І одержали параметричний запис майже з попереднього пункту. \parvskip

Використовуючи попередній пункт, запишемо загальний розв’язок у параметричному вигляді:
\begin{equation*}
	%\label{eq:2.2.14}
	\left\{
		\begin{aligned}
			x &= \psi(t, C_1), \\
			y &= \phi_n(t, C_2, \ldots, C_n).
		\end{aligned}
	\right.
\end{equation*}
 
\item Нехай рівняння вигляду
\begin{equation*}
	%\label{eq:2.2.15}
	F \left( y^{(n-2)}, y^{(n)} \right) = 0
\end{equation*}
можна розв'язати відносно старшої похідної
\begin{equation*}
	%\label{eq:2.2.16}
	y^{(n)} = f \left( y^{(n-2)} \right).
\end{equation*}
Домножимо його на $2 y^{(n-1)} \cdot \diff x$ й одержимо
\begin{equation*}
	%\label{eq:2.2.17}
	2 y^{(n-1)} \cdot y^{(n)} \cdot \diff x= 2 f \left( y^{(n-2)} \right) \cdot y^{(n-1)} \cdot \diff x.
\end{equation*}
Перепишемо його у вигляді
\begin{equation*}
	%\label{eq:2.2.18}
	\diff \left( y^{(n-1)} \right)^2 = 2 f \left( y^{(n-2)} \right) \cdot \diff y^{(n-2)}.
\end{equation*}
Проінтегрувавши, маємо
\begin{equation*}
	%\label{eq:2.2.19}
	\left( y^{(n-1)} \right)^2 = 2 \int f \left( y^{(n-2)} \right) \cdot \diff y^{(n-2)} + C_1,
\end{equation*}
тобто 
\begin{equation*}
	%\label{eq:2.2.20}
	y^{(n-1)}  = \pm \sqrt{2 \int f \left( y^{(n-2)} \right) \cdot \diff y^{(n-2)} + C_1},
\end{equation*}
або
\begin{equation*}
	%\label{eq:2.2.21}
	y^{(n-1)}  = \pm \psi_1 \left( y^{(n-2)}, C_1 \right).
\end{equation*}
Таким чином одержали повернулися до третього випадку.

\end{enumerate}

	\subsection{Найпростіші випадки зниження порядку в диференціальних рівняннях вищих порядків}
	Розглянемо деякі типи диференціальних рівнянь вищого порядку, що допускають зниження порядку.
\begin{enumerate}
\item Рівняння не містить шуканої функції і її похідних до $(k-1)$-го порядку включно:
\begin{equation}
	\label{eq:2.3.1}
	F \left( x, y^{(k)}, y^{(k + 1)}, \ldots, y^{(n)} \right) = 0.
\end{equation}
Зробивши заміну:
\begin{equation}
	\label{eq:2.3.2}
	y^{(k)} = z, \quad y^{(k + 1)} = z', \quad \ldots, \quad y^{(n)} = z^{(n - k)},
\end{equation}
одержимо рівняння $(n-k)$-го порядку
\begin{equation}
	\label{eq:2.3.3}
	F \left( x, z, z', \ldots, z^{(n - k)} \right) = 0.
\end{equation}

\item Рівняння не містить явно незалежної змінної
\begin{equation}
	\label{eq:2.3.4}
	F \left( y, y', \ldots, y^{(n)} \right) = 0.
\end{equation}

Будемо вважати, що $y$ -- нова незалежна змінна, а $y', \ldots, y^{(n)}$ -- функції від $y$. Тоді
\begin{align}
	\label{eq:2.3.5}
	y_x^\prime &= p(y), \\
	\label{eq:2.3.6}
	y_{x^2}^{\prime\prime} &= \frac{\diff}{\diff x} \cdot y_x^\prime = \frac{\diff}{\diff x} \cdot p(y) \cdot \frac{\diff y}{\diff x} = p_y^{\prime} \cdot  p(y), \\
	\label{eq:2.3.7}
	y_{x^3}^{\prime\prime\prime} &= \frac{\diff}{\diff x} \cdot y_{x^2}^{\prime\prime} = \frac{\diff}{\diff x} \cdot (p_y^\prime p) \cdot \frac{\diff y}{\diff x} = \left( p_{y^2}^{\prime\prime} \cdot p + \left( p_y^\prime \right)^2 \right) \cdot p,
\end{align}
і так далі до $y_{x^n}^{(n)}$. Після підстановки одержимо 
\begin{equation}
	\label{eq:2.3.8}
	F \left( y, p, p_y^{\prime} \cdot  p(y), \left( p_{y^2}^{\prime\prime} \cdot p + \left( p_y^\prime \right)^2 \right) \cdot p, \ldots, p^{(n - 1)} \right) = 0,
\end{equation}
диференціальне рівняння $(n-1)$-го порядку.
\item Нехай функція $F$ диференціального рівняння
\begin{equation}
	\label{eq:2.3.9}
	F \left( x, y, y', \ldots, y^{(n)} \right) = 0.
\end{equation}
є однорідної щодо аргументів  $y, y', \ldots, y^{(n)}$. \\

Робимо заміну $y = e^{\int u \diff x}$, де $u=u(x)$ -- нова невідома функція. Одержимо
\begin{align}
	\label{eq:2.3.10}
	y' &= e^{\int u \diff x} u, \\
	\label{eq:2.3.11}
	y^{\prime\prime} &= e^{\int u \diff x}  u^2 + e^{\int u \diff x} u' = e^{\int u \diff x} \left(u^2 + u'\right), \\
	\label{eq:2.3.12}
	y^{\prime\prime\prime} &= e^{\int u \diff x} u \left( u^2 + u' \right) + e^{\int u \diff x}  \left(2 u u' + u''\right) = \\ 
	&= e^{\int u \diff x} \left( u^3 + 3 u u' + u'' \right), \nonumber
\end{align}
і так далі до $y^{(n)}$. Після підстановки одержимо
\begin{equation}
	\label{eq:2.3.13}
	F \left( x, e^{\int u \diff x}, e^{\int u \diff x} u, e^{\int u \diff x} \left(u^2 + u'\right), e^{\int u \diff x} \left( u^3 + 3 u u' + u'' \right), \ldots \right) = 0.
\end{equation}

Оскільки \eqref{eq:2.3.9} (а отже і \eqref{eq:2.3.13}) однорідне відносно $e^{\int u\diff x}$, то цей член можна винести і на нього скоротити. Одержимо
\begin{equation}
	\label{eq:2.3.14}
	F \left( x, 1, u, u^2 + u', u^3 + 3 u u' + u'', \ldots \right) = 0,
\end{equation} 
диференціальне рівняння $(n-1)$-го порядку. 
\item Нехай ліва частина рівняння
\begin{equation}
	\label{eq:2.3.15}
	F \left( x, y, y', \ldots, y^{(n)} \right) = 0.
\end{equation}
є похідної деякого диференціального вираза ступеня $(n-1)$, тобто
\begin{equation}
	\label{eq:2.3.16}
	\frac{\diff}{\diff x} \cdot \Phi\left( x, y, y', \ldots, y^{(n-1)} \right) = F \left( x, y, y', \ldots, y^{(n)} \right).
\end{equation}
У цьому випадку легко обчислюється так званий перший інтеграл
\begin{equation}
	\label{eq:2.3.17}
	\Phi\left( x, y, y', \ldots, y^{(n-1)} \right) = C.
\end{equation}

\item Нехай диференціальне рівняння
\begin{equation}
	\label{eq:2.3.18}
	F \left( x, y, y', \ldots, y^{(n)} \right) = 0,
\end{equation}
розписано у вигляді диференціалів
\begin{equation}
	\label{eq:2.3.19}
	F \left( x, y, \diff y, \diff^2 y , \ldots, \diff^n y \right) = 0,
\end{equation}
і $F$ -- функція однорідна по всім змінним. Зробимо заміну $x = e^t$, $y = u \cdot e^t$, де $u$, $t$ -- нові змінні. Тоді одержуємо
\begin{align}
	\label{eq:2.3.20}
	\diff x &= e^t \diff t, \\
	\label{eq:2.3.21}
	y_x^\prime &=  \frac{y_t^\prime}{x_y^\prime} = \frac{u_t^\prime e^t + u e^t}{e^t} = u_t^\prime + u, \\
	\label{eq:2.3.22}
	y_{x^2}^{\prime\prime} &= \frac{\diff}{\diff x} \cdot y_x^\prime = \frac{\diff}{\diff t} \left( u_t^\prime + u \right) \cdot \frac{\diff t}{\diff x} = \frac{u_{t^2}^{\prime\prime} + u_t^\prime}{e^t}, \\
	\label{eq:2.3.23}
	y_{x^3}^{\prime\prime\prime} &= \frac{\diff}{\diff x} \cdot y_{x^2}^{\prime\prime} = \frac{\diff}{\diff t} \left( \frac{u_{t^2}^{\prime\prime} + u_t^\prime }{e^t} \right) \cdot \frac{\diff t}{\diff x} = \\
	&= \frac{\left( u_{t^3}^{\prime\prime\prime} + u_{t^2}^{\prime\prime} \right) e^t - \left( u_{t^2}^{\prime\prime} + u_t^\prime \right) e^t}{e^{3t}} = \frac{u_{t^3}^{\prime\prime\prime} - u_t^\prime}{e^{2t}}, \nonumber
\end{align}
і так далі до $y^{(n)}$. Підставивши, одержимо
\begin{multline}
	\label{eq:2.3.24}
	\Phi \left(x, y, \diff y, \diff^2 y, \ldots, \diff^n y\right) = \\
	= \Phi\left( e^t, u e^t, e^t \diff t, (u_t^\prime + u)e^t \diff t, \left(u_{t^2}^{\prime\prime} + u_t^\prime\right) e^t \diff t, \ldots\right) = 0.
\end{multline} 
Скоротивши на $e^t$ одержимо
\begin{equation}
	\label{eq:2.3.25}
	\Phi\left( 1, u, \diff t, u_t^\prime + u, u_{t^2}^{\prime\prime} + u_t^\prime, \ldots\right) = 0.
\end{equation} 
Тобто одержимо диференціальне рівняння вигляду \eqref{eq:2.3.4} і повертаємося до другого випадку.
\end{enumerate}

	\subsection{Вправи для самостійної роботи}
	Розглянемо приклади.
\begin{example}
	Розв’язати рівняння: $y'' = x + \sin x$.
\end{example}
\begin{solution}
	Інтегруємо два рази
	\begin{align*}
		y' &= \int (x + \sin x) \diff x + C_1 = \frac{x^2}{2} - \cos x + C_1; \\
		y &= \int \left( \frac{x^2}{2} - \cos x + C_1 \right) \diff x = \frac{x^3}{6} - \sin x + C_1 x+ C_2.
	\end{align*}
\end{solution}
\begin{example}
	Розв’язати рівняння $(y'')^3 - 2 y'' - x = 0$.
\end{example}
\begin{solution}
	Запишемо рівняння у параметричній формі \[ y'' = y, \quad x = t^3 - 2 t. \] Використовуючи співвідношення $\diff y' = y'' \diff x$, одержуємо \[ \diff y' = t (3t^2 - 2) \diff t,\] або \[ \diff y' = (3t^2 - 2t) \diff t.\] Звідси понижуємо порядок рівняння на одиницю \[ y' = \frac{3t^3}{4}-t^2+C_1, \quad x = t^3 - 2t.\] Знов використовуючи співвідношення $\diff y = y' \diff x$, одержуємо \[ \diff y = \left( \frac{3t^3}{4}-t^2+C_1 \right) \cdot (3t^2 - 2) \diff t,\] або \[ \diff y = \left( \frac{9t^5}{4} - 3 t^4 - \frac{3t^3}{2} + (2 + 3 C_1) t^2 - 2 C_1 \right) \diff t.\]	Звідси загальний розв’язок у параметричній формі має вигляд \[ x = t^3 - 2 t, \quad y = \frac{3t^6}{8} - \frac{3 t^5}{5} - \frac{3t^4}{8} + \frac{(2 + 3 C_1) t^3}{3} - 2 C_1 t + C_2.\]
\end{solution}
\begin{example}
	Розв’язати рівняння: $(y'')^3 + x y'' = y'$.
\end{example}
\begin{solution}
	Запишемо рівняння у параметричній формі \[ y'' = t, \quad y''' = e^{-t}. \] Використовуючи співвідношення $\diff y' = y'' \diff x$, одержуємо \[ \diff t = e^{-t} \diff x. \] Звідси $\diff x = e^t \diff t$ і $x = e^t + C_1$. Запишемо рівняння другого порядку \[ x = e^r + C_1, \quad y'' = t.\] Запишемо рівняння у параметричній формі \[ y'' = t, \quad x = t^3 - 2 t.\] Використовуючи співвідношення $\diff y' = y'' \diff x$, одержуємо \[ \diff y' = t e^t \diff t. \] Звідси \[y' = \int t e^t \diff t = e^t (t - 1) + C_2.\] Одержали диференціальне рівняння першого порядку \[ x = e^t + C_1, \quad y' = e^t (t - 1) + C_2.\] Використовуючи співвідношення $\diff y = y' \diff x$, запишемо \[ \diff y = (e^t (t - 1) + C_2) e^t \diff t.\] Звідси \[ y = \frac{e^{2t}(t-1)}{2} - \frac{e^{2t}}{4} + C_2 e^t + C_3.\] Остаточно загальний розв’язок має вигляд \[ x = e^t + C_1, \quad y = \frac{e^{2t}(2t - 3)}{4}+C_2e^t + C_3.\] Якщо вилучити параметр $t$, то одержимо загальний розв’язок \[ y = \frac{(x-C_1)^2}{4} \cdot \left( 2 \ln |x - C_1| - 3 \right) + C_2 (x - C_1) + C_3.\]
\end{solution}
\begin{example}
	Розв’язати рівняння: $3 \sqrt[3]{y} y'' = 1$.
\end{example}
\begin{solution}
	Запишемо рівняння у вигляді \[ y'' = \frac{1}{3\sqrt[3]{y}}. \] Помножимо обидві частини на $2 y' \diff x$. Одержимо \[ 2 y'' y' \diff x = \frac{2y'\diff x}{3\sqrt[3]{y}}, \] або \[ \diff (y')^2 = \frac{2\diff y}{3\sqrt[3]{y}}. \] Проінтегруємо і одержимо \[ (y')^2 = \sqrt[3]{y^2} + C_1.\] Звідси $y' = \pm \sqrt{y^{2/3} + C_1}$. Нехай початкові умови такі, що $C_1 = {\bar C_1}^2 > 0$. Тобто рівняння має вигляд \[ \frac{\diff y}{\diff x} = \pm \sqrt{y^{2/3} + {\bar C_1}^2}. \] Розділимо змінні \[ \int \frac{\diff y}{\sqrt{y^{2/3} + {\bar C_1}^2}} = \pm \int \diff x + C_2. \] Робимо заміну $\sqrt{y^{2/3} + {\bar C_1}^2} = t$. Тоді \[ y = \left(t^2 - {\bar C_1}^2\right)^{3/2}, \quad \diff y = 3 \left(t^2 - {\bar C_1}^2\right)^{1/2} t \diff t,\]	і інтеграл має вигляд \[ \int \frac{\diff y}{\sqrt{y^{2/3} + {\bar C_1}^2}}=3\int \sqrt{t^2 - {\bar C_1}^2} \diff t = 3.\]
\end{solution}
Розв’язати рівняння:
\begin{multicols}{2}
\begin{problem}
	\[ y'' \cdot x \cdot \ln x = y';\]
\end{problem}
\begin{problem}
	\[y''' = x + \cos x;\]
\end{problem}
\end{multicols}
\begin{problem}
	$2 x y'' = y'$ при $x_0 = 0$, $y_0 = 0$, $y_0' = 0$, $y_0'' = 0$;
\end{problem}
\begin{problem}
	$x y'' + y' = x + 1$ при $x_0 = 0$, $y_0 = 0$, $y_0' = 0$, $y_0'' = 0$;
\end{problem}
\begin{problem}
	$\tan x \cdot y'' = y' + \frac{1}{\sin x} = 0$ при $x_0 = 0$, $y_0 = 2$, $y_0' = 1$, $y_0'' = 1$;
\end{problem}
\begin{multicols}{2}
\begin{problem}
	\[(y'')^4+y''-x=0;\]
\end{problem}
\begin{problem}
	\[y''+\ln y''-x=0;\]
\end{problem}
\begin{problem}
	\[y''-a\cdot(1+(y')^2)^{3/2}=0;\]
\end{problem}
\begin{problem}
	\[y'''-(y'')^3;\]
\end{problem}
\begin{problem}
	\[y'''-y''=0;\]
\end{problem}
\begin{problem}
	\[y''+2y''\cdot\ln y'-1=0;\]
\end{problem}
\begin{problem}
	\[(y''')^2+(y'')^2-1=0;\]
\end{problem}
\begin{problem}
	\[y''\cdot y^3-1=0;\]
\end{problem}
\begin{problem}
	\[y^3\cdot y''-y^4+0;\]
\end{problem}
\begin{problem}
	\[4\sqrt{y}\cdot y''=1;\]
\end{problem}
\begin{problem}
	\[3y''=y^{-5/3};\]
\end{problem}
\begin{problem}
	\[(y'')^2+(y')^2-(y')^4=0;\]
\end{problem}
\end{multicols}
\begin{example}
	Розв’язати рівняння: $(y'')^3 + x \cdot y'' = y'$.
\end{example}
\begin{solution}
	Позначимо $y'=z$, $y''=z'$. Одержимо рівняння $(z')^3+xz'=z$, тобто рівняння Клеро, що легко інтегрується введенням параметра. \\

	Нехай $z' = p$. Тоді $z=xp+p^3$. Продиференцюємо це співвідношення: \[\diff z + x \diff p + p \diff x + 3 p^2 \diff p.\] Підставивши $\diff z = p \diff z$, отримаємо $(x+3p^2)\diff p = 0$. Це рівняння розділяється на два:
	\begin{enumerate}
		\item $x + 3p^2 = 0$. Звідси маємо $x = -3p^2$, $z=-2p^2$. Повертаємось до вихідних змінних $x=-3p^2$, $y'=-2p^3$. Використовуємо основне співвідношення $\diff y = y' \diff x$. Одержуємо \[ \diff y = 12 p^4 \diff p \implies y = \frac{12p^5}{5} + C_1.\] Таким чином перша гілка дає розв’язок \[ x = -3p^2, \quad y=\frac{12p^5}{5} + C_1.\]
		\item $\diff p = 0$. Звідси маємо $z = C_1 x + C_1^3$. Повертаємось до вихідних змінних $y'=C_1x+C_1^3$. Проінтегруємо і отримаємо другу гілку розв’язків \[y=\frac{C_1x^2}{2}+C_1^3x+C_2.\]
	\end{enumerate}
\end{solution}
\begin{example}
	Розв’язати рівняння: $y^4 - y^3 \cdot y'' = 1$.
\end{example}
\begin{solution}
	Відсутній аргумент $x$, отже, його порядок знижується заміною: \[y'=p,\quad y''=p\cdot\frac{\diff p}{\diff y}.\] Звідси одержуємо \[ y^4 - y^3 \cdot p \cdot \frac{\diff p}{\diff y} = 1.\] Розділимо змінні: \[ \frac{y^4-1}{y^3} \cdot \diff y=p \cdot \diff p.\] Проінтегруємо \[ \frac{y^2}{2}+\frac{1}{2y^2}=\frac{p^2}{2}-\frac{C_1}{2}.\] Звідси одержали \[p^2=y^2+C_1+y^{-2}.\] Повертаємось до вихідних змінних \[(y')^2=y^2+C_1+y^{-2}.\] Розв’яжемо рівняння відносно похідної \[ y' = \pm \sqrt{y^2 + C_1 + y^{-2}}.\] Розділимо змінні \[ \pm\frac{\diff y}{\sqrt{y^2+C_1+y^{-2}}}=\diff x.\] Візьмемо інтеграл \begin{multline*} \pm \int \frac{\diff y}{\sqrt{y^2+C_1+y^{-2}}} = \pm \int \frac{y \diff y}{\sqrt{y^4+C_1y^2+1}} = \\ =\pm \int \frac{\diff \left(y^2 + \frac{C_1}{2}\right)}{\sqrt{\left(y^2+\frac{C_1}{2}\right)^2+\left(1-\frac{C_1}{4}\right)}} = \pm \frac12 \ln \left| y^2 + \frac{C_1}{2} + \sqrt{y^4+C_1y^2+1}\right|. \end{multline*} Таким чином загальний розв’язок має вигляд: \[x = \pm \frac12 \ln \left| y^2 + C_1/2 + \sqrt{y^4+C_1y^2+1}\right|.\]
\end{solution}
\begin{example}
	Розв’язати рівняння: $y \cdot y'' = (y')^2$.
\end{example}
\begin{solution}
	Оскільки рівняння однорідне по змінним $y, y', y''$, то робимо заміну \[ y = e^{\int u \diff x}, \quad y' = e^{\int u \diff x} u, \quad y'' = e^{\int u \diff x} (u^2 +u). \] Рівняння буде мати вигляд \[ e^{\int u \diff x} \cdot e^{\int u \diff x} (u^2 + u) = \left( e^{\int u \diff x} u\right)^2.\] Скоротимо на $e^{\int u \diff x}$. Маємо $u^2 + u' = u^2$, або $u'=0$. Звідси $u'=C_1$ і одержимо загальний розв’язок \[ y = e^{\int C_1 x \diff x} = e^{c_1 x^2 + \ln |c_2|}=c_2 e^{c_1x^2}.\]
\end{solution}
\begin{example}
	Розв’язати рівняння: $y \cdot y'' - (y')^2 = y^2$.
\end{example} 
\begin{solution}
	Розділимо рівняння на $y^2$: \[ \frac{y \cdot y'' - (y')^2}{y^2} = 1,\] і перепишемо у вигляді: \[ \frac{\diff}{\diff x} \left( \frac{y'}{y}\right)=1.\] Проінтегрувавши, одержимо загальний розв’зок \[\frac{y'}{y}=x+C_1\implies \ln |y| = \frac{x^2}{2}+C_1x+\ln |C_2|\implies y=C_2 e^{x^2/2+C_1x}.\]
\end{solution}
Розв’язати рівняння:
\begin{multicols}{2}
\begin{problem}
	\[x \cdot y'' = y' \cdot \ln (y'/x);\]
\end{problem}
\begin{problem}
	\[ 2 y \cdot y'' - 3 (y')^2 = 4 y^2;\]
\end{problem}
\begin{problem}
	\[2x \cdot y'' = y';\]
\end{problem}
\begin{problem}
	\[x \cdot y'' + y' = x + 1;\]
\end{problem}
\begin{problem}
	\[\tan x \cdot y'' - y' + \frac{1}{\sin x}=0;\]
\end{problem}
\begin{problem}
	\[x^2\cdot y''+x\cdot y'=1;\]
\end{problem}
\begin{problem}
	\[y'' \cdot \cot (2x) + 2y' = 0;\]
\end{problem}
\begin{problem}
	\[x^3 \cdot y'' + x^2 \cdot y = 0;\]
\end{problem}
\begin{problem}
	\[\tan x \cdot y'' = 2 y';\]
\end{problem}
\begin{problem}
	\[y \cdot y'' - (y')^2 - y^2 \cdot \ln y = 0;\]
\end{problem}
\begin{problem}
	\[x^4\cdot y''+x^3\cdot y'=1;\]
\end{problem}
\begin{problem}
	\[x\cdot y\cdot y''-x \cdot (y')^2 - 2 y \cdot y' = 0;\]
\end{problem}
\begin{problem}
	\[x\cdot y\cdot y''-x \cdot (y')^2 - y \cdot y'+\frac{x \cdot (y')^2}{\sqrt{1-x^2}}=0;\]
\end{problem}
\begin{problem}
	\[x^2\cdot y'''-x \cdot (y'')^2=0;\]
\end{problem}
\begin{problem}
	\[x^5\cdot y''+x^4 \cdot y'=1.\]
\end{problem}
\end{multicols}

\section{Лі\-ній\-ні ди\-фе\-рен\-ці\-аль\-ні рів\-ня\-н\-ня ви\-щих \allowbreak по\-ряд\-ків}
Рівняння вигляду
\begin{equation}
	\label{eq:3.1}
	a_0(x) \cdot y^{(n)} + a_1(x) \cdot y^{(n - 1)} + \ldots + a_n(x) \cdot y = b(x)
\end{equation} 
називається лінійним неоднорідним диференціальним рівнянням $n$-го порядку. \\

Рівняння вигляду
\begin{equation}
	\label{eq:3.2}
	a_0(x) \cdot y^{(n)} + a_1(x) \cdot y^{(n - 1)} + \ldots + a_n(x) \cdot y = 0
\end{equation} 
називається лінійним однорідним диференціальним рівнянням $n$-го порядку. \\

Якщо при $x \in [a, b]$, $a_0(x) \ne 0$ коефіцієнти $b(x)$, $a_i(x)$, $i=\overline{0,n}$ неперервні, то для рівняння
\begin{equation}
	\label{eq:3.3}
	y^{(n)} = - \frac{a_1(x)}{a_0(x)} \cdot y^{(n - 1)} - \ldots - \frac{a_n(x)}{a_0(x)} \cdot y + \frac{b(x)}{a_0(x)}.
\end{equation}
виконуються умови теореми існування та єдиності і існує єдиний розв’язок $y = y(x)$, що задовольняє початковим умовам
\begin{equation}
	\label{eq:3.4}
	y(x_0) = y_0, \quad y'(x_0) = y_0', \quad \ldots, \quad y^{(n - 1)} = y_0^{(n - 1)}.
\end{equation}

\subsection{Лінійні однорідні рівняння}

\subsubsection{Властивості лінійних однорідних рівнянь}

\begin{theorem}
	Лінійність і однорідність зберігаються при довільному перетворенні незалежної змінної $x = \phi(t)$.
\end{theorem}
\begin{proof}
	Справді, після заміни $x = \phi(t)$, одержимо
	\begin{align}
		\label{eq:3.1.1}
		y_x' &= \frac{\diff y}{\diff x} = \frac{\diff y}{\diff t} \cdot \frac{\diff t}{\diff x} = \frac{1}{\phi'(t)} \cdot \frac{\diff y}{\diff t}, \\
		\label{eq:3.1.2}
		y_{x^2}'' &= \frac{\diff}{\diff x} \cdot y_x' = \frac{\diff}{\diff t} \left( \frac{1}{\phi'(t)} \cdot \frac{\diff y}{\diff t} \right) \cdot \frac{1}{\phi'(t)} = \\
		&= - \frac{\phi''(t)}{(\phi'(t))^2} \cdot \frac{\diff y}{\diff t} + \frac{1}{(\phi'(t))^2} \cdot \frac{\diff^2 y}{\diff t^2}, \nonumber
	\end{align}
	і так далі до $n$-го порядку. Після підстановки і приведення подібних, знову отримуємо лінійне однорідне рівняння
	\begin{equation}
		\label{eq:3.1.3}
		A_0(t) \cdot \frac{\diff^n y}{\diff t^n} + A_1(t) \cdot \frac{\diff^{n - 1} y}{\diff t^{n - 1}} + \ldots + A_n(t) \cdot y = 0.
	\end{equation}
\end{proof}

\begin{theorem}
	Лінійність і однорідність зберігаються при лінійному перетворенні невідомої функції $y = \alpha (x) \cdot z$.
\end{theorem}
\begin{proof}
	Справді, після заміни $y = \alpha (x) \cdot z$, одержимо
	\begin{align}
		\label{eq:3.1.4}
		y_x' &= \alpha'(x) \cdot z + \alpha(x) \cdot z', \\
		\label{eq:3.1.5}
		y_{x^2}'' &= \alpha''(x) \cdot z + 2 \alpha'(x) \cdot z' + \alpha(x) \cdot z'',
	\end{align}
	і так далі до $n$-го порядку. Після підстановки знову отримаємо лінійне однорідне рівняння
	\begin{equation}
		\label{eq:3.1.6}
		\bar A_0(x) \cdot z^{(n)} + \bar A_1(x) \cdot z^{(n - 1)} + \ldots + \bar A_n(x) \cdot z = 0.
	\end{equation}
\end{proof}

	\subsection{Лінійні однорідні рівняння}
	\input{subsection-3-1.tex}

		\subsubsection{Властивості лінійних однорідних рівнянь}
		\begin{theorem}
	Лінійність і однорідність зберігаються при довільному перетворенні незалежної змінної $x = \phi(t)$.
\end{theorem}
\begin{proof}
	Справді, після заміни $x = \phi(t)$, одержимо
	\begin{align*}
		%\label{eq:3.1.1}
		y_x' &= \frac{\diff y}{\diff x} = \frac{\diff y}{\diff t} \cdot \frac{\diff t}{\diff x} = \frac{1}{\phi'(t)} \cdot \frac{\diff y}{\diff t}, \\
		%\label{eq:3.1.2}
		y_{x^2}'' &= \frac{\diff}{\diff x} \cdot y_x' = \frac{\diff}{\diff t} \left( \frac{1}{\phi'(t)} \cdot \frac{\diff y}{\diff t} \right) \cdot \frac{1}{\phi'(t)} = \\
		&= - \frac{\phi''(t)}{(\phi'(t))^2} \cdot \frac{\diff y}{\diff t} + \frac{1}{(\phi'(t))^2} \cdot \frac{\diff^2 y}{\diff t^2}, \nonumber
	\end{align*}
	і так далі до $n$-го порядку. Після підстановки і приведення подібних, знову отримуємо лінійне однорідне рівняння
	\begin{equation*}
		%\label{eq:3.1.3}
		A_0(t) \cdot \frac{\diff^n y}{\diff t^n} + A_1(t) \cdot \frac{\diff^{n - 1} y}{\diff t^{n - 1}} + \ldots + A_n(t) \cdot y = 0.
	\end{equation*}
\end{proof}

\begin{theorem}
	Лінійність і однорідність зберігаються при лінійному перетворенні невідомої функції $y = \alpha (x) \cdot z$.
\end{theorem}
\begin{proof}
	Справді, після заміни $y = \alpha (x) \cdot z$, одержимо
	\begin{align*}
		%\label{eq:3.1.4}
		y_x' &= \alpha'(x) \cdot z + \alpha(x) \cdot z', \\
		%\label{eq:3.1.5}
		y_{x^2}'' &= \alpha''(x) \cdot z + 2 \alpha'(x) \cdot z' + \alpha(x) \cdot z'',
	\end{align*}
	і так далі до $n$-го порядку. Після підстановки знову отримаємо лінійне однорідне рівняння
	\begin{equation*}
		%\label{eq:3.1.6}
		\bar A_0(x) \cdot z^{(n)} + \bar A_1(x) \cdot z^{(n - 1)} + \ldots + \bar A_n(x) \cdot z = 0.
	\end{equation*}
\end{proof}

		\subsubsection{Властивості роз\-в'яз\-ків лінійних однорідних рівнянь}
		\begin{theorem}
	Якщо $y = y_1(x)$ є розв'язком однорідного лінійного рівняння, то і $y = C y_1 (x)$, де $C$ --- довільна стала, теж буде розв'язком однорідного лінійного рівняння.
\end{theorem}

\begin{proof}
	Справді, нехай $y = y_1(x)$ --- розв'язок лінійного однорідного рівняння, тобто
	\begin{equation*}
		a_0(x) y_1^{(n)} (x) + a_1(x) y_1^{(n - 1)} (x) + \ldots + a_n(x) y_1(x) \equiv 0.
	\end{equation*}

	Тоді і
	\begin{multline*}
		a_0(x) (C y_1)^{(n)}(x) + a_1(x) (C y_1)^{(n - 1)}(x) + \ldots + a_n(x) (C y_1)(x) = \\
		= C \left( a_0(x) y_1^{(n)} (x) + a_1(x) y_1^{(n - 1)} (x) + \ldots + a_n(x) y_1(x) \right) \equiv 0,
	\end{multline*}
	оскільки вираз в дужках дорівнює нулю.
\end{proof}

\begin{theorem}
	Якщо $y_1(x)$ і $y_2(x)$ є розв'язками лінійного однорідного рівняння, то і $y = y_1(x) + y_2(x)$ теж буде розв'язком лінійного однорідного рівняння.
\end{theorem}

\begin{proof}
	Справді, нехай $y_1(x)$ і $y_2(x)$ --- розв'язки лінійного рівняння, тобто
	\begin{align*}
		a_0(x) y_1^{(n)} (x) + a_1(x) y_1^{(n - 1)} (x) + \ldots + a_n(x) y_1(x) &\equiv 0, \\
		a_0(x) y_2^{(n)} (x) + a_1(x) y_2^{(n - 1)} (x) + \ldots + a_n(x) y_2(x) &\equiv 0.
	\end{align*}

	Тоді і
	\begin{multline*}
		a_0(x) (y_1 + y_2)^{(n)} (x) + a_1(x) (y_1 + y_2)^{(n - 1)} (x) + \ldots + a_n(x) (y_1 + y_2) (x) = \\
		= \left( a_0(x) y_1^{(n)} (x) + a_1(x) y_1^{(n - 1)} (x) + \ldots + a_n(x) y_1(x) \right) + \\
		+ \left( a_0(x) y_2^{(n)} (x) + a_1(x) y_2^{(n - 1)} (x) + \ldots + a_n(x) y_2(x) \right) \equiv 0,
	\end{multline*}
	оскільки обидві дужки дорівнюють нулю.
\end{proof}

\begin{theorem}
	Якщо $y_1(x), y_2(x), \ldots, y_n(x)$ --- розв'язки однорідного лінійного рівняння, то і  $y = \sum_{i=1}^n C_i y_i(x)$, де $C_i$ --- довільні сталі, також буде розв'язком лінійного однорідного рівняння.
\end{theorem}

\begin{proof}
	Справді, нехай $y_1(x), y_2(x), \ldots, y_n(x)$ --- розв'язки лінійного однорідного рівняння, тобто
	\begin{equation*}
		a_0(x) y_i^{(n)} (x) + a_1(x) y_i^{(n - 1)} (x) + \ldots + a_n(x) y_i(x) \equiv 0, \quad i = \overline{1, n}.
	\end{equation*}
	
	Тоді і   
 	\begin{multline*}
 		a_0(x) \left(\sum_{i=1}^n C_i y_i\right)^{(n)} (x) + a_1(x) \left(\sum_{i=1}^n C_i y_i\right)^{(n - 1)} (x) + \ldots \\
 		\ldots + a_{n-1}(x) \left(\sum_{i=1}^n C_i y_i\right)'(x) + a_n(x) \left(\sum_{i=1}^n C_i y_i\right)(x) = \\
 		= \sum_{i=1}^n C_i  \left( a_0(x) y_i^{(n)} (x) + a_1(x) y_i^{(n - 1)} (x) + \ldots + a_n(x) y_i(x) \right) \equiv 0,
 	\end{multline*}
	оскільки кожна дужка дорівнює нулю.
\end{proof}

\begin{theorem}
	Якщо комплексна функція дійсного аргументу, тобто $y = u(x) + i v(x)$ є розв'язком лінійного однорідного рівняння, то окремо дійсна частина $u(x)$ і уявна $v(x)$ будуть також розв'язками цього рівняння.
\end{theorem}

\begin{proof}
	Справді, нехай $y = u(x) + i v(x)$ є розв'язком лінійного однорідного рівняння, тобто
	\begin{multline*}
		a_0(x) (u) + i v)^{(n)} (x) + a_1(x) (u + i v)^{(n - 1)} (x) + \ldots \\
		\ldots + a_{n - 1}(x) (u + i v)' (x) + a_n(x) (u + i v) (x) \equiv 0.
	\end{multline*}

	Розкривши дужки і перегрупувавши члени, одержимо
	\begin{multline*}
		\left( a_0(x) u^{(n)}(x) + a_1(x) u^{(n - 1)} (x) + \ldots + a_n(x) u(x) \right) + \\
		+ i \left( a_0(x) v^{(n)}(x) + a_1(x) v^{(n - 1)} (x) + \ldots + a_n(x) v(x) \right) \equiv 0.
	\end{multline*}

	Комплексний вираз дорівнює нулю тоді і тільки тоді, коли дорівнюють нулю дійсна і уявна частини, тобто
 	\begin{align*}
		a_0(x) u^{(n)}(x) + a_1(x) u^{(n - 1)} (x) + \ldots + a_n(x) u(x) &\equiv 0, \\
		a_0(x) v^{(n)}(x) + a_1(x) v^{(n - 1)} (x) + \ldots + a_n(x) v(x) &\equiv 0,
	\end{align*}
	або функції $u(x)$, $v(x)$ є розв'язками рівняння, що і було потрібно довести.
\end{proof}


		\subsubsection{Лінійна залежність і незалежність роз\-в'яз\-ків. Загальний роз\-в'яз\-ок лінійного однорідного рівняння вищого порядку}
		\begin{definition}
	Функції $y_0(x), y_1(x), \ldots, y_n(x)$ називаються лінійно залежними на відрізку $[a,b]$ якщо існують не всі рівні нулю сталі $C_0, \ldots, C_n$ такі, що при всіх $x \in [a,b]$: 
	\begin{equation}
		\label{eq:3.1.18}
		C_0 \cdot y_0(x) + C_1 \cdot y_1(x) + \ldots + C_n \cdot y_n(x) = 0.
	\end{equation}

	Якщо ж тотожність справедлива лише коли $C_0 = C_1 = \ldots = C_n = 0$, то функції $y_1(x), y_2(x), \ldots, y_n(x)$ називаються лінійно незалежними.
\end{definition}

\textbf{Приклади:}
\begin{enumerate}
	\item Функції $1, x, x^2, \ldots, x^n$ -- лінійно незалежні на будь-якому відрізку $[a,b]$, тому що вираз $C_0 + C_1 x + \ldots + C_n x^n$ є многочленом ступеню $n$ і має не більш, ніж $n$ дійсних коренів.
	\item Функції $e^{\lambda_1 x}, e^{\lambda_2 x}, \ldots, e^{\lambda_n x}$, де всі $\lambda_i$ -- дійсні різні числа -- лінійно незалежні. 
	\item Функції $1, \sin x, \cos x, \ldots, \sin nx, \cos nx$ -- лінійно незалежні.
\end{enumerate}

\begin{theorem}[необхідна умова лінійної незалежності функцій]
	Якщо функції $y_0(x), y_1(x), \ldots, y_n(x)$ -- лінійно залежні, то визначник Вронського $W[y_0, y_1, \ldots, y_n](x)$ тотожно дорівнює нулю при всіх $x \in [a,b]$:
	\begin{equation}
		\label{eq:3.1.19}
		W[y_0, y_1, \ldots, y_n](x) = \begin{vmatrix} y_0(x) & y_1(x) & \cdots & y_n(x) \\ y_0'(x) & y_1'(x) & \cdots & y_n'(x) \\ \vdots & \vdots & \ddots & \vdots \\ y_0^{(n)}(x) & y_1^{(n)}(x) & \cdots & y_n^{(n)}(x) \end{vmatrix} = 0.
	\end{equation}
\end{theorem}

\begin{proof}
	Нехай $y_0(x), y_1(x), \ldots, y_n(x)$ -- лінійно залежні. Тоді існують не всі рівні нулю сталі $C_0, \ldots, C_n$ такі, що при $x \in [a,b]$ буде тотожно виконуватися: \eqref{eq:3.1.18}.	Продиференціювавши $n$ разів, одержимо 
	\begin{equation}
		\label{eq:3.1.20}
		\left\{ \begin{aligned}
			C_0 \cdot y_0(x) + C_1 \cdot y_1(x) + \ldots + C_n \cdot y_n(x) &= 0, \\
			C_0 \cdot y_0'(x) + C_1 \cdot y_1'(x) + \ldots + C_n \cdot y_n(x) &= 0, \\
			\ldots \ldots \ldots \ldots \ldots \ldots \ldots \ldots \ldots \ldots \ldots \ldots \ldots & \ldots \ldots \\
			C_0 \cdot y_0^{(n)}(x) + C_1 \cdot y_1^{(n)}(x) + \ldots + C_n \cdot y_n^{(n)}(x) &= 0.
		\end{aligned} \right.
	\end{equation}
 
	Для кожного фіксованого $x \in [a,b]$ одержимо лінійну однорідну систему алгебраїчних рівнянь, що має ненульовий розв’язок $C_0, \ldots, C_n$. А це можливо тоді і тільки тоді, коли визначник системи дорівнює нулю, тобто $W[y_0, y_1, \ldots, y_n](x) = 0$ при всіх $x \in [a,b]$.
\end{proof}

\begin{theorem}[достатня умова лінійної незалежності розв’язків]
	Якщо розв’язки лінійного однорідного рівняння $y_0(x), y_1(x), \ldots, y_n(x)$ -- лінійно незалежні, то визначник Вронського $W[y_0, y_1, \ldots, y_n](x)$ не дорівнює нулю в жодній точці $x \in [a,b]$.
\end{theorem} 

\begin{proof}
	Припустимо, від супротивного, що існує $x_0 \in [a,b]$, при якому $W[y_0, y_1, \ldots, y_n](x_0) = 0$. Оскільки визначник дорівнює нулю, то існує ненульовий розв’язок $C_0^0, C_1^0, \ldots, C_n^0$ лінійної однорідної системи алгебраїчних рівнянь \eqref{eq:3.1.19}. Розглянемо лінійну комбінацію 
	\begin{equation}
		\label{eq:3.1.21}
		y(x) = C_0^0 y_0(x) + C_1 y_1(x) + \ldots + C_n y_n(x)
	\end{equation}
	з отриманими коефіцієнтами. \\

	У силу третьої властивості ця комбінація буде розв’язком. У силу вибору сталих $C_0^0, C_1^0, \ldots, C_n^0$, розв’язок буде задовольняти умовам
	\begin{equation}
		\label{eq:3.1.22}
		y(x_0) = y'(x_0) = \ldots = y^{(n)}(x_0) = 0.
	\end{equation}
 
	Але цим же умовам, як неважко перевірити простою підстановкою, задовольняє і тотожний нуль, тобто $y \equiv 0$. І в силу теореми існування та єдиності ці два розв’язки співпадають, тобто 
	\begin{equation}
		\label{eq:3.1.23}
		y(x) = C_0^0 y_0(x) + C_1 y_1(x) + \ldots + C_n y_n(x) = 0
	\end{equation}
	при $x \in [a,b]$, або система функцій $y_0(x), y_1(x), \ldots, y_n(x)$ лінійно залежна, що суперечить припущенню. Таким чином $W[y_0, y_1, \ldots, y_n](x_0) \ne 0$ у жодній точці $x_0 \in [a,b]$, що і було потрібно довести .
\end{proof}

На підставі попередніх двох теорем сформулюємо необхідні і достатні умови лінійної незалежності розв’язків лінійного однорідного рівняння.

\begin{theorem}
	Для того щоб розв’язки лінійного однорідного диференціального рівняння $y_0(x), y_1(x), \ldots, y_n(x)$ були лінійно незалежними, необхідно і достатньо, щоб визначник Вронського не дорівнював нулю в жодній точці $x \in [a,b]$, тобто $W[y_0, y_1, \ldots, y_n](x) \ne 0$.
\end{theorem}

\begin{theorem}
	Загальним розв’язком лінійного однорідного рівняння
	\begin{equation}
		\label{eq:3.1.24}
		a_0(x) \cdot y^{(n)} + a_1(x) \cdot y^{(n-1)} + \ldots + a_{n-1}(x) \cdot y' + a_n \cdot y = 0
	\end{equation}
 	є лінійна комбінація $n$ лінійно незалежних розв’язків $y = \sum_{i = 1}^n C_i \cdot y_i(x)$.
\end{theorem}

\begin{proof}
	Оскільки $y_i(x)$, $i = 1, 2, \ldots, n$ є розв’язками, то в силу третьої властивості їхня лінійна комбінація також буде розв’язком. \\

	Покажемо, що цей розв’язок загальний, тобто вибором сталих $C_1, \ldots, C_n$ можна розв’язати довільну задачу Коші
	\begin{equation}
		\label{eq:3.1.25}
		y(x_0) = y_0, \quad y'(x_0) = y_0', \quad \ldots, \quad y^{(n - 1)}(x_0) = y_0^{(n - 1)}.
	\end{equation}

	Дійсно, оскільки система розв’язків лінійно незалежна, то визначник Вронського відмінний від нуля й алгебраїчна система неоднорідних рівнянь
	\begin{equation}
		\label{eq:3.1.26}
		\left\{ \begin{aligned}
			C_1 \cdot y_1(x_0) + C_2 \cdot y_2(x_0) + \ldots + C_n \cdot y_n(x_0) &= y_0, \\
			C_1 \cdot y_1'(x_0) + C_2 \cdot y_2'(x_0) + \ldots + C_n \cdot y_n(x_0) &= y_0', \\
			\ldots \ldots \ldots \ldots \ldots \ldots \ldots \ldots \ldots \ldots \ldots \ldots \ldots \ldots & \ldots \ldots \\
			C_1 \cdot y_1^{(n - 1)}(x_0) + C_2 \cdot y_2^{(n - 1)}(x_0) + \ldots + C_n \cdot y_n^{(n)}(x_0) &= y_0^{(n - 1)},
		\end{aligned} \right.
	\end{equation}
	має єдиний розв’язок $C_1^0, C_2^0, \ldots, C_n^0$. І лінійна комбінація $y = \sum_{i = 1}^n C_i^0 \cdot y_i(x)$ є розв’язком, причому, як видно із системи алгебраїчних рівнянь, буде задовольняти довільно обраним умовам Коші.
\end{proof}

Зауважимо, що максимальне число лінійно незалежних розв’язків дорівнює порядку рівняння. Це випливає з попередньої теореми, тому що будь-який розв’язок виражається через лінійну комбінацію $n$ лінійно незалежних розв’язків.

\begin{definition}
	Будь-які $n$ лінійно незалежних розв’язків лінійного однорідного рівняння $n$-го порядку називаються фундаментальною системою розв’язків.
\end{definition}

		\subsubsection{Формула Остроградського-Ліувіля}
		Оскільки максимальне число лінійно незалежних розв’язків дорівнює $n$, то система $y_1(x) , \ldots, y_n(x), y(x)$ буде залежною і $W[y_1,\ldots,y_n,y]\equiv0$, тобто
\begin{equation*}
	%\label{eq:3.1.27}
	\begin{vmatrix}
		y_1 & \cdots & y_n & y \\
		y_1' & \cdots & y_n' & y' \\
		\vdots & \ddots & \vdots & \vdots \\
		y_1^{(n)} & \cdots & y_n^{(n)} & y'
	\end{vmatrix} \equiv 0.
\end{equation*}
 
Розкладаючи визначник по елементах останнього стовпця, одержимо
 
\begin{multline}
	%\label{eq:3.1.28}
	\begin{vmatrix}
		y_1 & y_2 & \cdots & y_n \\
		y_1' & y_2' & \cdots & y_n' \\
		\vdots & \vdots & \ddots & \vdots \\
		y_1^{(n - 1)} & y_2^{(n - 1)} & \cdots & y_n^{(n - 1)} \\
	\end{vmatrix}  y^{(n)} 
	- 
	\begin{vmatrix}
		y_1 & y_2 & \cdots & y_n \\
		\vdots & \vdots & \ddots & \vdots \\
		y_1^{(n - 2)} & y_2^{(n - 2)} & \cdots & y_n^{(n - 2)} \\
		y_1^{(n)} & y_2^{(n)} & \cdots & y_n^{(n)}
	\end{vmatrix}  y^{(n - 1)} + \ldots \\
	\ldots + (-1)^{n - 1}
	\begin{vmatrix}
		y_1 & y_2 & \cdots & y_n \\
		y_1'' & y_2'' & \cdots & y_n'' \\
		\vdots & \vdots & \ddots & \vdots \\
		y_1^{(n)} & y_2^{(n)} & \cdots & y_n^{(n)}
	\end{vmatrix}  y'
	+ (-1)^n  
	\begin{vmatrix}
		y_1' & y_2' & \cdots & y_n' \\
		y_1'' & y_2'' & \cdots & y_n'' \\
		\vdots & \vdots & \ddots & \vdots \\
		y_1^{(n)} & y_2^{(n)} & \cdots & y_n^{(n)}
	\end{vmatrix}  y\equiv 0.
\end{multline}

Порівнюючи з рівнянням 
\begin{equation*}
	%\label{eq:3.1.29}
	a_0(x) \cdot y^{(n)} + a_1(x) \cdot y^{(n - 1)} + \ldots + a_n(x) \cdot y = 0
\end{equation*}
одержимо, що
\begin{equation*}
	%\label{eq:3.1.30}
	\frac{a_1(x)}{a_0(x)} = - \frac{\begin{vmatrix}
		y_1(x) & y_2(x) & \cdots & y_n(x) \\
		\vdots & \vdots & \ddots & \vdots \\
		y_1^{(n - 2)}(x) & y_2^{(n - 2)}(x) & \cdots & y_n^{(n - 2)}(x) \\
		y_1^{(n)}(x) & y_2^{(n)}(x) & \cdots & y_n^{(n)}(x)
	\end{vmatrix}}{W[y_1, y_2, \ldots, y_n](x)}.
\end{equation*}
Але оскільки
\begin{multline}
	%\label{eq:3.1.31}
	\frac{\diff}{\diff x} W[y_1, y_2, \ldots, y_n] = \begin{vmatrix}
		y_1' & y_2' & \cdots & y_n' \\
		y_1' & y_2' & \cdots & y_n' \\
		\vdots & \vdots & \ddots & \vdots \\
		y_1^{(n - 2)} & y_2^{(n - 2)} & \cdots & y_n^{(n - 2)} \\
		y_1^{(n - 1)} & y_2^{(n - 1)} & \cdots & y_n^{(n - 1)}
	\end{vmatrix} + \\ + \begin{vmatrix}
		y_1 & y_2 & \cdots & y_n \\
		y_1'' & y_2'' & \cdots & y_n'' \\
		\vdots & \vdots & \ddots & \vdots \\
		y_1^{(n - 2)} & y_2^{(n - 2)} & \cdots & y_n^{(n - 2)} \\
		y_1^{(n - 1)} & y_2^{(n - 1)} & \cdots & y_n^{(n - 1)}
	\end{vmatrix} + \ldots + \begin{vmatrix}
		y_1 & y_2 & \cdots & y_n \\
		y_1' & y_2' & \cdots & y_n' \\
		\vdots & \vdots & \ddots & \vdots \\
		y_1^{(n - 2)} & y_2^{(n - 2)} & \cdots & y_n^{(n - 2)} \\
		y_1^{(n)} & y_2^{(n)} & \cdots & y_n^{(n)}
	\end{vmatrix} = \\ = 0 + 0 + \ldots + \begin{vmatrix}
		y_1 & y_2 & \cdots & y_n \\
		y_1' & y_2' & \cdots & y_n' \\
		\vdots & \vdots & \ddots & \vdots \\
		y_1^{(n - 2)} & y_2^{(n - 2)} & \cdots & y_n^{(n - 2)}\\
		y_1^{(n)} & y_2^{(n)} & \cdots & y_n^{(n)}
	\end{vmatrix}
\end{multline}   
то, підставивши в попередній вираз, одержимо
\begin{equation*}
	%\label{eq:3.1.32}
	- \frac{a_1(x)}{a_0(x)} = \frac{\frac{\diff}{\diff x} W[y_1, y_2, \ldots, y_n](x)}{W[y_1, y_2, \ldots, y_n](x)}.
\end{equation*}
Розділимо змінні
\begin{equation*}
	%\label{eq:3.1.33}
	- \frac{a_1(x)}{a_0(x)} \diff x = \frac{\diff W[y_1, y_2, \ldots, y_n](x)}{W[y_1, y_2, \ldots, y_n](x)}.
\end{equation*}
Проінтегрувавши, одержимо
\begin{equation*}
	%\label{eq:3.1.33}
	\ln W[y_1, y_2, \ldots, y_n](x) - \ln W[y_1, y_2, \ldots, y_n](x_0) = -\int_{x_0}^x \frac{a_1(x)}{a_0(x)} \diff x
\end{equation*}
або
\begin{equation*}
	%\label{eq:3.1.33}
	W[y_1, y_2, \ldots, y_n](x) = W[y_1, y_2, \ldots, y_n](x_0) \cdot \exp \left\{ -\int_{x_0}^x \frac{a_1(x)}{a_0(x)} \diff x \right\}.
\end{equation*}
Отримана формула називається формулою Остроградського-Ліувілля. Зокрема, якщо рівняння має вид
\begin{equation*}
	%\label{eq:3.1.34}
	y^{(n)} + p_1 \cdot y^{(n - 1)} + \ldots + p_n(x) \cdot y = 0,
\end{equation*}
то формула запишеться у вигляді
\begin{equation*}
	%\label{eq:3.1.35}
	W[y_1, y_2, \ldots, y_n](x) = W[y_1, y_2, \ldots, y_n](x_0) \cdot \exp \left\{ -\int_{x_0}^x p_1(x) \diff x \right\}.
\end{equation*}


		\subsubsection{Формула Абеля}
		Розглянемо застосування формули Остроградського-Ліувіля до рівняння 2-го порядку
\begin{equation*}
	y'' + p_1(x) y' + p_2(x) y = 0.
\end{equation*}

Нехай $y_1(x)$ --- один з розв'язків. Тоді
\begin{equation*}
	\begin{vmatrix}
		y_1(x) & y(x) \\
		y_1'(x) & y'(x)
	\end{vmatrix} = C_2 \exp \left\{ - \int p_1(x) \diff x \right\}.
\end{equation*}

Розкривши визначник, одержимо
\begin{equation*}
	y_1(x) y'(x) - y(x) y_1'(x) = C_2 \exp \left\{ - \int p_1(x) \diff x \right\}.
\end{equation*}
 
Розділивши на $y_1^2(x)$, запишемо
\begin{equation*}
	\frac{y_1(x) y'(x) - y(x) y_1'(x)}{y_1^2(x)} = \frac{C_2}{y_1^2(x)} \exp \left\{ - \int p_1(x) \diff x \right\},
\end{equation*}
або
\begin{equation*}
	\frac{\diff}{\diff x} \left( \frac{y(x)}{y_1(x)} \right) = \frac{C_2}{y_1^2(x)} \exp \left\{ - \int p_1(x) \diff x \right\},
\end{equation*}

Проінтегрувавши, одержимо
\begin{equation*}
	\frac{y(x)}{y_1(x)} = C_2 \int \left( \frac{1}{y_1^2(x)} \exp \left\{ - \int p_1(x) \diff x \right\} \right) \diff x + C_1,
\end{equation*}

Остаточно
\begin{equation*}
	y(x) = C_1 y_1(x) + C_2 y_1(x) \int \left( \frac{1}{y_1^2(x)} \exp \left\{ - \int p_1(x) \diff x \right\} \right) \diff x,
\end{equation*}

Отримана формула називається формулою Абеля. Вона дозволяє по одному відомому роз\-в'яз\-ку знайти загальний роз\-в'яз\-ок однорідного лінійного рівняння другого порядку.


		\subsubsection{Вправи для самостійної роботи}
		Розв’язати лінійне однорядне диференціальне рівняння другого порядку, якщо відомий один розв’язок

\begin{example}
	$(x^2 + 1) \cdot y'' - 2 x \cdot y' + 2 y = 0$, $y_1(x) = x$.
\end{example}
\begin{solution}
	За формулою Абеля маємо
	\begin{multline*}
		y_2(x) = x \cdot \int \left(\frac{1}{x} \exp \left\{ \int \frac{2x\diff x}{x^2+1} \right\} \right) \diff x = x \cdot \int \left(\frac{1}{x} e^{\ln |x^2 + 1|} \right) \diff x = \\ = x \cdot \int \left(\frac{x^2 + 1}{x} \right) \diff x = x \cdot \left( x - \frac1x \right) = x^2 - 1.
	\end{multline*}
	Загальний розв’язок має вигляд \[ y(x) = C_1 \cdot x + C_2 \cdot (x^2 - 1).\]
\end{solution}

Розв’язати рівняння: 
\begin{problem}
	\[x^2\cdot(x+1)\cdot y''-2y=0,\quad y_1(x)=1+\frac1x;\]
\end{problem}
\begin{problem}
	\[x\cdot y''+2y'-x\cdot y=0,\quad y_1(x)=\frac{e^x}{x};\]
\end{problem}
\begin{problem}
	\[y''-2\cdot(1+\tan^2(x))\cdot y=0,\quad y_1(x)=\tan x;\]
\end{problem}
\begin{problem}
	\[(e^x+1)\cdot y''-2y'+e^x\cdot y=0,\quad y_1(x)=e^x-1;\]
\end{problem}
\begin{problem}
	\[y''-y'\cdot\tan x+2y=0,\quad y_1(x)=\sin x;\]
\end{problem}
\begin{problem}
	\[y''+4x\cdot y'+(4x^2+2)\cdot y=0,\quad y_1(x)=e^{a x^2}.\]
\end{problem}

Знайти загальний розв’язок підібравши один частинний
\begin{problem}
	\[(2x+1)\cdot y''+4x\cdot y'-4y=0;\]
\end{problem}
\begin{problem}
	\[x\cdot y''-(2x+1)\cdot y'+(x+1)\cdot y=0;\]
\end{problem}
\begin{problem}
	\[x\cdot(x-1)\cdot y''-x\cdot y'+y=0.\]
\end{problem}

	\subsection{Лінійні однорідні рівняння зі сталими коефіцієнтами}
	
		\subsubsection{Загальна теорія}
		Розглянемо лінійні однорідні диференціальні рівняння з сталими коефіцієнтами
\begin{equation*}
	y^{(n)} + a_1 y^{(n - 1)} + \ldots + a_n y = 0
\end{equation*}

Розв'язок будемо шукати у вигляді $y = e^{\lambda x}$. Продиференціювавши, одержимо 
\begin{equation*}
 	y' = \lambda e^{\lambda x}, \quad y'' = \lambda^2 e^{\lambda x}, \quad \ldots \quad y^{(n)} = \lambda^n e^{\lambda x}.
\end{equation*}

Підставивши $y', y'', \ldots, y^{(n)}$ в диференціальне рівняння, отримаємо
\begin{equation*}
	\lambda^n e^{\lambda x} + a_1 \lambda^{n - 1}e^{\lambda x} + \ldots + a_n e^{\lambda x} = 0.
\end{equation*}

Скоротивши на $e^{\lambda x}$, одержимо характеристичне рівняння
\begin{equation*}
	\lambda^n + a_1 \lambda^{n - 1} + \ldots + a_n = 0.
\end{equation*}

Алгебраїчне рівняння $n$-го степеня має $n$ коренів. У залежності від їхнього вигляду будемо мати різні розв'язки.
\begin{enumerate}
    \item Нехай $\lambda_1, \lambda_2, \ldots, \lambda_n$ --- дійсні і різні. Тоді функції $e^{\lambda_1 x}, e^{\lambda_2 x}, \ldots, e^{\lambda_n x}$ є розв'язками й оскільки всі $\lambda_i$ різні, то $e^{\lambda_i x}$ --- розв'язки лінійно незалежні, тобто $\left\{ e^{\lambda_i x} \right\}_{i = 1}^n$ фундаментальна система розв'язків. Загальним розв'язком буде лінійна комбінація $y = \sum_{i=1}^n C_i e^{\lambda_i x}$.

    \item Нехай маємо комплексно спряжені корені $\lambda=p+iq$, $\bar\lambda=p-iq$. Їм відповідають розв'язки $e^{(p+iq)x}$, $e^{(p-iq)x}$ . Розкладаючи їх по формулі Ейлера, одержимо: 
    \begin{align*}
    	e^{(p+iq)x} &= e^{px} e^{iqx} = e^{px} (\cos qx + i \sin qx) = u(x) + i v(x), \\
    	e^{(p-iq)x} &= e^{px} e^{-iqx} = e^{px} (\cos qx - i \sin qx) = u(x) - i v(x).
    \end{align*}
    
    І, як випливає з властивості 4, функції $u(x)$ й $v(x)$ будуть окремими розв'язками. Таким чином, кореням $\lambda = p + iq$, $\bar\lambda = p - iq$ відповідають два лінійно незалежних розв'язки $u = e^{px} \cos qx$, $v = e^{px} \sin qx$. Загальним розв'язком, що відповідає цим двом кореням, буде $y = C_1 e^{px} \cos qx + C_2 e^{px} \sin x$.
    
    \item Нехай $\lambda$ --- кратний корінь, кратності $k$, тобто $\lambda_1 = \lambda_2 = \ldots = \lambda_k$, $k\le n$.
    
    \begin{enumerate}
        \item Розглянемо випадок $\lambda=0$. Тоді характеристичне рівняння вироджується в рівняння
        \begin{equation*}
        	\lambda^n + a_1 \lambda^{n - 1} + \ldots + a_{n - k} \lambda^k = 0.
        \end{equation*}
         
        Диференціальне рівняння, що відповідає цьому характеристичному, запишеться у вигляді
        \begin{equation*}
        	y^{(n)} + a_1 y^{(n - 1)} + \ldots + a_{n-k} y^{(k)} = 0
        \end{equation*}
        
        Неважко бачити, що частковими, лінійно незалежними роз\-в'я\-з\-ка\-ми цього рівняння, будуть функції $1, x, x^2, \ldots, x^{k-1}$. Загальним роз\-в'я\-з\-ком, що відповідає кореню $\lambda=0$ кратності $k$, буде лінійна комбінація цих функцій $y = C_1 + C_2 x + \ldots + C_k x^{k - 1}$.
        
        \item Нехай $\lambda = \nu \ne 0$ --- корінь дійсний. Зробивши заміну $y = e^{\nu x} z$, на підставі властивості 2 лінійних рівнянь після підстановки знову одержимо лінійне однорідне диференціальне рівняння 
        \begin{equation*}
        	z^{(k)} + b_1 z^{(k-1)} + \ldots + b_k z = 0.
        \end{equation*}
        
        Причому, оскільки $y_i(x) = e^{\lambda_i x}$ а $x_i(x) = e^{\mu_i x}$, то показники $\lambda_i$, $\mu_i$ зв'язані співвідношенням $\lambda_i = \nu + \mu_i$. Звідси кореню $\lambda = \nu$ кратності $k$ відповідає корінь $\mu=0$ кратності $k$. Як випливає з попереднього пункту, кореню $\mu=0$ кратності $k$ відповідає загальний розв'язок вигляду $z = C_1 + C_2 x + \ldots + C_k x^{k - 1}$. \parvskip
        
        З огляду на те, що $y = e^{\nu x} z$, одержимо, що кореню $\lambda=\nu$ кратності $k$ відповідає розв'язок
        \begin{equation*}
        	y = \left(C_1 + C_2 x + \ldots + C_k x^{k - 1} \right) e^{\nu x}.
        \end{equation*}
        
        \item Нехай характеристичне рівняння має корені $\lambda=p+iq$, $\bar\lambda=p-iq$ кратності $k$. Проводячи аналогічні викладки одержимо, що їм відповідають лінійно незалежні розв'язки
        \begin{equation*}
        	e^{px} \cos qx, \quad x e^{px} \cos qx, \quad \ldots, \quad x^{k-1} e^{px} \cos qx,
        \end{equation*}
        
        \begin{equation*}
        	e^{px} \sin qx, \quad x e^{px} \sin qx, \quad \ldots, \quad x^{k-1} e^{px} \sin qx.
        \end{equation*}
        
        І загальним розв'язком, що відповідає цим кореням буде
        \begin{multline*}
        	y = C_1 e^{px} \cos qx + C_2 x e^{px} \cos qx + C_k x^{k-1} e^{px} \cos qx + \\ + C_{k+1} e^{px} \sin qx + C_{k+2} x e^{px} \sin qx + \ldots + C_{2k} x^{k-1} e^{px} \sin qx.
        \end{multline*}
    \end{enumerate}
\end{enumerate}


		\subsubsection{Вправи для самостійної роботи}
		\begin{example}
	Розв’язати рівняння $y'' + y' - 2 y = 0$.
\end{example}
\begin{solution}
	Розв’язок шукаємо у вигляді $y = e^{\lambda x}$. Тоді \[y'=\lambda e^{\lambda x}, \quad y''=\lambda^2 e^{\lambda x}.\] Підставивши в диференціальне рівняння, одержуємо \[ \lambda^2 e^{\lambda x} + \lambda e^{\lambda x} - 2 e^{\lambda x} = 0.\] Скоротивши на $e^{\lambda x}$, одержуємо характеристичне рівняння \[\lambda^2+\lambda-2=0.\] Його коренями будуть $\lambda_1=-1$, $\lambda_2=2$. Їм відповідають два лінійно незалежні розв’язки $e^{-x}$, $e^{2x}$. І загальним розв’язком диференціального рівняння буде \[ y(x) = C_1 \cdot e^{-x} + C_2 \cdot e^{2x}.\]
\end{solution}
\begin{example}
	Розв’язати рівняння $y'' + y' + 2 y = 0$.
\end{example}
\begin{solution}
	Розв’язок шукаємо у вигляді $y = e^{\lambda x}$. Тоді \[ y' = \lambda e^{\lambda x}, \quad y'' =\lambda^2 e^{\lambda x}.\] Підставивши в диференціальне рівняння, одержуємо \[ \lambda^2 e^{\lambda x} + \lambda e^{\lambda x} + 2 e^{\lambda x}=0.\] Скоротимо на $e^{\lambda x}$: \[ \lambda^2+\lambda+2=0.\] Коренями характеристичного рівняння будуть $\lambda_1=-1\pm i$. Їм відповідають два лінійно незалежні розв’язки \[y_1(x)=e^{-x}\cdot\cos x, \quad y_2(x)=e^{-x}\cdot\sin x.\] І загальним розв’язком рівняння буде \[y(x)=C_1\cdot e^{-x}\cdot\cos x+C_2\cdot e^{-x}\cdot\sin x.\]
\end{solution}
\begin{example}
	Розв’язати рівняння $y'' + 4 y' + 4 y = 0$.
\end{example}
\begin{solution}
	Розв’язок шукаємо у вигляді $y = e^{\lambda x}$. Тоді \[y'=\lambda e^{\lambda x},\quad y''=\lambda^2 e^{\lambda x}.\] Підставляємо в диференціальне рівняння, одержуємо \[\lambda^2 e^{\lambda x}+4\lambda e^{\lambda x}+4e^{\lambda x}=0.\] Скоротимо на $e^{\lambda x}$:\[\lambda^2+4\lambda+4=0.\] Коренями характеристичного рівняння будуть $\lambda_1=\lambda_2=-2$. Оскільки вони кратні їм відповідають два лінійно незалежні розв’язки \[y_1(x)=e^{-2x},\quad y_2(x)=x\cdot e^{-2x}.\] І загальним розв’язком рівняння буде \[y(x)=C_1\cdot e^{-2x}+C_2\cdot x\cdot e^{-2x}.\]
\end{solution}

Розв’язати рівняння:
\begin{multicols}{2}
\begin{problem}
	\[y''-5y'+6y=0;\]
\end{problem}
\begin{problem}
	\[y''-9y=0;\]
\end{problem}
\begin{problem}
	\[y''-y'=0;\]
\end{problem}
\begin{problem}
	\[y''+2y'+y=0;\]
\end{problem}
\begin{problem}
	\[2y''+5y'+2y=0;\]
\end{problem}
\begin{problem}
	\[y''-4y=0;\]
\begin{problem}
	\[y''+3y'=0;\]
\end{problem}
\end{problem}
\begin{problem}
	\[y''-y'-2y=0;\]
\end{problem}
\begin{problem}
	\[y''-4y'+2y=0;\]
\end{problem}
\begin{problem}
	\[y''+6y'+13y=0;\]
\end{problem}
\begin{problem}
	\[y''-4y'+15y=0;\]
\end{problem}
\begin{problem}
	\[y''-6y'+34y=0;\]
\end{problem}
\begin{problem}
	\[y''+4y=0;\]
\end{problem}
\begin{problem}
	\[y''+2y'+10y=0;\]
\end{problem}
\begin{problem}
	\[y''+y=0.\]
\end{problem}
\end{multicols}

Знайти частинні розв’язки, що задовольняють зазначеним початковим умовам при $x=0$:
\begin{problem}
	\[y''-5y'+4y=0,\quad y=5,\quad y'=8;\]
\end{problem}
\begin{problem}
	\[y''+3y'+2y=0,\quad y=1,\quad y'=-1;\]
\end{problem}
\begin{problem}
	\[y''+4y=0,\quad y=0,\quad y'=2;\]
\end{problem}
\begin{problem}
	\[y''+2y'=0,\quad y=1,\quad y'=0;\]
\end{problem}
\begin{problem}
	\[y''-4y'+4y=0,\quad y=3,\quad y'=-1;\]
\end{problem}
\begin{problem}
	\[y''+4y'+29y=0,\quad y=0,\quad y'=15;\]
\end{problem}
\begin{problem}
	\[y''+3y=0,\quad y=0,\quad y'=1;\]
\end{problem}
\begin{problem}
	\[y''-2y'+y=0,\quad y=4,\quad y'=2;\]
\end{problem}

Розв’язати рівняння:
\begin{multicols}{2}
\begin{problem}
	\[y'''-13y''+12y'=0;\]
\end{problem}
\begin{problem}
	\[y''-y'=0;\]
\end{problem}
\begin{problem}
	\[y^{(4)}-2y''=0;\]
\end{problem}
\begin{problem}
	\[y'''-3y''+3y-y=0;\]
\end{problem}
\begin{problem}
	\[y^{(4)}+4y=0;\]
\end{problem}
\begin{problem}
	\[y'''+y=0;\]
\end{problem}
\begin{problem}
	\[y^{(4)}+8y''+16y=0;\]
\end{problem}
\begin{problem}
	\[y^{(4)}+y'=0;\]
\end{problem}
\begin{problem}
	\[y^{(4)}-2y''+y=0;\]
\end{problem}
\begin{problem}
	\[y^{(4)}-a^4y=0;\]
\end{problem}
\begin{problem}
	\[y^{(4)}-6y''+9y=0;\]
\end{problem}
\begin{problem}
	\[y^{(4)}+a^2y''=0;\]
\end{problem}
\begin{problem}
	\[y^{(4)}+2y'''+y''=0;\]
\end{problem}
\begin{problem}
	\[y^{(4)}+2y''+y=0;\]
\end{problem}
\begin{problem}
	\[y'''+9y'=0;\]
\end{problem}
\begin{problem}
	\[y'''-3y'-2y=0;\]
\end{problem}
\begin{problem}
	\[y^{(4)}+10y''+9y=0.\]
\end{problem}
\end{multicols}
Знайти частинні розв’язки диференціальних рівнянь:
\begin{problem}
	\[y'''+y'=0, \quad y(0)=2, \quad y'(0)=0, \quad y''(0)=-1;\]
\end{problem}
\begin{problem}
	\[y^{(5)}-y'=0, \quad y(0)=y''(0)=0, \quad y'(0)=1, \quad y'''(0)=1, \quad y^{(4)}=2;\]
\end{problem}
\begin{problem}
	\[y'''+2y''+10y'=0, \quad y(0)=2, \quad y'(0)=y''(0)=1;\]
\end{problem}
\begin{problem}
	\[y'''-y'=0, \quad y(0)=3, \quad y'(0)=-1, \quad y''(0)=1;\]
\end{problem}
\begin{problem}
	\[y'''+y'=0, \quad y(0)=2, \quad y'(0)=0, \quad y''(0)=-1.\]
\end{problem}

	\subsection{Лінійні неоднорідні диференціальні рівняння}
	Загальний вигляд лінійних неоднорідних диференціальних рівнянь наступний
\begin{equation*}
	%\label{eq:3.3.1}
	a_0(x) \cdot y^{(n)}(x) + a_1(x) \cdot y^{(n - 1)}(x) + \ldots + a_n(x) \cdot y(x) = b(x).
\end{equation*}

\subsubsection{Властивості розв’язків лінійних неоднорідних рівнянь. Загальний розв’язок лінійного неоднорідного рівняння}

\begin{property}
	\label{prop:3.3.1}
	Якщо $y_0(x)$ -- розв’язок лінійного однорідного рівняння, $y_1(x)$ -- розв’язок неоднорідного рівняння, то $y(x) = y_0(x) + y_1(x)$ буде розв’язком лінійного неоднорідного диференціального рівняння.
\end{property}

\begin{proof}
	Дійсно, нехай $y_0(x)$ і $y_1(x)$ -- розв’язки відповідно однорідного і неоднорідного рівнянь, тобто
	\begin{align}
		%\label{eq:3.3.2}
		a_0(x) \cdot y_0^{(n)}(x) + a_1(x) \cdot y_0^{(n - 1)}(x) + \ldots + a_n(x) \cdot y_0(x) &= 0, \\
		%\label{eq:3.3.3}
		a_0(x) \cdot y_1^{(n)}(x) + a_1(x) \cdot y_1^{(n - 1)}(x) + \ldots + a_n(x) \cdot y_1(x) &= b(x).
	\end{align}	
	
	Тоді 
	\begin{multline}
		%\label{eq:3.3.4}
		a_0(x) (y_0 + y_1)^{(n)}(x) + a_1(x) (y_0 + y_1)^{(n - 1)}(x) + \ldots + a_n(x) (y_0 + y_1)(x) = \\ = \left( a_0(x) \cdot y_0^{(n)}(x) + a_1(x) \cdot y_0^{(n - 1)}(x) + \ldots + a_n(x) \cdot y_0(x) \right) + \\ + \left( a_0(x) \cdot y_1^{(n)}(x) + a_1(x) \cdot y_1^{(n - 1)}(x) + \ldots + a_n(x) \cdot y_1(x) \right) = \\ = 0 + b(x) = b(x),
	\end{multline}
	тобто $y(x) = y_0(x) + y_1(x)$ -- розв’язок неоднорідного диференціального рівняння.
\end{proof}

\begin{property}[Принцип суперпозиції]
	Якщо $y_i(x)$, $i = \overline{1, n}$ -- розв’язки лінійних неоднорідних диференціальних рівнянь
	\begin{equation*}
		%\label{eq:3.3.5}
		a_0(x) \cdot y^{(n)}(x) + a_1(x) \cdot y^{(n - 1)}(x) + \ldots + a_n(x) \cdot y(x) = b_i(x), \quad i = \overline{1, n}
	\end{equation*}
	то $y(x) = \sum_{i = 1}^n C_i \cdot y_i(x)$ з довільними сталими $C_i$ буде розв’язком лінійного неоднорідного рівняння
	\begin{equation*}
		%\label{eq:3.3.6}
		a_0(x) y^{(n)}(x) + a_1(x) y^{(n - 1)}(x) + \ldots + a_n(x) y(x) = \sum_{i = 1}^n C_i  b_i(x)
	\end{equation*}
\end{property}

\begin{proof}
	Дійсно, нехай $y_i(x)$, $i = \overline{1, n}$ -- розв’язки відповідних неоднорідних рівнянь, тобто
	\begin{equation*}
		%\label{eq:3.3.7}
		a_0(x) \cdot y_i^{(n)}(x) + a_1(x) \cdot y_i^{(n - 1)}(x) + \ldots + a_n(x) \cdot y_i(x) = b_i(x), \quad i = \overline{1, n}
	\end{equation*}

	Склавши лінійну комбінацію з рівнянь і їхніх правих частин з коефіцієнтами $C_i$ одержимо
	\begin{multline}
		%\label{eq:3.3.8}
		\sum_{i = 1}^n C_i \cdot \left( a_0(x) \cdot y_i^{(n)}(x) + a_1(x) \cdot y_i^{(n - 1)}(x) + \ldots + a_n(x) \cdot y_i(x) \right) = \\ = \sum_{i = 1}^n C_i \cdot b_i(x),
	\end{multline}
	або, перегрупувавши, запишемо
	\begin{multline}
		%\label{eq:3.3.9}
		a_0(x) \cdot \left( \sum_{i = 1}^n C_i \cdot y_i^{(n)}(x) \right) + a_1(x) \cdot \left( \sum_{i = 1}^n C_i \cdot y_i^{(n - 1)}(x)\right) + \ldots \\ \ldots + a_n(x) \cdot \left( \sum_{i = 1}^n C_i \cdot y_i(x) \right) = \sum_{i = 1}^n C_i \cdot b_i(x),
	\end{multline}
	що і було потрібно довести.
\end{proof}

\begin{property}
	Якщо комплексна функція $y(x) = u(x) + i v(x)$ з дійсними елементами є розв’язком лінійного неоднорідного рівняння з комплексною правою частиною $b(x) = f(x) + i p(x)$, то дійсна частина $u(x)$ є розв’язком рівняння з правою частиною $f(x)$, а уявна $v(x)$ є розв’язком рівняння з правою частиною $p(x)$.
\end{property}

\begin{proof}
	Дійсно, як випливає з умови,
	\begin{multline}
	 	%\label{eq:3.3.10}
	 	a_0(x) \cdot (u + i v)^{(n)}(x) + a_1(x) \cdot (u + i v)^{(n - 1)}(x) + \ldots + a_n(x) \cdot (u + i v)(x) = \\ = f(x) + i p(x).
	\end{multline}
	
	Розкривши дужки, одержимо
	\begin{multline}
		%\label{eq:3.3.11}
		\left( a_0(x) \cdot u^{(n)}(x) + a_1(x) \cdot u^{(n - 1)}(x) + \ldots + a_n(x) \cdot u(x) \right) + \\ + i \left( a_0(x) \cdot v^{(n)}(x) + a_1(x) \cdot v^{(n - 1)}(x) + \ldots + a_n(x) \cdot v(x) \right) = \\ = f(x) + i p(x).
	\end{multline}

	А комплексні вирази рівні між собою тоді і тільки тоді, коли дорівнюють окремо дійсні та уявні частини, тобто
	\begin{align}
		%\label{eq:3.3.12}
		a_0(x) \cdot u^{(n)}(x) + a_1(x) \cdot u^{(n - 1)}(x) + \ldots + a_n(x) \cdot u(x) &= f(x), \\ 
		%\label{eq:3.3.13}
		a_0(x) \cdot v^{(n)}(x) + a_1(x) \cdot v^{(n - 1)}(x) + \ldots + a_n(x) \cdot v(x) &= p(x),
	\end{align}
	що і було потрібно довести.
\end{proof}

\begin{theorem}
	Загальний розв’язок лінійного неоднорідного диференціального рівняння складається з загального розв’язку лінійного однорідного рівняння і частинного розв’язку неоднорідного рівняння.
\end{theorem}
\begin{proof}
	Нехай $y_{\text{homo}}(x) = \sum_{i = 1}^n C_i \cdot y_i(x)$ -- загальний розв’язок однорідного\footnote{Homogeneous equation* -- однорідне рівняння.} рівняння, а $y_{\text{hetero}}(x)$ -- частинний розв’язок неоднорідного\footnote{Heterogeneous equation* -- неоднорідне рівняння.} рівняння. \\

	Тоді, як випливає з властивості \ref{prop:3.3.1}, $y(x) = \sum_{i = 1}^n C_i \cdot y_i(x) + y_{\text{hetero}}(x)$, буде розв’язком неоднорідного рівняння. Покажемо, що цей розв’язок  загальний, тобто вибором коефіцієнтів $C_i$ можна розв’язати довільну задачу Коші
	\begin{equation*}
		%\label{eq:3.3.14}
		y(x_0) = y_0, \quad y'(x_0) = y_0', \quad \ldots, \quad y^{(n - 1)}(x_0) = y_0^{(n - 1)}.
	\end{equation*}

	Дійсно, оскільки $y_{\text{homo}}$ загальний розв’язок однорідного рівняння, то $y_i$, $i = \overline{1, n}$ лінійно незалежні, а отже визначник Вронського $W[y_1, y_2, \ldots, y_n] \ne 0$. Звідси, неоднорідна система лінійних алгебраїчних рівнянь 
	\begin{equation*}
		%\label{eq:3.3.15}
		\left\{ \begin{aligned}
			C_1 \cdot y_1(x_0) + C_2 \cdot y_2(x_0) + \ldots + C_n \cdot y_n(x_0) &= y_0 - y_{\text{hetero}}(x_0), \\
			C_1 \cdot y_1'(x_0) + C_2 \cdot y_2'(x_0) + \ldots + C_n \cdot y_n'(x_0) &= y_0' - y_{\text{hetero}}'(x_0), \\
			\ldots \ldots \ldots \ldots \ldots \ldots \ldots \ldots \ldots \ldots \ldots \ldots \ldots \ldots & \ldots \ldots \ldots \ldots \ldots \ldots 	\\
			C_1 \cdot y_1^{(n - 1)}(x_0) + C_2 \cdot y_2^{(n - 1)}(x_0) + \ldots + C_n \cdot y_n^{(n - 1)}(x_0) &= y_0 - y_{\text{hetero}}^{(n - 1)}(x_0),
		\end{aligned} \right.
	\end{equation*}
	має єдиний розв’язок для довільних наперед обраних $y_0, y_0', \ldots, y_0^{(n - 1)}$. Нехай розв’язком системи буде $C_1^0, C_2^0, \ldots, C_n^0$. Тоді, як випливає з вигляду системи, функція $y(x) = \sum_{i = 1}^n C_i^0 \cdot y_i(x) + y_{\text{hetero}}$ є розв’язком поставленому задачі Коші.
\end{proof}

Як випливає з теореми для знаходження загального розв’язку лінійного неоднорідного рівняння треба шукати загальний розв’язок однорідного рівняння, тобто будь-які $n$ лінійно незалежні розв’язкі і якийсь частинний розв’язок неоднорідного рівняння. Розглянемо три методи побудови частинного розв’язку лінійного неоднорідного рівняння.

		\subsubsection{Властивості розв’язків лінійних неоднорідних рівнянь. Загальний розв’язок лінійного неоднорідного рівняння}
		\begin{property}
	Якщо $y_0(x)$ --- розв'язок лінійного однорідного рівняння, $y_1(x)$ --- розв'язок неоднорідного рівняння, то $y(x) = y_0(x) + y_1(x)$ буде розв'язком лінійного неоднорідного диференціального рівняння.
\end{property}

\begin{proof}
	Дійсно, нехай $y_0(x)$ і $y_1(x)$ --- розв'язки відповідно однорідного і неоднорідного рівнянь, тобто
	\begin{align*}
		a_0(x) y_0^{(n)}(x) + a_1(x) y_0^{(n - 1)}(x) + \ldots + a_n(x) y_0(x) &= 0, \\
		a_0(x) y_1^{(n)}(x) + a_1(x) y_1^{(n - 1)}(x) + \ldots + a_n(x) y_1(x) &= b(x).
	\end{align*}	
	
	Тоді 
	\begin{multline*}
		a_0(x) (y_0 + y_1)^{(n)}(x) + a_1(x) (y_0 + y_1)^{(n - 1)}(x) + \ldots + a_n(x) (y_0 + y_1)(x) = \\ = \left( a_0(x) y_0^{(n)}(x) + a_1(x) y_0^{(n - 1)}(x) + \ldots + a_n(x) y_0(x) \right) + \\ + \left( a_0(x) y_1^{(n)}(x) + a_1(x) y_1^{(n - 1)}(x) + \ldots + a_n(x) y_1(x) \right) = \\ = 0 + b(x) = b(x),
	\end{multline*}
	тобто $y(x) = y_0(x) + y_1(x)$ --- розв'язок неоднорідного диференціального рівняння.
\end{proof}

\begin{property}[Принцип суперпозиції]
	Якщо $y_i(x)$, $i = \overline{1, n}$ --- розв'язки лінійних неоднорідних диференціальних рівнянь
	\begin{equation*}
		a_0(x) y^{(n)}(x) + a_1(x) y^{(n - 1)}(x) + \ldots + a_n(x) y(x) = b_i(x), \quad i = \overline{1, n}
	\end{equation*}
	то $y(x) = \sum_{i = 1}^n C_i y_i(x)$ з довільними сталими $C_i$ буде розв'язком лінійного неоднорідного рівняння
	\begin{equation*}
		a_0(x) y^{(n)}(x) + a_1(x) y^{(n - 1)}(x) + \ldots + a_n(x) y(x) = \sum_{i = 1}^n C_i  b_i(x)
	\end{equation*}
\end{property}

\begin{proof}
	Дійсно, нехай $y_i(x)$, $i = \overline{1, n}$ --- розв'язки відповідних неоднорідних рівнянь, тобто
	\begin{equation*}
		a_0(x) y_i^{(n)}(x) + a_1(x) y_i^{(n - 1)}(x) + \ldots + a_n(x) y_i(x) = b_i(x), \quad i = \overline{1, n}
	\end{equation*}

	Склавши лінійну комбінацію з рівнянь і їхніх правих частин з коефіцієнтами $C_i$ одержимо
	\begin{equation*}
		\sum_{i = 1}^n C_i \left( a_0(x) y_i^{(n)}(x) + a_1(x) y_i^{(n - 1)}(x) + \ldots + a_n(x) y_i(x) \right) = \sum_{i = 1}^n C_i b_i(x),
	\end{equation*}
	або, перегрупувавши, запишемо
	\begin{multline*}
		a_0(x) \left( \sum_{i = 1}^n C_i y_i^{(n)}(x) \right) + a_1(x) \left( \sum_{i = 1}^n C_i y_i^{(n - 1)}(x)\right) + \ldots \\ \ldots + a_n(x) \left( \sum_{i = 1}^n C_i y_i(x) \right) = \sum_{i = 1}^n C_i b_i(x),
	\end{multline*}
	що і було потрібно довести.
\end{proof}

\begin{property}
	Якщо комплексна функція $y(x) = u(x) + i v(x)$ з дійсними елементами є розв'язком лінійного неоднорідного рівняння з комплексною правою частиною $b(x) = f(x) + i p(x)$, то дійсна частина $u(x)$ є розв'язком рівняння з правою частиною $f(x)$, а уявна $v(x)$ є розв'язком рівняння з правою частиною $p(x)$.
\end{property}

\begin{proof}
	Дійсно, як випливає з умови,
	\begin{multline*}
	 	a_0(x) (u + i v)^{(n)}(x) + a_1(x) (u + i v)^{(n - 1)}(x) + \ldots + a_n(x) (u + i v)(x) = \\ = f(x) + i p(x).
	\end{multline*}
	
	Розкривши дужки, одержимо
	\begin{multline*}
		\left( a_0(x) u^{(n)}(x) + a_1(x) u^{(n - 1)}(x) + \ldots + a_n(x) u(x) \right) + \\ + i \left( a_0(x) v^{(n)}(x) + a_1(x) v^{(n - 1)}(x) + \ldots + a_n(x) v(x) \right) = \\ = f(x) + i p(x).
	\end{multline*}

	А комплексні вирази рівні між собою тоді і тільки тоді, коли дорівнюють окремо дійсні та уявні частини, тобто
	\begin{align*}
		a_0(x) u^{(n)}(x) + a_1(x) u^{(n - 1)}(x) + \ldots + a_n(x) u(x) &= f(x), \\ 
		a_0(x) v^{(n)}(x) + a_1(x) v^{(n - 1)}(x) + \ldots + a_n(x) v(x) &= p(x),
	\end{align*}
	що і було потрібно довести.
\end{proof}

\begin{theorem}
	Загальний розв'язок лінійного неоднорідного диференціального рівняння складається з загального розв'язку лінійного однорідного рівняння і частинного розв'язку неоднорідного рівняння.
\end{theorem}
\begin{proof}
	Нехай $y_{\text{homo}}(x) = \sum_{i = 1}^n C_i y_i(x)$ --- загальний розв'язок однорідного\footnote{Homogeneous equation --- однорідне рівняння.} рівняння, а $y_{\text{hetero}}(x)$ --- частинний розв'язок неоднорідного\footnote{Heterogeneous equation --- неоднорідне рівняння.} рівняння. \parvskip

	Тоді, як випливає з першої властивості, $y(x) = \sum_{i = 1}^n C_i y_i(x) + y_{\text{hetero}}(x)$, буде розв'язком неоднорідного рівняння. Покажемо, що цей розв'язок  загальний, тобто вибором коефіцієнтів $C_i$ можна розв'язати довільну задачу Коші
	\begin{equation*}
		y(x_0) = y_0, \quad y'(x_0) = y_0', \quad \ldots, \quad y^{(n - 1)}(x_0) = y_0^{(n - 1)}.
	\end{equation*}

	Дійсно, оскільки $y_{\text{homo}}$ загальний роз\-в'яз\-ок однорідного рівняння, то система функцій $y_i$, $i = \overline{1, n}$ лінійно незалежна, тому визначник Вронського $W[y_1, y_2, \ldots, y_n] \ne 0$. Звідси, неоднорідна система лінійних алгебраїчних рівнянь 
	\begin{equation*}
		\left\{ \begin{array}{rl}
			C_1 y_1(x_0) + C_2 y_2(x_0) + \ldots + C_n y_n(x_0) &= y_0 - y_{\text{hetero}}(x_0), \\
			C_1 y_1'(x_0) + C_2 y_2'(x_0) + \ldots + C_n y_n'(x_0) &= y_0' - y_{\text{hetero}}'(x_0), \\
			\hdotsfor{2} \\
			C_1 y_1^{(n - 1)}(x_0) + C_2 y_2^{(n - 1)}(x_0) + \ldots + C_n y_n^{(n - 1)}(x_0) &= y_0 - y_{\text{hetero}}^{(n - 1)}(x_0),
		\end{array} \right.
	\end{equation*}
	має єдиний роз\-в'яз\-ок для довільних наперед обраних $y_0, y_0', \ldots, y_0^{(n - 1)}$. Нехай роз\-в'яз\-ком системи буде $C_1^0, C_2^0, \ldots, C_n^0$. Тоді, як випливає з вигляду системи, функція $y(x) = \sum_{i = 1}^n C_i^0 y_i(x) + y_{\text{hetero}}$ є роз\-в'яз\-ком поставленої задачі Коші.
\end{proof}

Як випливає з теореми для знаходження загального розв'язку лінійного неоднорідного рівняння треба шукати загальний розв'язок однорідного рівняння, тобто будь-які $n$ лінійно незалежні розв'язки і якийсь частинний розв'язок неоднорідного рівняння. Розглянемо три методи побудови частинного розв'язку лінійного неоднорідного рівняння.

		\subsubsection{Метод варіації довільної сталої побудови частинного роз\-в'яз\-ку лінійного неоднорідного диференціального рівняння}
		Метод варіації довільної сталої полягає в тому, що розв’язок неоднорідного рівняння шукається в такому ж вигляді, як і розв’язок однорідного, але сталі $C_i$, $i = \overline{1, n}$ вважаються невідомими функціями. Нехай загальний розв’язок лінійного однорідного рівняння
\begin{equation*}
	%\label{eq:3.3.16}
	a_0(x) \cdot y^{(n)}(x) + a_1(x) \cdot y^{(n - 1)}(x) + \ldots + a_n(x) \cdot y(x) = 0.
\end{equation*}
записано у вигляді $y(x) = C_1 \cdot y_1(x) + C_2 \cdot y_2(x) + \ldots + C_n \cdot y_n(x)$. \parvskip

Розв’язок лінійного неоднорідного рівняння
\begin{equation*}
	%\label{eq:3.3.17}
	a_0(x) \cdot y^{(n)}(x) + a_1(x) \cdot y^{(n - 1)}(x) + \ldots + a_n(x) \cdot y(x) = b(x).
\end{equation*}
шукаємо у вигляді $y(x) = C_1(x) \cdot y_1(x) + C_2(x) \cdot y_2(x) + \ldots + C_n(x) \cdot y_n(x)$, де $C_i(x)$, $i = \overline{1, n}$ --- невідомі функції. Оскільки підбором $n$ функцій необхідно задовольнити одному рівнянню, тобто одній умові, то $n - 1$ умову можна накласти довільно. Розглянемо першу похідну від записаного розв’язку
\begin{equation*}
	%\label{eq:3.3.18}
	y'(x) = \sum_{i = 1}^n C_i(x) \cdot y_i'(x) + \sum_{i = 1}^n C_i'(x) \cdot y_i(x).
\end{equation*}
і зажадаємо, щоб $\sum_{i = 1}^n C_i'(x) \cdot y_i(x) = 0$. Розглянемо другу похідну
\begin{equation*}
	%\label{eq:3.3.19}
	y'(x) = \sum_{i = 1}^n C_i(x) \cdot y_i''(x) + \sum_{i = 1}^n C_i'(x) \cdot y_i'(x).
\end{equation*}
і зажадаємо, щоб $\sum_{i = 1}^n C_i'(x) \cdot y_i'(x) = 0$. Продовжимо процес взяття похідних до $(n - 1)$-ої 
\begin{equation*}
	%\label{eq:3.3.20}
	y^{(n - 1)}(x) = \sum_{i = 1}^n C_i(x) \cdot y_i^{(n - 1)}(x) + \sum_{i = 1}^n C_i'(x) \cdot y_i^{(n - 2)}(x).
\end{equation*}
і зажадаємо, щоб $\sum_{i = 1}^n C_i'(x) \cdot y_i^{(n - 2)}(x)$. На цьому $(n - 1)$ умова вичерпалася. І для $n$-ої похідної справедливо
\begin{equation*}
	%\label{eq:3.3.21}
	y^{(n)}(x) = \sum_{i = 1}^n C_i(x) \cdot y_i^{(n)}(x) + \sum_{i = 1}^n C_i'(x) \cdot y_i^{(n - 1)}(x).
\end{equation*}

Підставимо взяту функцію та її похідні в неоднорідне диференціальне рівняння
\begin{multline*}
 	%\label{eq:3.3.22}
 	a_0(x) \cdot \left( \sum_{i = 1}^n C_i(x) \cdot y_i^{(n)}(x) \right) + a_0(x) \cdot \left( \sum_{i = 1}^n C_i'(x) \cdot y_i^{(n - 1)}(x) \right) + \\ + a_1(x) \cdot \left( \sum_{i = 1}^n C_i(x) \cdot y_i^{(n - 1)}(x) \right) + \ldots + a_n(x) \cdot \left( \sum_{i = 1}^n C_i(x) \cdot y_i(x) \right) = b(x).
\end{multline*} 
Оскільки $y(x) = \sum_{i = 1}^n C_i(x) \cdot y_i(x)$ --- розв’язок однорідного диференціального рівняння, то після скорочення одержимо $n$-у умову
\begin{equation*}
	%\label{eq:3.3.23}
	\left( \sum_{i = 1}^n C_i'(x) \cdot y_i^{(n - 1)}(x) \right) = \frac{b(x)}{a_0(x)}.
\end{equation*}
Додаючи перші $(n -1 )$ умови, одержимо систему
\begin{equation*}
	%\label{eq:3.3.24}
	\left\{ \begin{aligned}
		C_1'(x) \cdot y_1(x) + C_2'(x) \cdot y_2(x) + \ldots + C_n'(x) \cdot y_n(x) &= 0, \\
		C_1'(x) \cdot y_1'(x) + C_2'(x) \cdot y_2'(x) + \ldots + C_n'(x) \cdot y_n'(x) &= 0, \\
		\ldots \ldots \ldots \ldots \ldots \ldots \ldots \ldots \ldots \ldots \ldots \ldots \ldots \ldots \ldots \ldots & . \ldots . \\
		C_1'(x) \cdot y_1^{(n - 2)}(x) + C_2'(x) \cdot y_2^{(n - 2)}(x) + \ldots + C_n'(x) \cdot y_n^{(n - 2)}(x) &= 0, \\
		C_1'(x) \cdot y_1^{(n - 1)}(x) + C_2'(x) \cdot y_2^{(n - 1)}(x) + \ldots + C_n'(x) \cdot y_n^{(n - 1)}(x) &= \frac{b(x)}{a_0(x)}.
	\end{aligned} \right.
\end{equation*}
 
Оскільки визначником системи є визначник Вронського і він відмінний від нуля, то система має єдиний роз\-в'яз\-ок
\begin{equation*}
	%\label{eq:3.3.25}
	\begin{aligned}
		C_1(x) &= \int \frac{\begin{vmatrix} 0 & y_2(x) & \cdots & y_{n - 1}(x) & y_n'(x) \\ 0 & y_2'(x) & \cdots & y_{n - 1}'(x) & y_n'(x) \\ \vdots & \vdots & \ddots & \vdots & \vdots \\ 0 & y_2^{(n - 2)}(x) & \cdots & y_{n - 1}^{(n - 2)}(x) & y_n^{(n - 2)}(x) \\ \frac{b(x)}{a_0(x)} & y_2^{(n - 1)}(x) & \cdots & y_{n - 1}^{(n - 1)}(x) & y_n^{(n - 1)}(x) \end{vmatrix}}{W[y_1, y_2, \ldots, y_n]} \diff x, \\
		\ldots \ldots & \ldots \ldots \ldots \ldots \ldots \ldots \ldots \ldots \ldots \ldots \ldots \ldots \ldots \ldots \ldots \ldots \\
		C_n(x) &= \int \frac{\begin{vmatrix} y_1(x) & y_2(x) & \cdots & y_{n - 1}(x) & 0 \\ y_1'(x) & y_2'(x) & \cdots & y_{n - 1}'(x) & 0 \\ \vdots & \vdots & \ddots & \vdots & \vdots \\ y_1^{(n - 2)} & y_2^{(n - 2)}(x) & \cdots & y_{n - 1}^{(n - 2)}(x) & 0 \\ y_1^{(n - 1)}(x) & y_2^{(n - 1)}(x) & \cdots & y_{n - 1}^{(n - 1)}(x) & \frac{b(x)}{a_0(x)} \end{vmatrix}}{W[y_1, y_2, \ldots, y_n]} \diff x.
	\end{aligned}
\end{equation*}

І загальний розв’язок лінійного неоднорідного диференціального рівняння запишеться у вигляді
\begin{equation*}
	%\label{eq:3.3.26}
	y(x) = \bar C_1 \cdot y_1(x) + \bar C_2 \cdot y_2(x) + \ldots + \bar C_n \cdot y_n(x) + y_{\text{hetero}}(x),
\end{equation*}
де $\bar C_i$ --- довільні сталі, а
\begin{equation*}
	%\label{eq:3.3.27}
	y_{\text{hetero}}(x) = C_1(x) \cdot y_1(x) + C_2(x) \cdot y_2(x) + \ldots + C_n(x) \cdot y_n(x).
\end{equation*}

Якщо розглядати диференціальне рівняння другого порядку
\begin{equation*}
	%\label{eq:3.3.27}
	a_0(x) \cdot y''(x) + a_1(x) \cdot y'(x) + a_2(x) \cdot y(x) = b(x),
\end{equation*}
і загальний розв’язок однорідного рівняння має вигляд
\begin{equation*}
	%\label{eq:3.3.28}
	y_{\text{homo}}(x) = C_1 \cdot y_1(x) + C_2 \cdot y_2(x),
\end{equation*}
то частинний розв’язок неоднорідного має вигляд 
\begin{equation*}
	%\label{eq:3.3.29}
	y_{\text{hetero}}(x) = C_1(x) \cdot y_1(x) + C_2(x) \cdot y_2(x).
\end{equation*}
І для знаходження функцій $C_1(x), C_2(x)$ маємо систему
\begin{equation*}
	%\label{eq:3.3.30}
	\left\{ \begin{aligned}
		C_1'(x) \cdot y_1(x) + C_2'(x) \cdot y_2(x) &= 0, \\
		C_1'(x) \cdot y_1'(x) + C_2'(x) \cdot y_2'(x) &= \frac{b(x)}{a_0(x)}.
	\end{aligned} \right.
\end{equation*}

Звідси
\begin{equation*}
	%\label{eq:3.3.31}
	C_1(x) = \int \frac{\begin{vmatrix} 0 & y_2(x) \\ \frac{b(x)}{a_0(x)} & y_2'(x) \end{vmatrix}}{\begin{vmatrix} y_1(x) & y_2(x) \\ y_1'(x) & y_2'(x) \end{vmatrix}} \diff x, \quad C_2(x) = \int \frac{\begin{vmatrix} y_1(x) & 0 \\ y_1'(x) & \frac{b(x)}{a_0(x)} \end{vmatrix}}{\begin{vmatrix} y_1(x) & y_2(x) \\ y_1'(x) & y_2'(x) \end{vmatrix}} \diff x
\end{equation*}

І одержуємо $y_{\text{hetero}}(x) = C_1(x) \cdot y_1(x) + C_2(x) \cdot y_2(x)$ з обчисленими функціями $C_1(x)$ і $C_2(x)$.


		\subsubsection{Метод Коші}
		Нехай $y(x) = K(x, s)$ --- розв’язок однорідного диференціального рівняння, що задовольняє умовам
\begin{equation*}
	K(s, s) = K_x'(s, s) = \ldots = K_{x^{n - 2}}^{(n - 2)}(s, s) = 0, \quad K_{x^{n - 1}}^{(n - 1)}(s, s) = 1.
\end{equation*}

Тоді функція
\begin{equation*}
	y(x) = \int_{x_0}^x K(x, s) \cdot \frac{b(s)}{a_0(s)} \diff s
\end{equation*}
буде розв’язком неоднорідного рівняння, що задовольняє початковим умовам
\begin{equation*}
	y(x_0) = y'(x_0) = \ldots = y^{(n - 1)}(x_0) = 0.
\end{equation*}

Дійсно, розглянемо похідні від функції $y(x)$:
\begin{equation*}
	y'(x) = \int_{x_0}^x K_x'(x, s) \cdot \frac{b(s)}{a_0(s)} \diff s + K(x, x) \cdot \frac{b(x)}{a_0(x)}.
\end{equation*}

І, оскільки $K(x, x) = 0$, то
\begin{equation*}
	y'(x) = \int_{x_0}^x K_x'(x, s) \cdot \frac{b(s)}{a_0(s)} \diff s.
\end{equation*}

Аналогічно
\begin{equation*}
	y''(x) = \int_{x_0}^x K_{x^2}''(x, s) \cdot \frac{b(s)}{a_0(s)} \diff s + K_x'(x, x) \cdot \frac{b(x)}{a_0(x)} = \int_{x_0}^x K_{x^2}''(x, s) \cdot \frac{b(s)}{a_0(s)} \diff s,
\end{equation*}
і так далі до
\begin{align*}
	y^{(n - 1)}(x) &= \int_{x_0}^x K_{x^{n - 1}}^{(n - 1)}(x, s) \cdot \frac{b(s)}{a_0(s)} \diff s + K_{x^{n - 2}}^{(n - 2)}(x, x) \cdot \frac{b(x)}{a_0(x)} = \\ &= \int_{x_0}^x K_{x^{n - 1}}^{(n - 1)}(x, s) \cdot \frac{b(s)}{a_0(s)} \diff s, \\
	y^{(n)}(x) &= \int_{x_0}^x K_{x^n}^{(n)}(x, s) \cdot \frac{b(s)}{a_0(s)} \diff s + K_{x^{n - 1}}^{(n - 1)}(x, x) \cdot \frac{b(x)}{a_0(x)}.
\end{align*}

І, оскільки $K_{x^{n - 1}}^{(n - 1)}(x, x) = 1$, то
\begin{equation*}
	y^{(n)}(x) = \int_{x_0}^x K_{x^n}^{(n)}(x, s) \cdot \frac{b(s)}{a_0(s)} \diff s + \frac{b(x)}{a_0(x)}.
\end{equation*}

Підставивши функцію $y(x)$ і її похідні у вихідне диференціальне рівняння, одержимо
\begin{multline*}
	a_0(x) \cdot \left( \int_{x_0}^x K_{x^n}^{(n)} (x, s) \cdot \frac{b(s)}{a_0(s)} \diff s + \frac{b(x)}{a_0(x)} \right) + \\ + a_1(x) \cdot \left( \int_{x_0}^x K_{x^{n - 1}}^{(n - 1)} (x, s) \cdot \frac{b(s)}{a_0(s)} \diff s \right) + \ldots + a_n(x) \int_{x_0}^x K_x' (x, s)  \cdot \frac{b(s)}{a_0(s)} \diff s = \\ = \int_{x_0}^x \left( a_0(x) \cdot K_{x^n}^{(n)}(x, s) + a_1(x) \cdot K_{x^{n - 1}}^{(n - 1)}(x, s) + \ldots + a_n(x) \cdot K(x, s) \right).
\end{multline*}

Оскільки $K(x, s)$ -- є розв’язком лінійного однорідного рівняння і, отже,
\begin{equation*}
	a_0(x) \cdot K_{x^n}^{(n)}(x, s) + a_1(x) \cdot K_{x^{n - 1}}^{(n - 1)}(x, s) + \ldots + a_n(x) \cdot K(x, s) = 0.
\end{equation*}
 
У такий спосіб показано, що $y(x) = \int_{x_0}^x K(x, s) \cdot \frac{b(s)}{a_0(s)} \diff s$ -- є розв’язком лінійного неоднорідного рівняння. \parvskip

Підставляючи $x = x_0$ в значення $y(x), y'(x), \ldots, y^{(n)}(x)$ одержимо, що
\begin{equation*}
	y(x_0) = y'(x_0) = \ldots = y^{(n - 1)}(x_0) = 0.
\end{equation*}

Для знаходження функції $K(x, s)$ (інтегрального ядра) можна використати такий спосіб. Якщо $y_1(x), y_2(x), \ldots, y_n(x)$ лінійно незалежні роз\-в'яз\-ки однорідного рівняння, то загальний роз\-в'яз\-ок однорідного рівняння має вигляд 
\begin{equation*}
	y_{\text{homo}}(x) = C_1 \cdot y_1(x) + C_2 \cdot y_2(x) + \ldots + C_n \cdot y_n(x).
\end{equation*}
Оскільки $K(x, s)$ є розв’язком однорідного рівняння, то його слід шукати у вигляді
\begin{equation*}
	K(x, s) = C_1(s) \cdot y_1(x) + C_2(s) \cdot y_2(x) + \ldots + C_n(s) \cdot y_n(x).
\end{equation*}
Відповідні початкові умови мають вигляд
\begin{align*}
	K(s, s) = 0 &\Rightarrow C_1(s) \cdot y_1(s) + C_2(s) \cdot y_2(s) + \ldots + C_n(s) \cdot y_n(s) = 0, \\
	K_x'(s, s) = 0 &\Rightarrow C_1(s) \cdot y_1'(s) + C_2(s) \cdot y_2'(s) + \ldots + C_n(s) \cdot y_n'(s) = 0,
\end{align*}
і так далі до
\begin{multline*}
	K_{x^{n - 2}}^{(n - 2)}(s, s) = 0 \Rightarrow \\ \Rightarrow C_1(s) \cdot y_1^{(n - 2)}(s) + C_2(s) \cdot y_2^{(n - 2)}(s) + \ldots + C_n(s) \cdot y_n^{(n - 2)}(s) = 0,
\end{multline*}
і
\begin{multline*}
	K_{x^{n - 1}}^{(n - 1)}(s, s) = 1 \Rightarrow \\ \Rightarrow C_1(s) \cdot y_1^{(n - 1)}(s) + C_2(s) \cdot y_2^{(n - 1)}(s) + \ldots + C_n(s) \cdot y_n^{(n - 1)}(s) = 0. 
\end{multline*}
Звідси
\begin{align*}
	C_1(s) &= \int \frac{\begin{vmatrix} 0 & y_2(s) & \cdots & y_n(s) \\ \vdots & \vdots & \ddots & \vdots \\ 0 & y_2^{(n - 2)}(s) & \cdots & y_n^{(n - 2)}(s) \\ 1 & y_2^{(n - 1)}(s) & \cdots & y_n^{(n - 1)}(s) \end{vmatrix}}{W[y_1, y_2, \ldots, y_n](s)} \diff s, \\
	C_2(s) &= \int \frac{\begin{vmatrix} y_1(s) & 0 & \cdots & y_n(s) \\ \vdots & \vdots & \ddots & \vdots \\ y_2^{(n - 2)} & 0 & \cdots & y_n^{(n - 2)}(s) \\ y_2^{(n - 1)}(s) & 1 & \cdots & y_n^{(n - 1)}(s) \end{vmatrix}}{W[y_1, y_2, \ldots, y_n](s)} \diff s,
\end{align*}
і так далі до
\begin{align*}
	C_n(s) &= \int \frac{\begin{vmatrix} y_1(s) & y_2(s) & \cdots & 0 \\ \vdots & \vdots & \ddots & \vdots \\ y_1^{(n - 2)}(s) & y_2^{(n - 2)}(s) & \cdots & 0 \\ y_1^{(n - 1)}(s) & y_2^{(n - 1)}(s) & \cdots & 1 \end{vmatrix}}{W[y_1, y_2, \ldots, y_n](s)} \diff s.
\end{align*}
 
І ядро $K(x, s)$ має вигляд
\begin{equation*}
	K(x, s) = C_1(s) \cdot y_1(x) + C_2(s) \cdot y_2(x) + \ldots + C_n(s) \cdot y_n(x)
\end{equation*}
з одержаними функціями $C_1(s), C_2(s), \ldots, C_n(s)$. \parvskip

Якщо розглядати диференціальне рівняння другого порядку 
\begin{equation*}
	a_0(x) \cdot y''(x) + a_1(x) \cdot y'(x) + a_2(x) \cdot y(x) = b(x),
\end{equation*}
то функція  має вигляд
\begin{equation*}
	K(x, s) = C_1(s) \cdot y_1(x) + C_2(s) \cdot y_2(x),
\end{equation*}
де
\begin{equation*}
	C_1(s) = \frac{\begin{vmatrix} 0 & y_2(s) \\ 1 & y_2'(s) \end{vmatrix}}{\begin{vmatrix} y_1(s) & y_2(s) \\ y_1'(s) & y_2'(s) \end{vmatrix}}, \quad C_1(s) = \frac{\begin{vmatrix} y_1(s) & 0 \\ y_1'(s) & 1 \end{vmatrix}}{\begin{vmatrix} y_1(s) & y_2(s) \\ y_1'(s) & y_2'(s) \end{vmatrix}}.
\end{equation*}
Звідси
\begin{equation*}
	K(x, s) = \frac{\begin{vmatrix} 0 & y_2(s) \\ 1 & y_2'(s) \end{vmatrix} y_1(x) + \begin{vmatrix} y_1(s) & 0 \\ y_1'(s) & 1 \end{vmatrix} y_2(x) }{W[y_1, y_2](s)} = \frac{y_1(s) \cdot y_2(x) - y_1(x) \cdot y_2(s)}{W[y_1, y_2](s)}
\end{equation*}

		\subsubsection{Метод невизначених коефіцієнтів}
		Якщо лінійне диференціальне рівняння є рівнянням з сталими коефіцієнтами, а функція $b(x)$ спеціального виду, то частинний розв’язок можна знайти за допомогою методу невизначених коефіцієнтів.

\begin{enumerate}
	\item Нехай $b(x)$ має вид многочлена, тобто
	\begin{equation*}
		b(x) = A_0 \cdot x^s + A_1 \cdot x^{s - 1} + \ldots + A_{s - 1} \cdot x + A_s.
	\end{equation*}

	\begin{enumerate}
		\item Розглянемо випадок, коли характеристичне рівняння не має нульового кореня, тобто $\lambda \ne 0$. Частинний розв’язок неоднорідного рівняння шукаємо вигляді:
		\begin{equation*}
			y_{\text{part}} = B_0 \cdot x^s + B_1 \cdot x^{s - 1} + \ldots + B_{s - 1} + B_s,
		\end{equation*}
		де $B_0, \ldots, B_s$ -- невідомі сталі. Тоді
		\begin{align*}
			y_{\text{part}}' &= s \cdot B_0 \cdot x^{s - 1} + (s - 1) \cdot B_1 \cdot x^{s - 2} + \ldots + 1 \cdot B_{s - 1}, \\
			y_{\text{part}}'' &= s \cdot (s - 1) \cdot B_0 \cdot x^{s - 2} + (s - 1) \cdot (s - 2) \cdot B_1 \cdot x^{s - 3} + \ldots \\ & \quad \ldots + 2 \cdot 1 \cdot B_{s - 2},
		\end{align*}
		і так далі. \\

		Підставляючи у вихідне диференціальне рівняння, одержимо
		\begin{multline*}
			a_0 \left( s! B_s \right) + \ldots \\ + a_{n - 2} \left( s (s - 1) B_0 x^{s - 2} + (s - 1) (s - 2) B_1 x^{s - 3} + \ldots + 2 B_{s - 1} \right) + \\ + a_{n - 1} \left( s B_0 x^{s - 1} + (s - 1) B_1 x^{s - 2} + \ldots + B_{s - 1} \right) + \\ + a_n \left( B_0 x^s + B_1 x^{s - 1} + \ldots + B_{s - 1} + B_s \right) = \\ = A_0 x^s + A_1 x^{s - 1} + \ldots + A_{s - 1} x + A_s.
		\end{multline*}

		Прирівнявши коефіцієнти при однакових степенях $x$ запишемо: 
		\begin{table}[H]
			\centering
			\begin{tabular}{c|l}
				$x^s$ & $a_n \cdot B_0 = A_0$ \\
				$x^{s - 1}$ & $a_n \cdot B_1 + s \cdot a_{n - 1} \cdot B_0 = A_1$ \\
				$x^{s - 2}$ & $a_n \cdot B_2 + (s - 1) \cdot a_{n - 1} \cdot B_1 + s \cdot (s - 1) \cdot a_{n - 2} \cdot B_0 = A_2$
			\end{tabular}
		\end{table}
		і так далі. \\

		Оскільки характеристичне рівняння не має нульового кореня, то $a_n \ne 0$. Звідси одержимо $B_0 = \frac{A_0}{a_n}$, $B_1 = \frac{A_1 - s \cdot a_{n - 1} \cdot B_0}{a_n}$, і так далі.

		\item Розглянемо випадок, коли характеристичне рівняння має нульовий корінь кратності $r$. Тоді диференціальне рівняння має вигляд
		\begin{equation*}
			a_0 \cdot y^{(n)} + a_1 \cdot y^{(n - 1)} + \ldots + a_{n - r} \cdot y^{(r)} = A_0 \cdot x^s + A_1 \cdot x^{s - 1} + \ldots + A_s.
		\end{equation*}

		Зробивши заміну $y^{(r)} = z$ одержимо диференціальне рівняння 
		\begin{equation*}
			a_0 \cdot z^{(n - r)} + a_1 \cdot z^{(n - r - 1)} + \ldots + a_{n - r} \cdot z = A_0 \cdot x^s + A_1 \cdot x^{s - 1} + \ldots + A_s,
		\end{equation*}
		характеристичне рівняння якого вже не має нульового кореня, тобто повернемося до попереднього випадку. Звідси частинний розв’язок шукається у вигляді
		\begin{equation*}
			z_{\text{part}} = \bar B_0 \cdot x^s + \bar B_1 \cdot x^{s - 1} + \ldots + \bar B_s.
		\end{equation*}

 		Проінтегрувавши його $r$-разів, одержимо, що частиний роз\-в'яз\-ок вихідного однорідного рівняння має вигляд
 		\begin{equation*}
			y_{\text{part}} = \left(B_0 \cdot x^s + B_1 \cdot x^{s - 1} + \ldots + B_s\right) \cdot x^r.
		\end{equation*}
 	\end{enumerate}
	\item Нехай $b(x)$ має вигляд $b(x) = e^{px} \cdot \left( A_0 \cdot x^s + A_1 \cdot x^{s - 1} + \ldots + A_s \right)$.
	\begin{enumerate}
		\item Розглянемо випадок, коли $p$ не є коренем характеристичного рівняння. Зробимо заміну
		\begin{align*}
			y &= e^{p x} \cdot z, \\
			y' &= p \cdot e^{p x} \cdot z + e^{p x} \cdot z = e^{p x} \cdot (p z + z'), \\
			y'' &= p \cdot e^{p x} \cdot (p z + z') + e^{p x} \cdot (p z' + z'') = e^{p x} \cdot (p^2 z + 2 p z' + z''),
		\end{align*}
 		і так далі до
 		\begin{equation*}
 			y^{(n)} = e^{p x} \cdot \left( p^n \cdot z + n \cdot p^{n - 1} \cdot z' + \ldots + z^{(n)} \right).
 		\end{equation*}
		
		Підставивши отримані вирази у вихідне диференціальне рівняння, одержимо
		\begin{multline*}
			e^{p x} \cdot \left( B_0 \cdot z^{(n)} + B_1 \cdot z^{(n - 1)} + \ldots B_n \cdot z \right) = \\ = e^{p z} \cdot \left( A_0 \cdot x^s + A_1 \cdot x^{s - 1} + \ldots + A_s \right).
		\end{multline*}
		де $B_i$ -- сталі коефіцієнти, що виражаються через $a_i$ і $p$. Скоротивши на $e^{p x}$, одержимо рівняння 
 		\begin{equation*}
			B_0 \cdot z^{(n)} + B_1 \cdot z^{(n - 1)} + \ldots B_n \cdot z = A_0 \cdot x^s + A_1 \cdot x^{s - 1} + \ldots + A_s.
		\end{equation*}
		
		Причому, оскільки $p$ не є коренем характеристичного рівняння, то після заміни $y = e^{px} \cdot z$, отримане диференціальне рівняння не буде мати коренем характеристичного рівняння $\mu = 0$. Таким чином, повернулися до випадку 1.a). Частинний розв’язок неоднорідного рівняння шукаємо у вигляді
		\begin{equation*}
			z_{\text{part}} = B_0 \cdot x^s + B_1 \cdot x^{s - 1} + \ldots + B_{s - 1} + B_s,
		\end{equation*}

		А частинний розв’язок вихідного неоднорідного диференціального рівняння у вигляді:
		\begin{equation*}
			y_{\text{part}} = e^{px} \cdot \left( B_0 \cdot x^s + B_1 \cdot x^{s - 1} + \ldots + B_{s - 1} + B_s \right),
		\end{equation*}

		\item Розглянемо випадок, коли $p$ -- корінь характеристичного рівняння кратності $r$. Це значить, що після, заміни $y = e^{px} \cdot z$ і скорочення на $e^{p x}$, вийде диференціальне рівняння, що має коренем характеристичного рівняння, число $\mu = 0$ кратності $r$, тобто
		\begin{equation*}
			B_0 \cdot z^{(n)} + B_1 \cdot z^{(n - 1)} + \ldots B_{n - r} \cdot z^{(r)} = A_0 \cdot x^s + A_1 \cdot x^{s - 1} + \ldots + A_s.
		\end{equation*}

		Як випливає з пункту 1.б) частинний розв’язок шукається у вигляді
		\begin{equation*}
			z_{\text{part}} = \left( B_0 \cdot x^s + B_1 \cdot x^{s - 1} + \ldots + B_s \right) \cdot x^r,
		\end{equation*}
		а частинний розв’язок вихідного неоднорідного диференціального рівняння у вигляді
		\begin{equation*}
			y_{\text{part}} = e^{p x} \cdot \left( B_0 \cdot x^s + B_1 \cdot x^{s - 1} + \ldots + B_s \right) \cdot x^r,
		\end{equation*}
 	\end{enumerate}
	\item Нехай $b(x)$ має вигляд:
	\begin{equation*}
		b(x) = e^{px} \cdot \left( P_s(x) \cdot \cos (qx) + Q_\ell (x) \cdot \sin(q x) \right),
	\end{equation*}
	де $P_s(x)$, $Q_\ell(x)$ -- многочлени степеня $s$ і $\ell$, відповідно, і, наприклад, $\ell \le s$. Використовуючи формулу Ейлера, перетворимо вираз до вигляду:
	\begin{equation*}
		b(x) = e^{(p + i q) \cdot x} \cdot R_s(x) + e^{(p - i q) \cdot x} \cdot T_s(x),
	\end{equation*}
	де $R_s(x)$, $T_s(x)$ -- многочлени степеня не вище, ніж $s$. Використовуючи властивості 2, 3 розв’язків неоднорідних диференціальних рівнянь, а також випадки 2.а), 2.б) знаходження частинного розв’язку лінійних неоднорідних рівнянь, одержимо, що частинний розв’язок шукається у виглядах:
 	\begin{enumerate}
 		\item 
 		\begin{multline*}
 			y_{\text{part}} = e^{p x} \cdot \left( \left( A_0 \cdot x^s + A_1 \cdot x^{s - 1} + \ldots + A_s \right) \cdot \cos (qx) \right. + \\ + \left. \left( B_0 \cdot x^s + B_1 \cdot x^{s - 1} + \ldots + B_s \right) \cdot \sin (q x) \right),
 		\end{multline*}
		якщо $p \pm i q$ не є коренем характеристичного рівняння;
		\item
		\begin{multline*}
 			y_{\text{part}} = e^{p x} \cdot \left( \left( A_0 \cdot x^s + A_1 \cdot x^{s - 1} + \ldots + A_s \right) \cdot \cos (qx) \right. + \\ + \left. \left( B_0 \cdot x^s + B_1 \cdot x^{s - 1} + \ldots + B_s \right) \cdot \sin (q x) \right) \cdot x^r,
 		\end{multline*}
 		якщо $p \pm i q$ є коренем характеристичного рівняння кратності $r$.
 	\end{enumerate}
\end{enumerate}

		\subsubsection{Вправи для самостійної роботи}
		\begin{example}
	Знайти загальний розв'язок рівняння $y'' - 2 y' + y = \frac{e^x}{x}$.
\end{example}

\begin{solution}
	Загальний розв'язок складається з суми загального роз\-в'яз\-ку однорідного та частинного роз\-в'яз\-ку неоднорідного рівнянь. \\

	Розглянемо однорідне рівняння
	\begin{equation*}
		y'' - 2 y' + y = 0.
	\end{equation*}
	
	Його характеристичне рівняння має вигляд
	\begin{equation*}
		\lambda^2 - 2 \lambda + 1 = 0.
	\end{equation*}

	Його коренями будуть $\lambda_1 = 1$, $\lambda_2 = 1$. І загальний роз\-в'яз\-ок однорідного має вигляд $y_{\text{homo}}(x) = C_1 \cdot e^x + C_2 \cdot x \cdot e^x$.  \\

	Частинний розв’язок неоднорідного рівняння шукаємо методом варіації довільної сталої у вигляді $y_{\text{part}}(x) = C_1(x) \cdot e^x + C_2(x) \cdot x \cdot e^x$. Для знаходження функцій $C_1(x)$, $C_2(x)$ отримаємо систему 
	\begin{equation*}
		\left\{
			\begin{aligned}
				C_1'(x) \cdot e^x + C_2'(x) \cdot x \cdot e^x &= 0, \\
				C_1'(x) \cdot e^x + C_2'(x) \cdot \left( x \cdot e^x + e^x \right) &= \frac{e^x}{x}.
			\end{aligned}
		\right.
	\end{equation*}
	Звідси
	\begin{align*}
		C_1(x) &= \int \frac{\begin{vmatrix} 0 & x \cdot e^x \\ \frac{e^x}{x} & x \cdot e^x + e^x \end{vmatrix}}{\begin{vmatrix} e^x & x \cdot e^x \\ e^x & x \cdot e^x + e^x \end{vmatrix}} \diff x = \int \frac{e^{2x}}{e^{2x}} \diff x = x + \bar C_1, \\
		C_2(x) &= \int \frac{\begin{vmatrix} e^x & 0 \\ e^x & \frac{e^x}{x} \end{vmatrix}}{\begin{vmatrix} e^x & x \cdot e^x \\ e^x & x \cdot e^x + e^x \end{vmatrix}} \diff x = \int \frac{e^{2x}}{x \cdot e^{2x}} \diff x = \ln |x| + \bar C_2.
	\end{align*}

	Поклавши (для зручності) $\bar C_1 = 0$, $\bar C_2 = 0$, одержимо
	\begin{equation*}
		y_{\text{part}}(x) = x \cdot e^x + \ln |x| \cdot x \cdot e^x.
	\end{equation*}
	Загальний розв’язок має вигляд
	\begin{equation*}
		y_{\text{hetero}}(x) = C_1 \cdot e^x + C_2 \cdot x \cdot e^x + \ln |x| \cdot x \cdot e^x.
	\end{equation*}
\end{solution}

\begin{example}
	Знайти загальний розв’язок рівняння \[y'' + 3 y' + 2 y = \frac{1}{e^x + 1}.\]
\end{example}
\begin{solution}
	Загальний розв’язок складається з суми загального роз\-в'яз\-ку однорідного та частинного роз\-в'яз\-ку неоднорідного. Розглянемо однорідне рівняння
	\begin{equation*}
		y'' + 3 y ' + 2 y = 0.
	\end{equation*}
	Його характеристичне рівняння має вигляд
	\begin{equation*}
		\lambda^2 + 3 \lambda + 2 = 0.
	\end{equation*}
	Його коренями будуть $\lambda_1 = - 1$, $\lambda_2 = -2$. І загальний розв’язок однорідного має вигляд $y_{\text{homo}}(x) = C_1 \cdot e^{-x} + C_2 \cdot e^{-2x}$. \\

	Частинний розв’язок неоднорідного рівняння шукаємо методом Коші. Враховуючи вигляд загального роз\-в'яз\-ку однорядного рівняння функцію $K(x, s)$ шукаємо у вигляді
	\begin{equation*}
		K(x, s) = C_1(s) \cdot e^{-x} + C_2(s) \cdot e^{-2x}.
	\end{equation*}

	Початкові умови дають наступне
	\begin{align*}
		K(s, s) = 0 &\implies C_1(s) \cdot e^{-x} + C_2(s) \cdot e^{-2s} = 0, \\
		K_x'(s, s) = 1 &\implies C_1(s) \cdot e^{-x} - 2 C_2(s) \cdot e^{-2s} = 1, \\
	\end{align*}

	Звідси
	\begin{align*}
		C_1(s) &= \frac{\begin{vmatrix} e^{-s} & 0 \\ -e^{-s} & 1 \end{vmatrix}}{\begin{vmatrix} e^{-s} & e^{-2s} \\ -e^{-s} & -2e^{-2s} \end{vmatrix}} = \frac{e^{-s}}{-e^{-3s}} = -e^{2s}.
	\end{align*}

	Таким чином $K(x, s) = e^{s - x} - e^{2(s - x)}$. І частинний роз\-в'яз\-ок, що задовольняє нульовим початковим умовам, має вигляд
	\begin{align*}
		y_{\text{part}}(x) &= \int \frac{e^{s - x} - e^{2(s - x)}}{e^s + 1} \diff s = e^{-x} \int_{x_0}^x \frac{e^s}{e^s + 1} \diff s - e^{-2x} \frac{e^{2 s}}{e^s + 1} \diff s = \\ &= e^{-x} \cdot \left. \ln |e^s + 1| \right|_{s = x_0}^{s = x} - e^{-2x} \cdot \int_{x_0}^x \frac{e^s + 1 - 1}{e^s + 1} \diff (e^s) = \\ &= e^{-x} \cdot \left( \ln |e^x + 1| - \ln |e^{x_0} - 1| \right) + \\ & \quad + e^{-2x} \cdot \left( e^x - e^{x_0} - \ln |e^x + 1| + \ln |e^{x_0} + 1| \right).
	\end{align*}
	Враховуючи, що початкові дані не задані, остаточно отримаємо
	\begin{equation*}
		y_{\text{hetero}}(x) = C_1 \cdot e^{-x} + C_2 \cdot e^{-2x} + e^{-x} \cdot \ln |e^x + 1| + e^{-2x} \cdot \ln |e^x + 1|.
	\end{equation*}
\end{solution}

Розв’язати лінійні неоднорідні рівняння
\begin{multicols}{2}
\begin{problem}
	\[y'' + y = \frac{1}{\sin x};\]
\end{problem}
\begin{problem}
	\[y'' + 4 y = 2 \tan (x);\]
\end{problem}
\begin{problem}
	\[y'' + 2 y' + y = 3 \cdot e^{-x \cdot \sqrt{x + 1}};\]
\end{problem}
\begin{problem}
	\[y'' + y = 2 \sec^3(x);\]
\end{problem}
\begin{problem}
	\[y'' - y= \frac{x^2 - 2}{x^3}.\]
\end{problem}
\end{multicols}

Якщо рівняння зі сталими коефіцієнтами, а функція $b(x)$ спеціального вигляду, то зручніше використовувати метод невизначених коефіцієнтів.

\begin{example}
	Розв’язати лінійне неоднорідне рівняння \[y'' + 2 y' + y = x^2 + 1.\]
\end{example}
\begin{solution}
	Спочатку розв’язуємо однорідне рівняння
	\begin{equation*}
		y'' + 2 y' + y = 0.	
	\end{equation*}

	Його характеристичне рівняння має вигляд
	\begin{equation*}
		\lambda^2 + 2 \lambda + 1 = 0.
	\end{equation*}

	Його коренями будуть $\lambda_1 = -1$, $\lambda_2 = -1$. І загальним роз\-в'яз\-ком однорідного рівняння буде $y_{\text{homo}}(x) = C_1 \cdot e^{-x} + C_2 \cdot x \cdot e^{-x}$. Оскільки справа стоїть многочлени другого ступеня і характеристичне рівняння не містить нульових коренів, то частинний роз\-в'яз\-ок має вигляд
	\begin{equation*}
		y_{\text{part}}(x) = a x^2 + b x + c.
	\end{equation*}

	Звідси
	\begin{equation*}
		y_{\text{part}}'(x) = 2 a x + b.	
	\end{equation*}

	Підставляємо одержані вирази в диференціальне рівняння
	\begin{equation*}
		2 a + 2 (2 a x + b) + (a x^2 + b x + c) = x^2 + 1
	\end{equation*}

	Прирівнюємо коефіцієнти при однакових степенях
	\begin{table}[H]
		\centering
		\begin{tabular}{c|l}
			$x^2$ & $a = 1$ \\
			$x$ & $4 a + b = 0$ \\
			$1$ & $a + 2 b + c = 1$
		\end{tabular}
	\end{table}

	Звідси $a = 1$, $b = - 4$, $c = 7$. \\

	Таким чином загальний розв’язок має вигляд
	\begin{equation*}
		y_{\text{hetero}}(x) = C_1 \cdot e^{-x} + C_2 \cdot x \cdot e^{-x} + x^2 - 4 x + 7.
	\end{equation*}
\end{solution}

\begin{example}
	Розв’язати лінійне неоднорідне рівняння \[y''' + y'' = x + 1.\]
\end{example}

\begin{solution}
	Розв’язуємо однорідне рівняння
	\begin{equation*}
		y''' + y'' = 0.
	\end{equation*}

	Його характеристичне рівняння має вигляд
	\begin{equation*}
		\lambda^3 + \lambda^2 = 0
	\end{equation*}
	Його коренями будуть $\lambda_1 = \lambda_2 = 0$, $\lambda_3 = 1$. І загальним роз\-в'яз\-ком однорідного рівняння буде
	\begin{equation*}
		y_{\text{homo}}(x) = C_1 + C_2 \cdot x + C_3 \cdot e^{-x}.
	\end{equation*}

	Оскільки справа стоїть многочлен другого порядку, а характеристичне рівняння має нульовий корінь кратності два, то частинний розв’язок має вигляд 
	\begin{equation*}
		y_{\text{part}}(x) = x^2 \cdot (a x + b),
	\end{equation*}
	або
	\begin{equation*}
		y_{\text{part}}(x) = a x^3 + b x^2.
	\end{equation*}

	Звідси
	\begin{align*}
		y_{\text{part}}'(x) &= 3 a x^2 + 2 b x, \\
		y_{\text{part}}''(x) &= 6 a x + 2 b.
	\end{align*}

	Підставляємо одержані вирази в диференціальне рівняння
	\begin{equation*}
		6 a + (6 a x + 2 b) = x + 1.
	\end{equation*}

	Прирівнюємо коефіцієнти при однакових ступенях
	\begin{table}[H]
		\centering
		\begin{tabular}{c|l}
			$x$ & $6 a = 1$ \\
			$1$ & $6 a + 2 b = 1$
		\end{tabular}
	\end{table}

	Звідси $a = \frac16$, $b = 0$. \\

	Таким чином загальний розв’язок має вигляд
	\begin{equation*}
		y_{\text{hetero}}(x) = C_1 + C_2 \cdot x + C_3 \cdot e^{-x} + \frac{x^3}{6}
	\end{equation*}
\end{solution}

\begin{example}
	Розв’язати лінійне неоднорідне рівняння $y'' + y = e^x \cdot x$.
\end{example}
\begin{solution}
	Розв’язуємо лінійне однорідне рівняння
	\begin{equation*}
		y'' + y = 0.
	\end{equation*}
	
	Характеристичне рівняння має вигляд
	\begin{equation*}
		\lambda^2 + 1 = 0.
	\end{equation*}

	Його коренями будуть $\lambda_{1, 2} = \pm i$. І загальним роз\-в'яз\-ком однорідного рівняння буде
	\begin{equation*}
		y_{\text{homo}}(x) = C_1 \cdot \cos (x) + C_2 \cdot \sin (x).
	\end{equation*}

	Оскільки справа стоїть многочлен першого порядку, помножений на експоненту, то частинний роз\-в'яз\-ок має вигляд
	\begin{equation*}
		y_{\text{part}}(x) = e^x \cdot (a x + b).
	\end{equation*}

	Звідси
	\begin{align*}
		y_{\text{part}}'(x) &= e^x \cdot (a x + a + b), \\
		y_{\text{part}}'(x) &= e^x \cdot (a x + 2 a + b).
	\end{align*}

	Підставляємо одержані вирази у диференціальне рівняння
	\begin{equation*}
		e^x \cdot (a x + 2 a + b) + e^x \cdot (a x + b) = e^x \cdot x.
	\end{equation*}

	Прирівнюємо коефіцієнти при однакових членах
	\begin{table}[H]
		\centering
		\begin{tabular}{c|l}
			$x \cdot e^x$ & $2 a = 1$ \\
			$e^x$ & $2 a + 2 b = 0$
		\end{tabular}
	\end{table}

	Звідси $a = \frac12$, $b = - \frac12$. \\

	Таким чином загальний розв’язок має вигляд
	\begin{equation*}
		y_{\text{hetero}}(x) = C_1 \cdot \cos (x) + C_2 \cdot \sin (x) + \frac{e^x \cdot (x - 1)}{2}.
	\end{equation*}
\end{solution}

\begin{example}
	Розв’язати лінійне неоднорідне рівняння \[ y'' - 2 y' + y = e^x \cdot x.\]
\end{example}
\begin{solution}
	Розв’язуємо однорідне рівняння
	\begin{equation*}
		y'' - 2 y' + y = 0.
	\end{equation*}
	
	Характеристичне рівняння має вигляд
	\begin{equation*}
		\lambda^2 - 2 \lambda + 1 = 0.
	\end{equation*}
	
	Його коренями будуть $\lambda_1 = 1$, $\lambda_2 = 1$. І загальним роз\-в'яз\-ком однорідного рівняння буде
	\begin{equation*}
		y_{\text{homo}}(x) = C_1 \cdot e^x + C_2 \cdot x \cdot e^x.
	\end{equation*}

	Оскільки справа стоїть многочлен першого порядку, а показник при експоненті є двократним коренем характеристичного рівняння, частинний розв’язок має вигляд
	\begin{equation*}
		y_{\text{part}}(x) = x^2 \cdot e^x \cdot (a x + b),
	\end{equation*}
	або
	\begin{equation*}
		y_{\text{part}}(x) = e^x \cdot (a x^3 + b x^2),
	\end{equation*}

	Звідси
	\begin{align*}
		y_{\text{part}}'(x) &= e^x \cdot (a x^3 + (3 a + b) x^2 + 2 b x), \\
		y_{\text{part}}''(x) &= e^x \cdot (a x^3 + (6 a + b) x^2 + (6 a + 4 b) x + 2 b).
	\end{align*}

	Підставляємо одержані вирази в диференціальне рівняння
	\begin{multline*}
		e^x \cdot (a x^3 + (6 a + b) x^2 + (6 a + 4 b) x + 2 b) - 2 e^x \cdot (a x^3 + (3 a + b) x^2 + 2 b x) + \\ + e^x \cdot (a x^3 + b x^2) = e^x \cdot x.
	\end{multline*}

	Прирівнюємо коефіцієнти при однакових членах
	\begin{table}[H]
		\centering
		\begin{tabular}{c|l}
			$x \cdot e^x$ & $6 a + 4 b + 2 b = 1$ \\
			$e^x$ & $2 b = 0$
		\end{tabular}
	\end{table}

	Звідси $a = \frac16$, $b = 0$. \\

	Таким чином загальний розв’язок має вигляд
	\begin{equation}
		y_{\text{hetero}}(x) = C_1 \cdot e^x + C_2 \cdot x \cdot e^x + \frac{x^3 \cdot e^x}{6}.
	\end{equation}
\end{solution}

\begin{example}
	Розв’язати лінійне неоднорідне рівняння \[ y'' - y = x \cdot \cos(x) + \sin (x).\]
\end{example}
\begin{solution}
	Розв’язуємо однорідне рівняння
	\begin{equation*}
		y'' - y = 0.
	\end{equation*}

	Характеристичне рівняння має вигляд
	\begin{equation*}
		\lambda^2 - 1 = 0.
	\end{equation*}

	Його коренями будуть $\lambda_1 = 1$, $\lambda_2 = -1$. І загальним роз\-в'яз\-ком однорідного рівняння буде
	\begin{equation*}
		y_{\text{homo}}(x) = C_1 \cdot e^x + C_2 \cdot e^{-x}.
	\end{equation*}

	Частинний розв’язок неоднорідного має вигляд
	\begin{equation*}
		y_{\text{part}}(x) = (a x + b) \cdot \cos (x) + (c x + d) \cdot \sin(x).
	\end{equation*}

	Звідси
	\begin{align*}
		y_{\text{part}}'(x) &= (c x + a + d) \cdot \cos (x) + (-a x - b + c) \cdot \sin(x), \\
		y_{\text{part}}''(x) &= (-a x - b + 2 c) \cdot \cos (x) + (- c x - 2 a - d) \cdot \sin(x)
	\end{align*}

	Підставляємо одержані вирази в диференціальне рівняння
	\begin{multline*}
		(-a x - b + 2 c) \cdot \cos (x) + (- c x - 2 a - d) \cdot \sin(x) - \\
		- (a x + b) \cdot \cos (x) - (c x + d) \cdot \sin(x) = x \cdot \cos(x) + \sin (x).
	\end{multline*}
		 
	Прирівнюємо коефіцієнти при однакових виразах
	\begin{table}[H]
		\centering
		\begin{tabular}{c|l}
			$x \cdot \cos (x)$ & $- 2 a = 1$ \\
			$x \cdot \sin (x)$ & $- 2 c = 0$ \\
			$\cos (x)$ & $- b + 2c - b = 0$ \\
			$\sin (x)$ & $- 2 a - d - d = 1$
		\end{tabular}
	\end{table}

	Звідси $a = - \frac12$, $b = c = d = 0$. \\

	Таким чином загальний розв’язок має вигляд
	\begin{equation*}
		y_{\text{hetero}}(x) = C_1 \cdot e^x + C_2 \cdot x \cdot e^{x} - \frac{\cos (x)}{2}.
	\end{equation*}
\end{solution}

\begin{example}
	Розв’язати диференціальне рівняння \[y'' + 2 y' + 2 y = e^{-x} \cdot \sin (x).\]
\end{example}
\begin{solution}
	Розв’язуємо однорідне рівняння
	\begin{equation*}
		y'' + 2 y' + 2y = 0.
	\end{equation*}

	Характеристичне рівняння $\lambda^2 + 2 \lambda + 2 = 0$ має корені $\lambda_{1,2} = -1\pm i$. І загальним розв’язком однорідного рівняння буде
	\begin{equation*}
		y_{\text{homo}}(x) = C_1 \cdot e^{-x} \cdot \cos(x) + C_2 \cdot e^{-x} \cdot \sin(x).
	\end{equation*}

	Оскільки $\lambda_1 = 1 + i$ корінь кратності один, то частинний роз\-в'яз\-ок неоднорідного має вигляд
	\begin{equation*}
		y_{\text{part}}(x) = x \cdot e^{-x} \cdot (a \cdot \cos(x) + b \cdot \sin(x)).
	\end{equation*}

	Звідси
	\begin{align*}
		y_{\text{part}}'(x) &= e^{-x} \cdot ((b - a x) \cdot \sin(x) + (a - (a - b) \cdot x) \cdot \cos(x)) \\
		y_{\text{part}}'(x) &= -2 e^{-x} \cdot ((a + b - a x) \cdot \sin(x) + ((a - b) + b \cdot x) \cdot \cos(x))
	\end{align*}

	Підставляємо одержані вирази в диференціальне рівняння
	\begin{multline*}
		-2 e^{-x} \cdot ((a + b - a x) \cdot \sin(x) + ((a - b) + b \cdot x) \cdot \cos(x)) + \\ + 2 e^{-x} \cdot ((b - a x) \cdot \sin(x) + (a - (a - b) \cdot x) \cdot \cos(x)) + \\ + 2 x \cdot e^{-x} \cdot (a \cdot \cos(x) + b \cdot \sin(x)) = e^{-x} \cdot \sin(x). 
	\end{multline*}

	Прирівнюємо коефіцієнти при однакових членах
	\begin{table}[H]
		\centering
		\begin{tabular}{c|l}
			$e^{-x} \cdot \cos(x)$ & $2a + 2b = 0$ \\
			$e^{-x} \cdot \sin(x)$ & $-2 a - 2 b + c = 1$
		\end{tabular}
	\end{table}

	Звідси $a = -1$, $b = 1$. \\

	Таким чином загальний розв’язок має вигляд
	\begin{equation*}
		y_{\text{hetero}}(x) = C_1 \cdot e^{-x} \cdot \cos(x) + C_2 \cdot e^{-x} \cdot \sin(x) + x \cdot e^{-x} \cdot \left( \sin(x) - \cos(x) \right).
	\end{equation*}
\end{solution}

Знайти загальний розв’язок рівнянь:

\begin{multicols}{2}
\begin{problem}
	\[y'''-4y''+5y'-2y=2x+3;\]
\end{problem}
\begin{problem}
	\[y'''-3y'+2y=e^{-x}(4x^2+4x-10);\]
\end{problem}
\begin{problem}
	\[y^{(4)}+8y''+16y=\cos(x);\]
\end{problem}
\begin{problem}
	\[y^{(5)}+y'''=x^2-1;\]
\end{problem}
\begin{problem}
	\[y^{(4)}-y=x\cdot e^x+\cos(x);\]
\end{problem}
\begin{problem}
	\[y^{(4)}+2y''+y=x^2\cdot\cos(x);\]
\end{problem}
\begin{problem}
	\[y^{(4)}-y=5e^x\cdot\sin(x)+x^4;\]
\end{problem}
\begin{problem}
	\[y^{(4)}+5y''+4y=\sin(x)\cdot\cos(2x);\]
\end{problem}
\begin{problem}
	\[y'''-4y''+3y'=x^3 \cdot e^{2x};\]
\end{problem}
\begin{problem}
	\[y^{(4)}+y''=7x-3\cos(x);\]
\end{problem}
\begin{problem}
	\[y'''-y''-y'+y=3e^x+5x\cdot\sin(x);\]
\end{problem}
\begin{problem}
	\[y'''-2y''+4y'-8y=e^{2x}\sin(2x)+2x^2;\]
\end{problem}
\begin{problem}
	\[y'''+y'=\sin(x)+x\cdot\cos(x);\]
\end{problem}
\begin{problem}
	\[y'''-y=x^3-1;\]
\end{problem}
\begin{problem}
	\[y'''+y''=x^2+1+3x\cdot e^x;\]
\end{problem}
\begin{problem}
	\[y'''+y''+y'+y=x\cdot e^x;\]
\end{problem}
\begin{problem}
	\[y'''-9y'=-9(e^{3x}-2\sin3x+\cos3x);\]
\end{problem}
\begin{problem}
	\[y'''-y'=10\sin(x)+6\cos(x)+4e^x;\]
\end{problem}
\begin{problem}
	\[y'''-6y''+9y'=4x\cdot e^x;\]
\end{problem}
\begin{problem}
	\[y'''+2y''-3y'=(8x+6)\cdot e^x;\]
\end{problem}
\begin{problem}
	\[y^{(4)}+y''=x^2+x;\]
\end{problem}
\begin{problem}
	\[y'''-3y'+2y=(2x^2-x) e^x+\cos(x);\]
\end{problem}
\begin{problem}
	\[y^{(4)}-y=5e^x\cdot\cos(x)+3;\]
\end{problem}
\begin{problem}
	\[y^{(5)}-y'''=x^2+\cos(x);\]
\end{problem}
\begin{problem}
	\[y^{(4)}-2y''+y'=e^x;\]
\end{problem}
\begin{problem}
	\[y^{(4)}-2y'''+y''=x^3;\]
\end{problem}
\begin{problem}
	\[y^{(4)}+y'''=\cos(3x).\]
\end{problem}
\end{multicols}

Знайти частинний розв’язок диференціальних рівнянь:
\begin{problem}
	\[y'''-2y''+y'=4(\sin(x)+\cos(x)),\quad y(0)=1,y'(0)=0,y''(0)=-1;\]
\end{problem}
\begin{problem}
	\[y'''+2y''+y'=-2e^{-2x},\quad y(0)=2,y'(0)=y''(0)=1;\]
\end{problem}
\begin{problem}
	\[y'''-3y'=3(2-x^2),\quad y(0)=y'(0)=y''(0)=1;\]
\end{problem}
\begin{problem}
	\[y'''+2y''+y'=5e^x,\quad y(0)=y'(0)=y''(0)=0;\]
\end{problem}
\begin{problem}
	\[y'''-y'=3(2-x^2),\quad y(0)=y'(0)=y''(0)=1;\]
\end{problem}
\begin{problem}
	\[y'''+2y''+2y'+y=x,\quad y(0)=y'(0)=y''(0)=0.\]
\end{problem}


\section{Системи диференціальних рівнянь}
\subsection{Загальна теорія}

Співвідношення вигляду
\begin{equation*}
	\left\{
		\begin{aligned}
			F_1(\dot x_1, \dot x_2, \ldots, \dot x_n, x_1, x_2, \ldots, x_n, t) &= 0, \\
			F_2(\dot x_1, \dot x_2, \ldots, \dot x_n, x_1, x_2, \ldots, x_n, t) &= 0, \\
			\ldots \ldots \ldots \ldots \ldots \ldots \ldots \ldots \ldots \ldots \ldots & \ldots . \\
			F_n(\dot x_1, \dot x_2, \ldots, \dot x_n, x_1, x_2, \ldots, x_n, t) &= 0
		\end{aligned}
	\right.
\end{equation*}
називається системою $n$ звичайних диференціальних рівнянь першого порядку. \\

Якщо система розв'язана відносно похідних і має вигляд
\begin{equation*}
	\left\{
		\begin{aligned}
			\dot x_1 &= f_1(x_1, x_2, \ldots, x_n, t) &= 0, \\
			\dot x_2 &= f_2(x_1, x_2, \ldots, x_n, t) &= 0, \\
			\ldots & \ldots \ldots \ldots \ldots \ldots \ldots \ldots \ldots &\ldots . \\
			\dot x_n &= f_n(x_1, x_2, \ldots, x_n, t) &= 0
		\end{aligned}
	\right.
\end{equation*}
то вона називається системою в нормальній формі.

\begin{definition}
	Розв'язком системи диференціальних рівнянь на\-зи\-ва\-є\-ть\-ся набір $n$ неперервно диференційованих функцій $x_1(t), \ldots, x_n(t)$ що тотожно задовольняють кожному з рівнянь системи.
\end{definition}

У загальному випадку розв'язок системи залежить від $n$ довільних сталих і має вигляд  $x_1(t, C_1, \ldots, C_n), \ldots, x_n(t, C_1, \ldots, C_n)$ і задача Коші для системи звичайних диференціальних рівнянь першого порядку ставиться в такий спосіб. Потрібно знайти розв'язок, що задовольняє початковим умовам (умовам Коші):
\begin{equation*}
	x_1(t_0) = x_1^0, \quad x_2(t_0) = x_2^0, \quad \ldots, \quad x_n(t_0) = x_n^0
\end{equation*}

\begin{definition}
	Розв'язок $x_1(t, C_1, \ldots, C_n)$, $\ldots$, $x_n(t, C_1, \ldots, C_n)$ на\-зи\-ва\-є\-ть\-ся загальним, якщо за рахунок вибору сталих $C_1, \ldots, C_n$ можна роз\-в'яз\-а\-ти довільну задачу Коші.
\end{definition}

Для систем звичайних диференціальних рівнянь досить важливим є поняття інтеграла системи. В залежності від гладкості (тобто диференційованості) можна розглядати два визначення інтеграла.

\begin{definition}
	\begin{enumerate}
		\item Функція $F(x_1, x_2, \ldots, x_n, t)$ стала уздовж розв'язків системи, називається інтегралом системи.
		\item Функція $F(x_1, x_2, \ldots, x_n, t)$ повна похідна, якої в силу системи тотожно дорівнює нулю, називається інтегралом системи.
	\end{enumerate}
\end{definition}

Для лінійних рівнянь існує поняття лінійної залежності і незалежності розв'язків. Для нелінійних рівнянь (систем рівнянь) аналогічним поняттям є функціональна незалежність.

\begin{definition}
	Інтеграли 
	\begin{equation*}
		F_1(x_1, x_2, \ldots, x_n, t), \quad F_2(x_1, x_2, \ldots, x_n, t), \quad \ldots, \quad F_n(x_1, x_2, \ldots, x_n, t)
	\end{equation*}
	називаються функціонально незалежними, якщо не існує функції $n$ змінних $\Phi(z_1, z_2, \ldots, z_n)$ такої, що
	\begin{equation*}
		\Phi(F_1(x_1, x_2, \ldots, x_n, t), F_2(x_1, x_2, \ldots, x_n, t), \ldots, F_n(x_1, x_2, \ldots, x_n, t)) = 0.
	\end{equation*}
\end{definition}

\begin{theorem}
	Для того щоб інтеграли 
	\begin{equation*}
		F_1(x_1, x_2, \ldots, x_n, t), \quad F_2(x_1, x_2, \ldots, x_n, t), \quad \ldots, \quad F_n(x_1, x_2, \ldots, x_n, t)
	\end{equation*}
	системи звичайних диференціальних рівнянь були функціонально незалежними, необхідно і достатньо, щоб визначник Якобі був відмінний від тотожного нуля, тобто 
	\begin{equation*}
		\frac{D(F_1, F_2, \ldots, F_n)}{D(x_1, x_2, \ldots, x_n)} \ne 0.
	\end{equation*}
\end{theorem}

\begin{definition}
	Якщо $F(x_1, x_2, \ldots, x_n, t)$ -- інтеграл системи диференціальних рівнянь, то рівність $F(x_1, x_2, \ldots, x_n, t) = C$ називається першим інтегралом.
\end{definition}

\begin{definition}
	Сукупність $n$ функціонально незалежних інтегралів називається загальним інтегралом системи диференціальних рівнянь.
\end{definition}

Власне кажучи загальний інтеграл -- це загальний роз\-в'яз\-ок системи диференціальних рівнянь у неявному вигляді.

\begin{theorem}[існування та єдиності розв'язку задачі Коші]
	Щоб система диференціальних рівнянь, розв'язаних відносно похідної, мала єдиний роз\-в'яз\-ок, що задовольняє умовам Коші: 
	\begin{equation*}
		x_1(t_0) = x_1^0, \quad x_2(t_0) = x_2^0, \quad \ldots, \quad x_n(t_0) = x_n^0
	\end{equation*}
	досить, щоб:
	\begin{enumerate}
		\item функції $f_1, f_2, \ldots, f_n$ були неперервними по змінним $x_1, x_2, \ldots, x_n, t$ в околі точки $\left(x_1^0, x_2^0, \ldots, x_n^0, t_0\right)$;
		\item функції $f_1, f_2, \ldots, f_n$ задовольняли умові Ліпшиця по аргументах $x_1, x_2, \ldots, x_n$ у тому ж околі.
	\end{enumerate}
\end{theorem}

\begin{remark}
	Умову Ліпшиця можна замінити більш грубою умовою, але такою, що перевіряється легше, існування обмежених частинних похідних, тобто 
	\begin{equation*}
		\left| \frac{\partial f_i}{\partial x_j} \right| \le M, \quad i,j = 1, 2, \ldots, n.
	\end{equation*}
\end{remark}

	\subsection{Загальна теорія}
	Співвідношення вигляду
\begin{equation*}
	\left\{
		\begin{aligned}
			F_1(\dot x_1, \dot x_2, \ldots, \dot x_n, x_1, x_2, \ldots, x_n, t) &= 0, \\
			F_2(\dot x_1, \dot x_2, \ldots, \dot x_n, x_1, x_2, \ldots, x_n, t) &= 0, \\
			\ldots \ldots \ldots \ldots \ldots \ldots \ldots \ldots \ldots \ldots \ldots & \ldots . \\
			F_n(\dot x_1, \dot x_2, \ldots, \dot x_n, x_1, x_2, \ldots, x_n, t) &= 0
		\end{aligned}
	\right.
\end{equation*}
називається системою $n$ звичайних диференціальних рівнянь першого порядку. \parvskip

Якщо система розв'язана відносно похідних і має вигляд
\begin{equation*}
	\left\{
		\begin{aligned}
			\dot x_1 &= f_1(x_1, x_2, \ldots, x_n, t) &= 0, \\
			\dot x_2 &= f_2(x_1, x_2, \ldots, x_n, t) &= 0, \\
			\ldots & \ldots \ldots \ldots \ldots \ldots \ldots \ldots \ldots &\ldots . \\
			\dot x_n &= f_n(x_1, x_2, \ldots, x_n, t) &= 0
		\end{aligned}
	\right.
\end{equation*}
то вона називається системою в нормальній формі.

\begin{definition}
	Розв'язком системи диференціальних рівнянь на\-зи\-ва\-є\-ть\-ся набір $n$ неперервно диференційованих функцій $x_1(t), \ldots, x_n(t)$ що тотожно задовольняють кожному з рівнянь системи.
\end{definition}

У загальному випадку розв'язок системи залежить від $n$ довільних сталих і має вигляд  $x_1(t, C_1, \ldots, C_n), \ldots, x_n(t, C_1, \ldots, C_n)$ і задача Коші для системи звичайних диференціальних рівнянь першого порядку ставиться в такий спосіб. Потрібно знайти розв'язок, що задовольняє початковим умовам (умовам Коші):
\begin{equation*}
	x_1(t_0) = x_1^0, \quad x_2(t_0) = x_2^0, \quad \ldots, \quad x_n(t_0) = x_n^0
\end{equation*}

\begin{definition}
	Розв'язок $x_1(t, C_1, \ldots, C_n)$, $\ldots$, $x_n(t, C_1, \ldots, C_n)$ на\-зи\-ва\-є\-ть\-ся загальним, якщо за рахунок вибору сталих $C_1, \ldots, C_n$ можна роз\-в'яз\-а\-ти довільну задачу Коші.
\end{definition}

Для систем звичайних диференціальних рівнянь досить важливим є поняття інтеграла системи. В залежності від гладкості (тобто диференційованості) можна розглядати два визначення інтеграла.

\begin{definition}
	\begin{enumerate}
		\item Функція $F(x_1, x_2, \ldots, x_n, t)$ стала уздовж розв'язків системи, називається інтегралом системи.
		\item Функція $F(x_1, x_2, \ldots, x_n, t)$ повна похідна, якої в силу системи тотожно дорівнює нулю, називається інтегралом системи.
	\end{enumerate}
\end{definition}

Для лінійних рівнянь існує поняття лінійної залежності і незалежності розв'язків. Для нелінійних рівнянь (систем рівнянь) аналогічним поняттям є функціональна незалежність.

\begin{definition}
	Інтеграли 
	\begin{equation*}
		F_1(x_1, x_2, \ldots, x_n, t), \quad F_2(x_1, x_2, \ldots, x_n, t), \quad \ldots, \quad F_n(x_1, x_2, \ldots, x_n, t)
	\end{equation*}
	називаються функціонально незалежними, якщо не існує функції $n$ змінних $\Phi(z_1, z_2, \ldots, z_n)$ такої, що
	\begin{equation*}
		\Phi(F_1(x_1, x_2, \ldots, x_n, t), F_2(x_1, x_2, \ldots, x_n, t), \ldots, F_n(x_1, x_2, \ldots, x_n, t)) = 0.
	\end{equation*}
\end{definition}

\begin{theorem}
	Для того щоб інтеграли 
	\begin{equation*}
		F_1(x_1, x_2, \ldots, x_n, t), \quad F_2(x_1, x_2, \ldots, x_n, t), \quad \ldots, \quad F_n(x_1, x_2, \ldots, x_n, t)
	\end{equation*}
	системи звичайних диференціальних рівнянь були функціонально незалежними, необхідно і достатньо, щоб визначник Якобі був відмінний від тотожного нуля, тобто 
	\begin{equation*}
		\frac{D(F_1, F_2, \ldots, F_n)}{D(x_1, x_2, \ldots, x_n)} \ne 0.
	\end{equation*}
\end{theorem}

\begin{definition}
	Якщо $F(x_1, x_2, \ldots, x_n, t)$ --- інтеграл системи диференціальних рівнянь, то рівність $F(x_1, x_2, \ldots, x_n, t) = C$ називається першим інтегралом.
\end{definition}

\begin{definition}
	Сукупність $n$ функціонально незалежних інтегралів називається загальним інтегралом системи диференціальних рівнянь.
\end{definition}

Власне кажучи загальний інтеграл --- це загальний роз\-в'яз\-ок системи диференціальних рівнянь у неявному вигляді.

\begin{theorem}[існування та єдиності розв'язку задачі Коші]
	Щоб система диференціальних рівнянь, розв'язаних відносно похідної, мала єдиний роз\-в'яз\-ок, що задовольняє умовам Коші: 
	\begin{equation*}
		x_1(t_0) = x_1^0, \quad x_2(t_0) = x_2^0, \quad \ldots, \quad x_n(t_0) = x_n^0
	\end{equation*}
	досить, щоб:
	\begin{enumerate}
		\item функції $f_1, f_2, \ldots, f_n$ були неперервними по змінним $x_1, x_2, \ldots, x_n, t$ в околі точки $\left(x_1^0, x_2^0, \ldots, x_n^0, t_0\right)$;
		\item функції $f_1, f_2, \ldots, f_n$ задовольняли умові Ліпшиця по аргументах $x_1, x_2, \ldots, x_n$ у тому ж околі.
	\end{enumerate}
\end{theorem}

\begin{remark}
	Умову Ліпшиця можна замінити більш грубою умовою, але такою, що перевіряється легше, існування обмежених частинних похідних, тобто 
	\begin{equation*}
		\left| \frac{\partial f_i}{\partial x_j} \right| \le M, \quad i,j = 1, 2, \ldots, n.
	\end{equation*}
\end{remark}

		\subsubsection{Геометрична інтерпретація розв'язків}
		Назвемо $(n+1)$-вимірний простір змінних $x_1, x_2, \ldots, x_n, t$ розширеним фазовим простором $\RR^{n + 1}$. Тоді розв’язок $x_1 = x_1(t), x_2 = x_2(t), \ldots, x_n = x_n(t)$ визначає в просторі $\RR^{n + 1}$ деяку криву, що називається інтегральною кривою. Загальний розв’язок (чи загальний інтеграл) визначає сім’ю інтегральних кривих, що всюди щільно заповнюють деяку область $D \subseteq \RR^{n + 1}$  (область існування та єдиності розв’язків). Задача Коші ставиться як виділення із сім’ї інтегральних кривих, окремої кривої, що проходить через задану початкову точку $M \left(x_1^0, x_2^0, \ldots, x_n^0, t_0\right) \in D$.

		\subsubsection{Механічна інтерпретація розв'язків}
		В евклідовому просторі $\RR^n$ змінних $x_1(t), x_2(t), \ldots, x_n(t)$ розв’язок $x_1 = x_1(t), x_2 = x_2(t), \ldots, x_n = x_n(t)$ визначає закон руху по деякій траєкторії в залежності від часу $t$. При такій інтерпретації функції $f_1, f_2, \ldots, f_n$ є складовими швидкості руху, простір зміни перемінних називається фазовим простором, система динамічної, а крива, по якій відбувається рух $x_1 = x_1(t), x_2 = x_2(t), \ldots, x_n = x_n(t)$ -- фазовою траєкторією. Фазова траєкторія є проекцією інтегральної кривої на фазовий простір.


		\subsubsection{Зведення одного диференціального рівняння вищого порядку до системи рівнянь першого порядку}
		Нехай маємо диференціальне рівняння
\begin{equation*}
 	\frac{\diff^n y}{\diff x^n} = f \left( x, y, \frac{\diff y}{\diff x}, \frac{\diff^2 y}{\diff x^2}, \ldots, \frac{\diff^{n - 1} y}{\diff x^{n - 1}} \right).
\end{equation*}

Розглянемо заміну змінних
\begin{equation*}
	x \mapsto t, \quad y \mapsto x_1, \quad, \frac{\diff y}{\diff x} \mapsto x_2, \quad \ldots, \quad \frac{\diff^{n - 1} y}{\diff x^{n - 1}} \mapsto x_n.
\end{equation*}

Тоді одержимо систему рівнянь
\begin{equation*}
	\left\{
		\begin{aligned}
			\dot x_1 &= x_2, \\
			\dot x_2 &= x_3, \\
			\ldots & \ldots \ldots \\
			\dot x_{n - 1} &= x_n, \\
			\dot x_n &= f(t, x_1, x_2, \ldots, x_n).
		\end{aligned}
	\right.
\end{equation*}

		\subsubsection{Зведення системи диференціальних рівнянь до одного рівняння вищого порядку}
		Нехай маємо систему диференціальних рівнянь
\begin{equation*}
	\left\{
		\begin{aligned}
			\dot x_1 &= f_1 (x_1, x_2, \ldots, x_n, t), \\
			\dot x_2 &= f_2 (x_1, x_2, \ldots, x_n, t), \\
			\ldots & \ldots \ldots \ldots \ldots \ldots \ldots \ldots, \\
			\dot x_n &= f_n (x_1, x_2, \ldots, x_n, t).
		\end{aligned}
	\right.
\end{equation*}
і заданий її розв'язок $x_1 = x_1(t), x_2 = x_2(t), \ldots, x_n = x_n(t)$. Якщо цей розв'язок підставити в перше рівняння, то вийде тотожність і її можна диференціювати
\begin{equation*}
	\frac{\diff^2 x_1}{\diff t^2} = \frac{\partial f_1}{\partial t} + \sum_{i = 1}^n \frac{\partial f_1}{\partial x_i} \cdot \frac{\diff x_i(t)}{\diff t}.
\end{equation*}

Підставивши замість $\frac{\diff x_i(t)}{\diff t}$ їх значення, одержимо
\begin{equation*}
	\frac{\diff^2 x_1}{\diff t^2} = \frac{\partial f_1}{\partial t} + \sum_{i = 1}^n \frac{\partial f_1}{\partial x_i} \cdot f_i = F_2(t, x_1, x_2, \ldots, x_n).
\end{equation*}

Знову диференціюємо це рівняння й одержимо
\begin{equation*}
	\frac{\diff^3 x_1}{\diff t^3} = \frac{\partial F_2}{\partial t} + \sum_{i = 1}^n \frac{\partial F_2}{\partial x_i} \cdot \frac{\diff x_i(t)}{\diff t} = \frac{\partial F_2}{\partial t} + \sum_{i = 1}^n \frac{\partial F_2}{\partial x_i} \cdot f_i = F_3(t, x_1, x_2, \ldots, x_n).
\end{equation*}

Продовжуючи процес далі, одержимо
\begin{align*}
	\frac{\diff^{n - 1} x_1}{\diff t^{n - 1}} &= F_{n - 1}(t, x_1, x_2, \ldots, x_n), \\
	\frac{\diff^n x_1}{\diff t^n} &= F_n(t, x_1, x_2, \ldots, x_n).
\end{align*} 
 
Таким чином, маємо систему
\begin{equation*}
	\left\{
		\begin{aligned}
			\frac{\diff x_1}{\diff t} &= f_1 (x_1, x_2, \ldots, x_n, t), \\
			\frac{\diff^2 x_1}{\diff t^2} &= F_2(t, x_1, x_2, \ldots, x_n), \\
			\ldots \ldots & \ldots \ldots \ldots \ldots \ldots \ldots \ldots \ldots, \\
			\frac{\diff^{n - 1} x_1}{\diff t^{n - 1}} &= F_{n - 1}(t, x_1, x_2, \ldots, x_n), \\
			\frac{\diff^n x_1}{\diff t^n} &= F_n(t, x_1, x_2, \ldots, x_n).
		\end{aligned}
	\right.
\end{equation*}

Припустимо, що \[\frac{D(f_1, F_2, \ldots, F_{n - 1})}{D(x_2, x_3, \ldots, x_n)} \ne 0.\] Тоді систему перших $(n - 1)$ рівнянь 
\begin{equation*}
	\left\{
		\begin{aligned}
			\frac{\diff x_1}{\diff t} &= f_1 (x_1, x_2, \ldots, x_n, t), \\
			\frac{\diff^2 x_1}{\diff t^2} &= F_2(t, x_1, x_2, \ldots, x_n), \\
			\ldots \ldots & \ldots \ldots \ldots \ldots \ldots \ldots \ldots \ldots, \\
			\frac{\diff^{n - 1} x_1}{\diff t^{n - 1}} &= F_{n - 1}(t, x_1, x_2, \ldots, x_n).
		\end{aligned}
	\right.
\end{equation*}
можна розв'язати відносно останніх $(n - 1)$ змінних $x_2, x_3, \ldots, x_n$ і одержати
\begin{equation*}
	\left\{
		\begin{aligned}
			x_2 &= \phi_2 \left( t, x_1, \frac{\diff x_1}{\diff t}, \ldots, \frac{\diff^{n - 1} x_1}{\diff t^{n - 1}} \right), \\
			x_3 &= \phi_3 \left( t, x_1, \frac{\diff x_1}{\diff t}, \ldots, \frac{\diff^{n - 1} x_1}{\diff t^{n - 1}} \right), \\
			\ldots & \ldots \ldots \ldots \ldots \ldots \ldots \ldots \ldots \ldots \ldots, \\
			x_n &= \phi_n \left( t, x_1, \frac{\diff x_1}{\diff t}, \ldots, \frac{\diff^{n - 1} x_1}{\diff t^{n - 1}} \right), \\
		\end{aligned}
	\right.
\end{equation*}

Підставивши одержані вирази в останнє рівняння, запишемо
\begin{multline*}
	\frac{\diff^n x_1}{\diff t^n} = F_n \left( t, x_1, \phi_2 \left( t, x_1, \frac{\diff x_1}{\diff t}, \ldots, \frac{\diff^{n - 1} x_1}{\diff t^{n - 1}} \right), \ldots, \right. \\ \left. \phi_n \left( t, x_1, \frac{\diff x_1}{\diff t}, \ldots, \frac{\diff^{n - 1} x_1}{\diff t^{n - 1}} \right) \right).	
\end{multline*}

Або, після перетворень
\begin{equation*}
	\frac{\diff^n x_1}{\diff t^n} = \Phi \left( t, x_1, \frac{\diff x_1}{\diff t}, \ldots, \frac{\diff^{n - 1} x_1}{\diff t^{n - 1}} \right),
\end{equation*}
одержимо одне диференціальне рівняння $n$-го порядку. \parvskip

У загальному випадку, одержимо, що система диференціальних рівнянь першого порядку
\begin{equation*}
	\left\{
		\begin{aligned}
			\dot x_1 &= f_1 (x_1, x_2, \ldots, x_n, t), \\
			\dot x_2 &= f_2 (x_1, x_2, \ldots, x_n, t), \\
			\ldots & \ldots \ldots \ldots \ldots \ldots \ldots \ldots, \\
			\dot x_n &= f_n (x_1, x_2, \ldots, x_n, t).
		\end{aligned}
	\right.
\end{equation*}
зводиться до одного рівняння $n$-го порядку
\begin{equation*}
	\frac{\diff^n x_1}{\diff t^n} = \Phi \left( t, x_1, \frac{\diff x_1}{\diff t}, \ldots, \frac{\diff^{n - 1} x_1}{\diff t^{n - 1}} \right),
\end{equation*}
і системи $(n - 1)$ рівнянь зв'язку
\begin{equation*}
	\left\{
		\begin{aligned}
			x_2 &= \phi_2 \left( t, x_1, \frac{\diff x_1}{\diff t}, \ldots, \frac{\diff^{n - 1} x_1}{\diff t^{n - 1}} \right), \\
			x_3 &= \phi_3 \left( t, x_1, \frac{\diff x_1}{\diff t}, \ldots, \frac{\diff^{n - 1} x_1}{\diff t^{n - 1}} \right), \\
			\ldots & \ldots \ldots \ldots \ldots \ldots \ldots \ldots \ldots \ldots \ldots, \\
			x_n &= \phi_n \left( t, x_1, \frac{\diff x_1}{\diff t}, \ldots, \frac{\diff^{n - 1} x_1}{\diff t^{n - 1}} \right), \\
		\end{aligned}
	\right.
\end{equation*}
 
\begin{remark}
	Було зроблене припущення, що \[\frac{D(f_1, F_2, \ldots, F_{n - 1})}{D(x_2, x_3, \ldots, x_n)} \ne 0.\] Якщо ця умова не виконана, то можна зводити до рівняння щодо інших змінних, наприклад відносно $x_2$.
\end{remark}


		\subsubsection{Комбінації, що інтегруються}
		\begin{definition}
	Комбінацією, що інтегрується, називається диференціальне рівняння, отримане шляхом перетворень із системи, диференціальних рівнянь, але яке вже можна легко інтегрувати.
\end{definition}
\begin{equation*}
	\diff \Phi(t, x_1, x_2, \ldots, x_n) = 0.
\end{equation*}

Одна комбінація, що інтегрується, дає можливість одержати одне кінцеве рівняння
\begin{equation*}
	\Phi(t, x_1, x_2, \ldots, x_n) = C,
\end{equation*}
яке є першим інтегралом системи. \\

Геометрично перший інтеграл являє собою $n$-вимірну поверхню в $(n + 1)$-вимірному просторі, що цілком складається з інтегральних кривих. \\

Якщо знайдено $k$ комбінацій, що інтегруються, то одержуємо $k$ перших інтегралів
\begin{equation*}
	\left\{
		\begin{aligned}
			\Phi_1(t, x_1, x_2, \ldots, x_n) &= C_1, \\
			\Phi_2(t, x_1, x_2, \ldots, x_n) &= C_2, \\
			\ldots \ldots \ldots \ldots \ldots \ldots \ldots & \ldots \ldots, \\
			\Phi_n(t, x_1, x_2, \ldots, x_n) &= C_n.
		\end{aligned}
	\right.
\end{equation*}
 
І, якщо інтеграли незалежні, то хоча б один з визначників \[\frac{D(\Phi_1, \Phi_2, \ldots, \Phi_k)}{D(x_{i_1}, x_{i_2}, \ldots, x_{i_k})} \ne 0.\] Звідси з системи можна виразити $k$ невідомих функцій $x_{i_1}, x_{i_2}, \ldots, x_{i_k}$ через інші і підставивши їх у вихідну систему, понизити порядок до $(n - k)$ рівнянь. Якщо $n = k$ і всі інтеграли незалежні, то одержимо загальний інтеграл системи. \\

Особливо поширеним засобом знаходження комбінацій, що інтегруються, є використання систем у симетричному вигляді. \\

Систему диференціальних рівнянь, що записана в нормальній формі
\begin{equation*}
	\left\{
		\begin{aligned}
			\dot x_1 &= f_1 (x_1, x_2, \ldots, x_n, t), \\
			\dot x_2 &= f_2 (x_1, x_2, \ldots, x_n, t), \\
			\ldots & \ldots \ldots \ldots \ldots \ldots \ldots \ldots, \\
			\dot x_n &= f_n (x_1, x_2, \ldots, x_n, t).
		\end{aligned}
	\right.
\end{equation*}
можна переписати у вигляді
\begin{equation*}
	\frac{\diff x_1}{f_1(x_1, x_2, \ldots, x_n, t)} = \frac{\diff x_2}{f_2 (x_1, x_2, \ldots, x_n, t)} = \ldots = \frac{\diff x_n}{f_n (x_1, x_2, \ldots, x_n, t)} = \frac{\diff t}{1}.
\end{equation*}

При такій формі запису всі змінні $x_1, x_2, \ldots, x_n, t$ рівнозначні. \\

Система диференціальних рівнянь, що записана у вигляді
\begin{equation*}
	\frac{\diff x_1}{X_1(x_1, x_2, \ldots, x_n)} = \frac{\diff x_2}{X_2 (x_1, x_2, \ldots, x_n)} = \ldots = \frac{\diff x_n}{X_n (x_1, x_2, \ldots, x_n)}.
\end{equation*}
називається системою у симетричному вигляді. \\

При знаходженні комбінацій, що інтегруються, найбільш часто використовується властивість ``пропорційності''. А саме, для систем в симетричному вигляді справедлива рівність
\begin{multline*}
	\frac{\diff x_1}{X_1(x_1, x_2, \ldots, x_n)} = \frac{\diff x_2}{X_2 (x_1, x_2, \ldots, x_n)} = \ldots = \frac{\diff x_n}{X_n (x_1, x_2, \ldots, x_n)} = \\
	= \frac{k_1 \cdot \diff x_1 + k_2 \cdot \diff x_2 + \ldots k_n \cdot \diff x_n}{(k_1 \cdot X_1  + k_2 \cdot X_2  + \ldots + k_n \cdot X_n) (x_1, x_2, \ldots, x_n)}.
\end{multline*}


		\subsubsection{Вправи для самостійної роботи}
		\begin{example}
	Розв'язати систему диференціальних рівнянь зведенням до одного рівняння вищого порядку: \[ \frac{\diff y}{\diff x} = \frac{x}{z}, \quad \frac{\diff z}{\diff x} = -\frac{x}{y}.\]
\end{example}

\begin{solution}
	Диференціюємо перше рівняння по змінній $x$:
	\[ y'' = \frac{1}{z} - \frac{x}{z^2} \cdot z' = \frac{1}{z} + \frac{x}{z^2} \cdot \frac{x}{y}. \]
	
	Таким чином, одержали допоміжну систему:
	\[ y' = \frac{x}{z}, \quad y'' = \frac{1}{z} + \frac{x}{z^2} \cdot \frac{x}{y}. \]

	З першого рівняння отримаємо $z = \frac{x}{y'}$. Підставляємо одержане значення в другу систему \[ y'' = \frac{y'}{x} + \frac{(y')^2}{y}.\]

	Маємо однорідне (по $y, y', y''$) диференціальне рівняння другого порядку. Робимо заміну \begin{align*} y &= \exp\left\{\int u \diff x\right\}, \\ y' &= u \cdot \exp\left\{\int u \diff x\right\}, \\ y'' &= (u^2 + u') \cdot \exp\left\{\int u \diff x\right\}. \end{align*}

	Після підстановки та скорочення на $e^{\int u \diff x}$ одержуємо \[ u^2 + u' = \frac{u}{x} + u^2, \] або \[ \frac{\diff u}{\diff x} = \frac{u}{x}. \]

	Далі \[ \frac{\diff u}{u} = \frac{\diff x}{x} \implies u = \frac{C_1 \cdot x}{2}. \]

	Звідси \[ y = \exp\left\{\int \frac{C_1 \cdot x}{2} \diff x\right\} = e^{C_1 x^2 + \ln C_2}, \] або \[ y_2 = C_2 \cdot e^{C_1 x^2}. \]

	Змінна $z$ знаходиться з умови $z = \frac{x}{y'}$, або \[ z = \frac{x}{2 \cdot C_2 \cdot C_1 \cdot x \cdot e^{C_1 x^2}} = \frac{1}{2 \cdot C_1 \cdot C_2} \cdot e^{- C_1 x^2}. \]
\end{solution}

\begin{example}
	Розв'язати систему в симетричному вигляді за допомогою інтегрованих комбінацій \[\frac{\diff x}{y} = \frac{\diff y}{x} = \frac{\diff z}{z}.\]
\end{example}

\begin{solution}
	Використовуючи властивості ``пропорційності'', маємо \[ \frac{\diff x}{y} = \frac{\diff y}{x} = \frac{\diff z}{z} = \frac{\diff (x - y)}{-(x - y)} = \frac{\diff (x + y + z)}{(x + y + z)}. \]

	\begin{enumerate}
		\item Візьмемо \[\frac{\diff z}{z} = \frac{\diff (x - y)}{-(x - y)}.\] Звідси \[\ln |z| + \ln |x - y| = \ln C_1,\] і перший інтеграл має вигляд $z \cdot (x - y) = C_1$.
		\item Візьмемо \[\frac{\diff z}{z} = \frac{\diff (x + y + z)}{(x + y + z)}.\] Звідси \[\ln |z| + \ln |x + y + z| = \ln C_2,\] і ще один інтеграл має вигляд $\frac{x + y + z}{z} = C_2$.
	\end{enumerate}

	Умовою функціональної незалежності одержаних інтегралів є \[\frac{D(C_1, C_2)}{D(x, y)} \ne 0.\] Перевіряємо: \[ \frac{D(C_1, C_2)}{D(x, y)} = \begin{vmatrix} \frac{\partial C_1}{\partial x} & \frac{\partial C_1}{\partial y} \\ \frac{\partial C_2}{\partial x} & \frac{\partial C_2}{\partial y} \end{vmatrix} = \begin{vmatrix} z & -z \\ \frac{1}{z} & \frac{1}{z} \end{vmatrix} = 2 \ne 0. \]
	
	Таким чином \[ z \cdot (x - y) = C_1, \quad \frac{x + y + z}{z} = C_2 \] є загальним інтегралом системи.
\end{solution}

Розв'язати системи диференціальних рівнянь, зведенням до одного рівняння вищого порядку

\begin{multicols}{2}
\begin{problem}
	\[ \frac{\diff y}{\diff x} = \frac{y^2}{z - x}, \quad \frac{\diff z}{\diff x} = y + 1; \]
\end{problem}
\begin{problem}
	\[ \frac{\diff y}{\diff x} = \frac{z}{x}, \quad \frac{\diff z}{\diff x} = \frac{z (y + 2 z - 1)}{x (y - 1)}; \]
\end{problem}
\begin{problem}
	\[ \frac{\diff y}{\diff x} = y^2 \cdot z, \quad \frac{\diff z}{\diff x} = \frac{z}{x} - y \cdot z^2; \]
\end{problem}
\begin{problem}
	\[ \frac{\diff y}{\diff x} = \frac{y^2 - z^2 + 1}{2 z}, \quad \frac{\diff z}{\diff x} = z + y. \]
\end{problem}
\end{multicols}

Розв’язати системи диференціальних рівнянь за допомогою інтегрованих комбінацій.

\begin{multicols}{2}
\begin{problem}
	\[ \frac{\diff x}{y + z} = \frac{\diff y}{x + z} = \frac{\diff z}{x + y}; \]
\end{problem}
\begin{problem}
	\[ \frac{\diff x}{y - x} = \frac{\diff y}{x + y + z} = \frac{\diff z}{x - y}; \]
\end{problem}
\begin{problem}
	\[ \frac{\diff x}{z} = \frac{\diff y}{x \cdot z} = \frac{\diff z}{y}; \]
\end{problem}
\begin{problem}
	\[ \frac{\diff x}{z^2 - y^2} = \frac{\diff y}{z} = \frac{\diff z}{- y}; \]
\end{problem}
\begin{problem}
	\[ \frac{\diff x}{x} = \frac{\diff y}{y} = \frac{\diff z}{x \cdot y + z}; \]
\end{problem}
\begin{problem}
	\[ \frac{\diff x}{x + y^2 + z^2} = \frac{\diff y}{y} = \frac{\diff z}{z}; \]
\end{problem}
\begin{problem}
	\[ \frac{\diff x}{x (y + z)} = \frac{\diff y}{z (z - y)} = \frac{\diff z}{y  (y - z)}; \]
\end{problem}
\begin{problem}
	\[ \frac{\diff x}{-x^2} = \frac{\diff y}{x \cdot y - 2 z^2} = \frac{\diff z}{x \cdot z}; \]
\end{problem}
\begin{problem}
	\[ \frac{\diff x}{x (z - y)} = \frac{\diff y}{y (y - x)} = \frac{\diff z}{y^2 - x \cdot z}; \]
\end{problem}
\begin{problem}
	\[ \frac{\diff x}{x (y^2 - z^2)} = \frac{\diff y}{- y (z^2 + x^2)} = \frac{\diff z}{z (x^2 + y^2)}. \]
\end{problem}
\end{multicols}

	\subsection{Системи лінійних диференціальних рівнянь. Загальні положення}
	Система диференціальних рівнянь, що записана у вигляді
\begin{equation*}
	\left\{
		\begin{array}{rl}
			\dot x_1 &= a_{11}(t) \cdot x_1 + a_{12}(t) \cdot x_2 + \ldots + a_{1n}(t) \cdot x_n + f_1(t), \\
			\dot x_2 &= a_{21}(t) \cdot x_1 + a_{22}(t) \cdot x_2 + \ldots + a_{2n}(t) \cdot x_n + f_2(t), \\
			\hdotsfor{2}, \\
			\dot x_n &= a_{n1}(t) \cdot x_1 + a_{n2}(t) \cdot x_2 + \ldots + a_{nn}(t) \cdot x_n + f_n(t),
		\end{array}
	\right.
\end{equation*}
називається лінійною неоднорідною системою диференціальних рівнянь. Система 
\begin{equation*}
	\left\{
		\begin{array}{rl}
			\dot x_1 &= a_{11}(t) \cdot x_1 + a_{12}(t) \cdot x_2 + \ldots + a_{1n}(t) \cdot x_n, \\
			\dot x_2 &= a_{21}(t) \cdot x_1 + a_{22}(t) \cdot x_2 + \ldots + a_{2n}(t) \cdot x_n, \\
			\hdotsfor{2}, \\
			\dot x_n &= a_{n1}(t) \cdot x_1 + a_{n2}(t) \cdot x_2 + \ldots + a_{nn}(t) \cdot x_n,
		\end{array}
	\right.
\end{equation*}
називається лінійною однорідною системою диференціальних рівнянь. Якщо ввести векторні позначення
\begin{equation*}
	x = \begin{pmatrix} x_1 \\ x_2 \\ \vdots \\ x_n \end{pmatrix}, \quad 
	f(t) = \begin{pmatrix} f_1(t) \\ f_2(t) \\ \vdots \\ f_n(t) \end{pmatrix}, \quad
	A(t) = \begin{pmatrix} a_{11}(t) & a_{12}(t) & \ldots & a_{1n}(t) \\ a_{21}(t) & a_{22}(t) & \ldots & a_{2n}(t) \\ \vdots & \vdots & \ddots & \vdots \\ a_{n1}(t) & a_{n2}(t) & \ldots & a_{nn}(t) \end{pmatrix},
\end{equation*}
то лінійну неоднорідну систему можна переписати у вигляді
\begin{equation*}
	\dot x = A(t) \cdot x + f(t),
\end{equation*}
а лінійну однорідну систему у вигляді
\begin{equation*}
	\dot x = A(t) \cdot x.
\end{equation*}

Якщо функції $a_{ij}(t)$, $f_i(t)$, $i,j=\overline{1,n}$ неперервні в околі точки \[(x_0, t_0) = (x_1^0, x_2^0, \ldots, x_n^0, t_0),\] то виконані умови теореми існування та єдиності розв'язку задачі Коші, і існує єдиний розв'язок
\begin{equation*}
	x_1 = x_1(t), \quad x_2 = x_2(t), \quad \ldots, \quad x_n = x_n(t),
\end{equation*}
системи рівнянь, що задовольняє початковим даним
 \begin{equation*}
	x_1(t_0) = x_1^0, \quad x_2(t_0) = x_2^0, \quad \ldots, \quad x_n(t_0) = x_n^0.
\end{equation*}


		\subsubsection{Властивості розв'язків лінійних однорідних систем}
		\setcounter{property}{0}
\begin{property}
	Якщо вектор $x(t) = (x_1(t), x_2(t), \ldots, x_n(t))^T$ є розв'язком лінійної однорідної системи, то і \[ C \cdot x(t) = (C \cdot x_1(t), C \cdot x_2(t), \ldots, C \cdot x_n(t))^T,\] де $C$ --- стала скалярна величина, також є розв'язком цієї системи.
\end{property}

\begin{proof}
	Дійсно, за умовою 
	\begin{equation*}
		\dot x(t) - A(t) \cdot x(t) \equiv 0.
	\end{equation*}

	Але тоді і
	\begin{multline*}
		\frac{\diff}{\diff t} (C \cdot x(t)) - A(t) \cdot (C \cdot x(t)) = \\ = C \cdot (\dot x(t) - A(t) \cdot x(t)) \equiv 0
	\end{multline*}
	оскільки дорівнює нулю вираз в дужках. Тобто $C \cdot x(t)$ є розв'язком однорідної системи.
\end{proof}

\begin{property}
	Якщо дві векторні функції $x_1 = (x_{11}(t), x_{21}(t), \ldots, x_{n1}(t))^T$, $x_2 = (x_{12}(t), x_{22}(t), \ldots, x_{n2}(t))^T$ є розв'язками однорідної системи, то і їхня сума також буде розв’язком однорідної системи.
\end{property}

\begin{proof}
	Дійсно, за умовою
	\begin{align*}
		\dot x_1(t) - A(t) \cdot x_1(t) &\equiv 0, \\
		\dot x_2(t) - A(t) \cdot x_2(t) &\equiv 0.
	\end{align*}

	Але тоді і
	\begin{equation*}
		\frac{\diff}{\diff t} (x_1(t) + x_2(t)) - A(t) \cdot (x_1(t) + x_2(t)) = (\dot x_1(t) - A(t) \cdot x_1(t)) + (\dot x_2(t) - A(t) \cdot x_2(t)) \equiv 0
	\end{equation*}
	тому що дорівнюють нулю вирази в дужках, тобто $x_1(t) + x_2(t)$ є розв'язком однорідної системи.
\end{proof}

\begin{property}
	Якщо вектори $x_1 = (x_{11}(t), x_{21}(t), \ldots, x_{n1}(t))^T$, $\ldots$, $x_n = (x_{1n}(t), x_{2n}(t), \ldots, x_{nn}(t))^T$ є розв'язками однорідної системи, та і їхня лінійна комбінація з довільними коефіцієнтами також буде розв'язком однорідної системи. 
\end{property}

\begin{proof}
	Дійсно, за умовою
	\begin{equation*}
		\dot x_i(t) - A(t) \cdot x_i(t) \equiv 0, \quad i = \overline{1, n}.
	\end{equation*}
	
	Але тоді і
	\begin{multline*}
		\frac{\diff}{\diff t} \left( \sum_{i=1}^n C_i \cdot x_i(t) \right) - A(t) \cdot \left( \sum_{i = 1}^n C_i \cdot x_i(t) \right) = \\ = \sum_{i = 1}^n C_i \cdot \left( \dot x_i(t) - A(t) \cdot x_i(t) \right) \equiv 0
	\end{multline*}
	тому що дорівнює нулю кожний з доданків, тобто $\sum_{i=1}^n C_i \cdot x_i(t)$ є роз\-в'яз\-ком однорідної системи.
\end{proof}

\begin{property}
	Якщо комплексний вектор з дійсними елементами $u(t) + i \cdot v(t) = (u_1(t), \ldots, u_n(t))^T + i \cdot (v_1(t), \ldots, v_n(t))^T$ є розв’язком однорідної системи, то окремо дійсна та уявна частини є розв'язками системи.
\end{property}

\begin{proof}
	Дійсно за умовою
	\begin{equation*}
		\frac{\diff}{\diff t} (u(t) + i \cdot v(t)) - A(t) \cdot (u(t) + i \cdot v(t)) \equiv 0.
	\end{equation*}

	Розкривши дужки і зробивши перетворення, одержимо
	\begin{equation*}
		(\dot u(t) - A(t) \cdot u(t)) + i \cdot (\dot v(t) - A(t) \cdot v(t)) \equiv 0.
	\end{equation*}
	 
	А комплексний вираз дорівнює нулю тоді і тільки тоді, коли дорівнюють нулю дійсна і уявна частини, тобто
	\begin{align*}
		\dot u(t) - A(t) \cdot u(t) &\equiv 0, \\
		\dot v(t) - A(t) \cdot v(t) &\equiv 0.
	\end{align*}
	що і було потрібно довести.
\end{proof}

\begin{definition}
	Вектори \[ x_1 = \begin{pmatrix} x_{11}(t) \\ x_{21}(t) \\ \vdots \\ x_{n1}(t) \end{pmatrix}, \quad x_2 = \begin{pmatrix} x_{12}(t) \\ x_{22}(t) \\ \vdots \\ x_{n2}(t) \end{pmatrix}, \quad \ldots, \quad x_n = \begin{pmatrix} x_{1n}(t) \\ x_{2n}(t) \\ \vdots \\ x_{nn}(t) \end{pmatrix} \] називаються лінійно залежними на відрізку $t \in [a, b]$, якщо існують не всі рівні нулю сталі $C_1, C_2, \ldots, C_n$, такі, що 
	\begin{equation*}
		C_1 \cdot x_1(t) + C_2 \cdot x_2(t) + \ldots + C_n \cdot x_n(t) \equiv 0
	\end{equation*}
	при $t \in [a, b]$. \\

	Якщо тотожність справедлива лише при $C_i = 0$, $i = \overline{1, n}$, то вектори лінійно незалежні.
\end{definition}

\begin{definition}
	Визначник, що складається з векторів $x_1(t), x_2(t), \ldots, x_n(t)$, тобто
	\begin{equation*}
		W[x_1, x_2, \ldots, x_n](t) = \begin{vmatrix} x_{11}(t) & x_{12}(t) & \ldots & x_{1n}(t) \\ x_{21}(t) & x_{22}(t) & \ldots & x_{2n}(t) \\ \vdots & \vdots & \ddots & \vdots \\ x_{n1}(t) & x_{n2}(t) & \ldots & x_{nn}(t) \end{vmatrix}.
	\end{equation*}
	називається визначником Вронського.
\end{definition}

\begin{theorem}
	Якщо векторні функції $x_1(t), x_2(t), \ldots, x_n(t)$ лінійно залежні, то визначник Вронського тотожно дорівнює нулю.
\end{theorem}

\begin{proof}
	За умовою існують не всі рівні нулю $C_1, C_2, \ldots, C_n$, такі, що $C_1 \cdot x_1(t) + C_2 \cdot x_2(t) + \ldots + C_n \cdot x_n(t) \equiv 0$ при $t \in [a, b]$. \\

	Або, розписавши покоординатно, одержимо
	\begin{equation*}
		\left\{
			\begin{array}{rl}
				C_1 \cdot x_{11}(t) + C_2 \cdot x_{12}(t) + \ldots + C_n \cdot x_{1n}(t) &\equiv 0, \\
				C_1 \cdot x_{21}(t) + C_2 \cdot x_{22}(t) + \ldots + C_n \cdot x_{2n}(t) &\equiv 0, \\
				\hdotsfor{2}, \\
				C_1 \cdot x_{n1}(t) + C_2 \cdot x_{n2}(t) + \ldots + C_n \cdot x_{nn}(t) &\equiv 0.
			\end{array}
		\right.
	\end{equation*}

	А однорідна система має ненульовий розв'язок $C_1, C_2, \ldots, C_n$ тоді і тільки тоді, коли визначник дорівнює нулю, тобто
	\begin{equation*}
		W[x_1, x_2, \ldots, x_n](t) = \begin{vmatrix} x_{11}(t) & x_{12}(t) & \ldots & x_{1n}(t) \\ x_{21}(t) & x_{22}(t) & \ldots & x_{2n}(t) \\ \vdots & \vdots & \ddots & \vdots \\ x_{n1}(t) & x_{n2}(t) & \ldots & x_{nn}(t) \end{vmatrix} \equiv 0, \quad t \in [a, b].
	\end{equation*}
\end{proof}

\begin{theorem}
	Якщо розв'язки $x_1(t), x_2(t), \ldots, x_n(t)$ лінійної однорідної системи лінійно незалежні, то визначник Вронського не дорівнює нулю в жодній точці $t \in [a, b]$. 
\end{theorem}

\begin{proof}
	Нехай, від супротивного, існує точка $t_0 \in [a, b]$ і \[W[x_1, x_2, \ldots, x_n](t_0) = 0.\]

	Тоді виконується система однорідних алгебраїчних рівнянь \[C_1 \cdot x_1(t_0) + C_2 \cdot x_2(t_0) + \ldots + C_n \cdot x_n(t_0) = 0. \]

	Або, розписавши покоординатно, одержимо
	\begin{equation*}
		\left\{
			\begin{array}{rl}
				C_1 \cdot x_{11}(t_0) + C_2 \cdot x_{12}(t_0) + \ldots + C_n \cdot x_{1n}(t_0) &= 0, \\
				C_1 \cdot x_{21}(t_0) + C_2 \cdot x_{22}(t_0) + \ldots + C_n \cdot x_{2n}(t_0) &= 0, \\
				\hdotsfor{2}, \\
				C_1 \cdot x_{n1}(t_0) + C_2 \cdot x_{n2}(t_0) + \ldots + C_n \cdot x_{nn}(t_0) &= 0.
			\end{array}
		\right.
	\end{equation*}
 	має ненульовий розв'язок $C_1^0, C_2^0, \ldots, C_n^0$. Розглянемо лінійну комбінацію розв'язків з отриманими коефіцієнтами
 	\begin{equation*}
 		x(t) = C_1^0 \cdot x_1(t) + C_2^0 \cdot x_2(t) + \ldots + C_n^0 \cdot x_n(t).
 	\end{equation*}

	Відповідно до властивості 4, ця комбінація буде розв'язком. Крім того, як випливає із системи алгебраїчних рівнянь, для отриманих $C_1^0, C_2^0, \ldots, C_n^0$: $x(t_0) = 0$, $t_0 \in [a, b]$. Але розв'язком, що задовольняють таким умовам, є $x \equiv 0$. І в силу теореми існування та єдиності ці два розв’язки збігаються, тобто $x(t) \equiv 0$ при $t \in [a, b]$, або 
 	\begin{equation*}
 		C_1^0 \cdot x_1(t) + C_2^0 \cdot x_2(t) + \ldots + C_n^0 \cdot x_n(t) \equiv 0,
 	\end{equation*}
	або розв'язки $x_1(t), x_2(t), \ldots, x_n(t)$ лінійно залежні, що суперечить умові теореми.  \\

	Таким чином, $W[x_1, x_2, \ldots, x_n](t) \ne 0$ у жодній точці $t \in [a, b]$, що і було потрібно довести.
\end{proof}

\begin{theorem}
	Для того щоб розв'язки $x_1(t), \ldots, x_n(t)$ були лінійно незалежні, необхідно і достатно, щоб $W[x_1, \ldots, x_n](t) \ne 0$ у жодній точці $t \in [a, b]$.
\end{theorem}

\begin{proof}
	Випливає з попередніх двох теорем.
\end{proof}

\begin{theorem}
	Загальний розв'язок лінійної однорідної системи представляється у вигляді лінійної комбінації $n$ лінійно незалежних роз\-в'яз\-ків.
\end{theorem}

\begin{proof}
	Як випливає з властивості 3, лінійна комбінація розв'язків також буде розв'язком. Покажемо, що цей розв'язок загальний, тобто завдяки вибору коефіцієнтів $C_1, \ldots, C_n$ можна розв'язати будь-яку задачу Коші $x(t_0) = x_0$ або в координатній формі:
	\begin{equation*}
		x_1(t_0) = x_1^0, \quad x_2(t_0) = x_2^0, \quad \ldots, \quad x_n(t_0) = x_n^0.
	\end{equation*}

	Оскільки розв'язки $x_1(t), \ldots, x_n(t)$  лінійно незалежні, то визначник Вронського відмінний від нуля. Отже, система алгебраїчних рівнянь
	\begin{equation*}
		\left\{
			\begin{array}{rl}
				C_1 \cdot x_{11}(t_0) + C_2 \cdot x_{12}(t_0) + \ldots + C_n \cdot x_{1n}(t_0) &= x_1^0, \\
				C_1 \cdot x_{21}(t_0) + C_2 \cdot x_{22}(t_0) + \ldots + C_n \cdot x_{2n}(t_0) &= x_2^0, \\
				\hdotsfor{2}, \\
				C_1 \cdot x_{n1}(t_0) + C_2 \cdot x_{n2}(t_0) + \ldots + C_n \cdot x_{nn}(t_0) &= x_n^0.
			\end{array}
		\right.
	\end{equation*}
	має єдиний розв'язок $C_1^0, C_2^0, \ldots, C_n^0$. \\

	Тоді лінійна комбінація
	\begin{equation*}
		x(t) = C_1^0 \cdot x_1(t) + C_2^0 \cdot x_2(t) + \ldots + C_n^0 \cdot x_n(t)
	\end{equation*}
	є розв'язком поставленої задачі Коші. Теорема доведена.
\end{proof}

\begin{remark}
	Максимальне число незалежних розв'язків дорівнює кількості рівнянь $n$.
\end{remark}

\begin{proof}
	Це випливає з теореми про загальний розв'язок системи однорідних рівнянь, тому що будь-який інший розв'язок може бути представлений у вигляді лінійної комбінації $n$ лінійно незалежних розв'язків.
\end{proof}

\begin{definition}
	Матриця, складена з будь-яких $n$ лінійно незалежних роз\-в'яз\-ків, називається фундаментальною матрицею розв'язків системи.
\end{definition}

Якщо лінійно незалежними розв'язками будуть \[ x_1 = \begin{pmatrix} x_{11}(t) \\ x_{21}(t) \\ \vdots \\ x_{n1}(t) \end{pmatrix}, \quad x_2 = \begin{pmatrix} x_{12}(t) \\ x_{22}(t) \\ \vdots \\ x_{n2}(t) \end{pmatrix}, \quad \ldots, \quad x_n = \begin{pmatrix} x_{1n}(t) \\ x_{2n}(t) \\ \vdots \\ x_{nn}(t) \end{pmatrix} \] то матриця
\begin{equation*}
	X(t) = \begin{pmatrix} x_{11} (t) & x_{12} (t) & \ldots & x_{1n} (t) \\ x_{21} (t) & x_{22} (t) & \ldots & x_{2n} (t) \\ \vdots & \vdots & \ddots & \vdots \\ x_{n1} (t) & x_{n2} (t) & \ldots & x_{nn} (t) \end{pmatrix}
\end{equation*}
буде фундаментальною матрицею розв'язків. \\

Як випливає з попередньої теореми загальний розв'язок може бути представлений у вигляді
\begin{equation*}
	x_{\text{homo}} = \sum_{i = 1}^n C_i \cdot x_i(t),
\end{equation*}
де $C_i$ --- довільні сталі. Якщо ввести вектор $C = (C_1, C_2, \ldots, C_n)^T$, то загальний розв'язок можна записати у вигляді $x_{\text{homo}} = X(t) \cdot C$.


		\subsubsection{Формула Якобі}
		Нехай $x_1(t), \ldots, x_n(t)$ --- лінійно незалежні розв'язки однорідної системи, $W[x_1, \ldots, x_n]$ --- визначник Вронського. Обчислимо похідну визначника Вронського
\begin{multline*}
	\frac{\diff}{\diff t} W[x_1, \ldots, x_n] = \frac{\diff}{\diff t} \begin{vmatrix} x_{11}(t) & x_{12}(t) & \ldots & x_{1n}(t) \\ x_{21}(t) & x_{22}(t) & \ldots & x_{2n}(t) \\ \vdots & \vdots & \ddots & \vdots \\ x_{n1}(t) & x_{n2}(t) & \ldots & x_{nn}(t) \end{vmatrix} = \\
	= \begin{vmatrix} x_{11}'(t) & x_{12}(t) & \ldots & x_{1n}(t) \\ x_{21}'(t) & x_{22}(t) & \ldots & x_{2n}(t) \\ \vdots & \vdots & \ddots & \vdots \\ x_{n1}'(t) & x_{n2}(t) & \ldots & x_{nn}(t) \end{vmatrix} + \begin{vmatrix} x_{11}(t) & x_{12}'(t) & \ldots & x_{1n}(t) \\ x_{21}(t) & x_{22}'(t) & \ldots & x_{2n}(t) \\ \vdots & \vdots & \ddots & \vdots \\ x_{n1}(t) & x_{n2}'(t) & \ldots & x_{nn}(t) \end{vmatrix} + \ldots \\
	\ldots + \begin{vmatrix} x_{11}(t) & x_{12}(t) & \ldots & x_{1n}'(t) \\ x_{21}(t) & x_{22}(t) & \ldots & x_{2n}'(t) \\ \vdots & \vdots & \ddots & \vdots \\ x_{n1}(t) & x_{n2}(t) & \ldots & x_{nn}'(t) \end{vmatrix}.
\end{multline*}

Оскільки для похідних виконується співвідношення
\begin{equation*}
\begin{array}{rl}
	& \left\{
		\begin{array}{rl}
			x_{11}'(t) &= \alpha_{11} \cdot x_{11} (t) + \alpha_{12} \cdot x_{12}(t) + \ldots + \alpha_{1n} \cdot x_{1n}(t), \\
			x_{12}'(t) &= \alpha_{11} \cdot x_{21} (t) + \alpha_{12} \cdot x_{22}(t) + \ldots + \alpha_{1n} \cdot x_{2n}(t), \\
			\hdotsfor{2}, \\
			x_{1n}'(t) &= \alpha_{11} \cdot x_{n1} (t) + \alpha_{12} \cdot x_{n2}(t) + \ldots + \alpha_{1n} \cdot x_{nn}(t),
		\end{array}
	\right. \\
	& \left\{
		\begin{array}{rl}
			x_{21}'(t) &= \alpha_{21} \cdot x_{11} (t) + \alpha_{22} \cdot x_{12}(t) + \ldots + \alpha_{2n} \cdot x_{1n}(t), \\
			x_{22}'(t) &= \alpha_{21} \cdot x_{21} (t) + \alpha_{22} \cdot x_{22}(t) + \ldots + \alpha_{2n} \cdot x_{2n}(t), \\
			\hdotsfor{2}, \\
			x_{2n}'(t) &= \alpha_{21} \cdot x_{n1} (t) + \alpha_{22} \cdot x_{n2}(t) + \ldots + \alpha_{2n} \cdot x_{nn}(t),
		\end{array}
	\right. \\
	\hdotsfor{2}, \\
	& \left\{
		\begin{array}{rl}
			x_{n1}'(t) &= \alpha_{n1} \cdot x_{11} (t) + \alpha_{n2} \cdot x_{12}(t) + \ldots + \alpha_{nn} \cdot x_{1n}(t), \\
			x_{2n}'(t) &= \alpha_{n1} \cdot x_{21} (t) + \alpha_{n2} \cdot x_{22}(t) + \ldots + \alpha_{nn} \cdot x_{2n}(t), \\
			\hdotsfor{2}, \\
			x_{nn}'(t) &= \alpha_{n1} \cdot x_{n1} (t) + \alpha_{n2} \cdot x_{n2}(t) + \ldots + \alpha_{nn} \cdot x_{nn}(t),
		\end{array}
	\right.
\end{array}
\end{equation*}
то після підстановки одержимо
\begin{multline*}
	\frac{\diff}{\diff t} W[x_1, \ldots, x_n] = \\
	= \begin{vmatrix} \alpha_{11} \cdot x_{11} (t) + \alpha_{12} \cdot x_{12}(t) + \ldots + \alpha_{1n} \cdot x_{1n}(t) & x_{12}(t) & \ldots & x_{1n}(t) \\  \alpha_{11} \cdot x_{21} (t) + \alpha_{12} \cdot x_{22}(t) + \ldots + \alpha_{1n} \cdot x_{2n}(t) & x_{22}(t) & \ldots & x_{2n}(t) \\ \vdots & \vdots & \ddots & \vdots \\ \alpha_{11} \cdot x_{n1} (t) + \alpha_{12} \cdot x_{n2}(t) + \ldots + \alpha_{1n} \cdot x_{nn}(t) & x_{n2}(t) & \ldots & x_{nn}(t) \end{vmatrix} + \\
	+ \begin{vmatrix} x_{11}(t) & \alpha_{21} \cdot x_{11} (t) + \alpha_{22} \cdot x_{12}(t) + \ldots + \alpha_{2n} \cdot x_{1n}(t) & \ldots & x_{1n}(t) \\ x_{21}(t) & \alpha_{21} \cdot x_{21} (t) + \alpha_{22} \cdot x_{22}(t) + \ldots + \alpha_{2n} \cdot x_{2n}(t) & \ldots & x_{2n}(t) \\ \vdots & \vdots & \ddots & \vdots \\ x_{n1}(t) & \alpha_{21} \cdot x_{n1} (t) + \alpha_{22} \cdot x_{n2}(t) + \ldots + \alpha_{2n} \cdot x_{nn}(t) & \ldots & x_{nn}(t) \end{vmatrix} + \ldots \\
	\ldots + \begin{vmatrix} x_{11}(t) & x_{12}(t) & \ldots & \alpha_{n1} \cdot x_{11} (t) + \alpha_{n2} \cdot x_{12}(t) + \ldots + \alpha_{nn} \cdot x_{1n}(t) \\ x_{21}(t) & x_{22}(t) & \ldots & \alpha_{n1} \cdot x_{21} (t) + \alpha_{n2} \cdot x_{22}(t) + \ldots + \alpha_{nn} \cdot x_{2n}(t) \\ \vdots & \vdots & \ddots & \vdots \\ x_{n1}(t) & x_{n2}(t) & \ldots & \alpha_{n1} \cdot x_{n1} (t) + \alpha_{n2} \cdot x_{n2}(t) + \ldots + \alpha_{nn} \cdot x_{nn}(t) \end{vmatrix}.
\end{multline*}

Розкривши кожний з визначників, і з огляду на те, що визначники з однаковими стовпцями дорівнюють нулю, одержимо
\begin{multline*}
	\frac{\diff}{\diff t} W[x_1, \ldots, x_n] = a_{11} \cdot \begin{vmatrix} x_{11}(t) & x_{12}(t) & \ldots & x_{1n}(t) \\ x_{21}(t) & x_{22}(t) & \ldots & x_{2n}(t) \\ \vdots & \vdots & \ddots & \vdots \\ x_{n1}(t) & x_{n2}(t) & \ldots & x_{nn}(t) \end{vmatrix} + \\
	+ a_{22} \cdot \begin{vmatrix} x_{11}(t) & x_{12}(t) & \ldots & x_{1n}(t) \\ x_{21}(t) & x_{22}(t) & \ldots & x_{2n}(t) \\ \vdots & \vdots & \ddots & \vdots \\ x_{n1}(t) & x_{n2}(t) & \ldots & x_{nn}(t) \end{vmatrix} + \ldots + a_{nn} \cdot \begin{vmatrix} x_{11}(t) & x_{12}(t) & \ldots & x_{1n}(t) \\ x_{21}(t) & x_{22}(t) & \ldots & x_{2n}(t) \\ \vdots & \vdots & \ddots & \vdots \\ x_{n1}(t) & x_{n2}(t) & \ldots & x_{nn}(t) \end{vmatrix} = \\
	= (a_{11} + a_{22} + \ldots + a_{nn}) \cdot \begin{vmatrix} x_{11}(t) & x_{12}(t) & \ldots & x_{1n}(t) \\ x_{21}(t) & x_{22}(t) & \ldots & x_{2n}(t) \\ \vdots & \vdots & \ddots & \vdots \\ x_{n1}(t) & x_{n2}(t) & \ldots & x_{nn}(t) \end{vmatrix} = \\
	= \trace A \cdot \begin{vmatrix} x_{11}(t) & x_{12}(t) & \ldots & x_{1n}(t) \\ x_{21}(t) & x_{22}(t) & \ldots & x_{2n}(t) \\ \vdots & \vdots & \ddots & \vdots \\ x_{n1}(t) & x_{n2}(t) & \ldots & x_{nn}(t) \end{vmatrix} = \trace A \cdot W[x_1, \ldots, x_n].
\end{multline*}

Або
\begin{equation*}
	\frac{\diff}{\diff t} W[x_1, \ldots, x_n] = \trace A \cdot W[x_1, \ldots, x_n].
\end{equation*}

Розділивши змінні, одержимо
\begin{equation*}
	\frac{\diff W[x_1, \ldots, x_n]}{W[x_1, \ldots, x_n]}  = \trace A \cdot \diff t.
\end{equation*}

Проінтегруємо в межах $t_0 \le s \le t$,
\begin{equation*}
	\ln W[x_1, \ldots, x_n](t) - \ln W[x_1, \ldots, x_n](t_0) = \int_{t_0}^t \trace A \cdot \diff t,
\end{equation*}
або 
\begin{equation*}
	W[x_1, \ldots, x_n](t) = W[x_1, \ldots, x_n](t_0) \cdot \exp\left\{\int_{t_0}^t \trace A \cdot \diff t\right\}.
\end{equation*}

Взагалі кажучи, доведення проводилося в припущенні, що система рівнянь може залежати від часу, тобто
\begin{equation*}
	W[x_1, \ldots, x_n](t) = W[x_1, \ldots, x_n](t_0) \cdot \exp\left\{\int_{t_0}^t \trace A(t) \cdot \diff t\right\}.
\end{equation*}

Отримана формула називається формулою Якобі.


	\subsection{Системи лінійних однорідних диференціальних рівнянь з сталими коефіцієнтами}
	Система диференціальних рівнянь вигляду
\begin{equation*}
	\left\{
		\begin{array}{rl}
			\dot x_1 &= a_{11} x_1 + a_{12} x_2 + \ldots + a_{1n} x_n, \\
			\dot x_1 &= a_{21} x_1 + a_{22} x_2 + \ldots + a_{2n} x_n, \\
			\hdotsfor{2} \\
			\dot x_1 &= a_{n1} x_1 + a_{n2} x_2 + \ldots + a_{nn} x_n,
		\end{array}
	\right.
\end{equation*}
де $a_{ij}$, $i,j = \overline{1, n}$ --- сталі величини, називається лінійною однорідною системою з сталими коефіцієнтами. У матричному вигляді вона записується
\begin{equation*}
	\dot x = A x.
\end{equation*}


		\subsubsection{Розв'язування систем однорідних рівнянь з сталими коефіцієнтами методом Ейлера}
		Розглянемо один з методів побудови розв'язку систем з сталими коефіцієнтами. \parvskip

Розв'язок системи шукаємо у вигляді вектора \[x(t) = (\alpha_1 e^{\lambda t}, \alpha_2 e^{\lambda t}, \ldots, \alpha_n e^{\lambda t})^T. \]

Підставивши в систему диференціальних рівнянь, одержимо
\begin{equation*}
	\left\{
		\begin{array}{rl}
			\alpha_1 \lambda e^{\lambda t} &= a_{11} \alpha_1 e^{\lambda t} + a_{12} \alpha_2  e^{\lambda t} + \ldots + a_{1n} \alpha_n e^{\lambda t}, \\
			\alpha_2 \lambda e^{\lambda t} &= a_{21} \alpha_1 e^{\lambda t} + a_{22} \alpha_2 e^{\lambda t} + \ldots + a_{2n} \alpha_n e^{\lambda t}, \\
			\hdotsfor{2} \\
			\alpha_n \lambda e^{\lambda t} &= a_{n1} \alpha_1 e^{\lambda t} + a_{n2} \alpha_2 e^{\lambda t} + \ldots + a_{nn} \alpha_n e^{\lambda t}.
		\end{array}
	\right.
\end{equation*}
 
Скоротивши на $e^{\lambda t}$, і перенісши всі члени вправо, запишемо
\begin{equation*}
	\left\{
		\begin{array}{rl}
			(a_{11} - \lambda) \alpha_1 + a_{12} \alpha_2 + \ldots + a_{1n} \alpha_n &= 0, \\
			a_{21} \alpha_1 + (a_{22} - \lambda) \alpha_2 + \ldots + a_{2n} \alpha_n &= 0, \\
			\hdotsfor{2} \\
			a_{n1} \alpha_1 + a_{n2} \alpha_2 + \ldots + (a_{nn} - \lambda) \alpha_n &= 0.
		\end{array}
	\right.
\end{equation*}
 
Отримана однорідна система лінійних алгебраїчних рівнянь має роз\-в'яз\-ок тоді і тільки тоді, коли її визначник дорівнює нулю, тобто
\begin{equation*}
	\begin{vmatrix}
		a_{11} - \lambda & a_{12} & \cdots & a_{1n} \\
		a_{21} & a_{22} - \lambda & \cdots & a_{2n} \\
		\vdots & \vdots & \ddots & \vdots \\
		a_{n1} & a_{n2} & \cdots & a_{nn} - \lambda
	\end{vmatrix} = 0.
\end{equation*}

Це рівняння, може бути записаним у векторно-матричній формі
\begin{equation*}
	\det(A - \lambda E) = 0.
\end{equation*}
і воно називається характеристичним рівнянням. Розкриємо його
\begin{equation*}
	\lambda^n + p_1 \lambda^{n - 1} + \ldots + p_{n - 1} \lambda + p_n = 0.
\end{equation*}

Алгебраїчне рівняння $n$-го ступеня має $n$ коренів. Розглянемо різні випадки:
\begin{enumerate}
\item Всі корені характеристичного рівняння $\lambda_1, \lambda_2, \ldots, \lambda_n$ (власні числа матриці $A$) дійсні і різні. Підставляючи їх по черзі в систему алгебраїчних рівнянь
\begin{equation*}
	\left\{
		\begin{array}{rl}
			(a_{11} - \lambda_i) \alpha_1 + a_{12} \alpha_2 + \ldots + a_{1n} \alpha_n &= 0, \\
			a_{21} \alpha_1 + (a_{22} - \lambda_i) \alpha_2 + \ldots + a_{2n} \alpha_n &= 0, \\
			\hdotsfor{2} \\
			a_{n1} \alpha_1 + a_{n2} \alpha_2 + \ldots + (a_{nn} - \lambda_i) \alpha_n &= 0.
		\end{array}
	\right.
\end{equation*}

одержуємо відповідні ненульові розв'язки системи
\begin{equation*}
	\alpha^1 = \begin{pmatrix} \alpha_1^1 \\ \alpha_2^1 \\ \vdots \\ \alpha_n^1 \end{pmatrix}, \quad
	\alpha^2 = \begin{pmatrix} \alpha_1^2 \\ \alpha_2^2 \\ \vdots \\ \alpha_n^2 \end{pmatrix}, \quad
	\ldots, \quad
	\alpha^n = \begin{pmatrix} \alpha_1^n \\ \alpha_2^n \\ \vdots \\ \alpha_n^n \end{pmatrix}
\end{equation*}
що являють собою власні вектори, які відповідають власним числам $\lambda_i$, $i = \overline{1, n}$. \parvskip

У такий спосіб одержимо $n$ розв'язків
\begin{equation*}
	x_1(t) = \begin{pmatrix} \alpha_1^1 e^{\lambda_1 x} \\ \alpha_2^1 e^{\lambda_1 x} \\ \vdots \\ \alpha_n^1 e^{\lambda_1 x} \end{pmatrix}, 
	x_2(t) = \begin{pmatrix} \alpha_1^2 e^{\lambda_2 x} \\ \alpha_2^2 e^{\lambda_2 x} \\ \vdots \\ \alpha_n^2 e^{\lambda_2 x} \end{pmatrix},
	\ldots, 
	x_n(t) = \begin{pmatrix} \alpha_1^n e^{\lambda_n x} \\ \alpha_2^n e^{\lambda_n x} \\ \vdots \\ \alpha_n^n e^{\lambda_n x} \end{pmatrix}
\end{equation*}

Причому оскільки $\lambda_1, \lambda_2, \ldots, \lambda_n$ --- різні а $\alpha^1, \alpha^2, \ldots, \alpha^n$ -- відповідні їм власні вектори, то розв'язки $x_1(t), x_2(t), \ldots, x_n(t)$ --- лінійно незалежні, і загальний розв'язок системи має вигляд
\begin{equation*}
	x(t) = \sum_{i = 1}^n C_i x_i(t).
\end{equation*}

Або у векторно-матричній формі запису
\begin{equation*}
	\begin{pmatrix} x_1 \\ x_2 \\ \vdots \\ x_n \end{pmatrix} = 
	\begin{pmatrix}
		\alpha_1^1 e^{\lambda_1 t} & \alpha_1^2 e^{\lambda_2 t} & \cdots & \alpha_1^n e^{\lambda_n t} \\
		\alpha_2^1 e^{\lambda_1 t} & \alpha_2^2 e^{\lambda_2 t} & \cdots & \alpha_2^n e^{\lambda_n t} \\
		\vdots & \vdots & \ddots & \vdots \\
		\alpha_n^1 e^{\lambda_1 t} & \alpha_n^2 e^{\lambda_2 t} & \cdots & \alpha_n^n e^{\lambda_n t}
	\end{pmatrix}
	\cdot
	\begin{pmatrix} C_1 \\ C_2 \\ \vdots \\ C_n \end{pmatrix},
\end{equation*}
де $C_1, C_2, \ldots, C_n$ --- довільні сталі.

\item Нехай $\lambda_{1,2} = p \pm i q$ --- пара комплексно спряжених коренів. Візьмемо один з них, наприклад $\lambda = p + i q$. Комплексному власному числу відповідає комплексний власний вектор
\begin{equation*}
	\begin{pmatrix} \alpha_1 \\ \alpha_2 \\ \vdots \\ \alpha_n \end{pmatrix} =
	\begin{pmatrix} r_1 + i s_1 \\ r_2 + i s_2 \\ \vdots \\ r_n + i s_n \end{pmatrix}
\end{equation*}
і, відповідно, розв'язок
\begin{equation*}
	\begin{pmatrix} x_1 \\ x_2 \\ \vdots \\ x_n \end{pmatrix} =
	\begin{pmatrix} (r_1 + i s_1) e^{(p + i q) t} \\ (r_2 + i s_2) e^{(p + i q) t} \\ \vdots \\ (r_n + i s_n) e^{(p + i q) t} \end{pmatrix}
\end{equation*}

Використовуючи залежність $e^{(p + i q) t} = e^{pt} (\cos qt + i \sin qt)$, перетворимо розв'язок до вигляду:
\begin{multline*}
	\begin{pmatrix} x_1 \\ x_2 \\ \vdots \\ x_n \end{pmatrix} =
	\begin{pmatrix} (r_1 + i s_1) e^{p t} (\cos qt + i \sin qt) \\ (r_2 + i s_2) e^{p t} (\cos qt + i \sin qt) \\ \vdots \\ (r_n + i s_n) e^{p t} (\cos qt + i \sin qt) \end{pmatrix} = \\
	= \begin{pmatrix} e^{p t} (r_1 \cos qt - s_1 \sin qt) \\ e^{pt} (r_2 \cos qt - s_2 \sin qt) \\ \vdots \\ e^{pt} (r_n \cos qt - s_n \sin qt) \end{pmatrix} + i \begin{pmatrix} e^{p t} (s_1 \cos qt + r_1 \sin qt) \\ e^{pt} (s_2 \cos qt + r_2 \sin qt) \\ \vdots \\ e^{pt} (s_n \cos qt + r_n \sin qt) \end{pmatrix} = \\
	= u(t) + i v(t).
\end{multline*}

І, як випливає з властивості 4 розв'язків однорідних систем, якщо комплексна функція $u(t) + i v(t)$ дійсного аргументу є розв'язком однорідної системи, то окремо дійсна і уявна частини також будуть розв'язками, тобто комплексним власним числам $\lambda_{1,2} = p \pm i q$  відповідають лінійно незалежні розв'язки
\begin{align*}
	u(t) &= \begin{pmatrix} e^{p t} (r_1 \cos qt - s_1 \sin qt) \\ e^{pt} (r_2 \cos qt - s_2 \sin qt) \\ \vdots \\ e^{pt} (r_n \cos qt - s_n \sin qt) \end{pmatrix}, \\
	v(t) &= \begin{pmatrix} e^{p t} (s_1 \cos qt + r_1 \sin qt) \\ e^{pt} (s_2 \cos qt + r_2 \sin qt) \\ \vdots \\ e^{pt} (s_n \cos qt + r_n \sin qt) \end{pmatrix}
\end{align*}

\item Якщо характеристичне рівняння має кратний корінь $\lambda$ кратності $\gamma$, тобто $\lambda_1 = \lambda_2 = \ldots = \lambda_\gamma = \lambda$, то розв'язок системи рівнянь має вигляд
\begin{equation*}
	\begin{pmatrix} x_1 \\ x_2 \\ \vdots \\ x_n \end{pmatrix} = \begin{pmatrix} \left(\alpha_1^1 + \alpha_1^2 t + \ldots + \alpha_1^\gamma t^{\gamma - 1}\right) e^{\lambda t} \\ \left(\alpha_2^1 + \alpha_2^2 t + \ldots + \alpha_2^\gamma t^{\gamma - 1}\right) e^{\lambda t} \\ \vdots \\ \left(\alpha_n^1 + \alpha_n^2 t + \ldots + \alpha_n^\gamma t^{\gamma - 1}\right) e^{\lambda t} \end{pmatrix}
\end{equation*}

Підставивши його у вихідне диференціальне рівняння і прирівнявши коефіцієнти при однакових степенях, одержимо $\gamma n$ рівнянь, що містять $\gamma n$ невідомих. Тому що корінь характеристичного рівняння $\lambda$ має кратність $\gamma$, то ранг отриманої системи $\gamma n - \gamma = \gamma (n - 1)$. Уводячи $\gamma$ довільних сталих $C_1, C_2, \ldots, C_\gamma$ і розв'язуючи систему, одержимо
\begin{equation*}
	\alpha_i^j = \alpha_i^j(C_1, C_2, \ldots, C_\gamma), \quad i = \overline{1, n}, \quad j = \overline{1, \gamma}.
\end{equation*}
\end{enumerate}

		\subsubsection{Розв'язок систем однорідних рівнянь зі сталими коефіцієнтами матричним методом}
		Досить універсальним методом розв’язку лінійних однорідних систем з сталими коефіцієнтами є матричний метод. Він полягає в наступному. Розглядається лінійна система з сталими коефіцієнтами, що записана у векторно-матричному вигляді
\begin{equation*}
	\dot x(t) = A x, \quad x \in \RR^n.	
\end{equation*}

Робиться невироджене перетворення $x = S y$, $y \in \RR^n$, $\det S \ne 0$, де вектор $y(t)$ --- нова невідома векторна функція. Тоді рівняння прийме вигляд
\begin{equation*}
	S \dot y = A S y,
\end{equation*}
або
\begin{equation*}
	\dot y = S^{-1} A S y.
\end{equation*}

Для довільної матриці $A$ завжди існує неособлива матриця $S$, що приводить її до жорданової форми, тобто $S^{-1} A S = \Lambda$, де $\Lambda$ --- жорданова форма матриці $A$. І система диференціальних рівнянь прийме вигляд
\begin{equation*}
	\dot y = \Lambda y, \quad y \in \RR^n.
\end{equation*}

Складемо характеристичне рівняння матриці $A$
\begin{equation*}
	\det (D - \lambda E) = 0,
\end{equation*}
або
\begin{equation*}
	\lambda^n + p_1 \lambda^{n - 1} + \ldots + p_{n - 1} \lambda + p_n = 0.
\end{equation*}

Алгебраїчне рівняння $n$-го ступеня має $n$ коренів. Розглянемо різні випадки:
\begin{enumerate}
\item Нехай $\lambda_1, \lambda_2, \ldots, \lambda_n$ --- дійсні різні числа. Тоді матриця $\Lambda$ має вигляд
\begin{equation*}
	\Lambda = 
	\begin{pmatrix}
		\lambda_1 & 0 & \cdots & 0 \\
		0 & \lambda_2 & \cdots & 0 \\
		\vdots & \vdots & \ddots & \vdots \\
		0 & 0 & \cdots & \lambda_n.
	\end{pmatrix}
\end{equation*}

І перетворена система диференціальних рівнянь розпадається на $n$ незалежних рівнянь
\begin{equation*}
	\dot y_1 = \lambda_1 y_1, \quad \dot y_2 = \lambda_2 y_2, \quad \ldots, \quad \dot y_n = \lambda_n y_n.
\end{equation*}

Розв’язуючи кожне окремо, отримаємо
\begin{equation*}
	y_1 = C_1 e^{\lambda_1 t}, \quad y_2 = C_2 e^{\lambda_2 t}, \quad \ldots, \quad y_n = C_n e^{\lambda_n t}.
\end{equation*}

Або в матричному вигляді
\begin{equation*}
	y = e^{\Lambda t} C,
\end{equation*}
де
\begin{equation*}
	e^{\Lambda t} = 
	\begin{pmatrix}
		e^{\lambda_1 t} & 0 & \cdots & 0 \\
		0 & e^{\lambda_2 t} & \cdots & 0 \\
		\vdots & \vdots & \ddots & \vdots \\
		0 & 0 & \cdots & e^{\lambda_n t}
	\end{pmatrix}, \quad
	C = \begin{pmatrix} C_1 \\ C_2 \\ \vdots \\ C_n \end{pmatrix}.
\end{equation*}

Звідси розв’язок вихідного рівняння має вигляд $x = S e^{\Lambda t} C$. Для знаходження матриці $S$ треба розв’язати матричне рівняння
\begin{equation*}
	S^{-1} A S = \Lambda
\end{equation*}
або
\begin{equation*}
	A S = S \Lambda
\end{equation*}
де $\Lambda$ --- жорданова форма матриці $A$. Якщо матрицю $S$ записати у вигляді
\begin{equation*}
	S = 
	\begin{pmatrix} 
		\alpha_1^1 & \alpha_1^2 & \cdots & \alpha_1^n \\
		\alpha_2^1 & \alpha_2^2 & \cdots & \alpha_2^n \\
		\vdots & \vdots & \ddots & \vdots \\
		\alpha_n^1 & \alpha_n^2 & \cdots & \alpha_n^n
	\end{pmatrix},
\end{equation*}
то для кожного з стовпчиків $s_i = (\alpha_1^i, \alpha_2^i, \ldots, \alpha_n^i)^T$, матричне рівняння перетвориться до
\begin{equation*}
	A s_i = \lambda_i s_i, \quad i = \overline{1, n}.
\end{equation*}

Таким чином, у випадку різних дійсних власних чисел матриця $S$ являє собою набір $n$ власних векторів, що відповідають різним власним числам.

\item Нехай $\lambda_{1,2} = p \pm i q$ --- комплексний корінь. Тоді відповідна клітка Жордана має вигляд
\begin{equation*}
	\Lambda_{1,2} = \begin{pmatrix} p & q \\ -q & p \end{pmatrix},
\end{equation*}
а перетворена система диференціальних рівнянь
\begin{equation*}
	\left\{
		\begin{aligned}
			\dot y_1 &= p y_1 + q y_2, \\
			\dot y_2 &= - q y_1 + p y_2.
		\end{aligned}
	\right.
\end{equation*}

Неважко перевірити, що розв’язок отриманої системи диференціальних рівнянь має вигляд
\begin{align*}
	y_1 &= c_1 e^{pt} \cos  qt + c_2 e^{pt} \sin qt, \\
	y_2 &= c_2 e^{pt} \cos  qt - c_1 e^{pt} \sin qt.
\end{align*}

Або в матричному вигляді
\begin{equation*}
	\begin{pmatrix} y_1 \\ y_2 \end{pmatrix} =
	\begin{pmatrix}
		e^{pt} \cos qt & e^{pt} \sin qt \\
		- e^{pt} \sin qt & e^{pt} \cos qt
	\end{pmatrix}
	\begin{pmatrix} c_1 \\ c_2 \end{pmatrix}.
\end{equation*}

Таким чином, комплексно-спряженим власним числам $\lambda_{1,2}$ відповідає розв’язок  
\begin{equation*}
	y = e^{\Lambda t} C,
\end{equation*}
де
\begin{equation*}
	e^{\Lambda t} =
	\begin{pmatrix}
		e^{pt} \cos qt & e^{pt} \sin qt \\
		- e^{pt} \sin qt & e^{pt} \cos qt
	\end{pmatrix} 
\end{equation*}

\item Нехай $\lambda$ --- кратний корінь, кратності $m \le n$, тобто $\lambda_1 = \lambda_2 = \ldots = \lambda_m = \lambda$ і йому відповідають $r \le m$ лінійно незалежних векторів. Тоді клітка Жордана, що відповідає цьому власному числу, має вид
\begin{equation*}
	\Lambda = 
	\begin{pmatrix}
		\Lambda_1 & \textbf{0} \\
		\textbf{0} & \Lambda_2, 
	\end{pmatrix}
\end{equation*}
де
\begin{align*}
	\Lambda_1 &= 
	\begin{pmatrix} 
		\lambda & 0 & \cdots & 0 & 0 \\
		0 & \lambda & \cdots & 0 & 0 \\
		\vdots & \vdots & \ddots & \vdots & \vdots \\
		0 & 0 & \cdots & \lambda & 0 \\
		0 & 0 & \cdots & 0 & \lambda
	\end{pmatrix} \in \RR^{r \times r}, \\
	\Lambda_2 &= 
	\begin{pmatrix} 
		\lambda & 1 & \cdots & 0 & 0 \\
		0 & \lambda & \ddots & 0 & 0 \\
		\vdots & \vdots & \ddots & \ddots & \vdots \\
		0 & 0 & \cdots & \lambda & 1 \\
		0 & 0 & \cdots & 0 & \lambda
	\end{pmatrix} \in \RR^{(m - r) \times (m - r)}.
\end{align*}
 
І перетворена підсистема, що відповідає власному числу $\lambda$, розпадається не дві підсистеми
\begin{align*}
	\dot y_1 &= \Lambda_1 y_1, \quad y_1 \in \RR^r, \\
	\dot y_2 &= \Lambda_2 y_2, \quad y_2 \in \RR^{m - r},
\end{align*}

Розв’язок першої знаходиться з використанням зазначеного в першому пункті підходу. Розглянемо другу підсистему. Запишемо її в координатному вигляді
 
Розв’язок останнього рівняння цієї підсистеми має вигляд
\begin{equation*}
	y_{2, m - r} = c_{2, m - r} e^{\lambda t}
\end{equation*}

Підставимо його в передостаннє рівняння. Одержуємо
\begin{equation*}
	\dot y_{2, m - r - 1} = \lambda y_{2, m - r - 1} + c_{2, m - r} e^{\lambda t}.
\end{equation*}

Загальний розв’язок лінійного неоднорідного рівняння має вигляд суми загального розв’язку однорідного і частинного розв’язку неоднорідних рівнянь, тобто
\begin{equation*}
	y_{2, m - r - 1} = y_{2, m - r - 1, \text{homo}} + y_{2, m - r - 1, \text{hetero}}.
\end{equation*}

Загальний розв’язок однорідного має вигляд
\begin{equation*}
	\dot y_{2, m - r - 1, \text{homo}} = c_{2, m - r - 1} e^{\lambda t}.
\end{equation*}

Частинний розв’язок неоднорідного шукаємо методом невизначених коефіцієнтів у вигляді
\begin{equation*}
	y_{2, m - r - 1, \text{hetero}} = A t e^{\lambda t},
\end{equation*}
де $A$ --- невідома стала. Підставивши в неоднорідне рівняння, одержимо
\begin{equation*}
	A e^{\lambda t} + A \lambda t e^{\lambda t} = A \lambda t e^{\lambda t} + c_{2, m - r} e^{\lambda t}.
\end{equation*}

Звідси $A = c_{2, m - r}$ і загальний розв’язок неоднорідного рівняння має вигляд
\begin{equation*}
	y_{2, m - r - 1} = c_{2, m - r - 1} e^{\lambda t} + c_{2, m - r} t e^{\lambda t}.
\end{equation*}

Піднявшись ще на один крок нагору одержимо
\begin{equation*}
	y_{2, m - r - 1} = c_{2, m - r - 2} e^{\lambda t} + c_{2, m - r - 1} t e^{\lambda t} + c_{2, m - r } \frac{t^2}{2!} e^{\lambda t}.
\end{equation*}

Продовжуючи процес далі, маємо
\begin{equation*}
	y_{2, 1} = c_{2, 1} e^{\lambda t} + c_{2, 2} t e^{\lambda t} + \ldots + c_{2, m - r} \frac{t^{m - r - 1}}{(m - r - 1)!} e^{\lambda t}.
\end{equation*}

Або у векторно-матричному вигляді
\begin{equation*}
	y_2(t) = 
	\begin{pmatrix}
		e^{\lambda t} & t e^{\lambda t} & \cdots & \dfrac{t^{m - r - 2}}{(m - r - 2)!} & \dfrac{t^{m - r - 1}}{(m - r - 1)!} \\
		0 & e^{\lambda t} & \cdots & \dfrac{t^{m - r - 3}}{(m - r - 3)!} & \dfrac{t^{m - r - 2}}{(m - r - 2)!} \\
		\vdots & \vdots & \ddots & \vdots & \vdots \\
		0 & 0 & \cdots & e^{\lambda t} & t e^{\lambda t} \\
		0 & 0 & \cdots & 0 & e^{\lambda t}
	\end{pmatrix}
	\begin{pmatrix} c_{2,1} \\ c_{2,2} \\ \vdots \\ c_{2,m-r-1} \\ c_{2,m-r} \end{pmatrix}.
\end{equation*}

Додавши першу підсистему, одержимо
\begin{equation*}
	y = \begin{pmatrix} e^{\Lambda_1 t} & \textbf{0} \\ \textbf{0} & e^{\Lambda_2 t} \end{pmatrix} C,
\end{equation*}
де
\begin{align*}
	e^{\Lambda_1 t} &= 
	\begin{pmatrix} 
		e^{\lambda t} & 0 & \cdots & 0 & 0 \\
		0 & e^{\lambda t} & \cdots & 0 & 0 \\
		\vdots & \vdots & \ddots & \vdots & \vdots \\
		0 & 0 & \cdots & e^{\lambda t} & 0 \\
		0 & 0 & \cdots & 0 & e^{\lambda t}
	\end{pmatrix}, \\
	e^{\Lambda_2 t} &= 
	\begin{pmatrix}
		e^{\lambda t} & t e^{\lambda t} & \cdots & \dfrac{t^{m - r - 2}}{(m - r - 2)!} & \dfrac{t^{m - r - 1}}{(m - r - 1)!} \\
		0 & e^{\lambda t} & \cdots & \dfrac{t^{m - r - 3}}{(m - r - 3)!} & \dfrac{t^{m - r - 2}}{(m - r - 2)!} \\
		\vdots & \vdots & \ddots & \vdots & \vdots \\
		0 & 0 & \cdots & e^{\lambda t} & t e^{\lambda t} \\
		0 & 0 & \cdots & 0 & e^{\lambda t}
	\end{pmatrix}, \\
	C &= \begin{pmatrix} c_{1,1} & \cdots & c_{1,r} & c_{2,1} & \cdots & c_{2,m-r} \end{pmatrix}^T.
\end{align*}

Для останніх двох випадків матриця   знаходиться як розв’язок матричного рівняння
\begin{equation*}
	A S = S \Lambda	
\end{equation*}
\end{enumerate}


		\subsubsection{Вправи для самостійної роботи \todo}
		При розв'язуванні систем методом Ейлера складають характеристичне рівняння, і в залежності від його коренів для кожного $\lambda_i$,    $i = \overline{1, n}$ знаходять відповідний лінійно незалежний розв'язок.

\begin{example}
    Розв'язати систему:
    \[ \left\{ \begin{aligned}
        \dot x &= 2 x + 3 y, \\
        \dot y &= 3 x + 4 y.
    \end{aligned} \right. \]
\end{example}

\begin{solution}
    Характеристичне рівняння має вигляд
    \[ \begin{vmatrix}
        2 - \lambda & 1 \\
        3 & 4 - \lambda 
    \end{vmatrix} = 0, \]
    або $\lambda^2 - 6 \lambda + 5 = 0$. \parvskip
    
    Коренями будуть $\lambda_1 = 1$, $\lambda_2 = 5$.
    
    \begin{enumerate}
        \item Знайдемо власний вектор, що відповідає $\lambda_1 = 1$. Підставивши в систему
        \[ \left\{ \begin{aligned}
            (2 - \lambda) \alpha_1 + \alpha_2 &= 0, \\
            3 \alpha_1 + (4 - \lambda) \alpha_2 &= 0, 
        \end{aligned} \right. \]
        одержимо 
        \[ \left\{ \begin{aligned}
            \alpha_1 + \alpha_2 &= 0, \\
            3 \alpha_1 + 3 \alpha_2 &= 0.
        \end{aligned} \right. \]
        
        Звідси $\alpha_1 = 1$, $\alpha_2 = -1$.
        
        \item
        Знайдемо власний вектор, що відповідає $\lambda_2 = 5$. Підставивши в систему, одержимо
        \[ \left\{ \begin{aligned}
            - 3 \alpha_1 + \alpha_2 &= 0, \\
            3 \alpha_1 - \alpha_2 &= 0.
        \end{aligned} \right. \]
        
        Звідси $\alpha_1 = 1$, $\alpha_2 = 3$.
    \end{enumerate}
    
    Таким чином, одержимо розв'язок системи у вигляді
    \[ \begin{pmatrix} x \\ y \end{pmatrix} = c_1 e^t \begin{pmatrix} 1 \\ -1 \end{pmatrix} + C_2 e^{5t} \begin{pmatrix} 1 \\ 3 \end{pmatrix} = \begin{pmatrix} e^t & e^{5t} \\ - e^t & 3 e^{5t} \end{pmatrix} \begin{pmatrix} c_1 \\ c_2 \end{pmatrix}. \]
\end{solution}

\begin{example}
    Розв'язати систему:
    \[ \left\{ \begin{aligned}
        \dot x &= x + y, \\
        \dot y &= -2 x + 3 y.
    \end{aligned} \right. \]
\end{example}

\begin{solution}
    Характеристичне рівняння має вигляд
    \[ \begin{vmatrix}
        1 - \lambda & 1 \\
        -2 & 3 - \lambda 
    \end{vmatrix} = 0, \]
    або $\lambda^2 - 4 \lambda + 5 = 0$. \parvskip
    
    Коренями будуть $\lambda_{1,2} = 2 \pm i$. \parvskip
    
    Візьмемо $\lambda_1 = 2 + i$. Підставивши в систему
    \[ \left\{ \begin{aligned}
        (1 - \lambda) \alpha_1 + \alpha_2 &= 0, \\
        -2 \alpha_1 + (3 - \lambda) \alpha_2 &= 0, 
    \end{aligned} \right. \]
    одержимо 
    \[ \left\{ \begin{aligned}
        (-1 - i) \alpha_1 + \alpha_2 &= 0, \\
        -2 \alpha_1 + (1 - i) \alpha_2 &= 0.
    \end{aligned} \right. \]
    
    Звідси $\alpha_1 = 1$, $\alpha_2 = 1 + i$. \parvskip
    
    Запишемо вектор розв'язку
    \begin{multline*} \begin{pmatrix} x \\ y \end{pmatrix} = \begin{pmatrix} e^{(2 + i) t} \\ (1 + i) e^{(2 + i) t} \end{pmatrix} = \begin{pmatrix} e^{2 t} (\cos t + i \sin t) \\ e^{2 t} (1 + i) (\cos t + i \sin t) \end{pmatrix} = \\ = \begin{pmatrix} e^{2 t} \cos t \\ e^{2 t} (\cos t - \sin t) \end{pmatrix} + i \begin{pmatrix} e^{2 t} \sin t \\ e^{2 t} (\cos t + \sin t) \end{pmatrix}. \end{multline*}
    
    Оскільки комплексно-спряженому розв'язку відповідають два лінійно незалежних розв'язки, то загальний розв'язок має вигляд
    \begin{multline*} \begin{pmatrix} x \\ y \end{pmatrix} = c_1 \begin{pmatrix} e^{2 t} \cos t \\ e^{2 t} \cos t - \sin t \end{pmatrix} + c_2 \begin{pmatrix} e^{2 t} \sin t \\ e^{2 t} (\cos t + \sin t) \end{pmatrix} = \\ = \begin{pmatrix} e^{2 t} \cos t & e^{2 t} \sin t \\ e^{2 t} (\cos t - \sin t) & e^{2 t} (\cos t + \sin t) \end{pmatrix} \begin{pmatrix} c_1 \\ c_2 \end{pmatrix}. \end{multline*}
\end{solution}

\begin{example}
    Розв'язати систему:
    \[ \left\{ \begin{aligned}
        \dot x &= 2 x + y, \\
        \dot y &= -x + 4 y.
    \end{aligned} \right. \]
\end{example}
 
\begin{solution}
    Характеристичне рівняння має вигляд
    \[ \begin{vmatrix}
        2 - \lambda & 1 \\
        -1 & 4 - \lambda 
    \end{vmatrix} = 0, \]
    або $\lambda^2 - 6 \lambda + 9 = 0$. \parvskip
    
    Коренями будуть $\lambda_1 = \lambda_2 = 3$. Оскільки
    \[ \rang \left. \begin{pmatrix} 
        2 - \lambda & 1 \\
        -1 & 4 - \lambda 
    \end{pmatrix} \right|_{\lambda = 3} 
    = 
    \rang \begin{pmatrix} 
        -1 & 1 \\
        -1 & 1
    \end{pmatrix} = 1, \]
    то матриця має один власний вектор. Тому розв'язок шукаємо у вигляді
    \[ x = (a_1^1 + a_1^2 t) e^{3t}, \quad y = (a_2^1 + a_2^2 t) e^{3t}. \]
    
    Підставимо в систему
    \[ \left\{ \begin{aligned}
        3 e^{3t} (a_1^1 + a_1^2 t) + a_1^2 e^{3t} &= 2 (a_1^1 + a_1^2 t) e^{3t} + (a_2^1 + a_2^2) e^{3t}, \\
        3 e^{3t} (a_2^1 + a_2^2 t) + a_2^2 e^{3t} &= - (a_1^1 + a_1^2 t) e^{3t} + 4 (a_2^1 + a_2^2) e^{3t}.
    \end{aligned} \right. \]
    
    Прирівнявши коефіцієнти при однакових членах, одержимо дві системи
    \[ \left\{ \begin{aligned} 
        3 a_1^2 &= 2 a_1^2 + a_2^2, \\
        3 a_2^2 &= -a_1^2 + 4 a_2^2,
    \end{aligned} \right. 
    \qquad
    \left\{ \begin{aligned} 
        3 a_1^1 + a_1^2 &= 2 a_1^1 + a_2^1, \\
        3 a_2^1 + a_2^2 &= -a_1^1 + 4 a_2^1.
    \end{aligned} \right.\]

    Або
    \[ \left\{ \begin{aligned} 
        -a_1^2 + a_2^2 &= 0, \\
        -a_1^2 + a_2^2 &= 0,
    \end{aligned} \right. 
    \qquad
    \left\{ \begin{aligned} 
        -a_1^1 + a_2^1 &= a_1^2, \\
        -a_1^1 + a_2^1 &= a_1^2.
    \end{aligned} \right.\]

    З першої системи одержуємо $a_1^2 = a_2^2 = c_1$. Підставивши в другу, одержимо $-a_1^1 + a_2^1 = c_1$. Поклавши $a_1^1 = c_2$, одержимо $c_2^1 = c_1 + c_2$. Таким чином,
    \begin{multline*} \begin{pmatrix} x \\ y \end{pmatrix} = \begin{pmatrix} (c_2 + c_1 t) e^{3 t} \\ (c_1 + c_2 + c_1 t) e^{3 t} \end{pmatrix} = c_1 \begin{pmatrix} t e^{3 t} \\ (1 + t) e^{3 t} \end{pmatrix} + c_2 \begin{pmatrix} e^{3 t} \\ e^{3 t} \end{pmatrix} = \\ = \begin{pmatrix} t e^{3 t} & e^{3 t} \\ (1 + t) e^{3 t} & e^{3 t} \end{pmatrix} \begin{pmatrix} c_1 \\ c_2 \end{pmatrix}. \end{multline*}
\end{solution}

Розв'яжемо ці ж системи матричним методом.

\setcounter{problem}{0}
\begin{example}
    Розв'язати систему:
    \[ \left\{ \begin{aligned}
        \dot x &= 2 x + 3 y, \\
        \dot y &= 3 x + 4 y.
    \end{aligned} \right. \]
\end{example}

\begin{solution}
    Характеристичне рівняння має вигляд
    \[ \begin{vmatrix}
        2 - \lambda & 1 \\
        3 & 4 - \lambda 
    \end{vmatrix} = 0, \]
    або $\lambda^2 - 6 \lambda + 5 = 0$. \parvskip
    
    Його коренями будуть $\lambda_1 = 1$, $\lambda_2 = 5$. Тому 
    \[ \Lambda = \begin{pmatrix} 1 & 0 \\ 0 & 5 \end{pmatrix} \quad e^{\Lambda t} = \begin{pmatrix} e^t & 0 \\ 0 & e^{5 t} \end{pmatrix}. \]
    
    Розв'язуємо матричне рівняння $A S = S \Lambda$, або 
    \[ \begin{pmatrix} 2 & 1 \\ 3 & 4 \end{pmatrix} \begin{pmatrix} a_1^1 & a_1^2 \\ a_2^1 & a_2^2 \end{pmatrix} = \begin{pmatrix} a_1^1 & a_1^2 \\ a_2^1 & a_2^2 \end{pmatrix} \begin{pmatrix} 1 & 0 \\ 0 & 5 \end{pmatrix}. \]
    
    Воно розпадається на два 
    \[ \left\{ \begin{aligned} 
        2 a_1^1 + a_2^1 &= a_1^1, \\
        3 a_1^1 + 4 a_2^1 &= a_2^1,
    \end{aligned} \right. 
    \qquad
    \left\{ \begin{aligned} 
        2 a_1^2 + a_2^2 &= 5 a_1^2, \\
        3 a_1^2 + 4 a_2^2 &= 5 a_2^2,
    \end{aligned} \right.\]
    
    Після перенесення всіх членів уліво, одержимо
    \[ \left\{ \begin{aligned} 
        a_1^1 + a_2^1 &= 0, \\
        3  a_1^1 + 3 a_2^1 &= 0,
    \end{aligned} \right. 
    \qquad
    \left\{ \begin{aligned} 
        -3 a_1^2 + a_2^2 &= 0, \\
        3 a_1^2 - a_2^2 &= 0,
    \end{aligned} \right.\]
    
    Звідси $a_1^1 = 1$, $a_2^1 = - 1$, $a_1^2 = 1$, $a_2^2 = 3$. \parvskip
    
    Таким чином, загальний розв'язок має вигляд
    \[ S = \begin{pmatrix} 1 & 1 \\ -1 & 3 \end{pmatrix}, \quad \begin{pmatrix} x \\ y \end{pmatrix} = \begin{pmatrix} e^t & e^{5t} \\ -e^t & 3 e^{5 t} \end{pmatrix} \begin{pmatrix} c_1 \\ c_2 \end{pmatrix}. \]
\end{solution}

\begin{example}
    Розв'язати систему:
    \[ \left\{ \begin{aligned}
        \dot x &= x + y, \\
        \dot y &= -2 x + 3 y.
    \end{aligned} \right. \]
\end{example}
 
\begin{solution}
    Характеристичне рівняння має вигляд
    \[ \begin{vmatrix}
        1 - \lambda & 1 \\
        -2 & 3 - \lambda 
    \end{vmatrix} = 0, \]
    або $\lambda^2 - 4 \lambda + 5 = 0$. \parvskip
    
    Коренями будуть $\lambda_{1,2} = 2 \pm i$. Тому 
    \[ \Lambda = \begin{pmatrix} 2 & 1 \\ -1 & 2 \end{pmatrix} \quad e^{\Lambda t} = \begin{pmatrix} e^{2 t} \cos t & e^{2 t} \sin t \\ - e^{2 t} \sin t & e^{2 t} \cos t \end{pmatrix}. \]
    
    Матричне рівняння має вигляд $A S = S \Lambda$, чи
    \[ \begin{pmatrix} 1 & 1 \\ -2 & 3 \end{pmatrix} \begin{pmatrix} a_1^1 & a_1^2 \\ a_2^1 & a_2^2 \end{pmatrix} = \begin{pmatrix} a_1^1 & a_1^2 \\ a_2^1 & a_2^2 \end{pmatrix} \begin{pmatrix} 2 & 1 \\ -1 & 2 \end{pmatrix}. \]
    
    Розпишемо його поелементно
    \[ \left\{ \begin{aligned} 
        a_1^1 + a_2^1 &= 2 a_1^1 - a_1^2, \\
        -2 a_1^1 + 3 a_2^1 &= 2 a_2^1 - a_2^2, \\
        a_1^2 + a_2^2 &= a_1^1 + 2 a_1^2, \\
        -2 a_1^2 + 3 a_2^2 &= a_2^1 + 2 a_2^2.
    \end{aligned} \right.\]
    
    На відміну від попереднього пункту (і це істотно ускладнює обчислення) система не розщеплюється  на дві незалежні підсистеми. Після перенесення всіх членів в одну сторону, одержимо систему
    \[ \left\{ \begin{aligned} 
        - a_1^1 - a_1^2 + a_2^1 &= 0, \\
        -2 a_1^1 + a_2^1 + a_2^2 &= 0, \\
        -a_1^1 - a_1^2 + a_2^2 &= 0, \\
        -2 a_1^2 + a_2^1 + a_2^2 &= 0.
    \end{aligned} \right.\]
     
    Помножимо перше рівняння на $-2$ і, склавши з другим, підставимо на місце другого. Далі, помножимо перше рівняння на $-1$ і, склавши з третім, поставимо його на місце третього. Одержуємо систему
    \[ \left\{ \begin{aligned} 
        - a_1^1 + a_1^2 + a_2^1 &= 0, \\
        -2 a_1^2 - a_2^1 + a_2^2 &= 0, \\
        -2 a_1^2 - a_2^1 + a_2^2 &= 0, \\
        -2 a_1^2 - a_2^1 + a_2^2 &= 0.
    \end{aligned} \right.\]
     
    Останні два рівняння можна відкинути. Залишається
    \[ \left\{ \begin{aligned} 
        - a_1^1 + a_1^2 + a_2^1 &= 0, \\
        -2 a_1^2 - a_2^1 + a_2^2 &= 0.
    \end{aligned} \right.\]
    
    Покладаємо $a_1^2 = a_2^2 = 1$. Тоді $a_2^1 = -1$, $a_1^1 = 0$. Таким чином,
    \begin{multline*} S = \begin{pmatrix} 0 & 1 \\ -1 & 1 \end{pmatrix}, \quad \begin{pmatrix} x \\ y \end{pmatrix} = \begin{pmatrix} 0 & 1 \\ -1 & 1 \end{pmatrix} \begin{pmatrix} e^{2 t} \cos t & e^{2 t} \sin t \\ - e^{2 t} \sin t & e^{2 t} \cos t \end{pmatrix} \begin{pmatrix} c_1 \\ c_2 \end{pmatrix} = \\ = \begin{pmatrix} - e^{2 t} \sin t & e^{2 t} \cos t \\ - e^{2 t} (\cos t + \sin t) & e^{2 t} (\cos t - \sin t) \end{pmatrix} \begin{pmatrix} c_1 \\ c_2 \end{pmatrix}. \end{multline*}
\end{solution}

\begin{example}
    Розв'язати систему:
    \[ \left\{ \begin{aligned}
        \dot x &= 2 x + y, \\
        \dot y &= -x + 4 y.
    \end{aligned} \right. \]
\end{example}

\begin{solution}
    Характеристичне рівняння має вигляд
    \[ \begin{vmatrix}
        2 - \lambda & 1 \\
        -1 & 4 - \lambda 
    \end{vmatrix} = 0, \]
    або $\lambda^2 - 6 \lambda + 9 = 0$. \parvskip
    
    Коренями будуть $\lambda_1 = \lambda_2 = 3$. Оскільки
    \[ \rang \left. \begin{pmatrix} 
        2 - \lambda & 1 \\
        -1 & 4 - \lambda 
    \end{pmatrix} \right|_{\lambda = 3} 
    = 
    \rang \begin{pmatrix} 
        -1 & 1 \\
        -1 & 1
    \end{pmatrix} = 1, \]
    то матриця має один власний вектор і клітка Жордана має вигляд
    \[ \Lambda = \begin{pmatrix} 3 & 1 \\ 0 & 3 \end{pmatrix} \quad e^{\Lambda t} = \begin{pmatrix} e^{3 t} & t e^{3 t} \\ 0 t & e^{3 t} \cos t \end{pmatrix}. \]
    
    Матричне рівняння має вигляд $A S = S \Lambda$, чи
    \[ \begin{pmatrix} 2 & 1 \\ -1 & 4 \end{pmatrix} \begin{pmatrix} a_1^1 & a_1^2 \\ a_2^1 & a_2^2 \end{pmatrix} = \begin{pmatrix} a_1^1 & a_1^2 \\ a_2^1 & a_2^2 \end{pmatrix} \begin{pmatrix} 3 & 1 \\ 0 & 3 \end{pmatrix}. \]
    
    Розпишемо його поелементно
    \[ \left\{ \begin{aligned} 
        2 a_1^1 + a_2^1 &= 3 a_1^1, \\
        - a_1^1 + 4 a_2^1 &= 3 a_2^1, \\
    \end{aligned} \right. 
    \qquad
    \left\{ \begin{aligned} 
        2 a_1^2 + a_2^2 &= a_1^1 + 3 a_1^2, \\
        - a_1^2 + 4 a_2^2 &= a_2^1 + 3 a_2^2.
    \end{aligned} \right.\]
    
    На відміну від комплексних коренів, можна розв'язати спочатку першу підсистему, а потім другу. Перша має вид
    \[ \left\{ \begin{aligned} 
        - a_1^1 + a_2^1 &= 0, \\
        - a_1^1 + a_2^1 &= 0, \\
    \end{aligned} \right. \]
    
    Звідси $a_1^1 = a_2^1 = 1$. \parvskip
    
    Підставивши в другу, одержимо
    \[ \left\{ \begin{aligned} 
        - a_1^2 + a_2^2 &= 1, \\
        - a_1^2 + a_2^2 &= 1.
    \end{aligned} \right.\]
    
    Звідси $a_2^2 = 1$, $a_1^2 = 0$. Таким чином одержали
    \begin{multline*} S = \begin{pmatrix} 1 & 0 \\ 1 & 1 \end{pmatrix}, \quad \begin{pmatrix} x \\ y \end{pmatrix} = \begin{pmatrix} 1 & 0 \\ 1 & 1 \end{pmatrix} \begin{pmatrix} e^{3 t} & t e^{3 t} \\ -0 & e^{3 t} \end{pmatrix} \begin{pmatrix} c_1 \\ c_2 \end{pmatrix} = \\ = \begin{pmatrix} e^{3 t} & t e^{3 t} \\ e^{3 t} & (t + 1) e^{3 t} \end{pmatrix} \begin{pmatrix} c_1 \\ c_2 \end{pmatrix}. \end{multline*}
\end{solution}

\begin{remark}
    Якщо власні числа дійсні різні, то обидва методи еквівалентні. Якщо власні числа комплексні, переважніше метод Ейлера, якщо кратні, то матричний метод.
\end{remark}

Розв'язати лінійні однорідні системи методом Ейлера чи матричним методом.
\begin{multicols}{2}
    \begin{problem}
        \[ \left\{ \begin{aligned} 
            \dot x &= x - y, \\
            \dot y &= -4 x + y.
        \end{aligned} \right. \]
    \end{problem}
    
    \begin{problem}
        \[ \left\{ \begin{aligned} 
            \dot x &= -x + 8 y, \\
            \dot y &= x + y.
        \end{aligned} \right. \]
    \end{problem}
    
    \begin{problem}
        \[ \left\{ \begin{aligned} 
            \dot x &= x -3 y, \\
            \dot y &= 3 x + y.
        \end{aligned} \right. \]
    \end{problem}
    
    \begin{problem}
        \[ \left\{ \begin{aligned} 
            \dot x &= - x - 5 y, \\
            \dot y &= x + y.
        \end{aligned} \right. \]
    \end{problem}
    
    \begin{problem}
        \[ \left\{ \begin{aligned} 
            \dot x &= 3 x - y, \\
            \dot y &= 4 x - y.
        \end{aligned} \right. \]
    \end{problem}
    
    \begin{problem}
        \[ \left\{ \begin{aligned} 
            \dot x &= -3 x + 2 y, \\
            \dot y &= -2 x + y.
        \end{aligned} \right. \]
    \end{problem}
    
    \begin{problem}
        \[ \left\{ \begin{aligned} 
            \dot x &= 5 x + 3 y, \\
            \dot y &= -3 x - y.
        \end{aligned} \right. \]
    \end{problem}
\end{multicols}

Розв'язати лінійні однорідні системи методом Ейлера чи матричним методом (після системи вкзані власні числа для спрощення обчислень).
\begin{multicols}{2}
    \begin{problem}
        \[ \left\{ \begin{aligned} 
            \dot x &= x - y + z, \\
            \dot y &= x + y - z, \\
            \dot z &= 2 x - y.
        \end{aligned} \right. \]
        ($\lambda_1 = 1$, $\lambda_2 = 2$, $\lambda_3 = -1$)
    \end{problem}
    
    \begin{problem}
        \[ \left\{ \begin{aligned} 
            \dot x &= x - 2 y - z, \\
            \dot y &= -x + y + z, \\
            \dot z &= x - z.
        \end{aligned} \right. \]
        ($\lambda_1 = 0$, $\lambda_2 = 2$, $\lambda_3 = -1$)
    \end{problem}
    
    \begin{problem}
        \[ \left\{ \begin{aligned} 
            \dot x &= 2 x - y + z, \\
            \dot y &= x + 2 y - z, \\
            \dot z &= x - y + 2 z.
        \end{aligned} \right. \]
        ($\lambda_1 = 1$, $\lambda_2 = 2$, $\lambda_3 = 3$)
    \end{problem}
    
    \begin{problem}
        \[ \left\{ \begin{aligned} 
            \dot x &= 3 x - y + z, \\
            \dot y &= x + y + z, \\
            \dot z &= 4 x - y + 4 z.
        \end{aligned} \right. \]
        ($\lambda_1 = 1$, $\lambda_2 = 2$, $\lambda_3 = 5$)
    \end{problem}
    
    \begin{problem}
        \[ \left\{ \begin{aligned} 
            \dot x &= -3 x - 4 y - 2 z, \\
            \dot y &= x + z, \\
            \dot z &= 6 z - 6 y + 5 z.
        \end{aligned} \right. \]
        ($\lambda_1 = 1$, $\lambda_2 = 2$, $\lambda_3 = -1$)
    \end{problem}
    
    \begin{problem} 
        \[ \left\{ \begin{aligned} 
            \dot x &= x - y - z, \\
            \dot y &= x + y, \\
            \dot z &= 3 x + z.
        \end{aligned} \right. \]
        ($\lambda_1 = 1$, $\lambda_{2, 3} = 1 +\pm 3 i$)
    \end{problem}
    
    \begin{problem}
        \[ \left\{ \begin{aligned} 
            \dot x &= 2 x + y, \\
            \dot y &= x + 3 y - z, \\
            \dot z &= -x + y - z.
        \end{aligned} \right. \]
        ($\lambda_1 = 2$, $\lambda_{2, 3} = 3 \pm i$)
    \end{problem}
    
    \begin{problem}
        \[ \left\{ \begin{aligned} 
            \dot x &= 2 x - y + 2 z, \\
            \dot y &= x + z, \\
            \dot z &= -2 x - y + 2 z.
        \end{aligned} \right. \]
        ($\lambda_1 = 2$, $\lambda_{2, 3} = \pm i$)
    \end{problem}
    
    \begin{problem}
        \[ \left\{ \begin{aligned} 
            \dot x &= 4 x - y - z, \\
            \dot y &= x + 2 y - z, \\
            \dot z &= x - y + 2 z.
        \end{aligned} \right. \]
        ($\lambda_1 = 2$, $\lambda_2 = \lambda_3 = 3$)
    \end{problem}
    
    \begin{problem}
        \[ \left\{ \begin{aligned} 
            \dot x &= 2 x - y - z, \\
            \dot y &= 3 x - 2 y - 3 z, \\
            \dot z &= -x + y + 2 z.
        \end{aligned} \right. \]
        ($\lambda_1 = 0$, $\lambda_2 = \lambda_3 = 1$)
    \end{problem}
    
    \begin{problem}
        \[ \left\{ \begin{aligned} 
            \dot x &= - 2 x + y - 2 z, \\
            \dot y &= x - 2 y + 2 z, \\
            \dot z &= 3 x - 3 y + 5 z.
        \end{aligned} \right. \]
        ($\lambda_1 = 3$, $\lambda_2 = \lambda_3 = -1$)
    \end{problem}
    
    \begin{problem}
        \[ \left\{ \begin{aligned} 
            \dot x &= 3 x - 2 y - z, \\
            \dot y &= 3 x - 4 y + 2 z, \\
            \dot z &= 2 x - 4 y.
        \end{aligned} \right. \]
        ($\lambda_1 = \lambda_2 = 2$, $\lambda_3 = -5$)
    \end{problem}
    
    \begin{problem}
        \[ \left\{ \begin{aligned} 
            \dot x &= x - y + z, \\
            \dot y &= x + y - z, \\
            \dot z &= - y + 2 z.
        \end{aligned} \right. \]
        ($\lambda_1 = \lambda_2 = 1$, $\lambda_3 = 2$)
    \end{problem}
    
    \begin{problem}
        \[ \left\{ \begin{aligned} 
            \dot x &= - x + y - 2 z, \\
            \dot y &= 4 x + y, \\
            \dot z &= - y + 2 z.
        \end{aligned} \right. \]
        ($\lambda_1 = 1$, $\lambda_2 = \lambda_3 = -1$)
    \end{problem}
    
    \begin{problem}
        \[ \left\{ \begin{aligned} 
            \dot x &= 2 x + y, \\
            \dot y &= 2 y + 4 z, \\
            \dot z &= x - z.
        \end{aligned} \right. \]
        ($\lambda_1 = \lambda_2 = 0$, $\lambda_3 = 3$)
    \end{problem}
    
    \begin{problem}
        \[ \left\{ \begin{aligned} 
            \dot x &= 2 x - y - z, \\
            \dot y &= 2 x - y - z, \\
            \dot z &= - x + z.
        \end{aligned} \right. \]
        ($\lambda_1 = \lambda_2 = \lambda_3 = 2$)
    \end{problem}
\end{multicols}


	\subsection{Лінійні неоднорідні системи \todo}
	Система диференціальних рівнянь, що записана у вигляді 
\begin{equation*}
	\left\{
		\begin{array}{rl}
			\dot x_1 &= a_{11}(t) x_1 + a_{12}(t) x_2 + \ldots + a_{1n}(t) x_n + f_1(t), \\
			\dot x_2 &= a_{21}(t) x_1 + a_{22}(t) x_2 + \ldots + a_{2n}(t) x_n + f_2(t), \\
			\hdotsfor{2} \\
			\dot x_n &= a_{n1}(t) x_1 + a_{n2}(t) x_2 + \ldots + a_{nn}(t) x_n + f_2(t),
		\end{array}
	\right.
\end{equation*} 
чи у векторно-матричному вигляді
\begin{equation*}
	\dot x = A(t) x + f(t)
\end{equation*}
називається системою лінійних неоднорідних диференціальних рівнянь.


		\subsubsection{Властивості розв'язків лінійних неоднорідних систем \todo}
		\setcounter{property}{0}
\begin{property}
	Якщо вектор 
	\begin{equation*}
		x(t) = \begin{pmatrix} x_1(t) & \cdots & x_n(t) \end{pmatrix}^T
	\end{equation*}
	є розв'язком лінійної неоднорідної системи, a 
	\begin{equation*}
		y(t) = \begin{pmatrix} y_1(t) & \cdots & y_n(t) \end{pmatrix}^T
	\end{equation*}
	розв'язком відповідної лінійної однорідної системи, то сума $x(t) + y(t)$ є розв'язком лінійної неоднорідної системи.
\end{property}

\begin{proof}
	Дійсно, за умовою
	\begin{equation*}
		\dot x(t) - A(t) x(t) \equiv f(t)
	\end{equation*}
	і
	\begin{equation*}
		\dot y(t) - A(t) y(t) \equiv 0.
	\end{equation*}

	Але тоді і
	\begin{multline*}
		\frac{\diff}{\diff t} (x(t) + y(t)) - A(t) (x(t) + y(t)) = \left( \frac{\diff}{\diff t} x(t) - A(t) x(t) \right) + \\
		+ \left( \frac{\diff}{\diff t} y(t) - A(t) y(t) \right) \equiv f(t) + 0 \equiv f(t),
	\end{multline*}
	тобто $x(t) + y(t)$ є розв'язком неоднорідної системи.
\end{proof}

\begin{property}[Принцип суперпозиції]
	Якщо вектори 
	\begin{equation*}
		x_i(t) = \begin{pmatrix} x_{1i}(t) & \cdots & x_{ni}(t) \end{pmatrix}^T, \quad i = \overline{1, n}
	\end{equation*}
	є розв'язками лінійних неоднорідних систем
	\begin{equation*}
		\dot x(t) - A(t) x(t) \equiv f_i(t) \quad i = \overline{1, n}
	\end{equation*}
	де 
	\begin{equation*}
		f_i(t) = \begin{pmatrix} f_{1i}(t) & \cdots & f_{ni}(t) \end{pmatrix}^T, \quad i = \overline{1, n},
	\end{equation*}
	то вектор $x(t) = \sum_{i = 1}^n C_i x_i(t)$, де $C_i$ --- довільні сталі буде розв'язком лінійної неоднорідної системи
	\begin{equation*}
		\dot x(t) - A(t) x(t) \equiv \sum_{i = 1}^n C_i f_i(t) \quad i = \overline{1, n}.
	\end{equation*}
\end{property}

\begin{proof}
	Дійсно, за умовою виконуються $n$ тотожностей
	\begin{equation*}
		\dot x_i(t) - A(t) x_i(t) \equiv f_i(t) \quad i = \overline{1, n}.
	\end{equation*}

	Склавши лінійну комбінацію з лівих і правих частин, одержимо
	\begin{multline*}
		\frac{\diff}{\diff t} \left( \sum_{i = 1}^n C_i x_i(t) \right) - A(t) \left( \sum_{i = 1}^n C_i x_i(t) \right) = \\
		= \sum_{i = 1}^n C_i (\dot x_i(t) - A(t) x_i(t) ) \equiv \sum_{i = 1}^n C_i f_i(t),
	\end{multline*}
	тобто лінійна комбінація $x(t) = \sum_{i = 1}^n C_i x_i(t)$ буде розв'язком системи
	\begin{equation*}
		\dot x(t) - A(t) x(t) \equiv \sum_{i = 1}^n C_i f_i(t) \quad i = \overline{1, n}.
	\end{equation*}
\end{proof}

\begin{property}
	Якщо комплексний вектор з дійсними елементами 
	\begin{equation*}
		x(t) = u(t) + i v(t) = \begin{pmatrix} u_1(t) & \cdots & u_n(t) \end{pmatrix}^T + i \begin{pmatrix} v_1(t) & \cdots & v_n(t) \end{pmatrix}^T
	\end{equation*}
	є розв'язком неоднорідної системи $\dot x = A(t) x + f(t)$, де 
	\begin{equation*}
		f(t) = p(t) + i q(t) = \begin{pmatrix} p_1(t) & \cdots & p_n(t) \end{pmatrix}^T + i \begin{pmatrix} q_1(t) & \cdots & q_n(t) \end{pmatrix}^T,
	\end{equation*}
	то окремо дійсна і уявна частини є розв'язками систем $\dot x = A(t) x + p(t)$ і $\dot x = A(t) x + q(t)$ відповідно.
\end{property}

\begin{proof}
	Дійсно, за умовою
	\begin{equation*}
		\frac{\diff}{\diff t} (u(t) + i v(t)) - A(t) (u(t) + i v(t) \equiv p(t) + i q(t).
	\end{equation*}

	Розкривши дужки і перетворивши, одержимо
	\begin{equation*}
		(\dot u(t) - A(t) u(t)) + i (\dot v(t) - A(t) v(t)) \equiv p(t) + i q(t).
	\end{equation*}
	
	Але комплексні вирази рівні між собою тоді і тільки тоді, коли рівні дійсні та уявні частини, що і було потрібно довести.
\end{proof}

\begin{theorem}[про загальний розв'язок лінійної неоднорідної системи]
	Загальний розв'язок лінійної неоднорідної системи складається із суми загального розв'язку однорідної системи і якого-небудь частинного розв'язку неоднорідної системи.
\end{theorem}

\begin{proof}
	Нехай $x(t) = \sum_{i = 1}^n C_i x_i(t)$ --- загальний розв'язок однорідної системи і $y(t)$ --- частинний розв'язок неоднорідної. Тоді, як випливає з властивості 1, їхня сума $x(t) + y(t)$ буде розв'язком неоднорідної системи. \parvskip

	Покажемо, що цей розв'язок загальний, тобто підбором сталих $C_i$, $i = \overline{1, n}$  можна розв'язати довільну задачу Коші
	\begin{equation*}
		x_1(t_0) = x_1^0, \quad x_2(t_0) = x_2^0, \quad \ldots, \quad x_n(t_0) = x_n^0.
	\end{equation*}

	Оскільки $x(t) = \sum_{i = 1}^n C_i x_i(t)$ --- загальний розв'язок однорідного рівняння, то вектори $x_1(t), \ldots, x_n(t)$ лінійно незалежні, $W[x_1, \ldots, x_n](t) \ne 0$ і система алгебраїчних рівнянь
	\begin{equation*}
		\left\{
			\begin{array}{rl}
				C_1 x_{11}(t_0) + C_2 x_{12}(t_0) + \ldots + C_n x_{1n}(t_0) &= x_1^0 - y_1(t_0), \\
				C_1 x_{21}(t_0) + C_2 x_{22}(t_0) + \ldots + C_n x_{2n}(t_0) &= x_2^0 - y_2(t_0), \\
				\hdotsfor{2} \\
				C_1 x_{n1}(t_0) + C_2 x_{n2}(t_0) + \ldots + C_n x_{nn}(t_0) &= x_n^0 - y_n(t_0)
			\end{array}
		\right.
	\end{equation*}
 	має єдиний розв'язок $C_i^0$, $i = \overline{1, n}$. І лінійна комбінація $z(t) = y(t) + \sum_{i = 1}^n C_i^0 x_i(t)$ з отриманими сталими є розв'язком поставленої задачі Коші.
\end{proof}

		\subsubsection{Побудова частинного розв'язку неоднорідної системи методом варіації довільних сталих \todo}
		Як випливає з останньої теореми, для побудови загального розв'язку неоднорідної системи потрібно розв'язати однорідну і яким-небудь засобом знайти частинний розв'язок неоднорідної системи. Розглянемо метод, який називається методом варіації довільної сталої. \parvskip

Нехай маємо систему
\begin{equation*}
 	\dot x = A(t) x + f(t)
\end{equation*}
і $x(t) = \sum_{i = 1}^n C_i x_i(t)$ --- загальний розв'язок однорідної системи. Розв'язок неоднорідної будемо шукати в такому ж вигляді, але вважати $C_i$ не сталими, а невідомими функціями, тобто $C_i = C_i(t)$ і 
\begin{equation*}
	x_{\text{hetero}}(t) = \sum_{i = 1}^n C_i(t) x_i(t),
\end{equation*}
чи в матричній формі
\begin{equation*}
	x_{\text{hetero}}(t) = X(t) C(t),
\end{equation*}
де $X(t)$ --- фундаментальна матриця розв'язків, $C(t)$ --- вектор з невідомих функцій. Підставивши в систему, одержимо
\begin{equation*}
	\frac{\diff}{\diff t} X(t) C(t) + X(t) \frac{\diff C(t)}{\diff t} = A(t) X(t) C(t) + f(t),
\end{equation*}
чи
\begin{equation*}
	\left( \frac{\diff}{\diff t} X(t) - A(t) X(t) \right) C(t) + X(t) \frac{\diff C(t)}{\diff t} = f(t).
\end{equation*}

Оскільки $X(t)$ --- фундаментальна матриця, тобто матриця складена з розв'язків, то
\begin{equation*}
	\frac{\diff}{\diff t} X(t) - A(t) X(t) \equiv 0
\end{equation*}
і залишається система рівнянь
\begin{equation*}
	X(t) C'(t) = f(t).
\end{equation*}

Розписавши покоординатно, одержимо
\begin{equation*}
	\left\{
		\begin{array}{rl}
			C_1' x_{11}(t) + C_2' x_{12}(t) + \ldots + C_n' x_{1n}(t) &= f_1(t), \\
			C_1' x_{21}(t) + C_2' x_{22}(t) + \ldots + C_n' x_{2n}(t) &= f_2(t), \\
			\hdotsfor{2} \\
			C_1' x_{n1}(t) + C_2' x_{n2}(t) + \ldots + C_n' x_{nn}(t) &= f_n(t).
		\end{array}
	\right.
\end{equation*}

Оскільки визначником системи є визначник Вронського і він не дорівнює нулю, то система має єдиний розв'язок і функції  визначаються в такий спосіб
\begin{equation*}
	\begin{array}{rl}
		C_1(t) &= \displaystyle \int \frac{\begin{vmatrix} f_1(t) & x_{12}(t) & \cdots & x_{1n}(t) \\ f_2(t) & x_{22}(t) & \cdots & x_{2n}(t) \\ \vdots & \vdots & \ddots & \vdots \\ f_n(t) & x_{n2}(t) & \cdots & x_{nn}(t) \end{vmatrix}}{W[x_1, \ldots, x_n](t)} \diff t, \\
		C_2(t) &= \displaystyle \int \frac{\begin{vmatrix} x_{11}(t) & f_1(t) & \cdots & x_{1n}(t) \\ x_{21}(t) & f_2(t) & \cdots & x_{2n}(t) \\ \vdots & \vdots & \ddots & \vdots \\ x_{n1}(t) & f_n(t) & \cdots & x_{nn}(t) \end{vmatrix}}{W[x_1, \ldots, x_n](t)} \diff t, \\
		\hdotsfor{2} \\
		C_n(t) &= \displaystyle \int \frac{\begin{vmatrix} x_{11}(t) & x_{12}(t) & \cdots & f_1(t) \\ x_{21}(t) & x_{22}(t) & \cdots & f_2(t) \\ \vdots & \vdots & \ddots & \vdots \\ x_{n1}(t) & x_{n2}(t) & \cdots & f_n(t) \end{vmatrix}}{W[x_1, \ldots, x_n](t)} \diff t.
	\end{array}
\end{equation*}

Звідси частинний розв'язок неоднорідної системи має вигляд
\begin{equation*}
	x_{\text{hetero}}(t) = \sum_{i = 1}^n C_i(t) x_i(t).
\end{equation*}

Для лінійної неоднорідної системи на площині 
\begin{equation*}
	\left\{
		\begin{aligned}
			\dot x_1 &= a_{11} x_1 + a_{12}(t) x_2 + f_1(t), \\
			\dot x_2 &= a_{21} x_1 + a_{22}(t) x_2 + f_2(t) 
		\end{aligned}
	\right.
\end{equation*}
метод варіації довільної сталої реалізується таким чином. \parvskip

Нехай
\begin{equation*}
	X(t) = \begin{pmatrix}
		x_{11}(t) & x_{12}(t) \\
		x_{21}(t) & x_{22}(t).
	\end{pmatrix}
\end{equation*}

Фундаментальна матриця розв'язків однорідної системи. Тоді частинний розв'язок неоднорідної шукається з системи
\begin{equation*}
	\left\{
		\begin{aligned}
			C_1' x_{11}(t) + C_2' x_{12}(t) &= f_1(t), \\
			C_2' x_{21}(t) + C_2' x_{22}(t) &= f_2(t).
		\end{aligned}
	\right.
\end{equation*}

Звідси
\begin{equation*}
	C_1(t) = \int \frac{\begin{vmatrix} f_1(t) & x_{12}(t) \\ f_2(t) & x_{22}(t) \end{vmatrix}}{\begin{vmatrix} x_{11}(t) & x_{12}(t) \\ x_{21}(t) & x_{22}(t) \end{vmatrix}}, \qquad C_2(t) = \int \frac{\begin{vmatrix} x_{11}(t) & f_1(t) \\ x_{21}(t) & f_2(t) \end{vmatrix}}{\begin{vmatrix} x_{11}(t) & x_{12}(t) \\ x_{21}(t) & x_{22}(t) \end{vmatrix}}
\end{equation*}
   
І загальний розв'язок має вигляд
\begin{equation*}
	\begin{pmatrix} x_1(t) \\ x_2(t) \end{pmatrix} = \begin{pmatrix} x_{11}(t) & x_{12}(t) \\ x_{21}(t) & x_{22}(t) \end{pmatrix} \begin{pmatrix} C_1 \\ C_2 \end{pmatrix} + \begin{pmatrix} x_{11}(t) & x_{12}(t) \\ x_{21}(t) & x_{22}(t) \end{pmatrix} \begin{pmatrix} C_1(t) \\ C_2(t) \end{pmatrix},
\end{equation*}
де $C_1, C_2$ --- довільні сталі.


		\subsubsection{Формула Коші \todo}
		Нехай $X(t, t_0)$ --- фундаментальна система, нормована при $t = t_0$ тобто $X(t_0, t_0) = E$, де $E$ --- одинична матриця. Загальний розв'язок однорідної системи має вигляд
\begin{equation*}
	x(t) = X(t, t_0) C.
\end{equation*}

Вважаючи $C$ невідомою вектором-функцією і повторюючи викладення методу варіації довільної постійний, одержимо
\begin{equation*}
	X(t, t_0) C'(t) = f(t).
\end{equation*}

Звідси
\begin{equation*}
	\frac{\diff C(t)}{\diff t} = X^{-1}(t, t_0) f(t).
\end{equation*}

Проінтегруємо отриманий вираз
\begin{equation*}
	C(t) = C + \int_{t_0}^t X^{-1}(\tau, t_0) f(\tau) \diff \tau.
\end{equation*}

Тут $C$ --- вектор із сталих, що отриманий при інтегруванні системи. Підставивши у вихідний вираз, одержимо:
\begin{multline*}
	x(t) = X(t, t_0) \left( C + \int_{t_0}^t X^{-1}(\tau, t_0) f(\tau) \diff \tau \right) = \\
	= X(t, t_0) C + \int_{t_0}^t X(t, t_0) X^{-1}(\tau, t_0) f(\tau) \diff \tau 
\end{multline*}
  
Якщо $X(t, t_0)$ --- фундаментальна матриця, нормована при $t = t_0$, то $X(t, t_0) = X(t) X^{-1} (t_0)$. Звідси
\begin{equation*}
	X(t, t_0) X^{-1}(\tau, t_0) = X(t) X^{-1}(t_0) \left( X(\tau) X^{-1}(t_0) \right)^{-1} = X(t) X^{-1} (\tau) = X(t, \tau).
\end{equation*}
 
Підставивши початкові значення $x(t_0 = x_0)$ і з огляду на те, що фундаметнальна матриця нормована, тобто $X(t_0, t_0) = E$, одержимо
\begin{equation*}
	x(t) = X(t, t_0) x_0 + \int_{t_0}^t X(t, \tau) f(\tau) \diff \tau.
\end{equation*}

Отримана формула називається формулою Коші загального розв'язку неоднорідного рівняння. \parvskip

Частинний розв'язок неоднорідного рівняння, що задовольняє нульовій початковій умові, має вид
\begin{equation*}
	x_{\text{hetero}}(t) = \int_{t_0}^t X(t, \tau) f(\tau) \diff \tau.
\end{equation*}

Якщо система з сталою матрицею $A$, то
\begin{equation*}
	X(t, t_0) = X(t - t_0), \qquad X(t, \tau) = X(t - \tau).
\end{equation*}

І формула Коші має вигляд
\begin{equation*}
	x(t) = X(t - t_0) x_0 + \int_{t_0}^t X(t - \tau) f(\tau) \diff \tau.
\end{equation*}


		\subsubsection{Метод невизначених коефіцієнтів \todo}
		Якщо система лінійних диференціальних рівнянь з сталими коефіцієнтами, а векторна функція $f(t)$ спеціального виду, то частинний розв'язок можна знайти методом невизначених коефіцієнтів. Доведення існування частинного розв'язку зазначеного виду аналогічно доведенню для лінійних рівнянь вищих порядків.

\begin{enumerate}
	\item Нехай кожна з компонент вектора $f(x)$ є многочленом степеню не більш ніж $s$, тобто
	\begin{equation*}
		\begin{pmatrix} f_1(t) \\ f_2(t) \\ \vdots \\ f_n(t) \end{pmatrix} =
		\begin{pmatrix} A_0^1 t^s + A_1^1 t^{s - 1} + \ldots + A_{s - 1}^1 t + A_s^1 \\ A_0^2 t^s + A_1^2 t^{s - 1} + \ldots + A_{s - 1}^2 t + A_s^2 \\ \vdots \\ A_0^n t^s + A_1^n t^{s - 1} + \ldots + A_{s - 1}^n t + A_s^n \end{pmatrix}.
	\end{equation*}

	\begin{enumerate}
		\item Якщо характеристичне рівняння не має нульового кореня, тобто $\lambda_i \ne 0$, $i = \overline{1, n}$, то частинний розв'язок шукається в такому ж вигляді, тобто
		\begin{equation*}
			\begin{pmatrix} x_1(t) \\ x_2(t) \\ \vdots \\ x_n(t) \end{pmatrix} =
			\begin{pmatrix} B_0^1 t^s + B_1^1 t^{s - 1} + \ldots + B_{s - 1}^1 t + B_s^1 \\ B_0^2 t^s + B_1^2 t^{s - 1} + \ldots + B_{s - 1}^2 t + B_s^2 \\ \vdots \\ B_0^n t^s + B_1^n t^{s - 1} + \ldots + B_{s - 1}^n t + B_s^n \end{pmatrix}.
		\end{equation*}

		\item Якщо характеристичне рівняння має нульовий корінь кратності $r$, тобто $\lambda_1 = \lambda_2 = \ldots = \lambda_r = 0$, то частинний розв'язок шукається у вигляді многочлена степеню $s + r$, тобто
		\begin{equation*}
			\begin{pmatrix} x_1(t) \\ x_2(t) \\ \vdots \\ x_n(t) \end{pmatrix} =
			\begin{pmatrix} B_0^1 t^{s + r} + B_1^1 t^{s + r - 1} + \ldots + B_{s + r - 1}^1 t + B_{s + r}^1 \\ B_0^2 t^{s + r} + B_1^2 t^{s + r - 1} + \ldots + B_{s + r - 1}^2 t + B_{s + r}^2 \\ \vdots \\ B_0^n t^{s + r} + B_1^n t^{s + r - 1} + \ldots + B_{s + r - 1}^n t + B_{s + r}^n \end{pmatrix}.
		\end{equation*}

		Причому перші $(s + 1) n$ коефіцієнти $B_i^j$, $i = \overline{0, s}$, $j = \overline{1, n}$ знаходяться точно, а інші $r n$ --- з точністю до сталих інтегрування $C_1, \ldots, C_n$, що входять у загальний розв'язок однорідних систем.
	\end{enumerate}

	\item Нехай $f(t)$ має вид
	\begin{equation*}
		\begin{pmatrix} f_1(t) \\ f_2(t) \\ \vdots \\ f_n(t) \end{pmatrix} =
		\begin{pmatrix} e^{pt} (A_0^1 t^s + A_1^1 t^{s - 1} + \ldots + A_{s - 1}^1 t + A_s^1) \\ e^{pt} (A_0^2 t^s + A_1^2 t^{s - 1} + \ldots + A_{s - 1}^2 t + A_s^2) \\ \vdots \\ e^{pt} (A_0^n t^s + A_1^n t^{s - 1} + \ldots + A_{s - 1}^n t + A_s^n) \end{pmatrix}.
	\end{equation*}

	\begin{enumerate}
		\item Якщо характеристичне рівняння не має коренем значення $p$, тобто $\lambda_i \ne p$, $i = \overline{1, n}$, то частинний розв'язок шукається в такому ж вигляді, тобто
		\begin{equation*}
			\begin{pmatrix} x_1(t) \\ x_2(t) \\ \vdots \\ x_n(t) \end{pmatrix} =
			\begin{pmatrix} e^{pt} (B_0^1 t^s + B_1^1 t^{s - 1} + \ldots + B_{s - 1}^1 t + B_s^1) \\ e^{pt} (B_0^2 t^s + B_1^2 t^{s - 1} + \ldots + B_{s - 1}^2 t + B_s^2) \\ \vdots \\ e^{pt} (B_0^n t^s + B_1^n t^{s - 1} + \ldots + B_{s - 1}^n t + B_s^n) \end{pmatrix}.
		\end{equation*}

		\item Якщо $p$ є коренем характеристичного рівняння кратності $r$, тобто $\lambda_1 = \lambda_2 = \ldots = \lambda_r = p$, то частинний розв'язок шукається у вигляді
		\begin{equation*}
			\begin{pmatrix} x_1(t) \\ x_2(t) \\ \vdots \\ x_n(t) \end{pmatrix} =
			\begin{pmatrix} e^{pt} (B_0^1 t^{s + r} + B_1^1 t^{s + r - 1} + \ldots + B_{s + r - 1}^1 t + B_{s + r}^1) \\ e^{pt} (B_0^2 t^{s + r} + B_1^2 t^{s + r - 1} + \ldots + B_{s + r - 1}^2 t + B_{s + r}^2) \\ \vdots \\ e^{pt} (B_0^n t^{s + r} + B_1^n t^{s + r - 1} + \ldots + B_{s + r - 1}^n t + B_{s + r}^n) \end{pmatrix}.
		\end{equation*}

		І, як у попередньому пункті, перші $(s + 1) n$ коефіцієнти $B_i^j$, $i = \overline{0, s}$, $j = \overline{1, n}$, а інші з точністю до сталих інтегрування $C_1, \ldots, C_n$.
	\end{enumerate}
	
	\item Нехай $f(t)$ має вигляд:
	\begin{multline*}
		\begin{pmatrix} f_1(t) \\ f_2(t) \\ \vdots \\ f_n(t) \end{pmatrix} =
		\begin{pmatrix} e^{pt} (A_0^1 t^s + A_1^1 t^{s - 1} + \ldots + A_{s - 1}^1 t + A_s^1) \cos qt \\ e^{pt} (A_0^2 t^s + A_1^2 t^{s - 1} + \ldots + A_{s - 1}^2 t + A_s^2) \cos qt \\ \vdots \\ e^{pt} (A_0^n t^s + A_1^n t^{s - 1} + \ldots + A_{s - 1}^n t + A_s^n) \cos qt \end{pmatrix} + \\
		+ \begin{pmatrix} e^{pt} (B_0^1 t^s + B_1^1 t^{s - 1} + \ldots + B_{s - 1}^1 t + B_s^1) \sin qt \\ e^{pt} (B_0^2 t^s + B_1^2 t^{s - 1} + \ldots + B_{s - 1}^2 t + B_s^2) \sin qt \\ \vdots \\ e^{pt} (B_0^n t^s + B_1^n t^{s - 1} + \ldots + B_{s - 1}^n t + B_s^n) \sin qt \end{pmatrix}.
	\end{multline*}

 
	\begin{enumerate}
		\item Якщо характеристичне рівняння не має коренем значення $p \pm i q$, то частинний розв'язок шукається в такому ж вигляді, тобто
		\begin{multline*}
			\begin{pmatrix} x_1(t) \\ x_2(t) \\ \vdots \\ x_n(t) \end{pmatrix} =
			\begin{pmatrix} e^{pt} (C_0^1 t^s + C_1^1 t^{s - 1} + \ldots + C_{s - 1}^1 t + C_s^1) \cos qt \\ e^{pt} (C_0^2 t^s + C_1^2 t^{s - 1} + \ldots + C_{s - 1}^2 t + C_s^2) \cos qt \\ \vdots \\ e^{pt} (C_0^n t^s + C_1^n t^{s - 1} + \ldots + C_{s - 1}^n t + C_s^n) \cos qt \end{pmatrix} + \\
			+ \begin{pmatrix} e^{pt} (D_0^1 t^s + D_1^1 t^{s - 1} + \ldots + D_{s - 1}^1 t + D_s^1) \sin qt \\ e^{pt} (D_0^2 t^s + D_1^2 t^{s - 1} + \ldots + D_{s - 1}^2 t + D_s^2) \sin qt \\ \vdots \\ e^{pt} (D_0^n t^s + D_1^n t^{s - 1} + \ldots + D_{s - 1}^n t + D_s^n) \sin qt \end{pmatrix}.
		\end{multline*}
 
		\item Якщо $p \pm iq$ є коренем характеристичного рівняння кратності $r$, то частинний розв'язок має вигляд
		\begin{multline*}
			\begin{pmatrix} x_1(t) \\ x_2(t) \\ \vdots \\ x_n(t) \end{pmatrix} = \\
			= \begin{pmatrix} e^{pt} (C_0^1 t^{s + r} + C_1^1 t^{s + r - 1} + \ldots + C_{s + r - 1}^1 t + C_{s + r}^1) \cos qt \\ e^{pt} (C_0^2 t^{s + r} + C_1^2 t^{s + r - 1} + \ldots + C_{s + r - 1}^2 t + C_{s + r}^2) \cos qt \\ \vdots \\ e^{pt} (C_0^n t^{s + r} + C_1^n t^{s + r - 1} + \ldots + C_{s + r - 1}^n t + C_{s + r}^n) \cos qt \end{pmatrix} + \\
			+ \begin{pmatrix} e^{pt} (D_0^1 t^{s + r} + D_1^1 t^{s + r - 1} + \ldots + D_{s + r - 1}^1 t + D_{s + r}^1) \sin qt \\ e^{pt} (D_0^2 t^{s + r} + D_1^2 t^{s + r - 1} + \ldots + D_{s + r - 1}^2 t + D_{s + r}^2) \sin qt \\ \vdots \\ e^{pt} (D_0^n t^{s + r} + D_1^n t^{s + r - 1} + \ldots + D_{s + r - 1}^n t + D_{s + r}^n) \sin qt \end{pmatrix}.
		\end{multline*}
	\end{enumerate} 
\end{enumerate}

		\subsubsection{Вправи для самостійної роботи \todo}
		\begin{example}
	Розв'язати систему неоднорідних рівнянь методом варіації довільної сталої
	\begin{equation*}
		\left\{
			\begin{aligned}
				\dot x &= - 4 x - 2 y + \frac{2}{e^t - 1}, \\
				\dot y &= 6 x + 3 y - \frac{3}{e^t - 1}.
			\end{aligned}
		\right.
	\end{equation*}
\end{example}

\begin{solution}
	Розв'язуємо спочатку однорідну систему. Її характеристичне рівняння має вигляд
	\begin{equation*}
		\det (A - \lambda E) = \begin{vmatrix} - 4 - \lambda & -2 \\ 6 & 3 - \lambda \end{vmatrix} = \lambda^2 + \lambda = 0 \implies \lambda_1 = 0, \lambda_2 = -1.
	\end{equation*}

	Розв'язуємо (наприклад) матричним методом. Маємо
	\begin{equation*}
		\Lambda = \begin{pmatrix}
			0 & 0 \\ 0 & -1
		\end{pmatrix}, \quad 
		e^{\Lambda t} = \begin{pmatrix}
			1 & 0 \\ 0 & e^{-t}
		\end{pmatrix}
	\end{equation*}

	Матричне рівняння $A S = S \Lambda$ має вигляд
	\begin{equation*}
		\begin{pmatrix} -4 & -2 \\ 6 & 3 \end{pmatrix} \begin{pmatrix} a_1^1 & a_1^2 \\ a_2^1 & a_2^2 \end{pmatrix} = \begin{pmatrix} a_1^1 & a_1^2 \\ a_2^1 & a_2^2 \end{pmatrix} \begin{pmatrix} 0 & 0 \\ 0 & -1 \end{pmatrix}.
	\end{equation*}

	Звідси маємо дві системи рівнянь
	\begin{equation*}
		\left\{
			\begin{aligned}
				- 4 a_1^1 - 2 a_2^1 &= 0, \\
				6 a_1^1 + 3 a_2^1 &= 0,
			\end{aligned}
		\right.
		\qquad
		\left\{
			\begin{aligned}
				- 4 a_1^2 - 2 a_2^2 &= -a_1^2, \\
				6 a_1^2 + 3 a_2^2 &= -a_2^2,
			\end{aligned}
		\right.
	\end{equation*}

	Їх розв'язками будуть
	\begin{equation*}
		a_1^1 = 1, \quad a_2^1 = -2, \quad a_1^2 = -2, \quad a_2^2 = 3.
	\end{equation*}

	І розв'язок однорідної системи має вигляд
	\begin{equation*}
		\begin{pmatrix} x_1(t) \\ x_2(t) \end{pmatrix} = \begin{pmatrix} 1 & -2 \\ -2 & 3 \end{pmatrix} \begin{pmatrix} 1 & 0 \\ 0 & e^t \end{pmatrix} \begin{pmatrix} C_1 \\ C_2 \end{pmatrix} = \begin{pmatrix} 1 & -2e^{-t} \\ -2 & 3e^{-t} \end{pmatrix} \begin{pmatrix} C_1 \\ C_2 \end{pmatrix}.
	\end{equation*}

	Частинний розв'язок неоднорідної системи має вигляд
	\begin{equation*}
		\begin{pmatrix} x_1(t) \\ x_2(t) \end{pmatrix} = \begin{pmatrix} 1 & -2e^{-t} \\ -2 & 3e^{-t} \end{pmatrix} \begin{pmatrix} C_1(t) \\ C_2(t) \end{pmatrix}.
	\end{equation*}

	Функції $C_1(t), C_2(t)$ задовольняють системі рівнянь
	\begin{equation*}
		\left\{
			\begin{aligned}
				C_1'(t) - 2 C_2(r) e^{-t} = \frac{2}{e^t - 1}, \\
				-2 C_1'(t) + 3 C_2(r) e^{-t} = - \frac{3}{e^t - 1}.
			\end{aligned}
		\right.
	\end{equation*}

	Звідси
	\begin{align*}
		C_1(t) &= \int \frac{\begin{vmatrix} \rfrac{2}{e^t - 1} & - 2 e^{-t} \\ \rfrac{-3}{e^t - 1} & 3 e^{-t} \end{vmatrix}}{\begin{vmatrix} 1 & - 2 e^{-t} \\ -2 & 3 e^{-t} \end{vmatrix}} \diff t = 0 + \bar C_1, \\
		C_2(t) &= \int \frac{\begin{vmatrix} 1 & \rfrac{2}{e^t - 1} \\ - 2 & \rfrac{-3}{e^t - 1} \end{vmatrix}}{\begin{vmatrix} 1 & - 2 e^{-t} \\ -2 & 3 e^{-t} \end{vmatrix}} \diff t =\int \frac{\frac{1}{e^t - 1}}{- e^{-t}} \diff t = - \int \frac{e^t}{e^t - 1} \diff t = \\
		&= - \ln |e^t - 1| + \bar C_2.
	\end{align*}

	Поклавши $\bar C_1 = \bar C_2 = 0$, одержуємо $C_1(t) \equiv 0$, $C_2(t) = - \ln |e^t - 1|$. Таким чином, частинний розв'язок має вигляд
	\begin{equation*}
		\begin{pmatrix} x_1(t) \\ x_2(t) \end{pmatrix} = \begin{pmatrix} 1 & -2e^{-t} \\ -2 & 3e^{-t} \end{pmatrix} \begin{pmatrix} 0 \\ - \ln |e^t - 1| \end{pmatrix} = \begin{pmatrix} 2 e^{-t} \ln |e^t - 1| \\ - 3 e^{-t} \ln |e^t - 1| \end{pmatrix}
	\end{equation*}

	А загальний розв'язок 
	\begin{equation*}
		\begin{pmatrix} x_1(t) \\ x_2(t) \end{pmatrix} = \begin{pmatrix} 1 & -2e^{-t} \\ -2 & 3e^{-t} \end{pmatrix} \begin{pmatrix} C_1 \\ C_2 \end{pmatrix} + \begin{pmatrix} 2 e^{-t} \ln |e^t - 1| \\ - 3 e^{-t} \ln |e^t - 1| \end{pmatrix}
	\end{equation*}
\end{solution}

\begin{example}
	Розв'язати систему неоднорідних рівнянь за допомогою формули Коші
	\begin{equation*}
		\left\{
			\begin{aligned}
				\dot x &= - x + 2 y, \\
				\dot y &= 3 x + 4 y + \frac{e^{3 t}}{e^{2 t} + 1}.
			\end{aligned}
		\right.
	\end{equation*}
\end{example}

\begin{solution}
	Розв'язуємо спочатку однорідну систему. Характеристичне рівняння має вигляд
	\begin{equation*}
		\det (A - \lambda E) = \begin{vmatrix} - 1 - \lambda & 2 \\ 3 & 4 - \lambda \end{vmatrix} = \lambda^2 - 3 \lambda + 2 = 0 \implies \lambda_1 = 1, \lambda_2 = -1.
	\end{equation*}

	Розв'язуємо матричним методом. Маємо
	\begin{equation*}
		\Lambda = \begin{pmatrix}
			1 & 0 \\ 0 & 2
		\end{pmatrix}, \quad 
		e^{\Lambda t} = \begin{pmatrix}
			e^t & 0 \\ 0 & e^{2t}
		\end{pmatrix}
	\end{equation*}

	Матричне рівняння $A S = S \Lambda$ має вигляд
	\begin{equation*}
		\begin{pmatrix} -1 & 2 \\ 3 & 4 \end{pmatrix} \begin{pmatrix} a_1^1 & a_1^2 \\ a_2^1 & a_2^2 \end{pmatrix} = \begin{pmatrix} a_1^1 & a_1^2 \\ a_2^1 & a_2^2 \end{pmatrix} \begin{pmatrix} 1 & 0 \\ 0 & 2 \end{pmatrix}.
	\end{equation*}
 
	Одержуємо дві системи 
	\begin{equation*}
		\left\{
			\begin{aligned}
				- a_1^1 + 2 a_2^1 &= a_1^1, \\
				3 a_1^1 + 4 a_2^1 &= a_2^1,
			\end{aligned}
		\right.
		\qquad
		\left\{
			\begin{aligned}
				- a_1^2 + 2 a_2^2 &= 2 a_1^2, \\
				3 a_1^2 + 4 a_2^2 &= 2 a_2^2,
			\end{aligned}
		\right.
	\end{equation*}

	Їх розв'язками будуть
	\begin{equation*}
		a_1^1 = 1, \quad a_2^1 = 1, \quad a_1^2 = 2, \quad a_2^2 = 3.
	\end{equation*}

	І розв'язок однорідної системи має вигляд
	\begin{equation*}
		\begin{pmatrix} x_1(t) \\ x_2(t) \end{pmatrix} = \begin{pmatrix} 1 & 2 \\ 1 & 3 \end{pmatrix} \begin{pmatrix} e^t & 0 \\ 0 & e^{2 t} \end{pmatrix} \begin{pmatrix} C_1 \\ C_2 \end{pmatrix} = \begin{pmatrix} e^t & 2 e^{2 t} \\ e^t & 3 e^{2t} \end{pmatrix} \begin{pmatrix} C_1 \\ C_2 \end{pmatrix}.
	\end{equation*}

	Фундаментальна матриця лінійної однорідної системи, нормована в точці $t = 0$, має вигляд
	\begin{equation*}
		X(t) = \begin{pmatrix} e^t & 2 e^{2 t} \\ e^t & 3 e^{2t} \end{pmatrix} \begin{pmatrix} 1 & 2 \\ 1 & 3 \end{pmatrix}^{-1} = \begin{pmatrix} (3 - 2 e^t) e^t & -2 (1 - e^t) e^t \\ 3 (1 - e^t) e^t & (-2 + 3 e^t) e^t \end{pmatrix}.
	\end{equation*}

	Використовуючи формулу Коші, одержуємо частинний розв'язок, який задовольняє нульовим початковим умовам
	\begin{align*}
		\begin{pmatrix} x_1(t) \\ x_2(t) \end{pmatrix} &= \int_0^t \begin{pmatrix} (3 - 2 e^s) e^s & -2 (1 - e^s) e^s \\ 3 (1 - e^s) e^s & (-2 + 3 e^s) e^s \end{pmatrix} \begin{pmatrix} C_1 \\ C_2 \end{pmatrix} \diff s = \\
		&= \begin{pmatrix} \displaystyle \int_0^t \frac{-2 (1 - e^{t - s}) e^{t + 2 s}}{e^{2 s}} \diff s \\ \\ \displaystyle  \int_0^t \frac{(-2 +3 e^{t - s}) e^{t + 2 s}}{e^{2 s}} \diff s \end{pmatrix} = \\
		&= \begin{pmatrix} - 2 e^t \displaystyle \int_0^t \frac{e^{2 s}}{e^{2 s}} \diff s + 2 e^{2 t} \displaystyle \int_0^t \frac{e^{2 s}}{e^{2 s}} \diff s \\ \\ - 2 e^t \displaystyle \int_0^t \frac{e^{2 s}}{e^{2 s}} \diff s + 3 e^{2 t} \displaystyle \int_0^t \frac{e^{2 s}}{e^{2 s}} \diff s \end{pmatrix} = \\
		&= \left. \begin{pmatrix} -e^t \ln |e^{2 s} + 1| + 2 e^{2 t} \arctan e^s \\ -e^t \ln |e^{2 s} + 1| + 3 e^{2 t} \arctan e^s \end{pmatrix} \right|_{s = 0}^{s = t} = \\
		&= \begin{pmatrix} -e^t (\ln |e^{2 t} + 1| - \ln 2) + 2 e^{2 t} \left( \arctan e^t - \frac\pi4 \right) \\ -e^t (\ln |e^{2 t} + 1| - \ln 2) + 3 e^{2 t} \left( \arctan e^t - \frac\pi4 \right) \end{pmatrix}.
	\end{align*}

	І загальний розв'язок системи у формі Коші має вигляд
	\begin{multline*}
		\begin{pmatrix} x_1(t) \\ x_2(t) \end{pmatrix} = \begin{pmatrix} (3 - 2 e^t) e^t & -2 (1 - e^t) e^t \\ 3 (1 - e^t) e^t & (-2 + 3 e^t) e^t \end{pmatrix} \begin{pmatrix} x_1(0) \\ x_2(0) \end{pmatrix} + \\
		+ \begin{pmatrix} -e^t (\ln |e^{2 t} + 1| - \ln 2) + 2 e^{2 t} \left( \arctan e^t - \frac\pi4 \right) \\ -e^t (\ln |e^{2 t} + 1| - \ln 2) + 3 e^{2 t} \left( \arctan e^t - \frac\pi4 \right) \end{pmatrix}.
	\end{multline*}
\end{solution}

\begin{remark}
	Якщо шукати розв'язок не в формі Коші, то він має більш простіший вигляд
	\begin{equation*}
		\begin{pmatrix} x_1(t) \\ x_2(t) \end{pmatrix} = \begin{pmatrix} e^t & 2 e^t \\ e^t & 3 e^t \end{pmatrix} \begin{pmatrix} C_1 \\ C_2 \end{pmatrix} + \begin{pmatrix} -e^t \ln |e^{2 t} + 1| + 2 e^{2 t} \arctan e^t \\ -e^t \ln |e^{2 t} + 1| + 3 e^{2 t} \arctan e^t \end{pmatrix}.
	\end{equation*}
\end{remark}

\begin{example}
	Знайти загальний розв'язок системи лінійних неоднорідних рівнянь за допомогою методу невизначених коефіцієнтів:
	\begin{equation*}
		\left\{
			\begin{aligned}
				\dot x_1 &= x_2, \\
				\dot x_2 &= x_1 + t.
			\end{aligned}
		\right.
	\end{equation*}
\end{example}

\begin{solution}
	Складаємо характеристичне рівняння
	\begin{equation*}
		\det (A - \lambda E) = \begin{vmatrix} - \lambda & 1 \\ 1 & - \lambda \end{vmatrix} = \lambda^2 - 1 = 0 \implies \lambda_1 = 1, \lambda_2 = -1.
	\end{equation*}

	Оскільки рівняння не містить нульових коренів, частинний розв'язок шукаємо у вигляді
	\begin{equation*}
		\begin{pmatrix} x_1(t) \\ x_2(t) \end{pmatrix} = \begin{pmatrix} a t + b \\ c t + d \end{pmatrix}.
	\end{equation*}

	Підставивши в систему, отримаємо
	\begin{equation*}
		\left\{
			\begin{aligned}
				a &= c t + d, \\
				c &= a t + b + t.
			\end{aligned}
		\right.
	\end{equation*}

	Прирівнявши коефіцієнти при членах з однаковими степенями, отримаємо
	\begin{equation*}
		0 = c, \quad 0 = a + 1, \quad a = d, \quad c = b.
	\end{equation*}

	Звідси $a = -1$, $b = c = 0$, $d = -1$. І частинний розв'язок має вигляд
	\begin{equation*}
		\begin{pmatrix} x_1(t) \\ x_2(t) \end{pmatrix} = \begin{pmatrix} - t \\ - 1 \end{pmatrix}.
	\end{equation*}
\end{solution}

\begin{example}
	Знайти загальний розв'язок системи лінійних неоднорідних рівнянь за допомогою методу невизначених коефіцієнтів:
	\begin{equation*}
		\left\{
			\begin{aligned}
				\dot x_1 &= x_1 + 2 x_2, \\
				\dot x_2 &= 2 x_1 + 4 x_2 + t.
			\end{aligned}
		\right.
	\end{equation*}
\end{example}

\begin{solution}
	Складаємо характеристичне рівняння
	\begin{equation*}
		\det (A - \lambda E) = \begin{vmatrix} 1 - \lambda & 2 \\ 2 & 4 - \lambda \end{vmatrix} = \lambda^2 - 5 \lambda = 0 \implies \lambda_1 = 0, \lambda_2 = 5.
	\end{equation*}

	Оскільки є один нульовий корінь, то частинний розв'язок шукаємо у вигляді
	\begin{equation*}
		\begin{pmatrix} x_1(t) \\ x_2(t) \end{pmatrix} = \begin{pmatrix} a t^2 + b t + c \\ d t^2 + e t + f \end{pmatrix}.
	\end{equation*}

	Підставляємо в неоднорідну систему
	\begin{equation*}
		\left\{
			\begin{aligned}
				2 a t + b &= a t^2 + b t + c + 2 (d t^2 + e t + f), \\
				2 d t + e &= 2 (a t^2 + b t + c) + 4 (d t^2 + e t + f) + t.
			\end{aligned}
		\right.
	\end{equation*}

	Прирівнюємо коефіцієнти при членах з однаковими степенями.
	\begin{equation*}
		\left\{
			\begin{aligned}
				0 &= a + 2 d, \\
				0 &= 2 a + 4 d, 
			\end{aligned}
		\right. \qquad \left\{
			\begin{aligned}
				2 a &= b + 2 e, \\
				2 d &= 2 b + 4 e + 1, 
			\end{aligned}
		\right. \qquad \left\{
			\begin{aligned}
				b &= c + 2 f, \\
				e &= 2 c + 4 f.
			\end{aligned}
		\right.
	\end{equation*}

	Помноживши перше рівняння у другій підсистемі на мінус два і склавши з другим рівнянням, одержуємо $-4 a + 2 d = 1$. Разом з першим рівнянням першої системи маємо
	\begin{equation*}
		\left\{
			\begin{aligned}
				a + 2 d &= 0, \\
				- 4 a + 2 d &= 1.
			\end{aligned}
		\right.
	\end{equation*}

	Звідси $a = - 1 / 5, d = - 1 / 10$. І перше рівняння другої підсистеми має вигляд*
	\begin{equation*}
		b - 2 e = - 2 / 5.
	\end{equation*}

	Помноживши перше рівняння останньої підсистеми на два і віднявши друге рівняння, маємо
	\begin{equation*}
		2 b - e = 0.
	\end{equation*}

	З одержаних двох рівнянь дістаємо $b = - 2 / 25, e = - 4 / 25$. І остання підсистема дає співвідношення $c = - 2 / 25 - 2 f$. Таким чином частинний розв'язок має вигляд
	\begin{equation*}
		\begin{pmatrix} x_1(t) \\ x_2(t) \end{pmatrix} = \begin{pmatrix} - t^2 / 5 - 2 t / 25 - 2 / 25 - 2 f \\ - t^2 / 10 - 4 t / 25 + f \end{pmatrix}.
	\end{equation*}

	Стала $f$ входить в загальний розв'язок однорідної системи і точно не визначається. Поклавши $f = 0$, одержуємо
	\begin{equation*}
		\begin{pmatrix} x_1(t) \\ x_2(t) \end{pmatrix} = \begin{pmatrix} - t^2 / 5 - 2 t / 25 - 2 / 25 \\ - t^2 / 10 - 4 t / 25 \end{pmatrix}.
	\end{equation*}
\end{solution}

\begin{example}
	Знайти частинний розв'язок системи за допомогою методу невизначених коефіцієнтів:
	\begin{equation*}
		\left\{
			\begin{aligned}
				\dot x_1 &= x_2 + e^t, \\
				\dot x_2 &= - x_1 + t e^t.
			\end{aligned}
		\right.
	\end{equation*}
\end{example}

\begin{solution}
	Складаємо характеристичне рівняння однорідної системи
	\begin{equation*}
		\det (A - \lambda E) = \begin{vmatrix} - \lambda & 1 \\ -1 & - \lambda \end{vmatrix} = \lambda^2 + 1 = 0 \implies \lambda_{1, 2} = \pm i.
	\end{equation*}

	Оскільки одиниця не є коренем, то частинний розв'язок шукаємо у вигляді
	\begin{equation*}
		\begin{pmatrix} x_1(t) \\ x_2(t) \end{pmatrix} = \begin{pmatrix} (a t + b) e^t \\ (c t + d) e^t \end{pmatrix}.
	\end{equation*}

	Підставляємо в неоднорідну систему, одержуємо
	\begin{equation*}
		\left\{
			\begin{aligned}
				a e^t + (a t + b) e^t &= (c t + d) e^t + e^t, \\
				c e^t + (c t + d) e^t &= - (a t + b) e^t + t e^t.
			\end{aligned}
		\right.
	\end{equation*}

	Прирівнюємо коефіцієнти при однакових членах, одержуємо
	\begin{equation*}
		\left\{
			\begin{aligned}
				a &= c, \\
				c &= - a + 1, 
			\end{aligned}
		\right. \qquad \left\{
			\begin{aligned}
				a + b &= d + 1, \\
				e &= 2 c + 4 f.
			\end{aligned}
		\right.
	\end{equation*}

	Розв'язавши, одержуємо: $b = 0, a = c = d = 1 / 2$. Таким чином частинний розв'язок має вигляд
	\begin{equation*}
		\begin{pmatrix} x_1(t) \\ x_2(t) \end{pmatrix} = \begin{pmatrix} t e^t / 2 \\ (t + 1) e^t / 2 \end{pmatrix}.
	\end{equation*}
\end{solution}

\begin{example}
	Знайти частинний розв'язок системи за допомогою методу невизначених коефіцієнтів:
	\begin{equation*}
		\left\{
			\begin{aligned}
				\dot x_1 &= x_2 + e^t, \\
				\dot x_2 &= x_1 + t e^t.
			\end{aligned}
		\right.
	\end{equation*}
\end{example}

\begin{solution}
	Складаємо характеристичне рівняння
	\begin{equation*}
		\det (A - \lambda E) = \begin{vmatrix} - \lambda & 1 \\ 1 & - \lambda \end{vmatrix} = \lambda^2 - 1 = 0 \implies \lambda_1 = 1, \lambda_2 = -1.
	\end{equation*}

	Оскільки характеристичне рівняння має коренем одиницю кратності \allowbreak один, то частинний роз\-в'я\-зок шукаємо у вигляді
	\begin{equation*}
		\begin{pmatrix} x_1(t) \\ x_2(t) \end{pmatrix} = \begin{pmatrix} (a t^2 + b t + c) e^t \\ (d t^2 + e t + f) e^t \end{pmatrix}.
	\end{equation*}

	Підставляємо в неоднорідну систему, одержуємо
	\begin{equation*}
		\left\{
			\begin{aligned}
				(2 a t + b) e^t + (a t^2 + b t + c) e^t &= (d t^2 + e t + f) e^t + e^t, \\
				(2 d t + e) e^t + (d t^2 + e t + f) e^t &= (a t^2 + b t + c) e^t + t e^t.
			\end{aligned}
		\right.
	\end{equation*}

	Прирівнюємо коефіцієнти при однакових членах, одержуємо
	\begin{equation*}
		\left\{
			\begin{aligned}
				a &= d, \\
				d &= a,
			\end{aligned}
		\right. \qquad \left\{
			\begin{aligned}
				2 a + b &= e, \\
				2 d + e &= b + 1,
			\end{aligned}
		\right. \qquad \left\{
			\begin{aligned}
				b + c &= f + 1, \\
				e + f &= c.
			\end{aligned}
		\right.
	\end{equation*}

	З першої підсистеми одержуємо $a = d$. Підставляємо в другу
	\begin{equation*}
		\left\{
			\begin{aligned}
				2 a + b - e&= 0, \\
				2 a + e - b &= 1,
			\end{aligned}
		\right.
	\end{equation*}

	Склавши два рівняння, одержуємо: $a = 1 / 4$, $b - e = - 1 / 2$. Склавши два рівняння останньої підсистеми, маємо $b + e = 1$. Звідси  $b = 1 / 4, e = 3 / 4$, $f - c = 3 / 4$. Таким чином частинний розв'язок має вигляд
	\begin{equation*}
		\begin{pmatrix} x_1(t) \\ x_2(t) \end{pmatrix} = \begin{pmatrix} (t^2 / 4 + t / 4 + c) e^t \\ (t^2 / 4 + 3 t / 4 - 3 / 4 + c) e^t \end{pmatrix}.
	\end{equation*}

	Поклавши $c = 0$, одержуємо
	\begin{equation*}
		\begin{pmatrix} x_1(t) \\ x_2(t) \end{pmatrix} = \begin{pmatrix} (t^2 + t) e^t / 4 \\ (t^2 + 3 t - 3) e^t / 4 \end{pmatrix}.
	\end{equation*}
\end{solution}

Знайти загальний розв'язок неоднорідної системи.
\begin{multicols}{2}
	\begin{problem}
		\[ \left\{ \begin{aligned}
			\dot x &= y + \tan^2 t - 1, \\
			\dot y &= - x + \tan t.
		\end{aligned} \right. \]
	\end{problem}

	\begin{problem}
		\[ \left\{ \begin{aligned}
			\dot x &= 3 x - y, \\
			\dot y &= 2 x - y + 15 e^t \sqrt{t}.
		\end{aligned} \right. \]
	\end{problem}

	\begin{problem}
		\[ \left\{ \begin{aligned}
			\dot x &= y - 5 \cos t, \\
			\dot y &= 2 x + y.
		\end{aligned} \right. \]
	\end{problem}
	
	\begin{problem}
		\[ \left\{ \begin{aligned}
			\dot x &= 2 x - 4 y + 4 e^{-2t}, \\
			\dot y &= 2 x - 2 y.
		\end{aligned} \right. \]
	\end{problem}

	\begin{problem}
		\[ \left\{ \begin{aligned}
			\dot x &= - x + 2 y + 1, \\
			\dot y &= 2 x - 2 y.
		\end{aligned} \right. \]
	\end{problem}

	\begin{problem}
		\[ \left\{ \begin{aligned}
			\dot x &= 2 x + y + e^t, \\
			\dot y &= - 2 x + 2 t.
		\end{aligned} \right. \]
	\end{problem}
	
	\begin{problem}
		\[ \left\{ \begin{aligned}
			\dot x &= 3 x - 4 y, \\
			\dot y &= x - 3 y + 3 e^t.
		\end{aligned} \right. \]
	\end{problem}
	
	\begin{problem}
		\[ \left\{ \begin{aligned}
			\dot x &= x + 2 y + 16 t e^t, \\
			\dot y &= 2 x - 2 y.
		\end{aligned} \right. \]
	\end{problem}

	\begin{problem}
		\[ \left\{ \begin{aligned}
			\dot x &= 2 x - 3 y, \\
			\dot y &= x - 2 y + 2 \sin t.
		\end{aligned} \right. \]
	\end{problem}

	\begin{problem}
		\[ \left\{ \begin{aligned}
			\dot x &= 2 x - y, \\
			\dot y &= x + 2 e^t.
		\end{aligned} \right. \]
	\end{problem}

	\begin{problem}
		\[ \left\{ \begin{aligned}
			\dot x &= 2 x + y + 2 e^t, \\
			\dot y &= x + 2 y - 3 e^{4 t}.
		\end{aligned} \right. \]
	\end{problem}
	
	\begin{problem}
		\[ \left\{ \begin{aligned}
			\dot x &= x - y + 1 / \cos t, \\
			\dot y &= 2 x - y.
		\end{aligned} \right. \]
	\end{problem}
	
	\begin{problem}
		\[ \left\{ \begin{aligned}
			\dot x &= y + 2 e^t, \\
			\dot y &= x + t^2.
		\end{aligned} \right. \]
	\end{problem}

	\begin{problem}
		\[ \left\{ \begin{aligned}
			\dot x &= 3 x + 2 y + 4 e^{5 t}, \\
			\dot y &= x + 2 y.
		\end{aligned} \right. \]
	\end{problem}
	
	\begin{problem}
		\[ \left\{ \begin{aligned}
			\dot x &= 4 x + y - e^{2 t}, \\
			\dot y &= - 2 x + t.
		\end{aligned} \right. \]
	\end{problem}
	
	\begin{problem}
		\[ \left\{ \begin{aligned}
			\dot x &= 5 x - 3 y + 2 e^{3 t}, \\
			\dot y &= x + y + 5 e^{-t}.
		\end{aligned} \right. \]
	\end{problem}

	\begin{problem}
		\[ \left\{ \begin{aligned}
			\dot x &= x + 2 y, \\
			\dot y &= x - 5 \sin t.
		\end{aligned} \right. \]
	\end{problem}

	\begin{problem}
		\[ \left\{ \begin{aligned}
			\dot x &= 2 x - y, \\
			\dot y &= - 2 x + y + 18 t.
		\end{aligned} \right. \]
	\end{problem}
\end{multicols}

\begin{multicols}{2}
	\begin{problem}
		\[ \left\{ \begin{aligned}
			\dot x &= 2 x + 4 t - 8, \\
			\dot y &= 3 x + 4 y.
		\end{aligned} \right. \]
	\end{problem}
	
	\begin{problem}
		\[ \left\{ \begin{aligned}
			\dot x &= x - y + 2 \sin t, \\
			\dot y &= 2 x - y.
		\end{aligned} \right. \]
	\end{problem}
	
	\begin{problem}
		\[ \left\{ \begin{aligned}
			\dot x &= 4 x - 3 y + \sin t, \\
			\dot y &= 2 x - y - 2 \cos t.
		\end{aligned} \right. \]
	\end{problem}

	\begin{problem}
		\[ \left\{ \begin{aligned}
			\dot x &= 2 x - y, \\
			\dot y &= - x + 2 y - 5 e^t \sin t.
		\end{aligned} \right. \]
	\end{problem}
\end{multicols}


\newpage

\nocite{*}

\bibliographystyle{plain}

\bibliography{literature}

\end{document}