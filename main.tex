% cd ..\..\Users\NikitaSkybytskyi\Desktop\differential-equations
% pdflatex main.tex && pdflatex main.tex && del main.toc, main.aux, main.log, main.out && start main.pdf
\documentclass[a4paper, 12pt]{article}
\usepackage[T2A,T1]{fontenc}
\usepackage[utf8]{inputenc}
\usepackage[english, ukrainian]{babel}
\usepackage{amsmath, amssymb}

\allowdisplaybreaks
\setlength\parindent{0pt}

\title{{\Huge ДИФЕРЕНЦІАЛЬНІ РІВНЯННЯ}}
\author{Скибицький Нікіта}
\date{\today}

\usepackage{float, multirow, multicol, xcolor, hyperref}
\hypersetup{unicode=true, colorlinks=true, linktoc=all, linkcolor=red}

\usepackage{amsthm}
\newtheorem{theorem}{Теорема}[section]
\newtheorem{lemma}{Лема}[section]
\theoremstyle{definition}
\newtheorem*{definition}{Визначення}
\newtheorem{problem}{Задача}[subsection]
\newtheorem{example}[problem]{Приклад}
\newtheorem*{solution}{Розв'язок}
\newtheorem*{remark}{Зауваження}

\renewcommand{\phi}{\varphi}
\renewcommand{\epsilon}{\varepsilon}
\newcommand{\RR}{\mathbb{R}}
\newcommand{\NN}{\mathbb{N}}

\newcommand*\diff{\mathop{}\!\mathrm{d}}
\newcommand*\rfrac[2]{{}^{#1}\!/_{\!#2}}

\numberwithin{equation}{section}% reset equation counter for sections
\numberwithin{equation}{subsection}
% Omit `.0` in equation numbers for non-existent subsections.
\renewcommand*{\theequation}{%
  \ifnum\value{subsection}=0 %
    \thesection
  \else
    \thesubsection
  \fi
  .\arabic{equation}%
}

\begin{document}

\maketitle \thispagestyle{empty} \newpage 

У ваших руках конспект лекцій з нормативного курсу ``Диференціальні рівняння'' прочитаного проф., д.ф.-м.н. Хусаїновим Денисом Ях'євичем на другому курсі спеціальності ``прикладна математика'' факультету ком\-п'ю\-тер\-них наук та кібернетики Київського національного університету імені Тараса Шевченка восени 2017-го та навесні 2018-го року. \\

Конспект у компактній формі відображає матеріал курсу, допомагає сформувати загальне уявлення про предмет вивчення, правильно зорієнтуватися в даній галузі знань. Конспект лекцій з названої дисципліни сприятиме більш успішному вивченню дисципліни, причому більшою мірою для студентів заочної форми, екстернату, дистанційного та індивідуального навчання. \\

Комп'ютерний набір та верстка -- Скибицький Нікіта Максимович. \newpage

\tableofcontents \newpage

\section*{Вступ}

Наведемо декілька основних визначень теорії диференціальних рівнянь, що будуть використовуватися надалі.

\begin{definition}
	Рівняння, що містять похідні від шуканої функції та можуть містити шукану функцію та незалежну змінну, називаються диференціальними рівняннями.
\end{definition}

\begin{definition}
	Якщо в диференціальному рівнянні невідомі функції є функціями однієї змінної:
	\begin{equation*}
		F \left( x, y, y', y'', \ldots, y^{(n)} \right) = 0,
	\end{equation*}
	то диференціальне рівняння називається звичайним.
\end{definition}

\begin{definition}
	Якщо невідома функція, що входить в диференціальне рівняння, є функцією двох або більшої кількості незалежних змінних:
	\begin{equation*}
		F \left( x, y, z, \frac{\partial z}{\partial x}, \frac{\partial z}{\partial y}, \ldots, \frac{\partial^k z}{\partial x^\ell \partial y^{k - \ell}}, \ldots, \frac{\partial^n z}{\partial y^n} \right) = 0,
	\end{equation*}
	то диференціальне рівняння називається рівнянням у частинних похідних.
\end{definition}

\begin{definition}
	Порядком диференціального рівняння називається максимальний порядок похідної від невідомої функції, що входить в диференціальне рівняння.
\end{definition}

\begin{definition}
	Розв'язком диференціального рівняння називається функція, що має необхідний ступінь гладкості, і яка при підстановці в диференціальне рівняння обертає його в тотожність. 
\end{definition}

\begin{definition}
	Процес знаходження розв'язку диференціального рівняння називається інтегруванням диференціального рівняння.
\end{definition}


\section{Диференціальні рівняння першого порядку}

% version 1.0
Рівняння першого порядку, що розв'язане відносно похідної, має вигляд
\begin{equation*}
	\frac{\diff y}{\diff x} = f(x, y).	
\end{equation*}

Диференціальне рівняння встановлює зв'язок між координатами точки та кутовим коефіцієнтом дотичної $\frac{\diff y}{\diff x}$ до графіку розв'язку в цій же точці. Якщо знати $x$ та $y$, то можна обчислити $f(x, y)$ тобто $\frac{\diff y}{\diff x}$. \\

Таким чином, диференціальне рівняння визначає поле напрямків, і задача інтегрування рівнянь зводиться до знаходження кривих, що звуться інтегральними кривими, напрям дотичних до яких в кожній точці співпадає з напрямом поля.


\subsubsection{Рівняння, що зводяться до рівнянь зі змінними, що розділяються}

Розглянемо рівняння вигляду
\begin{equation}
	\label{eq:1.1.8}
	\frac{\diff y}{\diff x} = f(a x + b y + c)
\end{equation}
де $a$, $b$, $c$ -- сталі. \\

Зробимо заміну $a x + b y + c = z$. Тоді $a \cdot \diff x + b \cdot \diff y = \diff z$ і $\frac{\diff y}{\diff x} = \frac1b \cdot \left( \frac{\diff z}{\diff x} - a \right)$. \\

Підставивши в \eqref{eq:1.1.8}, одержимо
\begin{equation}
	\label{eq:1.1.9}
	\frac1b \cdot \left( \frac{\diff z}{\diff x} - a \right) = f(z),
\end{equation}
або
\begin{equation}
	\label{eq:1.1.10}
	\frac{\diff z}{\diff x} = a + b \cdot f (z),
\end{equation}

Розділивши змінні, запишемо
\begin{equation}
	\label{eq:1.1.11}
	\frac{\diff z}{a + b \cdot f (z)} - \diff x = 0
\end{equation}
і
\begin{equation}
	\label{eq:1.1.12}
	\int \frac{\diff z}{a + b \cdot f (z)} - x = C.
\end{equation}

Загальний інтеграл має вигляд $\Phi(a x + b y + c, x) = C$.

\subsubsection{Вправи для самостійної роботи}

Рівняння зі змінними, що розділяються могуть бути записані у вигляді $y' = f(x) \cdot g(y)$ або  $f_1(x) \cdot f_2(y) \cdot \diff x + g_1(x) \cdot g_2(y) \cdot \diff y$. Для розв’язків такого рівняння необхідно обидві частини помножити або розділити на такий вираз, щоб в одну частину входило тільки $x$, а в другу -- тільки $y$. Тоді обидві частини рівняння можна проінтегрувати.
Якщо ділити на вираз, що містить $x$ та $y$, може бути загублений розв’язок, що обертає цей вираз в нуль.

\begin{example}
	Розв’язати рівняння
	\begin{equation}
		\label{eq:1.1.13}
		x^2 y^2 y' + y = 1.
	\end{equation}
\end{example}

\begin{solution}
	Підставивши $y = \frac{\diff y}{\diff x}$ в \eqref{eq:1.1.13}, отримаємо 
	\begin{equation}
		\label{eq:1.1.14}
		x^2 y^2 \cdot \frac{\diff y}{\diff x} + y = 1.
	\end{equation}

	Помножимо обидві частини рівняння на $\diff x$ і розділимо на $x^2 \cdot (y - 1)$. Перевіримо, що $y = 1$ при цьому є розв’язком, а $x = 0$ цим розв’язком не є:
	\begin{equation}
		\label{eq:1.1.15}
		\frac{y^2}{y - 1} \cdot \diff y = - \frac{\diff x}{x^2}.
	\end{equation}

	Проінтегрируємо обидві частини рівняння:
	\begin{align}
		\label{eq:1.1.16}
		\int \frac{y^2}{y - 1} \cdot \diff y &= - \int \frac{\diff x}{x^2}. \\
		\frac{y^2}{2} + y + \ln |y - 1| &= \frac{1}{x} + C
	\end{align}
\end{solution}

\begin{example}
	Розв’язати рівняння
	\begin{equation}
		\label{eq:1.1.17}
		y' = \sqrt{4x + 2y - 1}.
	\end{equation}
\end{example}

\begin{solution}
	Введемо заміну змінних $z = 4 x + 2 y - 1$. Тоді $x' = 4 + 2 y'$. Рівняння \eqref{eq:1.1.17} перетвориться до вигляду $z' - 4 = 2 \sqrt{z}$; $z' = 4 + 2 \sqrt{z}$; $\frac{\diff z}{2 + \sqrt{z}} = 2 \diff x$. Проінтегруємо обидві частини рівняння:
	\begin{equation}
		\label{eq:1.1.18}
		\int \frac{\diff z}{2 + \sqrt{z}} = \int 2 \diff x
	\end{equation}
	Обчислимо інтеграл, що стоїть зліва. При обчисленні будемо використовувати таку заміну: 
	\begin{equation*}
		\sqrt{z} = t, \quad \diff z = 2 t \diff t, \quad 2 + \sqrt{z} = 2 + t,
	\end{equation*}
	\begin{multline}
		\label{eq:1.1.19}
		\int \frac{\diff z}{2 + \sqrt{z}} = \int \frac{2 t \diff t}{2 + t} = 2 \int \frac{t + 2 - 2}{t + 2} \cdot \diff t = \\
		= 2 t - 4 \ln |2 + t| = 2 \sqrt{z} - 4 \ln \left(2 + \sqrt{z}\right).
	\end{multline}
	Після інтегрування отримаємо $2 \sqrt{z} - 4 \ln \left(2 + \sqrt{z}\right) = 2 x + 2 C$. Зробимо обернену заміну: $z = 4x + 2y - 1$;
	\begin{equation}
		\label{eq:1.1.20}
		\sqrt{4x + 2y - 1} - 2 \ln \left(2 + \sqrt{4x + 2y - 1}\right) = x + C.
	\end{equation}
\end{solution}

Розв’язати рівняння:
\begin{multicols}{2}
\begin{problem}
	\[ x y \cdot \diff x + (x + 1) \cdot \diff y = 0; \]
\end{problem}
\begin{problem}
	\[ x \cdot (1 + y) \cdot \diff x = y \cdot (1 + x^2) \cdot \diff y; \]
\end{problem}
\begin{problem}
	\[ y' = 10^{x + y}; \]
\end{problem}
\begin{problem}
	\[ y' - xy^2 = 2xy; \]
\end{problem}
\begin{problem}
	\[ \sqrt{y^2 + 1} \diff x = x y \cdot \diff y; \]
\end{problem}
\begin{problem}
	\[ y' = x \tan (y); \]
\end{problem}
\begin{problem}
	\[ y y' + x = 1; \]
\end{problem}
\begin{problem}
	\[ 3 y^2 y' + 15 x = 2 x y^3; \]
\end{problem}
\begin{problem}
	\[ y' = \cos (y - x); \]
\end{problem}
\begin{problem}
	\[ y' - y = 2x - 3; \]
\end{problem}
\begin{problem}
	\[ x y' + y = y^2; \]
\end{problem}
\begin{problem}
	\[ e^{-y} \cdot (1 + y') = 1; \]
\end{problem}
\begin{problem}
	\[ 2 x^2 y y' + y^2 = 2; \]
\end{problem}
\begin{problem}
	\[ y' - x y^3 = 2 x y^2. \]
\end{problem}
\end{multicols}

Знайти частинні розв’язки, що задовольняють заданим початковим умовам:
\begin{problem}
	\[ (x^2 - 1) \cdot y' + 2 x y^2 = 0, \quad y(0) = 1; \]
\end{problem}
\begin{problem}
	\[ y' \cdot \cot (x) + y = 2, \quad y(0) = - 1; \]
\end{problem}
\begin{problem}
	\[ y' = 3 \sqrt[3]{y^2}, \quad y(2) = 0. \]
\end{problem}

\subsection{Однорідні рівняння}

\subsubsection{Загальна теорія}
Нехай рівняння має вигляд
\begin{equation}
	\label{eq:1.2.1}
	M(x, y) \cdot \diff	x + N(x, y) \cdot \diff y = 0.
\end{equation}

Якщо функції $M(x, y)$ та $N(x, y)$ однорідні одного ступеня, то рівняння називається однорідним. Нехай функції $M(x, y)$ та $N(x, y)$ однорідні ступеня $k$, тобто
\begin{equation}
	\label{eq:1.2.2}
	M(t \cdot x, t \cdot y) = t^k \cdot M(x, y), \qquad N(t \cdot x, t \cdot y) = t^k \cdot N(x, y).
\end{equation}

Робимо заміну 
\begin{equation}
	\label{eq:1.2.2_5}
	y = u x, \quad \diff y = u \diff x + x \diff u.
\end{equation}
Після підстановки одержуємо
\begin{equation}
	\label{eq:1.2.3}
	M(x, u x) \cdot \diff x + N(x, u x) \cdot (u \diff x + x \diff u) = 0,
\end{equation}
або 
\begin{equation}
	\label{eq:1.2.4}
	x^k M(1, u) \cdot \diff x + x^k N(1, u) \cdot (u \diff x + x \diff u) = 0.
\end{equation}

Скоротивши на $x^k$ і розкривши дужки, запишемо 
\begin{equation}
	\label{eq:1.2.5}
	M(1, u) \cdot \diff x + N(1, u) \cdot u \diff x + N(1, u) \cdot x \diff u = 0.
\end{equation}
Згрупувавши, одержимо рівняння зі змінними, що розділяються
\begin{equation}
	\label{eq:1.2.6}
	(M(1, u) + N(1, u) \cdot u) \diff x + N(1, u) \cdot x \diff u = 0,
\end{equation}
або 
\begin{equation}
	\label{eq:1.2.7}
	\int \frac{\diff x}{x} + \int \frac{N(1, u) \cdot \diff u}{M(1, u) + N(1, u) \cdot u} = C.
\end{equation}

Взявши інтеграли та замінивши $u = y / x$, отримаємо загальний інтеграл $\Phi(x, y / x) = C$.

\subsubsection{Рівняння, що зводяться до однорідних}

Нехай маємо рівняння вигляду
\begin{equation}
	\label{eq:1.2.8}
	\frac{\diff y}{\diff x} = f \left( \frac{a_1 x + b_1 y + c_1}{a_2 x + b_2 y + c_2} \right).
\end{equation}

Розглянемо два випадки
\begin{enumerate}
	\item 
	\begin{equation}
		\label{eq:1.2.9}
		\Delta = \begin{vmatrix} a_1 & b_1 \\ a_2 & b_2 \end{vmatrix} \ne 0.
	\end{equation}

	Тоді система алгебраїчних рівнянь
	\begin{equation}
		\label{eq:1.2.10}
		\left\{
			\begin{aligned}
				a_1 x + b_1 y + c_1 &= 0, \\
				a_2 x + b_2 y + c_2 &= 0,
			\end{aligned}
		\right.
	\end{equation}
	має єдиний розв’язок $(x_0, y_0)$. Проведемо заміну 
	\begin{equation}
		\label{eq:1.2.10_5}
		\left\{\begin{aligned}
			x &= x_1 + x_0, \\
			y &= y_1 + y_0
		\end{aligned}\right.
	\end{equation}
	та отримаємо
	\begin{multline}
		\label{eq:1.2.11}
		\frac{\diff y_1}{\diff x_1} = f \left( \frac{a_1 \cdot (x_1 + x_0) + b_1 \cdot (y_1 + y_0) + c_1}{a_2 \cdot (x_1 + x_0) + b_2 \cdot (y_1 + y_0) + c_2} \right) = \\
		= f \left( \frac{a_1 x_1 + b_1 y_1 + (a_1 x_0 + b_1 y_0 + c_1)}{a_2 x_1 + b_2 y_1 + (a_2 x_0 + b_2 y_0 + c_2)} \right)
	\end{multline}

	Оскільки $(x_0, y_0)$ -- розв’язок \eqref{eq:1.2.10}, то \eqref{eq:1.2.8} набуде вигляду
	\begin{equation}
		\label{eq:1.2.12}
		\frac{\diff y_1}{\diff x_1} = f \left( \frac{a_1 x_1 + b_1 y_1}{a_2 x_1 + b_2 y_1} \right)
	\end{equation}
	і є однорідним нульового ступеня. Робимо заміну 
	\begin{equation}
		\label{eq:1.2.12_5}
		y_1 = u x_1, \quad \diff y_1 = u \cdot \diff x_1 + x_1 \cdot \diff u.
	\end{equation}

	Підставимо в \eqref{eq:1.2.12}
	\begin{equation}
		\label{eq:1.2.13}
		u + x_1 \cdot \frac{\diff u}{\diff x_1} = f \left( \frac{a_1 x_1 + b_1 u x_1}{a_2 x_1 + b_2 u x_1} \right).
	\end{equation}
	
	Одержимо
	\begin{equation}
		\label{eq:1.2.14}
		x_1 \cdot \diff u + \left( u - f \left( \frac{a_1 x_1 + b_1 u x_1}{a_2 x_1 + b_2 u x_1} \right) \right) \diff x_1 = 0.
	\end{equation}

	Розділивши змінні, маємо
	\begin{equation}
		\label{eq:1.2.15}
		\int \frac{\diff u}{u - f \left( \frac{a_1 x_1 + b_1 u x_1}{a_2 x_1 + b_2 u x_1} \right)} + \ln (x_1) = C.
	\end{equation}

	І загальний інтеграл диференціального рівняння має вигляд $\Phi(u, x_1) = C$. Повернувшись до вихідних змінних, запишемо
	\begin{equation}
		\label{eq:1.2.16}
		\Phi \left( \frac{y - y_0}{x - x_0}, x - x_0 \right) = C.
	\end{equation}

	\item Нехай 
	\begin{equation}
		\label{eq:1.2.17}
		\Delta = \begin{vmatrix} a_1 & b_1 \\ a_2 & b_2 \end{vmatrix} = 0,
	\end{equation}
	тобто коефіцієнти рядків лінійно залежні і
	\begin{equation}
		\label{eq:1.2.18}
		a_1 x + b_1 y = \alpha \cdot (a_2 x + b_2 y).
	\end{equation}

	Робимо заміну $a_2 x + b_2 y = z$. Звідси $\frac{\diff y}{\diff x} = \frac{1}{b_2} \cdot \left( \frac{\diff z}{\diff x} - a_2 \right)$. \\

	Підставивши в диференціальне рівняння, одержимо
	\begin{equation}
		\label{eq:1.2.19}
		\frac{1}{b_2} \cdot \left( \frac{\diff z}{\diff x} - a_2 \right) = f \left ( \frac{\alpha z + c_1}{z + c_2} \right),
	\end{equation}
	або
	\begin{equation}
		\label{eq:1.2.20}
		\frac{\diff z}{\diff x} = a_2 + b_2 \cdot f \left ( \frac{\alpha z + c_1}{z + c_2} \right),
	\end{equation}
	Розділивши змінні, отримаємо
	\begin{equation}
		\label{eq:1.2.21}
		\int \frac{\diff z}{a_2 + b_2 \cdot f \left ( \frac{\alpha z + c_1}{z + c_2} \right)} - x = C,
	\end{equation}

	Загальний інтеграл має вигляд $\Phi(a_2 x + b_2 y, x) = C$
\end{enumerate}

\subsubsection{Вправи для самостійної роботи}

Однорідні рівняння можуть бути записані у вигляді \[y' = f \left( \frac{y}{x} \right)\] або  \[ M(x, y)\cdot\diff y + N(x, y) \cdot \diff y = 0, \] де $M(x, y)$ і $N(x, y)$ -- однорідні функції одного й того ж ступеня. Для того, щоб розв’язати однорідне рівняння, необхідно провести заміну \[y = u x, \quad \diff y = u \cdot \diff x + x \cdot \diff u,\] в результаті якої отримаємо рівняння зі змінними, що розділяються. 

\begin{example}
	Розв’язати рівняння $x \cdot \diff y = (x + y) \cdot \diff y$. 
\end{example}

\begin{solution}
	Дане рівняння однорідне, оскільки $x$ та $x + y$ є однорідними функціями першого ступеня. \\

	Проведемо заміну: $y = u x$. Тоді $\diff y = u \diff x + x \diff y$. Підставивши $y$ та $\diff y$ в задане рівняння, отримаємо  
	\begin{align*}
		x \cdot (x \diff u + u \diff x) &= (x + x u) \diff x, \\
		x^2 \diff u &= x \diff x
	\end{align*}

	Розв’яжемо це рівняння зі змінними, що розділяються:
	\begin{align*}
		\diff u &= \frac{\diff x}{x}, \\
		u &= \ln |x| + C.
	\end{align*}
	Повернувшись до вихідних змінних $u = y / x$, отримаємо \[y = x \cdot (\ln |x| + C).\] Крім того розв’язком є $x = 0$, що було загублене при поділенні рівняння на $x$.
\end{solution}

Розв’язати рівняння:
\begin{multicols}{2}
\begin{problem}
	\[ (x + 2y) \cdot \diff x - x \diff y = 0; \]
\end{problem}
\begin{problem}
	\[ (x - y) \cdot \diff x + (x + y) \cdot \diff y = 0; \]
\end{problem}
\begin{problem}
	\[y^2 + x^2 y' = x y y'; \]
\end{problem}
\begin{problem}
	\[ (x^2 + y^2) \cdot y' = 2 xy; \]
\end{problem}
\begin{problem}
	\[ xy' - y = x \cdot \tan \left( \frac{y}{x} \right); \]
\end{problem}
\begin{problem}
	\[ x y' = y - x e^{y / x}; \]
\end{problem}
\begin{problem}
	\[x y' - y = (x + y) \cdot \ln \left( \frac{x + y}{x} \right); \]
\end{problem}
\begin{problem}
	\[ (3x + y) \cdot \diff x - (2x + 3y) \cdot \diff y = 0; \]
\end{problem}
\begin{problem}
	\[ x y' = y \cos \left(\ln \left(\frac{y}{x} \right)\right); \]
\end{problem}
\begin{problem}
	\[ \left(y+\sqrt{xy}\right)\cdot \diff x = x \diff y; \]
\end{problem}
\begin{problem}
	\[ xy' = \sqrt{x^2 - y^2} + y; \]
\end{problem}
\begin{problem}
	\[ x^2 y' = y \cdot (x + y); \]
\end{problem}
\begin{problem}
	\[ y \cdot (-y + xy') = \sqrt{x^4 + y^4}; \]
\end{problem}
\begin{problem}
	\[ x \diff y - y \diff x = \sqrt{x^2 + y^2} \diff x; \]
\end{problem}
\begin{problem}
	\[ (y^2 - 2xy) \cdot \diff x + x^2 \cdot \diff y = 0; \]
\end{problem}
\begin{problem}
	\[ 2x^3 y' = y \cdot (2x^2 - y^2); \]
\end{problem}
\begin{problem}
	\[ \left( x - y \cos \left(\rfrac{y}{x}\right)\right) \diff x + x \cos \left(\rfrac{y}{x}\right) \diff y = 0; \]
\end{problem}
\begin{problem}
	\[ y' (xy - x^2) = y^2; \]
\end{problem}
\begin{problem}
	\[ 2xyy' = x^2 + y^2; \]
\end{problem}
\begin{problem}
	\[ (6x + 3y) \cdot \diff x = (7x - 2y) \cdot \diff y; \]
\end{problem}
\begin{problem}
	\[ y^2 x \diff x = y \cdot (xy - 2y^2) \cdot \diff y; \]
\end{problem}
\begin{problem}
	\[ x^2 y \diff x = y \cdot (xy - 2y^2) \cdot \diff y; \]
\end{problem}
\begin{problem}
	\[ 2y^3 = xy' \cdot (2y^2 - x^2); \]
\end{problem}
\begin{problem}
	\[ \left( x + \sqrt{xy}\right) \cdot \diff y = y \diff x; \]
\end{problem}
\begin{problem}
	\[ y = \left( \sqrt{y^2 - x^2} + x\right) y'; \]
\end{problem}
\begin{problem}
	\[ (3x - 2y) \cdot \diff x - (2x + y) \cdot \diff y = 0; \]
\end{problem}
\begin{problem}
	\[ (7x + 6y) \cdot \diff x - (x + 3y) \cdot \diff y = 0; \]
\end{problem}
\begin{problem}
	\[ xy' = y + x \cdot \cot \left(\frac{y}{x}\right). \]
\end{problem}
\end{multicols}

Знайти частинні розв’язки, що задовольняють задані початкові умови:
\begin{problem}
	\[ x y' = 4 \sqrt{2x^2 + y^2} + y, \quad y(1) = 2; \]
\end{problem}
\begin{problem}
	\[ (2y^2 + 3x^2) \cdot xy' = 3y^3 + 6yx^2, \quad y(2) = 1; \]
\end{problem}
\begin{problem}
	\[ y' (x^2 - 2xy) = x^2 + xy - y^2, \quad y(3) = 0; \]
\end{problem}
\begin{problem}
	\[ 2 y' = \frac{y^2}{x^2} + 8 \cdot \frac{y}{x} + 8, \quad y(1) = 1; \]
\end{problem}
\begin{problem}
	\[ y' (x^2 - 4xy) = x^2 + xy - 3 y^2, \quad y(1) = 1; \]
\end{problem}
\begin{problem}
	\[ xy' = 3 \sqrt{2x^2 + y^2} + y, \quad y(1) = 1; \]
\end{problem}
\begin{problem}
	\[ (2y^2 + 7x^2)\cdot x y' = 3y^3 + 14yx^2, \quad y(1) = 1; \]
\end{problem}
\begin{problem}
	\[ 2y' = \frac{y^2}{x^2} + 6 \cdot \frac{y}{x} + 3, \quad y(3) = 1; \]
\end{problem}
\begin{problem}
	\[ x^2 y' = y^2 + 4xy + 2x^2, \quad y(1) = 1; \]
\end{problem}
\begin{problem}
	\[ xy' = \sqrt{2x^2 + y^2} + y, \quad y(1) = 1; \]
\end{problem}
\begin{problem}
	\[ xy' = 3 \sqrt{x^2 + y^2} + y, \quad y(3) = 4; \]
\end{problem}
\begin{problem}
	\[ xy' = 2 \sqrt{x^2 + y^2} + y, \quad y(4) = 3; \]
\end{problem}
\begin{problem}
	\[ 2y' = \frac{y^2}{x^2} + 8 \cdot \frac{y}{x} + 8, \quad y(1) = 1; \]
\end{problem}
\begin{problem}
	\[ y' = \frac{x + 2y}{2x - y}, \quad y(3) = 8. \]
\end{problem}

\subsection{Лінійні рівняння першого порядку}

\subsubsection{Загальна теорія}

Рівняння, що є лінійним відносно невідомої функції та її похідної, називається лінійним диференціальним рівнянням. Його загальний вигляд такий:
\begin{equation*}
	%\label{eq:1.3.1}
	\frac{\diff y}{\diff x} + p(x) \cdot y = q(x).
\end{equation*}

Якщо $q(x) \equiv 0$, тобто рівняння має вигляд
\begin{equation*}
	%\label{eq:1.3.2}
	\frac{\diff y}{\diff x} + p(x) \cdot y = 0,
\end{equation*}
то воно зветься однорідним. Однорідне рівняння є рівнянням зі змінними, що розділяються і розв’язується таким чином:
\begin{align}
	%\label{eq:1.3.2_25}
	\frac{\diff y}{y} &= -p(x) \cdot \diff x, \\
	\int \frac{\diff y}{y} &= - \int p(x) \cdot \diff x, \\
	\ln y &= - \int p(x) \cdot \diff x + \ln C.
\end{align}
Нарешті 
\begin{equation*}
	%\label{eq:1.3.2_5}
	y = C \cdot \exp \left\{ - \int p(x) \cdot \diff x \right\}
\end{equation*}

Розв’язок неоднорідного рівняння будемо шукати методом варіації довільних сталих (методом невизначених множників Лагранжа). Він складається в тому, що розв’язок неоднорідного рівняння шукається в такому ж вигляді, як і розв’язок однорідного, але $C$ вважається невідомою функцією від $x$, тобто $C = C(x)$ і 
\begin{equation*}
	%\label{eq:1.3.2_75}
	y = C(x) \cdot \exp \left\{ - \int p(x) \cdot \diff x \right\}	
\end{equation*}

Для знаходження $C(x)$ підставимо $y$ у рівняння
\begin{multline} 
	%\label{eq:1.3.3}
	\frac{\diff C(x)}{\diff x} \cdot \exp \left\{ - \int p(x) \cdot \diff x \right\} = - C(x) \cdot p(x) \cdot \exp \left\{ - \int p(x) \cdot \diff x \right\} + \\
	+ p(x) \cdot C(x) \cdot \exp \left\{ - \int p(x) \cdot \diff x \right\} = q(x).
\end{multline}

Звідси
\begin{equation*} 
	%\label{eq:1.3.4}
	\diff C(x) = q(x) \cdot \exp \left\{\int p(x) \cdot \diff x \right\} \cdot \diff x.
\end{equation*}

Проінтегрувавши, одержимо
\begin{equation*} 
	%\label{eq:1.3.5}
	C(x) = \int q(x) \cdot \exp \left\{\int p(x) \cdot \diff x \right\} \cdot \diff x + C.
\end{equation*}

І загальний розв’язок неоднорідного рівняння має вигляд
\begin{multline} 
	%\label{eq:1.3.6}
	y = \exp \left\{ - \int p(x) \cdot \diff x \right\} \cdot \\
	\cdot \left( \int q(x) \cdot \exp \left\{\int p(x) \cdot \diff x \right\} \cdot \diff x + C\right).
\end{multline}

Якщо використовувати початкові умови $y(x_0) = y_0$, то розв’язок можна записати у формі Коші:
\begin{multline} 
	%\label{eq:1.3.7}
	y(x, x_0, y_0) = \exp \left\{ - \int_{x_0}^x p(t) \cdot \diff t \right\} \cdot \\
	\cdot \left( \int_{x_0}^x q(t) \cdot \exp \left\{\int_t^x p(\xi) \cdot \diff \xi \right\} \cdot \diff t + y_0\right).
\end{multline}

\subsubsection{Рівняння Бернуллі}

Рівняння вигляду
\begin{equation*}
	%\label{eq:1.3.8}
	\frac{\diff y}{\diff x} + p(x) \cdot y = q(x) \cdot y^m, \quad m \ne 1
\end{equation*}
називається рівнянням Бернуллі. Розділимо на $y^m$ і одержимо 
\begin{equation*}
	%\label{eq:1.3.9}
	y^{-m} \cdot \frac{\diff y}{\diff x} + p(x) \cdot y^{1-m} = q(x).
\end{equation*}

Зробимо заміну: 
\begin{equation*}
	%\label{eq:1.3.9_5}
	y^{1-m} = z, \quad (1 - m) \cdot y^{-m} \cdot \frac{\diff y}{\diff x} = \diff z.
\end{equation*}

Підставивши в рівняння, отримаємо
\begin{equation*}
	%\label{eq:1.3.10}
	\frac{1}{1-m} \cdot \frac{\diff z}{\diff x} + p(x) \cdot z = q(x).
\end{equation*}

Одержали лінійне диференціальне рівняння. Його розв’язок має вигляд
\begin{multline}
	%\label{eq:1.3.11}
	z = \exp\left\{ -(1 - m) \cdot \int p(x) \diff x \right\} \cdot \\
	\cdot \left( (1-m) \cdot \int q(x) \cdot \exp\left\{ (1 - m) \cdot \int p(x) \diff x \right\} + C\right).
\end{multline}
 
\subsubsection{Рівняння Рікатті}

Рівняння вигляду 
\begin{equation*}
	%\label{eq:1.3.12}
	\frac{\diff y}{\diff x} + p(x) \cdot y + r(x) \cdot y^2 = q(x)
\end{equation*} 
називається рівнянням Рікатті. В загальному випадку рівняння Рікатті не інтегрується. Відомі лише деякі частинні випадки рівнянь Рікатті, що інтегруються в квадратурах. Розглянемо один з них. Нехай відомий один частинний розв’язок $y = y_1(x)$. Робимо заміну $y = y_1(x) + z$ і одержуємо
\begin{equation*}
	%\label{eq:1.3.13}
	\frac{\diff y_1(x)}{\diff x} + \frac{\diff z}{\diff x} + p(x) \cdot (y_1(x) + z) + r(x) \cdot (y_1(x) + z)^2 = q(x).
\end{equation*}

Оскільки $y_1(x)$ -- частинний розв’язок, то
\begin{equation*}
	%\label{eq:1.3.14}
	\frac{\diff y_1(x)}{\diff x} + p(x) \cdot y_1 + r(x) \cdot y_1^2 = q(x).
\end{equation*}

Розкривши в попередній рівності дужки і використовуючи останнє зауваження, одержуємо
\begin{equation*}
	%\label{eq:1.3.15}
	\frac{\diff z}{\diff x} + p(x) \cdot z + 2 r(x) \cdot y_1(x) \cdot z + r(x) \cdot z^2 = 0.
\end{equation*}

Перепишемо одержане рівняння у вигляді
\begin{equation*}
	%\label{eq:1.3.16}
	\frac{\diff z}{\diff x} + \left(p(x) + 2 r(x) \cdot y_1(x)\right) \cdot z = - r(x) \cdot z^2 ,
\end{equation*}
це рівняння Бернуллі з $m = 2$.

\subsubsection{Вправи для самостійної роботи}

\begin{example}
	Розв’язати рівняння \[ y' - y \cdot \tan x = \cos x.\]
\end{example}
\begin{solution}
	Використовуючи вигляд загального розв’язку, отримаємо
	\[ y = \exp\left\{\int \tan x \diff x\right\} \cdot \left(\int \exp\left\{-\int\tan x\diff x\right\}\cdot \cos x \diff x+C\right). \]

	Оскільки \[\int \tan x \diff x = - \ln |\cos x|,\] то отримаємо
	\begin{align*}
		y &= e^{-\ln|\cos x|} \cdot \left(\int e^{\ln|\cos x|} \cdot \cos x \diff x+C\right) = \\
		&= \frac{1}{\cos x} \cdot \left( \int \cos^2 x \diff x + C \right) = \\
		&= \frac{1}{\cos x} \cdot \left( \frac x2 + \frac{\sin 2x}{4} + C \right).
	\end{align*}
	Або
	\[ y = \frac{C}{\cos x} + \frac{x}{2 \cos x} + \frac{\sin x}{2}. \]
\end{solution}

\begin{example}
	Знайти частинний розв’язок рівняння \[ y' - \frac yx = x^2,\] що задовольняє початковій умові $y(2) = 2$.
\end{example}
\begin{solution}
	Використовуючи вигляд загального розв’язку, отримаємо
	\begin{align*}
		y &= \exp\left\{\int \frac1x \diff x\right\} \cdot \left(\int \exp\left\{-\int\frac1x\diff x\right\}\cdot x^2 \diff x+C\right) = \\
		&= e^{\ln|x|} \cdot \left(\int e^{-\ln|x|} \cdot x^2 \diff x+C\right) = \\
		&= x \cdot \left(\int x \diff x+C\right) = \\
		&= x \cdot \left(\frac{x^2}{2} + C\right).
	\end{align*}
	Таким чином 
	\[ y = C x + \frac{x^3}{2}.\]
	Підставивши початкові умови $y(2) = 2$, одержимо $2 = 2C + 4$. Звідси $C = -1$ і частинний розв’язок має вигляд \[ y_{\text{част.}} = \frac{x^3}{2} - x.\]
\end{solution}

Розв’язати рівняння:
\begin{multicols}{2}
\begin{problem}
	\[x y' + (x + 1) \cdot y = 3 x^2 e^{-x};\]
\end{problem}
\begin{problem}
	\[(2x + 1) \cdot y' =4x+2y;\]
\end{problem}
\begin{problem}
	\[y'=2x\cdot(x^2+y);\]
\end{problem}
\begin{problem}
	\[x^2y'+xy+1=0;\]
\end{problem}
\begin{problem}
	\[y'+y\cdot\tan x=\sec x;\]
\end{problem}
\begin{problem}
	\[x\cdot(y'-y)=e^x;\]
\end{problem}
\begin{problem}
	\[(xy'-1)\cdot\ln x=2y;\]
\end{problem}
\begin{problem}
	\[(y+x^2)\cdot \diff x=x \cdot\diff y;\]
\end{problem}
\begin{problem}
	\[(2e^x-y)\cdot\diff x=\diff y;\]
\end{problem}
\begin{problem}
	\[\sin^2 y + x \cdot \cot y = \frac1{y^2};\]
\end{problem}
\begin{problem}
	\[(x+y^2)\cdot y'=y;\]
\end{problem}
\begin{problem}
	\[(3e^y-x)\cdot y' = 1;\]
\end{problem}
\begin{problem}
	\[y = x\cdot(y'- x \cdot \cos x).\]
\end{problem}
\end{multicols}

Знайти частинні розв’язки рівняння з заданими початковими умовами:
\begin{problem}
	\[y'-\frac yx=-\frac{\ln x}x, \quad y(1)=1;\]
\end{problem}
\begin{problem}
	\[y'-\frac{2xy}{1+x^2}=1+x^2, \quad y(1)=3;\]
\end{problem}
\begin{problem}
	\[y'-\frac{2y}{x+1}=e^{x}\cdot(x+1)^2, \quad y(0)=1;\]
\end{problem}
\begin{problem}
	\[xy'+2y=x64,\quad y(1)=-\frac58;\]	
\end{problem}
\begin{problem}
	\[ y' - \frac yx  = x \cdot \sin x, \quad y\left(\frac\pi2\right)=1;\]
\end{problem}
\begin{problem}
	\[y'+\frac yx=\sin x, \quad y(\pi)=\frac1\pi;\]
\end{problem}
\begin{problem}
	\[(13y^3-x)\cdot y'=4y, \quad y(5)=1;\]
\end{problem}
\begin{problem}
	\[2\cdot(x+\ln^2y-\ln y)\cdot y'= y, \quad y(2)=1.\]
\end{problem}

Розв’язати рівняння Бернуллі:
\begin{multicols}{2}
\begin{problem}
	\[y'+xy=(1+x)\cdot e^{-x}\cdot y^2;\]
\end{problem}
\begin{problem}
	\[xy'+y=2y^2\cdot \ln x;\]
\end{problem}
\begin{problem}
	\[2\cdot(2xy'+y)=xy^2;\]
\end{problem}
\begin{problem}
	\[3\cdot(xy'+y)=y^2\cdot \ln x;\]
\end{problem}
\begin{problem}
	\[2\cdot(y'+y)=xy^2.\]
\end{problem}
\end{multicols}

Розв’язати рівняння Рікатті:
\begin{multicols}{2}
\begin{problem}
	\[x^2\cdot y' + xy +x^2y^2=4;\]
\end{problem}
\begin{problem}
	\[3y'+y^2+\frac2x=0;\]
\end{problem}
\begin{problem}
	\[xy'-(2x+1)\cdot y+y^2=5-x^2;\]
\end{problem}
\begin{problem}
	\[y'-2xy+y^2=5-x^2;\]
\end{problem}
\begin{problem}
	\[y'+2y \cdot e^x - y^2 = e^{2x} + e^x.\]
\end{problem}
\end{multicols}

\subsection{Рівняння в повних диференціалах}

\subsubsection{Загальна теорія}
Якщо ліва частина диференціального рівняння
\begin{equation}
	\label{eq:1.4.1}
	M(x, y) \cdot \diff x + N(x, y) \cdot \diff y = 0,
\end{equation}
є повним диференціалом деякої функції $u(x, y)$, тобто
\begin{equation}
	\label{eq:1.4.2}
	\diff u(x, y) = M(x, y) \cdot \diff x + N(x, y) \cdot \diff y,
\end{equation}
і, таким чином, \eqref{eq:1.4.1} приймає вигляд $\diff u (x, y) = 0$ то рівняння називається рівнянням в повних диференціалах. Звідси вираз
\begin{equation}
	\label{eq:1.4.3}
	u(x, y) = C
\end{equation}
є загальним інтегралом диференціального рівняння. \\

Критерієм того, що рівняння є рівнянням в повних диференціалах, тобто необхідною та достатньою умовою, є виконання рівності
\begin{equation}
	\label{eq:1.4.4}
	\frac{\partial M(x, y)}{\partial y} = \frac{\partial N(x, y)}{\partial x}.
\end{equation}
 
Нехай маємо рівняння в повних диференціалах. Тоді
\begin{equation}
	\label{eq:1.4.5}
	\frac{\partial u(x, y)}{\partial x} = M(x, y), \quad \frac{\partial u(x, y)}{\partial y} = N(x, y).
\end{equation}
Звідси $u(x, y) = \int M(x, y) \diff x + \phi(y)$ де $\phi(y)$ -- невідома функція. Для її визначення продиференціюємо співвідношення по $y$ і прирівняємо $N(x, y)$:
\begin{equation}
	\label{eq:1.4.6}
	\frac{\partial u(x, y)}{\partial y} = \frac{\partial}{\partial y} \left( \int M(x, y) \diff x \right) + \frac{\diff \phi(y)}{\diff y} = N(x, y).
\end{equation}
Звідси
\begin{equation}
	\label{eq:1.4.7}
	\phi(y) = \int \left( N(x, y) - \frac{\partial}{\partial y} \left( \int M(x, y) \diff x \right) \right) \diff y.
\end{equation}
Остаточно, загальний інтеграл має вигляд
\begin{equation}
	\label{eq:1.4.8}
	\int M(x, y) \diff x + \int \left( N(x, y) - \frac{\partial}{\partial y} \left( \int M(x, y) \diff x \right) \right) \diff y = C.
\end{equation}
Як відомо з математичного аналізу, якщо відомий повний диференціал \eqref{eq:1.4.2}, то $u(x, y)$ можна визначити, взявши криволінійний інтеграл по довільному контуру, що з’єднує фіксовану точку $(x_0, y_0)$ і точку із змінними координатами $(x, y)$. \\

Більш зручно брати криву, що складається із двох відрізків прямих. В цьому випадку криволінійний інтеграл розпадається на два простих інтеграла
\begin{multline}
	\label{eq:1.4.9}
	u(x, y) = \int_{(x_0, y_0)}^{(x,y)} M(x,y) \cdot \diff x + N(x, y) \cdot \diff y = \\
	= \int_{(x_0, y_0)}^{(x,y_0)} M(x,y) \cdot \diff x + \int_{(x,y_0)}^{(x,y)} N(x, y) \cdot \diff y = \\
	= \int_{x_0}^{x} M(x,y_0) \cdot \diff x + \int_{y_0}^{y} N(x, y) \cdot \diff y.
\end{multline}
В цьому випадку одразу одержуємо розв’язок задачі Коші.
\begin{equation}
	\label{eq:1.4.10}
	\int_{x_0}^{x} M(x,y_0) \cdot \diff x + \int_{y_0}^{y} N(x, y) \cdot \diff y = 0.
\end{equation}

\subsubsection{Множник, що Інтегрує}
В деяких випадках рівняння \eqref{eq:1.4.1} не є рівнянням в повних диференціалах, але існує функція $\mu = \mu(x,y)$ така, що рівняння
\begin{equation}
	\label{eq:1.4.11}
	\mu(x,y) \cdot M(x, y) \cdot \diff x + \mu(x,y) \cdot N(x, y) \cdot \diff y = 0,
\end{equation}
вже буде рівнянням в повних диференціалах. Необхідною та достатньою умовою цього є рівність
\begin{equation}
	\label{eq:1.4.12}
	\frac{\partial}{\partial y} (\mu(x,y) \cdot M(x, y)) = \frac{\partial}{\partial x} (\mu(x,y) \cdot N(x, y)),
\end{equation}
або
\begin{equation}
	\label{eq:1.4.13}
	\frac{\partial \mu}{\partial y} \cdot M + \mu \cdot \frac{\partial M}{\partial y} = \frac{\partial \mu}{\partial x} \cdot N + \mu \cdot \frac{\partial N}{\partial x}.
\end{equation}
Таким чином замість звичайного диференціального рівняння відносно функції $y(x)$ одержимо диференціальне рівняння в частинних похідних відносно функції $\mu(x, y)$. \\

Задача інтегрування його значно спрощується, якщо відомо в якому вигляді шукати функцію $\mu(x,y)$, наприклад $\mu = \mu(\omega(x,y))$ де $\omega(x,y)$ -- відома функція. В цьому випадку одержуємо
\begin{equation}
	\label{eq:1.4.14}
	\frac{\partial \mu}{\partial y} = \frac{\diff \mu}{\diff \omega} \cdot \frac{\partial \omega}{\partial y}, \quad \frac{\partial \mu}{\partial x} = \frac{\diff \mu}{\diff \omega} \cdot \frac{\partial \omega}{\partial x}
\end{equation}
Після підстановки в \eqref{eq:1.4.13} маємо
\begin{equation}
	\label{eq:1.4.15}
	\frac{\diff \mu}{\diff \omega} \cdot \frac{\partial \omega}{\partial y} \cdot M + \mu \cdot \frac{\partial M}{\partial y} = \frac{\diff \mu}{\diff \omega} \cdot \frac{\partial \omega}{\partial x} \cdot N + \mu \cdot \frac{\partial N}{\partial x}.
\end{equation}
або
\begin{equation}
	\label{eq:1.4.16}
	\frac{\diff \mu}{\diff \omega} \left( \frac{\partial \omega}{\partial x} \cdot N - \frac{\partial \omega}{\partial y} \cdot M \right) = \mu \left( \frac{\partial M}{\partial y} - \frac{\partial N}{\partial x} \right).
\end{equation}
Розділимо змінні
\begin{equation}
	\label{eq:1.4.17}
	\frac{\diff \mu}{\mu} = \frac{\frac{\partial M}{\partial y} - \frac{\partial N}{\partial x} }{\frac{\partial \omega}{\partial x} \cdot N - \frac{\partial \omega}{\partial y} \cdot M} \cdot \diff \omega.
\end{equation}
Проінтегрувавши і поклавши сталу інтегрування одиницею, одержимо:
\begin{equation}
	\label{eq:1.4.17}
	\mu(\omega(x,y)) = \exp\left\{\int \frac{\frac{\partial M}{\partial y} - \frac{\partial N}{\partial x} }{\frac{\partial \omega}{\partial x} \cdot N - \frac{\partial \omega}{\partial y} \cdot M} \cdot \diff \omega\right\}.
\end{equation}
Розглянемо частинні випадки.
\begin{enumerate}
	\item Нехай $\omega(x, y) = x$. Тоді $\frac{\partial \omega}{\partial x} = 1$, $\frac{\partial \omega}{\partial y} = 0$, $\diff \omega = \diff x$ і формула має вигляд
	\begin{equation}
		\label{eq:1.4.18}
		\mu(\omega(x,y)) = \exp\left\{\int \frac{\frac{\partial M}{\partial y} - \frac{\partial N}{\partial x} }{N} \cdot \diff x\right\}.
	\end{equation}	
	\item Нехай $\omega(x, y) = y$. Тоді $\frac{\partial \omega}{\partial x} = 0$, $\frac{\partial \omega}{\partial y} = 1$, $\diff \omega = \diff y$ і формула має вигляд
	\begin{equation}
		\label{eq:1.4.19}
		\mu(\omega(x,y)) = \exp\left\{\int \frac{\frac{\partial M}{\partial y} - \frac{\partial N}{\partial x} }{-M} \cdot \diff y\right\}.
	\end{equation}
	\item Нехай $\omega(x, y) = x^2 \pm y^2$. Тоді $\frac{\partial \omega}{\partial x} = 2 x$, $\frac{\partial \omega}{\partial y} = \pm 2y$, $\diff \omega = \diff (x^2 \pm y^2)$ і формула має вигляд
	\begin{equation}
		\label{eq:1.4.20}
		\mu(\omega(x,y)) = \exp\left\{\int \frac{\frac{\partial M}{\partial y} - \frac{\partial N}{\partial x} }{2 x N \mp 2 y M} \cdot \diff (x^2 \pm y^2)\right\}.
	\end{equation}
	\item Нехай $\omega(x, y) = x y$. Тоді $\frac{\partial \omega}{\partial x} = y$, $\frac{\partial \omega}{\partial y} = x$, $\diff \omega = \diff (xy)$ і формула має вигляд
	\begin{equation}
		\label{eq:1.4.21}
		\mu(\omega(x,y)) = \exp\left\{\int \frac{\frac{\partial M}{\partial y} - \frac{\partial N}{\partial x} }{yN-xM} \cdot \diff (xy)\right\}.
	\end{equation}
\end{enumerate}

\subsubsection{Вправи для самостійної роботи}
Як вже було сказано, рівняння \[M(x, y) \cdot \diff x + N(x, y) \cdot \diff y = 0\] буде рівнянням в повних диференціалах, якщо його ліва частина є повним диференціалом деякої функції. Це має місце при \[\frac{\partial M(x, y)}{\partial y} = \frac{\partial N(x, y)}{\partial x}.\]

\begin{example}
	Розв’язати рівняння \[(2x + 3x^2y) \cdot \diff x + (x^3 - 3y^2) \cdot \diff y = 0.\]
\end{example}
\begin{solution}
	Перевіримо, що це рівняння є рівнянням в повних диференціалах. Обчислимо
	\[ \frac{\partial}{\partial y} (2x + 3x^2y) = 3x^2, \quad \frac{\partial}{\partial x} (x^3 - 3y^2) = 3x^2. \]
	Таким чином існує функція $u(x,y)$, що \[\frac{\partial u(x,y)}{\partial x} = 2x + 3x^2y.\] Проінтегруємо по $x$. Отримаємо
	\[ u(x,y) = \int(2x+3x^2y)\cdot\diff x+\Phi(y)=x^2+x^3y+\Phi(y).\]
	Для знаходження функції $\Phi(y)$ візьмемо похідну від $u(x,y)$ по $y$ і прирівняємо до $x^3-3y^2$. Отримаємо
	\[ \frac{\partial u(x,y)}{\partial y} = x^3 + \Phi'(y) = x^3 - 3y^2.\]
	Звідси $\Phi'(y) = -3y^2$ і $\Phi(y) = -y^3$. Таким чином, \[u(x,y)=x^2+x^3y-y^3\] і загальний інтеграл диференціального рівняння має вигляд \[x^2+x^3y-y^3=C.\]
\end{solution}

Перевірити, що дані рівняння є рівняннями в повних диференціалах, і розв’язати їх:
\begin{problem}
	\[ 2 x y \cdot \diff x + (x^2 - y^2) \cdot \diff y = 0;\]
\end{problem}
\begin{problem}
	\[(2-9xy^2)\cdot x\cdot \diff x + (4y^2-6x^3)\cdot y\cdot \diff y=0;\]
\end{problem}
\begin{problem}
	\[e^{-y}\cdot\diff x-(2y+x\cdot e^{-y})\cdot\diff y=0;\]
\end{problem}
\begin{problem}
	\[ \frac yx\cdot\diff x+(y^3+\ln x)\cdot\diff y=0;\]
\end{problem}
\begin{problem}
	\[\frac{3x^2+y^2}{y^2}\cdot\diff x-\frac{2x^3+5y}{y^3}\cdot\diff y=0;\]
\end{problem}
\begin{problem}
	\[2x\cdot\left(1+\sqrt{x^2-y}\right)\cdot\diff x-\sqrt{x^2-y}\cdot\diff y=0;\]
\end{problem}
\begin{problem}
	\[(1+y^2\cdot\sin 2x)\cdot\diff x-2y\cdot\cos^2x\cdot\diff y=0;\]
\end{problem}
\begin{problem}
	\[3x^2\cdot(1+\ln y)\cdot\diff x=\left(2y-\frac{x^3}y\right)\cdot\diff y;\]
\end{problem}
\begin{problem}
	\[\left(\frac x{\sin y}+2\right)\cdot\diff x+\frac{(x^2+1)\cdot\cos y}{\cos2y-1}\cdot\diff y=0;\]
\end{problem}
\begin{problem}
	\[(2x+y\cdot e^{xy})\cdot\diff x+(x\cdot e^{xy}+3y^2)\cdot\diff y=0;\]
\end{problem}
\begin{problem}
	\[\left(2+\frac{1}{x^2+y^2}\right)\cdot x\cdot\diff x+\frac{y}{x^2+y^2}\cdot \diff y=0;\]
\end{problem}
\begin{problem}
	\[\left(3y^2-\frac{y}{x^2+y^2}\right)\cdot\diff x+\left(6xy+\frac{x}{x^2+y^2}\right)\cdot \diff y=0.\]
\end{problem}

Розв’язати, використовуючи інтегруючий множник:
\begin{problem} $\mu=\mu(x-y)$,
	\[(2x^3+3x^2y+y^2-y^3)\cdot\diff x+(2y^3+3xy^2+x^2-x^3)\cdot\diff x=0;\]
\end{problem}
\begin{problem}
	\[ \left(y-\frac{ay}{x}+x\right)\cdot\diff x+a\cdot\diff y=0, \quad \mu=\mu(x+y);\]
\end{problem}
\begin{problem}
	\[(x^2+y)\cdot\diff y+x\cdot(1-y)\cdot\diff x=0, \quad \mu=\mu(xy);\]
\end{problem}
\begin{problem}
	\[(x^2-y^2+y)\cdot\diff x+x\cdot(2y-1)\cdot\diff y=0;\]
\end{problem}
\begin{problem}
	\[(2x^2y^2+y)\cdot\diff x+(x^3y-x)\cdot\diff y=0.\]
\end{problem}

\subsection{Диференціальні рівняння першого порядку, не розв’язані відносно похідної}

Диференціальне рівняння першого порядку, не розв’язане відносно похідної, має такий вигляд
\begin{equation}
	\label{eq:1.5.1}
	F(x, y, y') = 0. 	
\end{equation}

\subsubsection{Частинні випадки рівнянь, що інтегруються в квадратурах}

Розглянемо ряд диференціальних рівнянь, що інтегруються в квадратурах.
\begin{enumerate}
	\item Рівняння вигляду 
	\begin{equation}
		\label{eq:1.5.2}
		F(y') = 0.
	\end{equation}
	Нехай алгебраїчне рівняння $F(k) = 0$ має принаймні один дійсний корінь $k = k_0$. Тоді, інтегруючи $y' = k_0$, одержимо $y = k_0 \cdot x + C$. Звідси $k_0 = (y - C) / x$ і вираз
	\begin{equation}
		\label{eq:1.5.3}
		F \left( \frac{y - c}{x} \right) = 0	
	\end{equation}
	містить всі розв’язки вихідного диференціального рівняння.
	\item Рівняння вигляду 
	\begin{equation}
		\label{eq:1.5.4}
		F(x, y') = 0.
	\end{equation}
	Нехай це рівняння можна записати у параметричному вигляді
	\begin{equation}
		\label{eq:1.5.5}
		\left\{\begin{aligned}
			x &= \phi(t), \\
			y' &= \psi(t).
		\end{aligned}\right.
	\end{equation}
	Використовуючи співвідношення $\diff y = y ' \cdot \diff x$, одержимо 
	\begin{equation}
		\label{eq:1.5.6}
		\diff y = \psi(t) \cdot \phi'(t) \cdot \diff t.
	\end{equation}
	Проінтегрувавши, запишемо
	\begin{equation}
		\label{eq:1.5.7}
		y = \int \psi(t) \cdot \phi'(t) \cdot \diff t + C.
	\end{equation}
	І загальний розв’язок в параметричній формі має вигляд
	\begin{equation}
		\label{eq:1.5.8}
		\left\{\begin{aligned}
		x &= \phi(t), \\
		y &= \int \psi(t) \cdot \phi'(t) \cdot \diff t + C.
		\end{aligned}\right.
	\end{equation}
	\item Рівняння вигляду 
	\begin{equation}
		\label{eq:1.5.9}
		F(y, y') = 0.
	\end{equation}
	Нехай це рівняння можна записати у параметричному вигляді
	\begin{equation}
		\label{eq:1.5.10}
		\left\{\begin{aligned}
			y &= \phi(t), \\
			y' &= \psi(t).
		\end{aligned}\right.
	\end{equation}
	Використовуючи співвідношення $\diff y = y ' \cdot \diff x$, одержимо 
	\begin{equation}
		\label{eq:1.5.11}
		\phi'(t) \cdot \diff t = \psi(t) \cdot \diff x
	\end{equation}
	і
	\begin{equation}
		\label{eq:1.5.12}
		\diff x = \frac{\phi'(t)}{\psi(t)} \cdot \diff t
	\end{equation}
	Проінтегрувавши, запишемо
	\begin{equation}
		\label{eq:1.5.13}
		x = \int \frac{\phi'(t)}{\psi(t)}\cdot \diff t + C.
	\end{equation}
	І загальний розв’язок в параметричній формі має вигляд
	\begin{equation}
		\label{eq:1.5.14}
		\left\{\begin{aligned}
		x &= \int \frac{\phi'(t)}{\psi(t)}\cdot \diff t + C, \\
		y &= \phi(t).
		\end{aligned}\right.
	\end{equation}
	\item Рівняння Лагранжа
	\begin{equation}
		\label{eq:1.5.15}
		y = \phi(y') \cdot x + \psi(y').
	\end{equation}
	Введемо параметр $y' = \frac{\diff y}{\diff x} = p$ і отримаємо
	\begin{equation}
		\label{eq:1.5.16}
		y = \phi(p) \cdot x + \psi(p).
	\end{equation}
	Продиференціювавши, запишемо
	\begin{equation}
		\label{eq:1.5.17}
		\diff y = \phi'(p) \cdot x \cdot \diff p + \phi(p) \cdot \diff x + \psi'(p) \cdot \diff p.
	\end{equation}
	Замінивши $\diff y = p \cdot \diff x$ одержимо
	\begin{equation}
		\label{eq:1.5.18}
		p \cdot \diff x = \phi'(p) \cdot x \cdot \diff p + \phi(p) \cdot \diff x + \psi'(p) \cdot \diff p.
	\end{equation}
	Звідси
	\begin{equation}
		\label{eq:1.5.19}
		(p - \phi(p)) \cdot \diff x - \phi'(p) \cdot x \cdot \diff p = \psi'(p) \cdot \diff p.
	\end{equation}
	І отримали лінійне неоднорідне диференціальне рівняння
	\begin{equation}
		\label{eq:1.5.20}
		\frac{\diff x}{\diff p} + \frac{\phi'(p)}{\phi(p)-p} \cdot x = \frac{\phi'(p)}{p-\phi(p)}.
	\end{equation}
	Його розв’язок
	\begin{multline}
		\label{eq:1.5.21}
		x = \exp\left\{\int \frac{\phi'(p)}{p-\phi(p)} \cdot \diff p\right\} \cdot \\
		\cdot \left(\int \frac{\phi'(p)}{p-\phi(p)} \cdot \exp\left\{\int \frac{\phi'(p)}{\phi(p)-p} \cdot \diff p\right\} \diff p + C \right) = \\
		= \Psi(p, C).
	\end{multline}
	І остаточний розв’язок рівняння Лагранжа в параметричній формі запишеться у вигляді
	\begin{equation}
		\label{eq:1.5.22}
		\left\{\begin{aligned}
			x &= \Psi(p,C), \\
			y &= \phi(p) \cdot \Phi(p, C) + \psi(p).
		\end{aligned}\right.
	\end{equation}
\end{enumerate}


\subsection{Існування та єдиність розв’язків диференціальних рівнянь першого порядку. Неперервна залежність та диференційованість}

Клас диференціальних рівнянь, що інтегруються в квадратурах, досить невеликий, тому мають велике значення наближені методи розв’язку диференціальних рівнянь. Але, щоб використовувати ці методи, треба бути впевненим в існуванні розв’язку шуканого рівняння та в його єдиності. \\

Зараз значна частина теорем існування  та єдиності розв’язків не тільки диференціальних, але й рівнянь інших видів доводиться методом стискаючих відображень. \\

\begin{definition} 
	Простір $M$ називається метричним, якщо для довільних двох точок $x,y\in M$ визначена функція $\rho(x,y)$, що задовольняє аксіомам:
	\begin{enumerate}
		\item $\rho(x, y)\ge0$, причому $\rho(x,y)=0$ тоді і тільки тоді, коли $x=y$;
		\item $\rho(x,y)=\rho(y,x)$ (комутативність);
		\item $\rho(x,y)+\rho(y,z)\ge\rho(x,z)$ (нерівність трикутника).
	\end{enumerate}
	Функція $\rho(x,y)$ називається відстанню (метрикою) в просторі $M$.
\end{definition}
\begin{example*} 
	Векторний $n$-вимірний простір $\RR^n$. \\

	Нехай $x=(x_1,x_2,\ldots,x_n)$, $y=(y_1,y_2,\ldots,y_n)$. За метрику можна взяти: 
	\begin{equation*}
		%\label{eq:1.6.1}
		\rho(x,y)=\left(\sum_{i=1}^n (x_i-y_i)^2\right)^{1/2},
	\end{equation*}
	або 
	\begin{equation*}
		%\label{eq:1.6.2}
		\rho(x,y)=\max_{i=\overline{1,n}}|x_i-y_i|.
	\end{equation*}
\end{example*}
\begin{example*} 
	Простір неперервних функцій на відрізку $[a,b]$ позначається $C([a,b])$. За метрику можна взяти
		\begin{equation*}
		%\label{eq:1.6.3}
		\rho(x(t), y(t)) = \left(\int_a^b (x(t)-y(t))^2 \diff t\right)^{1/2},
	\end{equation*}
	або
	\begin{equation*}
		%\label{eq:1.6.4}
		\rho(x(t), y(t)) = \max_{t\in[a,b]} |x(t)-y(t)|.
	\end{equation*}
\end{example*}
\begin{definition} 
	Послідовність $\{x_n\}_{n=1}^\infty$ називається фундаментальною, як\-що для довільного $\epsilon > 0$ існує $n \ge N(\epsilon)$ таке, що при $n \ge N(\epsilon)$ і довільному $m\in\NN$ буде $\rho(x_n,x_{n+m}) < \epsilon$.
\end{definition}
\begin{definition} 
	Метричний простір $M$ називається повним, якщо довільна фундаментальна послідовність точок $\{x_n\}$ простору $M$ збігається до деякої точки $x_0$ простору $M$.
\end{definition}
\begin{theorem}[принцип стискаючих відображень] 
	Нехай в повному метричному просторі $M$ задано оператор $A$, що задовольняє умовам.
	\begin{enumerate}
		\item Оператор $A$ переводить точки простору $M$ в точки цього ж простору, тобто якщо $x\in M$, то і $Ax \in M$.
		\item Оператор $A$ є оператором стиску, тобто $\rho(Ax,Ay)\le\alpha\rho(x,y)$, де $0<\alpha<1$, $x,y$ -- довільні точки $M$. 
	\end{enumerate}
	Тоді існує єдина нерухома точка $\bar x \in M$, яка є розв’язком операторного рівняння $A\bar x=\bar x$ і вона може бути знайдена методом послідовних відображень, тобто $x = \lim_{n\to\infty} x_n$, де $x_{n+1} = A x_n$, причому $x_0$ вибирається довільно.
\end{theorem}
\begin{proof}
	Візьмемо довільну точку $x_0\in M$ і побудуємо послідовність $A^nx_0$. Покажемо, що побудована послідовність є фундаментальною. Дійсно
	\begin{align*}
		%\label{eq:1.6.5}
		\rho(x_2, x_1) &= \rho(A x_1, A x_0) \le \alpha \rho (x_1, x_0), \\
		%\label{eq:1.6.6}
		\rho(x_3, x_2) &= \rho(A x_2, A x_1) \le \alpha \rho (x_2, x_1) \le \alpha^2 \rho(x_1, x_0), \\
		%\label{eq:1.6.7}
		\rho(x_{n+1}, x_n) &= \rho(A x_n, A x_{n-1}) \le \alpha \rho (x_n, x_{n-1}) \le \ldots \le \alpha^n \rho(x_1, x_0).
	\end{align*}
	Оцінимо $\rho(x_n, x_{n+m})$. Застосувавши $m-1$ раз нерівність трикутника, отримуємо 
	\begin{multline*}
		%\label{eq:1.6.8}
		\rho(x_n, x_{n+m}) \le \rho(x_n, x_{n+1}) + \rho(x_{n+1}, x_{n+2}) + \ldots + \rho(x_{n+m-1},x_{n+m}) \le \\
		\le \alpha^n \rho(x_1, x_0) + \alpha^{n+1} \rho(x_1, x_0) +\alpha^{n+m-1} \rho(x_1, x_0) = \\
		= (\alpha^n + \alpha^{n+1} + \ldots + \alpha^{n + m}) \cdot \rho(x_1, x_0) < \frac{\alpha^n}{1 - \alpha} \cdot \rho(x_1, x_0) \xrightarrow[n\to\infty]{} 0.
	\end{multline*}
	Тобто послідовність $\{x_n\}$ є фундаментальною і, в силу повноти простору $M$, збігається до деякого елемента цього ж простора $x$. \\

	Покажемо, що $x$ є нерухомою точкою $A$, тобто $Ax=x$.\\

	Нехай від супротивного $Ax=\bar x$ і $x\ne\bar x$. Застосувавши нерівність трикутника, одержимо $\rho(x,\bar x) < \rho(x, x_{n+1}) + \rho(x_{n+1}, \bar x)$. Оцінимо кожний з доданків.
	\begin{enumerate}
		\item $\rho(x, x_{n+1}) \xrightarrow[n\to\infty]{} 0$.
		\item $\rho(x_{n+1}, \bar x) = \rho(Ax_n, Ax) \le \alpha \rho(x_n, x) \xrightarrow[n\to\infty]{} 0$.
	\end{enumerate}
	Таким чином $\rho(x, \bar x) \le 0$, а в силу невід'ємності метрики це значить, що $x = \bar x$. \\

	Покажемо, що нерухома точка єдина. Нехай, від супротивного, існують дві точки $x$ і $y$: $A x = x$ і $A y = y$. Але тоді
	\begin{equation*}
		%\label{eq:1.6.9}
		\rho(x, y) = \rho(A x, A y) \le \alpha \rho(x, y) < \rho(x, y),
	\end{equation*}
	 що суперечить припущенню про стислість оператора. Таким чином, припущення про неєдиність нерухомої точки помилкове.
\end{proof}

З використанням теореми про нерухому точку доведемо теорему про існування та єдиність розв’язку задачі Коші диференціального рівняння, розв’язаного відносно похідної.

\begin{theorem}[про існування та єдиність розв’язку задачі Коші]
	Нехай у диференціальному рівнянні $\frac{\diff y}{\diff x} = f(x, y)$ функція $f(x,y)$ визначена в прямокутнику
	\begin{equation*}
		%\label{eq:1.6.10}
		D = \{(x,y) : x_0 - a \le x \le x_0 + a, y_0 - b \le y \le y_0 + b\},
	\end{equation*}
	і задовольняє умовам:
	\begin{enumerate}
		\item $f(x,y)$ неперервна по $x$ та $y$ у $D$;
		\item $f(x,y)$ задовольняє умові Ліпшиця по змінній $y$, тобто 
		\begin{equation*}
			%\label{eq:1.6.11}
			|f(x, y_1) - f(x, y_2)| \le N \cdot |y_1 - y_2|, \quad N = const.
		\end{equation*}
	\end{enumerate}
	Тоді існує єдиний розв’язок $y = y(x)$ диференціального рівняння, який визначений при $x_0 - h \le x \le x_0 + h$, і задовольняє умові $y(x_0) = y_0$, де $h < \min \{a, b / M, 1 / N\}$, $M = \max_{(x, y) \in D} |f(x,y)|$.
\end{theorem}

\begin{proof}
	Розглянемо простір, елементами якого є функції $y(x)$, неперервні на відрізку $[x_0 - h, x_0 + h]$ й обмежені $|y(x) - y_0| \le b$. Введемо метрику $\rho(y(x), z(x))$. Одержимо повний метричний простір $C([x_0 - h, x_0 + h])$. Замінимо диференціальне рівняння
	\begin{equation*}
		%\label{eq:1.6.12}
		\frac{\diff y}{\diff x} = f(x, y), \quad y(x_0) = y_0
	\end{equation*}
	еквівалентним інтегральним рівнянням
	\begin{equation*}
		%\label{eq:1.6.13}
		y(x) = \int_{x_0}^x f(t, y(t)) \diff t + y_0 = A y.
	\end{equation*}
	Розглянемо оператор $A$. Через те, що  
	\begin{equation*}
		%\label{eq:1.6.14}
		\left|\int_{x_0}^x f(t, y(t)) \diff t \right| \le \int_{x_0}^x |f(t, y(t))| \diff t \le M \cdot |x-x_0| \le Mh \le b,
	\end{equation*}
	то оператор $A$ ставить у відповідність кожній неперервній функції $y(x)$, визначеній при $x\in[x_0 - h, x_0 + h]$ й обмеженій $|y(x)-y_0|\le b$ також неперервну функцію $Ay$,  визначену при $x\in[x_0 - h, x_0 + h]$ й обмежену $|y(x)-y_0|\le b$. \\

	Перевіримо, чи є оператор $A$ оператором стиску:
	\begin{align*}
		%\label{eq:1.6.15}
		\rho(Ay, Az) &= \max_{x \in[x_0-h,x_0+h]} \left|y_0 + \int_{x_0}^x f(t, y(t)) \diff y - y_0 - \int_{x_0}^x f(t, z(t)) \diff t\right| \le \\
		&\le \max_{x \in[x_0-h,x_0+h]} \int_{x_0}^x |f(t, y(t)) - f(t, z(t))| \diff t \le \\
		&\le N \cdot \max_{x \in[x_0-h,x_0+h]} \int_{x_0}^x |y(t) - z(t)| \diff t \le \\
		&\le N \cdot \max_{x \in[x_0-h,x_0+h]} |y(t) - z(t)| \cdot \int_{x_0}^x \diff t \le N \cdot \rho(y, z) \cdot h.
	\end{align*}
	І оскільки $Nh < 1$, то оператор $A$ є оператором стиску. Відповідно до принципу стискаючих відображень операторне рівняння $Ay=y$ має єдиний розв’язок, тобто інтегральне рівняння чи початкова задача Коші також має єдиний розв’язок.
\end{proof}

\begin{remark}
	Умову Ліпшиця можна замінити іншою, більш грубою, але легше перевіряємою умовою існування обмеженої по модулю частинної похідної $f_y^\prime (x,y)$ в області $D$. Дійсно,
	\begin{equation*}
		%\label{eq:1.6.16}
		|f(x,y_1)-f(x,y_2)|=|f_y^\prime(x,\xi)|\cdot|y_1-y_2|\le N\cdot|y_1-y_2|,
	\end{equation*}
	де $N = \max_{(x,y)\in D} |f_y^\prime(x,y)|$.
\end{remark}

Використовуючи доведену теорему про існування та єдиність роз\-в'яз\-ку задачі Коші розглянемо ряд теорем, що описують якісну поведінку роз\-в'яз\-ків.

\begin{theorem}[про неперервну залежність роз\-в'яз\-ків від параметру]
	Якщо права частина диференціального рівняння
	\begin{equation*}
		%\label{eq:1.6.17}
		\frac{\diff y}{\diff x} = f(x, y, \mu)
	\end{equation*}
	неперервна по $\mu$ при $\mu \in [\mu_1, \mu_2]$ і при кожному фіксованому $\mu$ задовольняє умовам теореми існування й єдиності, причому стала Ліпшиця $N$ не залежить від $\mu$, то розв’язок $y = y(x, \mu)$, що задовольняє початковій умові $y(x_0)=y_0$, неперервно залежить від $\mu$.
\end{theorem}
\begin{proof} 
	Оскільки члени послідовності
	\begin{equation*}
		%\label{eq:1.6.18}
		y_n(x, \mu) = y_0 + \int_{x_0}^x f(t, y_n(t, \mu)) \diff t
	\end{equation*}
	є неперервними функціями змінних $x$ і $\mu$, а стала $N$ не залежить від $\mu$, то послідовність $\{y_n\}$ збігається до $y$ рівномірно по $\mu$. І, як випливає з математичного аналізу, якщо послідовність неперервних функцій збігається рівномірно, то вона збігається до неперервної функції, тобто $y=y(x,\mu)$ -- функція, неперервна по $\mu$.
\end{proof}

\begin{theorem}[про неперервну залежність від початкових умов]
	Нехай виконані умови теореми про існування та єдиність роз\-в'я\-зків рівняння
	\begin{equation*}
		%\label{eq:1.6.19}
		\frac{\diff y}{\diff x} = f(x,y)
	\end{equation*}
	з початковими умовами $y(x_0) = y_0$. Тоді, розв’язки $y=y(x_0,y_0,x)$, що записані у формі Коші, неперервно залежать від початкових умов. 
\end{theorem}
\begin{proof}
	Роблячи заміну $x = y(x_0, y_0, x) - y_0$, $t = x - x_0$ одержимо диференціальне рівняння  
	\begin{equation*}
		%\label{eq:1.6.20}
		\frac{\diff z}{\diff t} = f(t + x_0, z + y_0)
	\end{equation*}
	з нульовими початковими умовами. На підставі попередньої теореми маємо неперервну залежність розв’язків від $x_0$, $y_0$ як від параметрів.
\end{proof}

\begin{theorem}[про диференційованість розв’язків]
	Якщо в околі точки $(x_0,y_0)$ функція $f(x,y)$ має неперервні змішані похідні до $k$-го порядку, то розв’язок $y(x)$ рівняння
	\begin{equation*}
		%\label{eq:1.6.21}
		\frac{\diff y}{\diff x} = f(x, y)
	\end{equation*}
	з початковими умовами $y(x_0)=y_0$ в деякому околі точки $(x_0,y_0)$ буде $k$ разів неперервно диференційований.
\end{theorem}
\begin{proof} 
	Підставивши $y(x)$ в рівняння, одержимо тотожність
	\begin{equation*}
		%\label{eq:1.6.22}
		\frac{\diff y(x)}{\diff x} \equiv f(x, y(x)),
	\end{equation*}
	яку можна диференціювати
	\begin{equation*}
		%\label{eq:1.6.23}
		\frac{\diff^2 y}{{\diff x}^2} = \frac{\partial f}{\partial x} + \frac{\partial f}{\partial y} \cdot \frac{\diff y}{\diff x} = \frac{\partial f}{\partial x} + \frac{\partial f}{\partial y} \cdot f.
	\end{equation*}
	Якщо $k > 1$, то праворуч функція неперервно диференційована. Продиференціюємо її ще раз
	\begin{multline*}
		%\label{eq:1.6.24}
		\frac{\diff^3 y}{{\diff x}^3} = \frac{\partial^2 f}{{\partial x}^2} + \frac{\partial^2 f}{\partial x \partial y} \cdot \frac{\diff y}{\diff x} + \left( \frac{\partial^2 f}{\partial y \partial x} + \frac{\partial^2 f}{{\partial y}^2} \cdot \frac{\diff y}{\diff x} \right) \cdot f + \\
		+ \frac{\partial f}{\partial y} \cdot \left( \frac{\partial f}{\partial x} + \frac{\partial f}{\partial y} \cdot \frac{\diff y}{\diff x} \right),
	\end{multline*}
	або
	\begin{equation*}
		%\label{eq:1.6.25}
		\frac{\diff^3 y}{{\diff x}^3} = \frac{\partial^2 f}{{\partial x}^2} + 2 \cdot \frac{\partial^2 f}{\partial x \partial y} \cdot f + \frac{\partial^2 f}{{\partial y}^2} \cdot f^2 + \frac{\partial f}{\partial y} \cdot \left( \frac{\partial f}{\partial x} + \frac{\partial f}{\partial y} \cdot F \right),
	\end{equation*}
	Проробивши це $k$ разів, отримаємо твердження теореми.
\end{proof}

Розглянемо диференціальне рівняння, не розв’язане відносно похідної
\begin{equation*}
	%\label{eq:1.6.26}
	F(x, y, y') = 0.
\end{equation*}
Нехай $(x_0, y_0)$ -- точка на площині. Підставивши її в рівняння, одержимо відносно $y'$ алгебраїчне рівняння
\begin{equation*}
	%\label{eq:1.6.27}
	F(x_0, y_0, y') = 0.
\end{equation*}
Це рівняння має корені $y_0^\prime, y_1^\prime, \ldots, y_n^\prime$. Задача Коші для диференціального рівняння, не розв’язаного відносно похідної, ставиться в такий спосіб. \\

Потрібно знайти розв’язок $y=y(x)$ диференціального, що задовольняє умовам $y(x_0)=y_0$, $y'(x_0)=y_i^\prime$, де $x_0,y_0$ -- довільні значення, а $y_i^\prime$ -- один з вибраних наперед коренів алгебраїчного рівняння.

\begin{theorem}[існування й єдиність розв’язку задачі Коші рівняння, не розв’язаного  відносно похідної]
	Нехай у замкненому околі точки $(x_0, y_0, y_i^\prime)$ функція $F(x,y,y')$ задовольняє умовам:
	\begin{enumerate}
		\item $F(x,y,y')$ -- неперервна по всіх аргументах;
		\item $\frac{\partial F}{\partial y'}$ існує і відмінна від нуля;
		\item $\left| \frac{\partial F}{\partial y}\right| \le N_0$.
	\end{enumerate}
	Тоді при $x \in [x_0 - h, x_0 + h]$, де $h$ -- досить мал е, існує єдиний розв’язок $y=y(x)$ рівняння $F(x, y, y') =0$, що задовольняє початковій умові $y(x_0)=y_0$, $y'(x_0)=y_i^\prime$.
\end{theorem}
\begin{proof}
	Як випливає з математичного аналізу відповідно до теореми про неявну функцію можна стверджувати, що умови 1) і 2) гарантують існування єдиної неперервної в околі точки $(x_0,y_0,y_i^\prime)$ функції $y'=f(x,y)$, обумовленої рівнянням $F(x,y,y')=0$, для якої $y'(x_0)=y_i^\prime$. Перевіримо, чи задовольняє $f(x,y)$ умові Ліпшиця чи більш грубій $\left|\frac{\partial f}{\partial y}\right| \le N$. Диференціюємо $F(x,y,y')=0$ по $y$. Оскільки $y'=f(x,y)$, то одержуємо
	\begin{equation*}
		%\label{eq:1.6.28}
		\frac{\partial F}{\partial y} + \frac{\partial F}{\partial y'} \cdot \frac{\partial f}{\partial y} = 0.
	\end{equation*}
	Звідси
	\begin{equation*}
		%\label{eq:1.6.29}
		\frac{\partial f}{\partial y} = - \frac{\frac{\partial F}{\partial y}}{\frac{\partial F}{\partial y'}} 
	\end{equation*} 
	З огляду на умови 2), 3), одержимо, що в деякому околі точки $(x_0,y_0)$ буде $\left|\frac{\partial f}{\partial y}\right| \le N$ і для рівняння $y'=f(x,y)$ виконані умови теореми існування й єдиності розв’язку задачі Коші.
\end{proof}


\subsubsection{Особливі розв’язки}

\begin{definition}
	Розв’язок $y = \phi(x)$ диференціального рівняння, в кожній точці якого $M(x,y)$ порушена єдиність розв’язку задачі Коші, називається особливим розв’язком. 
\end{definition}

Очевидно, особливі розв’язки треба шукати в тих точках області $D$, де порушені умови теореми про існування й єдиність розв’язку задачі Коші. Але, оскільки умови теореми носять достатній характер, то їхнє не виконання для існування особливих розв’язків, носить необхідний характер. І точки $N(x,y)$ області $D$, у яких порушені умови теореми про існування та єдиність розв’язку диференціального рівняння, є лише "підозрілими" на особливі розв’язки. \\

Розглянемо рівняння 
\begin{equation*}
	%\label{eq:1.6.30}
	y' = f(x,y).
\end{equation*}
Неперервність $f(x,y)$ в області $D$ зазвичай виконується, і особливі роз\-в'яз\-ки варто шукати там, де $\frac{\partial f}{\partial y} = +\pm \infty$. \\

Для диференціального рівняння, не роз\-в'яз\-а\-но\-го відносно похідної 
\begin{equation*}
	%\label{eq:1.6.31}
	F(x, y, y') = 0,
\end{equation*}
умови неперервності $F(x,y,y')$ й обмеженості $\frac{\partial F}{\partial y}$ зазвичай виконуються. І особливі розв’язки варто шукати там, де задовольняється остання рівність і 
\begin{equation*}
	%\label{eq:1.6.32}
	\frac{\partial F(x,y,y')}{\partial y'} = 0.
\end{equation*}

Вилучаючи із системи $y'$, одержимо $\Phi(x,y)=0$. Однак не в кожній точці $M(x,y)$, у якій $\Phi(x,y)$, порушується єдиність роз\-в'яз\-ку, тому що умови теореми мають лише достатній характер і не є необхідними. Якщо ж яка-небудь гілка $y=\phi(x)$ кривої $\Phi(x,y)$ є інтегральною кривою, то $y=\phi(x)$ називається особливим роз\-в'яз\-ком. \\

Таким чином, для знаходження особливого роз\-в'яз\-ку $F(x, y, y') = 0$ треба
\begin{enumerate}
	\item знайти $p$-дискримінантну криву з $F(x, y, y') = 0$ та $\frac{\partial F(x,y,y')}{\partial y'} = 0$.
	\item з'я\-су\-ва\-ти шляхом підстановки чи є серед гілок $p$-дискримінантної кривої інтегральні криві;
	\item з'я\-су\-ва\-ти чи порушена умова одиничності в точках цих кривих.
\end{enumerate}


\subsubsection{Вправи для самостійної роботи}

\setcounter{problem}{0}

\begin{example}
	Побудувати послідовні наближення $y_0(x)$, $y_1(x)$, $y_2(x)$ для рівняння $y' = x - y^2$, $y(0) = 0$.
\end{example}
\begin{solution}
	Візьмемо початкову функцію $y_0(0) \equiv 0$. Підставивши в ітераційну залежність \[ y_{n+1} (x) = y(x_0) + \int_{x_0}^x f(s, y_n(s)) \diff s \] отримаємо 
	\begin{align*} 
		y_1(x) &= \int_0^x s \diff s = \frac{x^2}{2}, \\
		y_2(x) &= \int_0^x (s - y_1^2(s)) \diff s = \int_0^x \left(s - \frac{s^4}{4}\right) \diff s = \frac{x^2}{2} - \frac{x^5}{20}.
	\end{align*}
\end{solution}

Побудувати послідовні наближення $y_0(x)$, $y_1(x)$, $y_2(x)$ для рівнянь
\begin{problem}
	\[y' = y^2 + 3x^2 - 1, \quad y(0) = 1;\]
\end{problem}
\begin{problem}
	\[y'=y+e^{y-1},\quad y(0)=1;\]
\end{problem}
\begin{problem}
	\[y'=1+x\cdot\sin y, \quad y(\pi)=2\pi.\]
\end{problem}

\begin{example}
	Вказати на проміжок з $a=1$, $b=1$, на якому гарантується існування та єдиність розв’язку диференціального рівняння $y'=y^3+x$, $y(0)=1$.
\end{example}
\begin{solution}
	Як випливає з теореми про існування та єдиність розв’язку, проміжок, на якому гарантується існування та єдиність розв’язку задачі Коші дорівнює $h = \min \left\{ a, \frac{b}{M}, \frac{1}{L}\right\}$, де \[ M = \max_{(x,y)\in D} |f(x,y)|, \quad L = \max_{(x,y)\in D} \left|\frac{\partial f(x,y)}{\partial y}\right|.\] Для цієї задачі отримаємо $D = \{ (x,y): |x| \le 1, |y| \le 1\}$, $M=2$, $L=3$. Тому $h = 1/3$.
\end{solution}	

Вказати проміжки, де гарантується існування та єдиність розв’язку задачі Коші рівняння
\begin{problem}
	\[y'=y+e^y,\quad x_0=0, \quad y_0=0,\quad a=1,\quad b=2;\]
\end{problem}
\begin{problem}
	\[y'=2xy+y^3,\quad x_0=1, \quad y_0=1,\quad a=2,\quad b=1;\]
\end{problem}
\begin{problem}
	\[y'=2+\sqrt[3]{y-2x},\quad x_0=0, \quad y_0=1,\quad a=1,\quad b=1.\]
\end{problem}

\begin{example}
	Знайти особливий розв’язок рівняння $y' - \sqrt{y}$.
\end{example}
\begin{solution}
	Особливий розв’язок слід шукати там, де $\frac{\partial f(x,y)}{\partial y}=\pm\infty$. Оскільки $\frac{\partial f(x,y)}{\partial y}=\frac{1}{2\sqrt{y}}$, то отримаємо $\bar y(x)=0$ -- крива, що підозріла на особливу. Перевірка показує, що це дійсно інтегральна крива. Щоб до кінця переконатися, що ця крива особлива, розв’язуємо рівняння \[y'=\sqrt{y}\implies \frac{\diff y}{\sqrt{y}}=\diff x\implies 2\sqrt{y}=x+C\implies y(x)=\frac{(x+c)^2}{4}.\]
	Легко переконатися, що $\bar y(x)=0$ є кривою, що огинає сім’ю інтегральних кривих $y(x)=\frac{(x+c)^2}{4}$. 
\end{solution}

\begin{example}
	Знайти особливий розв’язок рівняння $y = x + y' - \ln y'$.
\end{example}
\begin{solution}
	Складаємо рівняння $p$-дискриминантної кривої \[ y = x + p - \ln p, \quad 0 = 1 - \frac{1}{p}.\] Із другого рівняння $p=1$. Підставивши в перше, отримаємо, що крива, що є підозрілою як особлива, має вигляд $\bar y(x)=x+1$. \\

	Підставивши у рівняння, отримаємо $x + 1=x+1-\ln 1$, тобто впевнились, що $\bar y(x)=x+1$ є інтегральною кривою. \\

	Розв’яжемо рівняння методом введення параметру. Його загальний розв’язок має вигляд \[y=Ce^x-\ln C.\] Можна переконатися, що $\bar y(x)=x+1$ є кривою, що огинає сім’ю інтегральних кривих. \\

	Щоб перевірити це аналітично, запишемо умову дотику кривої $y=x+1$ та $y=Ce^x-\ln C$ в точці $(x_0,y_0)$. Вона має вигляд: \[ \bar y(x_0) = y(x_0, C), \quad \bar{y}'(x_0) = y'(x_0,C).\] Тобто \[x_0+1=Ce^{x_0}-\ln C, \quad 1 = Ce^{x_0}.\] З другого рівняння отримаємо $C = e^{-x_0}$. Підставивши у перше рівняння, маємо $x_0 + 1 = 1 - \ln e^{-x_0}$, тобто $x_0+1=x_0+1$ -- тотожність. Таким чином при кожному $x_0$ відбувається дотик інтегральних кривих та $\bar y(x)=x+1$, що огинає сім’ю інтегральних кривих.
\end{solution}

Знайти особливі розв’язки та зробити рисунок.
\begin{multicols}{2}
\begin{problem}
	\[8\cdot(y')^3-27y=0;\]
\end{problem}
\begin{problem}
	\[(y'+1)^3-27\cdot(x+y)^2=0;\]
\end{problem}
\begin{problem}
	\[y^2\cdot((y')^2+1)=1;\]
\end{problem}
\begin{problem}
	\[(y')^2-4y^3=0.\]
\end{problem}
\end{multicols}

\end{document}