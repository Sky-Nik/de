Розглянемо деякі типи диференціальних рівнянь, що інтегруються в квадратурах.

\begin{enumerate}
\item Рівняння вигляду
\begin{equation*}
	%\label{eq:2.2.1}
	y^{(n)} = f(x).
\end{equation*}
Проінтегрувавши його $n$ разів одержимо загальний розв’язок у вигляді
\begin{equation*}
	%\label{eq:2.2.2}
	y = \underset{n}{\underbrace{\int \cdots \int}} f(x) \,\underset{n}{\underbrace{\diff x \cdots \diff x}} + C_1 x^{n-1} + C_2 x^{n-2} + \ldots + C_{n-1} x + C_n.
\end{equation*}
Якщо задані умови Коші
\begin{equation*}
	%\label{eq:2.1.3}
	y(x_0) = y_0, y'(x_0) = y_0', \ldots, y^{(n - 1)} (x_0) = y_0^{(n-1)},
\end{equation*}
то розв’язок має вигляд
\begin{multline*}
	%\label{eq:2.2.3}
	y = \underset{n}{\underbrace{\int_{x_0}^x \cdots \int_{x_0}^x}} f(t) \, \underset{n}{\underbrace{\diff t \cdots \diff t}} + \frac{y_0}{(n-1)!} \cdot (x-x_0)^{n-1} + \\ 
	+ \frac{y_0'}{(n-2)!} \cdot (x-x_0)^{n-2} + \ldots + y_0^{(n-2)} \cdot (x-x_0) + y_0^{(n-1)}.
\end{multline*}
\item Рівняння вигляду
\begin{equation*}
	%\label{eq:2.2.4}
	F\left(x, y^{(n)}\right) = 0.
\end{equation*}
Нехай це рівняння вдалося записати в параметричному вигляді
\begin{equation*}
	%\label{eq:2.2.5}
	\left\{
		\begin{aligned}
			x &= \phi(t), \\
			y^{(n)} &= \psi (t).
		\end{aligned}
	\right.
\end{equation*}
Використовуючи основне співвідношення $\diff y^{(n-1)} = y^{(n)} \cdot \diff x$, одержимо
\begin{equation*}
	%\label{eq:2.2.6}
	\diff y^{(n-1)} = \psi(t) \cdot \phi(t) \cdot \diff t
\end{equation*}
Проінтегрувавши його, маємо 
  \begin{equation*}
	%\label{eq:2.2.7}
	y^{(n-1)} = \int \psi(t) \cdot \phi(t) \cdot \diff t + C_1 = \psi_1(t, C_1).
\end{equation*}
І одержимо параметричний запис рівняння $(n-1)$-го порядку:
\begin{equation*}
	%\label{eq:2.2.8}
	\left\{
		\begin{aligned}
			x &= \phi(t), \\
			y^{(n-1)} &= \psi_1(t, C_1).
		\end{aligned}
	\right.
\end{equation*}
Проробивши зазначений процес ще $(n-1)$ раз, одержимо загальний розв’язок рівняння в параметричному вигляді
\begin{equation*}
	%\label{eq:2.2.9}
	\left\{
		\begin{aligned}
			x &= \phi(t), \\
			y &= \psi_n(t, C_1, \ldots, C_n).
		\end{aligned}
	\right.
\end{equation*}
 
\item Рівняння вигляду
\begin{equation*}
	%\label{eq:2.2.10}
	F \left( y^{(n-1)}, y^{(n)} \right) = 0.
\end{equation*}
Нехай це рівняння вдалося записати в параметричному вигляді 
\begin{equation*}
	%\label{eq:2.2.11}
	\left\{
		\begin{aligned}
			y^{(n-1)} &= \phi(t), \\
			y^{(n)} &= \psi(t).
		\end{aligned}
	\right.
\end{equation*}
Використовуючи основне співвідношення $\diff y^{(n-1)} = y^{(n)} \cdot \diff x$, одержуємо
\begin{equation*}
	%\label{eq:2.2.12}
	\diff x = \frac{\diff y^{(n-1)}}{y^{(n)}} = \frac{\phi'(t)}{\psi(t)} \cdot \diff t.
\end{equation*}
Проінтегрувавши, маємо
\begin{equation*}
	%\label{eq:2.2.13}
	x = \int \frac{\phi'(t)}{\psi(t)} \cdot \diff t + C_1 = \psi_1(t, C_1).
\end{equation*}
І одержали параметричний запис майже з попереднього пункту. \\

Використовуючи попередній пункт, запишемо загальний розв’язок у параметричному вигляді:
\begin{equation*}
	%\label{eq:2.2.14}
	\left\{
		\begin{aligned}
			x &= \psi(t, C_1), \\
			y &= \phi_n(t, C_2, \ldots, C_n).
		\end{aligned}
	\right.
\end{equation*}
 
\item Нехай рівняння вигляду
\begin{equation*}
	%\label{eq:2.2.15}
	F \left( y^{(n-2)}, y^{(n)} \right) = 0
\end{equation*}
можна розв'язати відносно старшої похідної
\begin{equation*}
	%\label{eq:2.2.16}
	y^{(n)} = f \left( y^{(n-2)} \right).
\end{equation*}
Домножимо його на $2 y^{(n-1)} \cdot \diff x$ й одержимо
\begin{equation*}
	%\label{eq:2.2.17}
	2 y^{(n-1)} \cdot y^{(n)} \cdot \diff x= 2 f \left( y^{(n-2)} \right) \cdot y^{(n-1)} \cdot \diff x.
\end{equation*}
Перепишемо його у вигляді
\begin{equation*}
	%\label{eq:2.2.18}
	\diff \left( y^{(n-1)} \right)^2 = 2 f \left( y^{(n-2)} \right) \cdot \diff y^{(n-2)}.
\end{equation*}
Проінтегрувавши, маємо
\begin{equation*}
	%\label{eq:2.2.19}
	\left( y^{(n-1)} \right)^2 = 2 \int f \left( y^{(n-2)} \right) \cdot \diff y^{(n-2)} + C_1,
\end{equation*}
тобто 
\begin{equation*}
	%\label{eq:2.2.20}
	y^{(n-1)}  = \pm \sqrt{2 \int f \left( y^{(n-2)} \right) \cdot \diff y^{(n-2)} + C_1},
\end{equation*}
або
\begin{equation*}
	%\label{eq:2.2.21}
	y^{(n-1)}  = \pm \psi_1 \left( y^{(n-2)}, C_1 \right).
\end{equation*}
Таким чином одержали повернулися до третього випадку.

\end{enumerate}