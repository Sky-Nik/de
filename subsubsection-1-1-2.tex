Розглянемо рівняння вигляду
\begin{equation}
	\label{eq:1.1.8}
	\frac{\diff y}{\diff x} = f(a x + b y + c)
\end{equation}
де $a$, $b$, $c$ -- сталі. \\

Зробимо заміну $a x + b y + c = z$. Тоді 
\begin{equation}
	\label{eq:1.1.8_5}
	a \cdot \diff x + b \cdot \diff y = \diff z, \quad \frac{\diff y}{\diff x} = \frac1b \cdot \left( \frac{\diff z}{\diff x} - a \right).
\end{equation}

Підставивши в \eqref{eq:1.1.8}, одержимо
\begin{equation}
	\label{eq:1.1.9}
	\frac1b \cdot \left( \frac{\diff z}{\diff x} - a \right) = f(z),
\end{equation}
або
\begin{equation}
	\label{eq:1.1.10}
	\frac{\diff z}{\diff x} = a + b \cdot f (z),
\end{equation}

Розділивши змінні, запишемо
\begin{equation}
	\label{eq:1.1.11}
	\frac{\diff z}{a + b \cdot f (z)} - \diff x = 0
\end{equation}
і
\begin{equation}
	\label{eq:1.1.12}
	\int \frac{\diff z}{a + b \cdot f (z)} - x = C.
\end{equation}

Загальний інтеграл має вигляд $\Phi(a x + b y + c, x) = C$.