Рівняння зі змінними, що розділяються могуть бути записані у вигляді $y' = f(x) \cdot g(y)$ або  $f_1(x) \cdot f_2(y) \cdot \diff x + g_1(x) \cdot g_2(y) \cdot \diff y$. Для розв’язків такого рівняння необхідно обидві частини помножити або розділити на такий вираз, щоб в одну частину входило тільки $x$, а в другу -- тільки $y$. Тоді обидві частини рівняння можна проінтегрувати.
Якщо ділити на вираз, що містить $x$ та $y$, може бути загублений розв’язок, що обертає цей вираз в нуль.

\begin{example}
	Розв’язати рівняння
	\begin{equation}
		\label{eq:1.13}
		x^2 y^2 y' + y = 1.
	\end{equation}
\end{example}

\begin{solution}
	Підставивши $y = \frac{\diff y}{\diff x}$ в задане рівняння, отримаємо 
	\begin{equation}
		\label{eq:1.14}
		x^2 y^2 \cdot \frac{\diff y}{\diff x} + y = 1.
	\end{equation}

	Помножимо обидві частини рівняння на $\diff x$ і розділимо на $x^2 \cdot (y - 1)$. Перевіримо, що $y = 1$ при цьому є розв’язком, а $x = 0$ цим розв’язком не є:
	\begin{equation}
		\label{eq:1.15}
		\frac{y^2}{y - 1} \cdot \diff y = - \frac{\diff x}{x^2}.
	\end{equation}

	Проінтегрируємо обидві частини рівняння:
	\begin{align}
		\label{eq:1.16}
		\int \frac{y^2}{y - 1} \cdot \diff y &= - \int \frac{\diff x}{x^2}. \\
		\frac{y^2}{2} + y + \ln |y - 1| &= \frac{1}{x} + C
	\end{align}
\end{solution}

\begin{example}
	Розв’язати рівняння
	\begin{equation}
		\label{eq:1.17}
		y' = \sqrt{4x + 2y - 1}.
	\end{equation}
\end{example}

\begin{solution}
	Введемо заміну змінних $z = 4 x + 2 y - 1$. Тоді $x' = 4 + 2 y'$. Рівняння перетвориться до вигляду $z' - 4 = 2 \sqrt{z}$; $z' = 4 + 2 \sqrt{z}$; $\frac{\diff z}{2 + \sqrt{z}} = 2 \diff x$. Проінтегруємо обидві частини рівняння:
	\begin{equation}
		\label{eq:1.18}
		\int \frac{\diff z}{2 + \sqrt{z}} = \int 2 \diff x
	\end{equation}
	Обчислимо інтеграл, що стоїть зліва. При обчисленні будемо використовувати таку заміну: 
	\begin{equation*}
		\sqrt{z} = t, \quad \diff z = 2 t \diff t, \quad 2 + \sqrt{z} = 2 + t,
	\end{equation*}
	\begin{multline}
		\label{eq:1.19}
		\int \frac{\diff z}{2 + \sqrt{z}} = \int \frac{2 t \diff t}{2 + t} = 2 \int \frac{t + 2 - 2}{t + 2} \cdot \diff t = \\
		= 2 t - 4 \ln |2 + t| = 2 \sqrt{z} - 4 \ln \left(2 + \sqrt{z}\right).
	\end{multline}
	Після інтегрування отримаємо $2 \sqrt{z} - 4 \ln \left(2 + \sqrt{z}\right) = 2 x + 2 C$. Зробимо обернену заміну: $z = 4x + 2y - 1$;
	\begin{equation}
		\label{eq:1.20}
		\sqrt{4x + 2y - 1} - 2 \ln \left(2 + \sqrt{4x + 2y - 1}\right) = x + C.
	\end{equation}
\end{solution}

Розв’язати рівняння:
\begin{multicols}{2}
\begin{problem}
	\[ x y \cdot \diff x + (x + 1) \cdot \diff y = 0; \]
\end{problem}
\begin{problem}
	\[ x \cdot (1 + y) \cdot \diff x = y \cdot (1 + x^2) \cdot \diff y; \]
\end{problem}
\begin{problem}
	\[ y' = 10^{x + y}; \]
\end{problem}
\begin{problem}
	\[ y' - xy^2 = 2xy; \]
\end{problem}
\begin{problem}
	\[ \sqrt{y^2 + 1} \diff x = x y \cdot \diff y; \]
\end{problem}
\begin{problem}
	\[ y' = x \tan (y); \]
\end{problem}
\begin{problem}
	\[ y y' + x = 1; \]
\end{problem}
\begin{problem}
	\[ 3 y^2 y' + 15 x = 2 x y^3; \]
\end{problem}
\begin{problem}
	\[ y' = \cos (y - x); \]
\end{problem}
\begin{problem}
	\[ y' - y = 2x - 3; \]
\end{problem}
\begin{problem}
	\[ x y' + y = y^2; \]
\end{problem}
\begin{problem}
	\[ e^{-y} \cdot (1 + y') = 1; \]
\end{problem}
\begin{problem}
	\[ 2 x^2 y y' + y^2 = 2; \]
\end{problem}
\begin{problem}
	\[ y' - x y^3 = 2 x y^2. \]
\end{problem}
\end{multicols}

Знайти частинні розв’язки, що задовольняють заданим початковим умовам:
\begin{problem}
	\[ (x^2 - 1) \cdot y' + 2 x y^2 = 0, \quad y(0) = 1; \]
\end{problem}
\begin{problem}
	\[ y' \cdot \cot (x) + y = 2, \quad y(0) = - 1; \]
\end{problem}
\begin{problem}
	\[ y' = 3 \sqrt[3]{y^2}, \quad y(2) = 0. \]
\end{problem}