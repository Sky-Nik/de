\begin{definition}
	Розв’язок $y = \phi(x)$ диференціального рівняння, в кожній точці якого $M(x,y)$ порушена єдиність розв’язку задачі Коші, називається особливим розв’язком. 
\end{definition}

Очевидно, особливі розв’язки треба шукати в тих точках області $D$, де порушені умови теореми про існування й єдиність розв’язку задачі Коші. Але, оскільки умови теореми носять достатній характер, то їхнє не виконання для існування особливих розв’язків, носить необхідний характер. І точки $N(x,y)$ області $D$, у яких порушені умови теореми про існування та єдиність розв’язку диференціального рівняння, є лише "підозрілими" на особливі розв’язки. \\

Розглянемо рівняння 
\begin{equation*}
	%\label{eq:1.6.30}
	y' = f(x,y).
\end{equation*}
Неперервність $f(x,y)$ в області $D$ зазвичай виконується, і особливі роз\-в'яз\-ки варто шукати там, де $\frac{\partial f}{\partial y} = +\pm \infty$. \\

Для диференціального рівняння, не роз\-в'яз\-а\-но\-го відносно похідної 
\begin{equation*}
	%\label{eq:1.6.31}
	F(x, y, y') = 0,
\end{equation*}
умови неперервності $F(x,y,y')$ й обмеженості $\frac{\partial F}{\partial y}$ зазвичай виконуються. І особливі розв’язки варто шукати там, де задовольняється остання рівність і 
\begin{equation*}
	%\label{eq:1.6.32}
	\frac{\partial F(x,y,y')}{\partial y'} = 0.
\end{equation*}

Вилучаючи із системи $y'$, одержимо $\Phi(x,y)=0$. Однак не в кожній точці $M(x,y)$, у якій $\Phi(x,y)$, порушується єдиність роз\-в'яз\-ку, тому що умови теореми мають лише достатній характер і не є необхідними. Якщо ж яка-небудь гілка $y=\phi(x)$ кривої $\Phi(x,y)$ є інтегральною кривою, то $y=\phi(x)$ називається особливим роз\-в'яз\-ком. \\

Таким чином, для знаходження особливого роз\-в'яз\-ку $F(x, y, y') = 0$ треба
\begin{enumerate}
	\item знайти $p$-дискримінантну криву з $F(x, y, y') = 0$ та $\frac{\partial F(x,y,y')}{\partial y'} = 0$.
	\item з'я\-су\-ва\-ти шляхом підстановки чи є серед гілок $p$-дискримінантної кривої інтегральні криві;
	\item з'я\-су\-ва\-ти чи порушена умова одиничності в точках цих кривих.
\end{enumerate}
