\begin{property}
	\label{prop:3.3.1}
	Якщо $y_0(x)$ -- розв’язок лінійного однорідного рівняння, $y_1(x)$ -- розв’язок неоднорідного рівняння, то $y(x) = y_0(x) + y_1(x)$ буде розв’язком лінійного неоднорідного диференціального рівняння.
\end{property}

\begin{proof}
	Дійсно, нехай $y_0(x)$ і $y_1(x)$ -- розв’язки відповідно однорідного і неоднорідного рівнянь, тобто
	\begin{align*}
		%\label{eq:3.3.2}
		a_0(x) \cdot y_0^{(n)}(x) + a_1(x) \cdot y_0^{(n - 1)}(x) + \ldots + a_n(x) \cdot y_0(x) &= 0, \\
		%\label{eq:3.3.3}
		a_0(x) \cdot y_1^{(n)}(x) + a_1(x) \cdot y_1^{(n - 1)}(x) + \ldots + a_n(x) \cdot y_1(x) &= b(x).
	\end{align*}	
	
	Тоді 
	\begin{multline*}
		%\label{eq:3.3.4}
		a_0(x) (y_0 + y_1)^{(n)}(x) + a_1(x) (y_0 + y_1)^{(n - 1)}(x) + \ldots + a_n(x) (y_0 + y_1)(x) = \\ = \left( a_0(x) \cdot y_0^{(n)}(x) + a_1(x) \cdot y_0^{(n - 1)}(x) + \ldots + a_n(x) \cdot y_0(x) \right) + \\ + \left( a_0(x) \cdot y_1^{(n)}(x) + a_1(x) \cdot y_1^{(n - 1)}(x) + \ldots + a_n(x) \cdot y_1(x) \right) = \\ = 0 + b(x) = b(x),
	\end{multline*}
	тобто $y(x) = y_0(x) + y_1(x)$ -- розв’язок неоднорідного диференціального рівняння.
\end{proof}

\begin{property}[Принцип суперпозиції]
	Якщо $y_i(x)$, $i = \overline{1, n}$ -- розв’язки лінійних неоднорідних диференціальних рівнянь
	\begin{equation*}
		%\label{eq:3.3.5}
		a_0(x) \cdot y^{(n)}(x) + a_1(x) \cdot y^{(n - 1)}(x) + \ldots + a_n(x) \cdot y(x) = b_i(x), \quad i = \overline{1, n}
	\end{equation*}
	то $y(x) = \sum_{i = 1}^n C_i \cdot y_i(x)$ з довільними сталими $C_i$ буде розв’язком лінійного неоднорідного рівняння
	\begin{equation*}
		%\label{eq:3.3.6}
		a_0(x) y^{(n)}(x) + a_1(x) y^{(n - 1)}(x) + \ldots + a_n(x) y(x) = \sum_{i = 1}^n C_i  b_i(x)
	\end{equation*}
\end{property}

\begin{proof}
	Дійсно, нехай $y_i(x)$, $i = \overline{1, n}$ -- розв’язки відповідних неоднорідних рівнянь, тобто
	\begin{equation*}
		%\label{eq:3.3.7}
		a_0(x) \cdot y_i^{(n)}(x) + a_1(x) \cdot y_i^{(n - 1)}(x) + \ldots + a_n(x) \cdot y_i(x) = b_i(x), \quad i = \overline{1, n}
	\end{equation*}

	Склавши лінійну комбінацію з рівнянь і їхніх правих частин з коефіцієнтами $C_i$ одержимо
	\begin{multline*}
		%\label{eq:3.3.8}
		\sum_{i = 1}^n C_i \cdot \left( a_0(x) \cdot y_i^{(n)}(x) + a_1(x) \cdot y_i^{(n - 1)}(x) + \ldots + a_n(x) \cdot y_i(x) \right) = \\ = \sum_{i = 1}^n C_i \cdot b_i(x),
	\end{multline*}
	або, перегрупувавши, запишемо
	\begin{multline*}
		%\label{eq:3.3.9}
		a_0(x) \cdot \left( \sum_{i = 1}^n C_i \cdot y_i^{(n)}(x) \right) + a_1(x) \cdot \left( \sum_{i = 1}^n C_i \cdot y_i^{(n - 1)}(x)\right) + \ldots \\ \ldots + a_n(x) \cdot \left( \sum_{i = 1}^n C_i \cdot y_i(x) \right) = \sum_{i = 1}^n C_i \cdot b_i(x),
	\end{multline*}
	що і було потрібно довести.
\end{proof}

\begin{property}
	Якщо комплексна функція $y(x) = u(x) + i v(x)$ з дійсними елементами є розв’язком лінійного неоднорідного рівняння з комплексною правою частиною $b(x) = f(x) + i p(x)$, то дійсна частина $u(x)$ є розв’язком рівняння з правою частиною $f(x)$, а уявна $v(x)$ є розв’язком рівняння з правою частиною $p(x)$.
\end{property}

\begin{proof}
	Дійсно, як випливає з умови,
	\begin{multline*}
	 	%\label{eq:3.3.10}
	 	a_0(x) \cdot (u + i v)^{(n)}(x) + a_1(x) \cdot (u + i v)^{(n - 1)}(x) + \ldots + a_n(x) \cdot (u + i v)(x) = \\ = f(x) + i p(x).
	\end{multline*}
	
	Розкривши дужки, одержимо
	\begin{multline*}
		%\label{eq:3.3.11}
		\left( a_0(x) \cdot u^{(n)}(x) + a_1(x) \cdot u^{(n - 1)}(x) + \ldots + a_n(x) \cdot u(x) \right) + \\ + i \left( a_0(x) \cdot v^{(n)}(x) + a_1(x) \cdot v^{(n - 1)}(x) + \ldots + a_n(x) \cdot v(x) \right) = \\ = f(x) + i p(x).
	\end{multline*}

	А комплексні вирази рівні між собою тоді і тільки тоді, коли дорівнюють окремо дійсні та уявні частини, тобто
	\begin{align*}
		%\label{eq:3.3.12}
		a_0(x) \cdot u^{(n)}(x) + a_1(x) \cdot u^{(n - 1)}(x) + \ldots + a_n(x) \cdot u(x) &= f(x), \\ 
		%\label{eq:3.3.13}
		a_0(x) \cdot v^{(n)}(x) + a_1(x) \cdot v^{(n - 1)}(x) + \ldots + a_n(x) \cdot v(x) &= p(x),
	\end{align*}
	що і було потрібно довести.
\end{proof}

\begin{theorem}
	Загальний розв’язок лінійного неоднорідного диференціального рівняння складається з загального розв’язку лінійного однорідного рівняння і частинного розв’язку неоднорідного рівняння.
\end{theorem}
\begin{proof}
	Нехай $y_{\text{homo}}(x) = \sum_{i = 1}^n C_i \cdot y_i(x)$ -- загальний розв’язок однорідного\footnote{Homogeneous equation* -- однорідне рівняння.} рівняння, а $y_{\text{hetero}}(x)$ -- частинний розв’язок неоднорідного\footnote{Heterogeneous equation* -- неоднорідне рівняння.} рівняння. \\

	Тоді, як випливає з властивості \ref{prop:3.3.1}, $y(x) = \sum_{i = 1}^n C_i \cdot y_i(x) + y_{\text{hetero}}(x)$, буде розв’язком неоднорідного рівняння. Покажемо, що цей розв’язок  загальний, тобто вибором коефіцієнтів $C_i$ можна розв’язати довільну задачу Коші
	\begin{equation*}
		%\label{eq:3.3.14}
		y(x_0) = y_0, \quad y'(x_0) = y_0', \quad \ldots, \quad y^{(n - 1)}(x_0) = y_0^{(n - 1)}.
	\end{equation*}

	Дійсно, оскільки $y_{\text{homo}}$ загальний роз\-в'яз\-ок однорідного рівняння, то $y_i$, $i = \overline{1, n}$ лінійно незалежні, тому визначник Вронського $W[y_1, y_2, \ldots, y_n] \ne 0$. Звідси, неоднорідна система лінійних алгебраїчних рівнянь 
	\begin{equation*}
		%\label{eq:3.3.15}
		\left\{ \begin{aligned}
			C_1 \cdot y_1(x_0) + C_2 \cdot y_2(x_0) + \ldots + C_n \cdot y_n(x_0) &= y_0 - y_{\text{hetero}}(x_0), \\
			C_1 \cdot y_1'(x_0) + C_2 \cdot y_2'(x_0) + \ldots + C_n \cdot y_n'(x_0) &= y_0' - y_{\text{hetero}}'(x_0), \\
			\ldots \ldots \ldots \ldots \ldots \ldots \ldots \ldots \ldots \ldots \ldots \ldots \ldots \ldots & \ldots \ldots \ldots \ldots \ldots \ldots 	\\
			C_1 \cdot y_1^{(n - 1)}(x_0) + C_2 \cdot y_2^{(n - 1)}(x_0) + \ldots + C_n \cdot y_n^{(n - 1)}(x_0) &= y_0 - y_{\text{hetero}}^{(n - 1)}(x_0),
		\end{aligned} \right.
	\end{equation*}
	має єдиний роз\-в'яз\-ок для довільних наперед обраних $y_0, y_0', \ldots, y_0^{(n - 1)}$. Нехай роз\-в'яз\-ком системи буде $C_1^0, C_2^0, \ldots, C_n^0$. Тоді, як випливає з вигляду системи, функція $y(x) = \sum_{i = 1}^n C_i^0 \cdot y_i(x) + y_{\text{hetero}}$ є роз\-в'яз\-ком поставленої задачі Коші.
\end{proof}

Як випливає з теореми для знаходження загального розв’язку лінійного неоднорідного рівняння треба шукати загальний розв’язок однорідного рівняння, тобто будь-які $n$ лінійно незалежні розв’язки і якийсь частинний розв’язок неоднорідного рівняння. Розглянемо три методи побудови частинного розв’язку лінійного неоднорідного рівняння.