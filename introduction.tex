% version 1.0
Наведемо декілька основних визначень теорії диференціальних рівнянь, що будуть використовуватися надалі.

\begin{definition}
	Рівняння, що містять похідні від шуканої функції та можуть містити шукану функцію та незалежну змінну, називаються диференціальними рівняннями.
\end{definition}

\begin{definition}
	Якщо в диференціальному рівнянні невідомі функції є функціями однієї змінної:
	\begin{equation*}
		%\label{eq:0.1}
		F \left( x, y, y', y'', \ldots, y^{(n)} \right) = 0,
	\end{equation*}
	то диференціальне рівняння називається звичайним.
\end{definition}

\begin{definition}
	Якщо невідома функція, що входить в диференціальне рівняння, є функцією двох або більшої кількості незалежних змінних:
	\begin{equation*}
		%\label{eq:0.2}
		F \left( x, y, z, \frac{\partial z}{\partial x}, \frac{\partial z}{\partial y}, \ldots, \frac{\partial^k z}{\partial x^\ell \partial y^{k - \ell}}, \ldots, \frac{\partial^n z}{\partial y^n} \right) = 0,
	\end{equation*}
	то диференціальне рівняння називається рівнянням в частинних похідних.
\end{definition}

\begin{definition}
	Порядком диференціального рівняння називається максимальний порядок похідної від невідомої функції, що входить в диференціальне рівняння.
\end{definition}

\begin{definition}
	Розв'язком диференціального рівняння називається функція, що має необхідну ступінь гладкості, і яка при підстановці в диференціальне рівняння обертає його в тотожність. 
\end{definition}

\begin{definition}
	Процес знаходження розв'язку диференціального рівняння називається інтегруванням диференціального рівняння.
\end{definition}
