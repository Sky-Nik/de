Розглянемо лінійні однорідні диференціальні рівняння з сталими коефіцієнтами
\begin{equation*}
	%\label{eq:3.2.1}
	y^{(n)} + a_1 \cdot y^{(n - 1)} + \ldots + a_n \cdot y = 0
\end{equation*}
Розв'язок будемо шукати у вигляді $y = e^{\lambda x}$. Продиференціювавши, одержимо 
\begin{equation*}
 	%\label{eq:3.2.2}
 	y' = \lambda \cdot e^{\lambda x}, \quad y'' = \lambda^2 \cdot e^{\lambda x}, \quad \ldots \quad y^{(n)} = \lambda^n \cdot e^{\lambda x}.
\end{equation*}
Підставивши $y', y'', \ldots, y^{(n)}$ в диференціальне рівняння, отримаємо
\begin{equation*}
	%\label{eq:3.2.3}
	\lambda^n \cdot e^{\lambda x} + a_1 \lambda^{n - 1}\cdot e^{\lambda x} + \ldots + a_n \cdot e^{\lambda x} = 0.
\end{equation*}
Скоротивши на $e^{\lambda x}$, одержимо характеристичне рівняння
\begin{equation*}
	%\label{eq:3.2.4}
	\lambda^n + a_1 \lambda^{n - 1} + \ldots + a_n = 0.
\end{equation*}
Алгебраїчне рівняння $n$-го степеня має $n$ коренів. У залежності від їхнього вигляду будемо мати різні розв'язки.
\begin{enumerate}
\item Нехай $\lambda_1, \lambda_2, \ldots, \lambda_n$ -- дійсні і різні. Тоді функції $e^{\lambda_1 x}, e^{\lambda_2 x}, \ldots, e^{\lambda_n x}$ є розв'язками й оскільки всі $\lambda_i$ різні, то $e^{\lambda_i x}$ -- розв'язки лінійно незалежні, тобто $\left\{ e^{\lambda_i x} \right\}_{i = 1}^n$ фундаментальна система розв'язків. Загальним розв'язком буде лінійна комбінація $y = \sum_{i=1}^n C_i \cdot e^{\lambda_i x}$.
\item Нехай маємо комплексно спряжені корені $\lambda=p+iq$, $\bar\lambda=p-iq$. Їм відповідають розв'язки $e^{(p+iq)x}$, $e^{(p-iq)x}$ . Розкладаючи їх по формулі Ейлера, одержимо: 
\begin{align*}
	%\label{eq:3.2.5}
	e^{(p+iq)x} &= e^{px} \cdot e^{iqx} = e^{px} \cdot (\cos qx + i \sin qx) = u(x) + i v(x), \\
	%\label{eq:3.2.6}
	e^{(p-iq)x} &= e^{px} \cdot e^{-iqx} = e^{px} \cdot (\cos qx - i \sin qx) = u(x) - i v(x).
\end{align*}
І, як випливає з властивості 4, функції $u(x)$ й $v(x)$ будуть окремими розв'язками. Таким чином, кореням $\lambda = p + iq$, $\bar\lambda = p - iq$ відповідають два лінійно незалежних розв'язки $u = e^{px} \cdot \cos qx$, $v = e^{px} \cdot \sin qx$. Загальним розв'язком, що відповідає цим двом кореням, буде $y = C_1 \cdot e^{px} \cdot \cos qx + C_2 \cdot e^{px} \cdot \sin x$.
\item Нехай $\lambda$ -- кратний корінь, кратності $k$, тобто $\lambda_1 = \lambda_2 = \ldots = \lambda_k$, $k\le n$.
\begin{enumerate}
\item Розглянемо випадок $\lambda=0$. Тоді характеристичне рівняння вироджується в рівняння
\begin{equation*}
	%\label{eq:3.2.7}
	\lambda^n + a_1 \lambda^{n - 1} + \ldots + a_{n - k} \lambda^k = 0.
\end{equation*}
 	Диференціальне рівняння, що відповідає цьому характеристичному, запишеться у вигляді
\begin{equation*}
	%\label{eq:3.2.8}
	y^{(n)} + a_1 \cdot y^{(n - 1)} + \ldots + a_{n-k} \cdot y^{(k)} = 0
\end{equation*}
 	Неважко бачити, що частковими, лінійно незалежними роз\-в'я\-з\-ка\-ми цього рівняння, будуть функції $1, x, x^2, \ldots, x^{k-1}$. Загальним роз\-в'я\-з\-ком, що відповідає кореню $\lambda=0$ кратності $k$, буде лінійна комбінація цих функцій $y = C_1 + C_2 \cdot x + \ldots + C_k \cdot x^{k - 1}$.
\item Нехай $\lambda = \nu \ne 0$ -- корінь дійсний. Зробивши заміну $y = e^{\nu x} \cdot z$, на підставі властивості 2 лінійних рівнянь після підстановки знову одержимо лінійне однорідне диференціальне рівняння 
\begin{equation*}
	%\label{eq:3.2.9}
	z^{(k)} + b_1 \cdot z^{(k-1)} + \ldots + b_k z = 0.
\end{equation*}
Причому, оскільки $y_i(x) = e^{\lambda_i x}$ а $x_i(x) = e^{\mu_i x}$, то показники $\lambda_i$, $\mu_i$ зв'язані співвідношенням $\lambda_i = \nu + \mu_i$. Звідси кореню $\lambda = \nu$ кратності $k$ відповідає корінь $\mu=0$ кратності $k$. Як випливає з попереднього пункту, кореню $\mu=0$ кратності $k$ відповідає загальний розв'язок вигляду $z = C_1 + C_2 \cdot x + \ldots + C_k \cdot x^{k - 1}$. \parvskip

З огляду на те, що $y = e^{\nu x} \cdot z$, одержимо, що кореню $\lambda=\nu$ кратності $k$ відповідає розв'язок
\begin{equation*}
	%\label{eq:3.2.10}
	y = \left(C_1 + C_2 \cdot x + \ldots + C_k \cdot x^{k - 1} \right) \cdot e^{\nu x}.
\end{equation*}

\item Нехай характеристичне рівняння має корені $\lambda=p+iq$, $\bar\lambda=p-iq$ кратності $k$. Проводячи аналогічні викладки одержимо, що їм відповідають лінійно незалежні розв'язки
\begin{equation*}
	%\label{eq:3.2.11}
	e^{px} \cdot \cos qx, \quad x \cdot e^{px} \cdot \cos qx, \quad \ldots, \quad x^{k-1} \cdot e^{px} \cdot \cos qx,
\end{equation*}
\begin{equation*}
	%\label{eq:3.2.12}
	e^{px} \cdot \sin qx, \quad x \cdot e^{px} \cdot \sin qx, \quad \ldots, \quad x^{k-1} \cdot e^{px} \cdot \sin qx.
\end{equation*}
І загальним розв'язком, що відповідає цим кореням буде
\begin{multline*}
	%\label{eq:3.2.13}
	y = C_1 e^{px} \cos qx + C_2 x e^{px} \cos qx + C_k x^{k-1} e^{px} \cos qx + \\ + C_{k+1} e^{px} \sin qx + C_{k+2} x e^{px} \sin qx + \ldots + C_{2k} x^{k-1} e^{px} \sin qx.
\end{multline*}
\end{enumerate}
\end{enumerate}
