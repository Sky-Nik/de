В евклідовому просторі $\RR^n$ змінних $x_1(t), x_2(t), \ldots, x_n(t)$ розв’язок $x_1 = x_1(t), x_2 = x_2(t), \ldots, x_n = x_n(t)$ визначає закон руху по деякій траєкторії в залежності від часу $t$. При такій інтерпретації функції $f_1, f_2, \ldots, f_n$ є складовими швидкості руху, простір зміни перемінних називається фазовим простором, система динамічної, а крива, по якій відбувається рух $x_1 = x_1(t), x_2 = x_2(t), \ldots, x_n = x_n(t)$ -- фазовою траєкторією. Фазова траєкторія є проекцією інтегральної кривої на фазовий простір.
